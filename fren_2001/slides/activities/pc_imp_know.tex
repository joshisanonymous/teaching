\begin{frame}{Savoir et connaître}
  \small
  Sur un papier, écris 4 phrases en suivant le modèle présenté (c'est-à-dire, avec une phrase qui utilise le verbe et le temps indiqué).
  Ensuite, circule dans la salle, et demande à tes camarades de classe si les phrases que tu as écrit sont vraies pour eux.
  Quand tu trouves une personne qui dit oui, écris son nom à côté de la phrase.
  \vspace{0.25cm}
  \begin{columns}
    \small
    \column{0.55\textwidth}
      Quand j'étais plus jeune, ...
      \begin{enumerate}
        \item {}(savoir / imparfait)
        \begin{itemize}
          \scriptsize
          \item[$\to$] \gloss{to know how}, suivi d'un verbe
        \end{itemize}
        \item {}(savoir / passé composé)
        \begin{itemize}
          \scriptsize
          \item[$\to$] \gloss{to find out}
        \end{itemize}
        \item {}(connaître / imparfait)
        \begin{itemize}
          \scriptsize
          \item[$\to$] \gloss{to be familiar with}, suivi d'un nom
        \end{itemize}
        \item {}(connaître / passé composé)
        \begin{itemize}
          \scriptsize
          \item[$\to$] \gloss{to meet}
        \end{itemize}
      \end{enumerate}
    \column{0.45\textwidth}
      \begin{itemize}
        \item[] \textbf{Modèle:} (savoir / imparfait)
        \item[] \emph{Tu écris:}
        \item Quand j'étais plus jeune, je savais jongler.
        \item[] \emph{Tu demandes:}
        \item Quand tu étais plus jeune, est-ce que tu savais jongler?
      \end{itemize}
  \end{columns}
\end{frame}