\begin{frame}[t]{Plus, moins ou aussi}
  \only<1>{
    \footnotesize
    Avec un(e) partenaire, écrivez des phrases à partir des fragments donnés pour comparer la vie \alert{en ville} à la vie \alert{à la campagne}.
    Il faudra mettre la phrases \alert{en ville} et \alert{à la campagne} dans le bon ordre quand vous les ajoutez.
    \begin{enumerate}
      \item[] \textbf{Modèle:} \emph{(\textcolor{red}{plus}) la circulation // être // intense}
      \item[] \emph{La circulation est \textcolor{red}{plus} intense \alert{en ville} qu'\alert{à la campagne}.}
      \item (moins) les maisons // coûter // cher
      \item (plus) la vie culturelle // être // actif
      \item (aussi) les gens // se comporter // poliment
      \item (plus) le travail // être // bien // payé
      \item (plus) la nourriture // être // bon et frais
      \item (moins) les rues // être // bruyant
      \item (plus) on // faire // des rénovations // souvent
      \item (aussi) les gens // travaillent // sérieusement
      \item (plus) on // pouvoir aller // fréquemment au cinéma
      \item (moins) les grands magasins // être // nombreux et élégant
    \end{enumerate}
  }
  \only<2>{
    Maintenant, discutez des phrases que vous avez écrites.
    Est-ce que vous êtes d'accord avec ces déclarations ou non? Pourquoi?
    \begin{description}
      \item[] \textbf{Modèle:}
      \item[] \emph{Le travail est mieux payé en ville qu'à la campagne.}
      \item[E1:] J'en suis d'accord. Ma mère travaille dans un hôpital et gagne beaucoup d'argent, mais il y a peu d'hôpitaux à la campagne.
      \item[E2:] C'est vrai, mais il y a d'autres emplois à la campagne. On peut être électricien, par exemple. Les maisons existent aussi à la campagne!
    \end{description}
  }
\end{frame}