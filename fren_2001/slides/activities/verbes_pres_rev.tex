\begin{frame}[t]{Le présent}
  \small
  Donnez la bonne conjugaison au présent.
  \vspace{0.25cm}
  \only<-13>{
    \begin{enumerate}
      \item Martin et Marie \underline{\uncover<2->{ont}} (avoir) deux cousines qui \underline{\uncover<3->{sont}} (être) polies.
      \item Si on \underline{\uncover<4->{fait}} (faire) le jardinage, on \underline{\uncover<5->{élève}} (élever) des légumes.
      \item Nous \underline{\uncover<6->{nous disputons}} (se disputer) de temps en temps, mais je \underline{\uncover<7->{m'entends}} (s'entendre) avec mes amis.
      \item Si vous \underline{\uncover<8->{n'obéissez pas}} (ne pas obéir) aux arrière-grands-parents, ils vous \underline{\uncover<9->{punissent}} (punir)!
      \item Au début, vous \underline{\uncover<10->{vous inscrivez}} (s'inscrire) à l'université, mais il y a d'autres gens qui \underline{\uncover<11->{obtiennent}} (obtenir) déjà des diplômes.
      \item Nous \underline{\uncover<12->{suivons}} (suivre) des cours, et nous \underline{\uncover<13->{réussissons}} (réussir) à chacun.
    \end{enumerate}
  }
  \only<14->{
    \begin{enumerate}
      \setcounter{enumi}{6}
      \item Sylvie \underline{\uncover<15->{époussette}} (épousseter) les étagères, pendant que Jean \underline{\uncover<16->{tond}} (tondre) la pelouse.
      \item Ils \underline{\uncover<17->{prennent}} (prendre) le manuel, puis ils le \underline{\uncover<18->{jettent}} (jeter)!
      \item Quand ma famille \underline{\uncover<19->{sort}} (sortir) à la plage, nous \underline{\uncover<20->{nageons}} (nager).
      \item Je dis que j'\underline{\uncover<21->{ouvre}} (ouvrir) la thèse pour la lire, mais je \underline{\uncover<22->{mens}} (mentir).
      \item Tu \underline{\uncover<23->{dois}} (devoir) souligner le texte, puis l'enseignant le \underline{\uncover<24->{voit}} (voir).
    \end{enumerate}
    \vspace{0.25cm}
    \raggedleft\raggedleft\hyperlink{début}{Au début}...
  }
\end{frame}