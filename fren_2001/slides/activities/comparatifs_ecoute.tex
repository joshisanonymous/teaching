\begin{frame}{}
  On va entendre des comparaisons entre deux villes (3-19, p.~97).
  Pour chaque comparaison, on va donner \emph{l'inverse} de ce qu'on entend.
  \vspace{0.5cm}
  \begin{columns}[t]
    \column{0.5\textwidth}
      \begin{itemize}
        \item[] \textbf{Modèle:}
        \item[] On entend: \emph{Paris est moins moderne que New York.}
        \item[] L'inverse: \emph{New York est plus moderne que Paris.}
      \end{itemize}
    \column{0.5\textwidth}
      Les villes:
      \begin{enumerate}
        \item Paris / Londres
        \item Cannes / Marseille
        \item Strasbourg / Bordeaux
        \item Clermont-Ferrand / Chamonix
        \item Rennes / Quimper
        \item Aix / Avignon
      \end{enumerate}
  \end{columns}
\end{frame}