\begin{frame}{La famille française}
  \only<1>{
    La famille française et le mariage changent actuellement.
    Le nombre de familles monoparentales augmente;
    beaucoup de femmes sont maintenant chefs de famille et travaillent au-dehors de la maison, tout en élevant leurs enfants.
    Il existe aussi beaucoup de familles recomposées dont les enfants vivent avec le parent biologique et un beau-parent.
    Le nombre de mariages diminue mais le nombre de couples se multiplie.
    De plus en plus d'hommes et de femmes décident de vivre ensemble et de signer un PACS.
    Par ailleurs, si les gens se marient plus tard -- les hommes à trente et un ans, les femmes à vingt-neuf -- le taux de divorce a augmenté et un mariage sur deux finit en divorce.
    La notion de famille se transforme donc, même si le sentiment de la famille perdure.
    Les liens familiaux restent très forts et les enfants vivent plus longtemps avec leurs parents.
  }
  \only<2->{
    \begin{enumerate}
      \item Quels types de famille augmentent?
      \item<3-> Qu'est-ce qu'une famille recomposée?
      \item<4-> Qu'est-ce que le PACS? Est-ce que quelque chose de pareil existe aux États-Unis?
      \item<5-> À quel âge est-ce que les hommes et les femmes se marient en général?
    \end{enumerate}
  }
\end{frame}