\begin{frame}{Adjectifs $\to$ Pronoms}
  \small
  Remplaçons les \emph{adjectifs} possessifs suivants avec des \emph{pronoms} possessifs.
  \begin{enumerate}
    \item \alert{\only<-1>{{Ses manières}}\only<2->{\textcolor{red}{Les siennes}}} sont moins raffinées que \alert{\only<-2>{nos manières}\only<3->{\textcolor{red}{les nôtres}}}.
    \item \alert{\only<-3>{Ton complexe d'infériorité}\only<4->{\textcolor{red}{Le tien}}} t'empêche de gravir les échelons de la hiérarchie sociale. \rule{3cm}{0pt}
    \item Nous vivons selon \alert{\only<-4>{nos moyens}\only<5->{\textcolor{red}{les nôtres}}}, mais ce jeune parvenu vit au-dessus de \alert{\only<-5>{ses moyens}\only<6->{\textcolor{red}{les siens}}}.
    \item \alert{\only<-6>{Votre sophistication}\only<7->{\textcolor{red}{La vôtre}}} vous aidera à être accepté par l'élite.
    \item Vous vous intéressez \alert{\only<-7>{au mode de vie des ouvriers}\only<8->{\textcolor{red}{au leur}}}.
    \item \alert{\only<-8>{Leur villa}\only<9->{\textcolor{red}{La leur}}} est mille fois plus chic que \alert{\only<-9>{ma villa}\only<10->{\textcolor{red}{la mienne}}}.
    \item \alert{\only<-10>{Son savoir-faire}\only<11->{\textcolor{red}{Le sien}}} est inférieur \alert{\only<-11>{à votre savoir-faire}\only<12->{\textcolor{red}{au vôtre}}}.
    \item \alert{\only<-12>{Mon revenu}\only<13->{\textcolor{red}{Le mien}}} ne correspond pas du tout à \alert{\only<-13>{ta classe sociale}\only<14->{\textcolor{red}{la tienne}}}.
    \item \alert{\only<-14>{Mon héritage}\only<15->{\textcolor{red}{Le mien}}} est plus important que \alert{\only<-15>{l'héritage d'oncle Joseph}\only<16->{\textcolor{red}{le sien}}}.
    \item \alert{\only<-16>{Le mode de vie de ce mendiant}\only<17->{\textcolor{red}{Le sien}}} nous fait de la peine.
  \end{enumerate}
\end{frame}