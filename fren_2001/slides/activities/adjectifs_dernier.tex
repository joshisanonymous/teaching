\begin{frame}{Adjectifs}
  \footnotesize
  Mettez l'adjectif dans sa place avec les bons accords.
  \begin{enumerate}
    \item C'est un \funderline[2cm]{} député \funderline[2cm]{\uncover<2->{réactionnaire}} (réactionnaire) qui a posé sa candidature.
    \item Mon neveu a épousé une \funderline[2cm]{} femme \funderline[2cm]{\uncover<3->{bourgeoise}} (bourgeois).
    \item On ne peut pas trouver une \funderline[2cm]{\uncover<4->{meilleure}} belle-mère \funderline[2cm]{} (meilleur) à respecter.
    \item Ils négligent leurs travaux parce qu'ils se considèrent des \funderline[2cm]{} hommes \funderline[2cm]{\uncover<5->{égaux}} (égal) à des cadres supérieurs.
    \item Sa \funderline[2cm]{\uncover<6->{vieille}} cousine \funderline[2cm]{} (vieux) s'épanouit dans son travail.
    \item Ce \funderline[2cm]{\uncover<7->{nouvel}} emploi \funderline[2cm]{} (nouveau) donne l'occasion de lutter contre la criminalité.
    \item Vers les gares, il y a trop de \funderline[2cm]{} circulation \funderline[2cm]{\uncover<8->{déroutante}} (déroutant).
  \end{enumerate}
  \raggedleft\raggedleft\hyperlink{début}{Au début}...
\end{frame}