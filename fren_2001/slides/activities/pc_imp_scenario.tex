\begin{frame}{À partir d'un scénario}
  \scriptsize
  En groupes de 4, écrivez un paragraphe qui décrit un scénario dans une histoire.
  L'histoire peut être une histoire que vous connaissez d'un film ou d'un livre ou une histoire que vous inventez, mais \alert{il faut utiliser l'imparfait} pour cette description (c'est-à-dire, imaginez que vous racontez à quelqu'un un scénario que vous avez véçu).
  La description peut comprendre ce qui suit:
  \begin{columns}
    \column{0.5\textwidth}
      \begin{enumerate}
        \item L'endroit
        \item Les choses dans l'endroit ou autour de l'endroit
        \item L'atmosphère (par ex., le temps ou l'éclairage)
        \item Les personnes présentes (une description, pas simplement des noms)
        \item La période (par ex., l'antiquité, les années 1960, le futur)
        \item La situation générale (par ex., pendant une guerre ou un discours public ou au travail)
      \end{enumerate}
      \emph{Donnez autant de details que possible!}
    \column{0.5\textwidth}
      \begin{itemize}
        \item<2-> Maintenant, donnez votre description à un autre groupe pour que chaque groupe ait \emph{un} scénario qui n'est pas le sien.
          Toujours dans vos groupes, écrivez \emph{ce qui s'est passé} dans ce scénario sur la même feuille de papier.
          Présentez les événements par ordre chronologique.
          \alert{Il faut utiliser le passé composé} pour cette partie de l'activité.
        \item<3-> Enfin, lisez votre histoire à la classe.
          Chaque personne doit lire un peu.
      \end{itemize}
  \end{columns}
\end{frame}