\begin{frame}{Français Américains: Ces différences qui nuos rapprochent}
  \only<1>{
    En groupes de 3 ou 4, lisez le texte suivant des pages 17 et 18 du manuel. Chaque personne devrait lire une phrase à tour de rôle \gloss{taking turns}.
    Répondez aux questions à mesure que vous lisez \gloss{as you read}.
  }
  \only<2>{
    En France, les parents ne se contentent pas <<d'élever>> ou de <<faire grandir \gloss{to grow}>> leurs enfants.
    Le mot qui vient le plus fréquemment aux lèvres pour décrire ce processus est <<éducation>>.
    On n'élève pas ses enfants, on les éduque, et ce mot a une connotation beaucoup plus large que sa traduction littérale américaine \gloss{to educate} qui se limite à l'éducation scolaire.
    L'éducation englobe \gloss{includes} à la fois l'apprentissage \gloss{learning} de la vie de famille, de le vie en société et également les études.
    La responsabilité de transmettre à l'enfant ce qui va lui servir et faire de lui un futur gagnant \gloss{winner} dans la vie incombe \gloss{is the responsibility} presque exclusivement aux parents. Ceci est beaucoup moins vrai aux États-Unis.
    \begin{itemize}
      \item {}<<On n'élève pas ses enfants, on les éduque>>, qu'est-ce que ça veut dire?
    \end{itemize}
  }
  \only<3>{
    En général, aux États-Unis, les parents dévouent \gloss{devote} toute leur attention à l'enfant; en France, l'enfant compte aussi, bien sûr, mais il s'inscrit \gloss{comes within the scope of} dans un cadre social et familial plus contraignant qui comporte ses propres règles et ses propres limites.
    Les parents américains donnent à l'enfant toute possibilité de s'exprimer.
    Ils essaient de se mettre à sa place et font tout ce qui est en leur pouvoir pour encourager son épanouissement \gloss{well-being}, quitte à\gloss{even if it means} renoncer à leur propre confort.
    \begin{itemize}
      \item {}<<[S]e mettre à sa place>>, qu'est-ce que ça veut dire?
    \end{itemize}
  }
  \only<4>{
    Les parents français de leur côté aiment tout autant leur enfant, mais ils pensent que leur devoir est avant tout de lui apprendre à vivre en société et à s'adapter au monde des adultes: ce sont eux qui ont la responsabilité de guider l'enfant jusqu'à sa maturité et de l'aider à appréhender le monde avec un regard adulte.
    En France, les enfants sont généralement propres \gloss{potty trained} dès deux ans -- sans, de toute évidence, d'effet nocif \gloss{harmful} sur leur sexualité adulte -- pour le meilleur confort et la réputation de leurs parents.
    Cet âge varie aux États-Unis en fonction des familles et des pratiques mais il se situe généralement aux alentours de trois ans.
    \begin{itemize}
      \item {}<<[S]'adapter au monde des adultes>>, qu'est-ce que ça veut dire?
    \end{itemize}
  }
  \only<5>{
    On a par ailleurs \gloss{moreover} en France une piètre \gloss{poor} opinion des méthodes et capacités d'éducation des parents dont les enfants ne se tiennent \gloss{behave} pas correctement en public.
    Les parents on un devoir à remplir envers la société: celui d'exercer un contrôle sur leurs enfants et de leur apprendre à vivre en société.
    S'ils n'y parviennent \gloss{achieve} pas, n'importe quel inconnu pourra se permettre de faire des réflexions sur le comportement du rejeton \gloss{offspring} et parfois même le réprimander directement pour compenser les lacunes \gloss{deficiencies} de l'éducation parentale.
    \begin{itemize}
      \item {}<<[N]'importe quel inconnu pourra se permettre de faire des réflexions sur le comportement>>, qu'est-ce que ça veut dire?
    \end{itemize}
  }
\end{frame}