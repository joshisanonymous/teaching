\begin{frame}{Associons des mots}
  \only<1>{
    Quels sont des mots, des phrases ou des idées que vous associez aux mots suivants?
    Pour chaque mot dans la liste, énumérez autant d'associations que vous pouvez sur une feuille de papier.
    Par exemple: \emph{un programme $\to$ l'économie, l'égalité, un parti, des problèmes sociaux, faire des réformes, être élu, ...}
  }
  \only<3>{
    Enfin, regroupez-vous avec une autre paire d'étudiants et discutez encore de vos associations.
    Est-ce que vos associations sont identiques ou différentes?
    Quelles sont les différences et les similarités entre vos idées?
  }
  \begin{columns}
    \column{0.5\textwidth}
      \begin{center}
        \includegraphics[scale=0.2]{réseau.png}
      \end{center}
    \column{0.5\textwidth}
      \begin{enumerate}
        \item l'Union européenne
        \item une grève
        \item une monarchie
        \item démissionner
        \item une crise
        \item la démocratie
        \item un citoyen
      \end{enumerate}
  \end{columns}
  \only<2>{
    Maintenant, avec un.e partenaire, combinez vos associations et discutez-en.
    Est-ce que vous avez choisi les mêmes associations? Pourquoi?
    Est-ce que vous avez choisi des associations différentes? Pourquoi?
  }
\end{frame}