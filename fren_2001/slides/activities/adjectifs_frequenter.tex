\begin{frame}{Un endroit fréquenté}
  \only<1>{
    Quels sont les endroits à Athens que vous fréquentez?
    Est-ce qu'il y a des bâtiments? Des magasins? Un quartier? Des cafés ou des bars?
    Cherchez un.e autre étudiant.e qui fréquente un endroit que vous fréquentez.
    \begin{description}
      \item[] \textbf{Modèle:}
      \item[E1:] Bonjour! Quels endroits fréquentes-tu?
      \item[E2:] Moi, je fréquente la bibliothèque, le café Walkers, le magasin Fringe et mon église.
      \item[E1:] Ah, moi, je fréquente la bibliothèque, aussi!
    \end{description}
  }
  \only<2-3>{
    \alert{Maintenant}, ensemble, discutez de l'endroit que vous fréquentez, vous deux.
    \begin{enumerate}
      \item Comment est-ce que vous décrivez cet endroit? Quels adjectifs sont appropriés?
      \item Et les gens qu'on trouve à cet endroit? Comment sont-ils?
      \item Qu'est-ce qu'on trouve dans cet endroit à part les gens?
    \end{enumerate}
    \uncover<3>{
      \vspace{0.25cm}
      \alert{Ensuite}, écrivez un paragraphe qui décrit cet endroit.
      Utilisez \emph{beaucoup d'adjectifs} pour l'endroit, les gens et les choses à cet endroit.
    }
  }
  \only<4->{
    \alert{Puis}, présentez votre endroit à la classe.
    Pour ceux qui écoutent:
    \begin{itemize}
      \item Notez les adjectifs utilisés pour décrire l'endroit.
      \item Écrivez-les pour aider à vous rappeler d'eux.
    \end{itemize}
    \uncover<5->{
      \alert{Enfin}, quels endroits ont quelles qualités?
    }
  }
\end{frame}