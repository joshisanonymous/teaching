\begin{frame}{Réflexion culturelle}
  \only<1>{
    En groupes de 3 ou 4, lisez le texte suivant de la page 17 du manuel. Chaque personne devrait lire une phrase à tour de rôle \gloss{taking turns}.
  }
  \only<2>{
    La famille française et la famille américaine ont une optique et des valeurs différentes que l'éducation des enfants révèle.
    Les parents américains sont généralement plus décontractés \gloss{laid back} et moins à cheval sur les principes \gloss{rigid about rules} que les parents français.
    Ils laissent leurs enfants explorer plus le monde environnant et s'adaptent au rythme de l'enfant.
    Au lieu de multiplier les interdits, ils ont tendance à créer un milieu qui permette à l'enfant d'observer et de se découvrir.
    En revanche, les parents français, eux, veulent que leur enfant soit très vite intégré à la société et sache bien se conduire en public, d'où les règles et les restrictions qui accompagnent l'enfance.
    Selon les Français, la façon dont un enfant se comporte reflète les aptitudes parentales de l'adulte, presque plus que son propre développement psychologique.  
  }
  \only<3>{
    Répondez à ces questions dans vos groupes.
    \begin{enumerate}
      \item Est-ce que les familles françaises et les familles américaines ont une même optique sur l'éducation des enfants?
      \item En général, quels parents sont les plus rigides?
      \item Qu'est-ce que les parents américains laissent leurs enfants faire?
      \item Pour les parents français, que doit faire un enfant?
      \item Que reflète le comportement d'un enfant selon les Français?
    \end{enumerate}
  }
\end{frame}