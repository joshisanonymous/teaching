\begin{frame}[t]{Un(e) candidat(e) idéal(e)}
  \footnotesize
  \begin{columns}
    \column{0.5\textwidth}
      Tu a de la chance!
      Une élection hypothétique arrive, et ton/ta candidat(e) idéal(e) est sur le bulletin de vote.
      Comment est ce(tte) candidat(e)?
      Réfléchis au sujets suivants, puis écris un paragraphe pour décrire ton/ta candidat(e) idéal(e).
    \column{0.5\textwidth}
      \begin{center}
        \includegraphics[scale=0.08]{candidat_ideal.jpg}
      \end{center}
  \end{columns}
  \only<1>{
    \begin{columns}[t]
      \column{0.5\textwidth}
        \begin{itemize}
          \item Est-ce que le/la candidat(e) est d'un parti spécifique?
          \item Le/La candidat(e) est de droite, de gauche ou centriste?
          \item Le/La candidat(e) s'intéresse à quels problèmes? Quel est son programme?
          \item Est-ce que le/la candidat(e) est riche ou pauvre?
        \end{itemize}
      \column{0.5\textwidth}
        \begin{itemize}
          \item Est-ce que le/la candidat(e) a une identité spécifique? Un genre? Une race ou ethnicité?
          \item Est-ce que la religion est importante? La laïcité?
          \item Est-ce que c'est un(e) citoyen(ne) typique ou un(e) politicien(ne) de carrière?
        \end{itemize}
    \end{columns}
  }
  \only<2>{
    \vspace{1cm}
    Maintenant, discute de ton/ta candidat(e) idéal(e) avec un(e) partenaire.
    Est-ce que vous votez pour des candidats similaires ou différentes? Pourquoi?
    \emph{Après finir}, discute de ton/ta candidat(e) avec encore une autre personne de la même façon, puis avec une autre personne, etc.
  }
\end{frame}