\begin{frame}[t]{Definissons des mots}
  Avec un.e partenaire, prenez cinq papiers sur lesquels s'écrient des mots du vocabulaire.
  Ensuite, \alert{en français}, écrivez des définitions pour ces mots ou des descriptions qui conduirait une autre personne à déterminer les mots.
  Soyez assez précis que seul un mot peut correspond à la définition/description!
  \only<1>{
    \begin{itemize}
      \item[] \textbf{Modèle:} \emph{un criminel}
      \item {}<<Cette personne agresse les gens dans les rues pour de l'argent.>>
      \item[] OU
      \item {}<<Samuel Bankman-Fried (SBF)>>
    \end{itemize}
    \begin{flushright}
      \includegraphics[scale=0.125]{sbf.jpg}
    \end{flushright}
  }
  \only<2->{

    \vspace{0.5cm}
    Maintenant, échangez vos définitions et faites ce qui suit:
    \begin{enumerate}
      \item Déterminez le mot du vocabulaire qui correspond à chaque définition.
      \item Choisissez une classe sociale parmi celles: \emph{la classe ouvrière}, \emph{la classe moyenne}, \emph{la bourgeoisie}, \emph{la grande bourgeoisie}.
      \item Ensemble, écrivez un paragraphe qui décrit la classe que vous avez choisie en utilisant \alert{tous} les mots que vous avez reçus.
    \end{enumerate}
  }
\end{frame}