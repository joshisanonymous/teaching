\begin{frame}{Richard Wright, écrivain des États-Unis}
  \begin{columns}
    \column{0.4\textwidth}
      \begin{center}
        \href{https://www.ina.fr/ina-eclaire-actu/video/i05133764/richard-wright-a-propos-du-livre-fishbelly}{\includegraphics[scale=0.25]{richard_wright.png}}
      \end{center}
    \column{0.6\textwidth}
      \small
      Richard Wright est né à Natchez, Mississippi en 1908.
      Comme adolescent, il a suivi \underline{\uncover<2->{des}} cours à Smith-Robertson Junior High School où il est devenu \underline{\uncover<3->{le}} Valedictorian.
      Son roman Native Son, publié en 1940, a rencontré \underline{\uncover<4->{un}} grand succès.
      Plus tard dans sa vie, pour échapper \only<-4>{à }\only<5->{\st{à} }\underline{\uncover<5->{aux}} poursuites \only<-5>{de }\only<6->{\st{de} }\underline{\uncover<6->{du}} gouvernement fédéral américain, il est parti se réfugier en France.
      En 1947, il a pris \underline{\uncover<7->{la}} nationalité française.
      Là où il a rencontré \underline{\uncover<8->{des}} figures telles que Sartre et Camus, il a passé \underline{\uncover<9->{le}} reste de sa vie.
  \end{columns}
\end{frame}