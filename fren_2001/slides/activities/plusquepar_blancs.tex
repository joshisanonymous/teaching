\begin{frame}{La défense d'un président}
  \scriptsize
  Ensemble, complétons les blancs avec l'\alert{imparfait}, le \alert{passé composé} ou le \alert{plus-que-parfait}.
  \begin{enumerate}
    \item Hier, le Président \underline{\uncover<2->{a fait}} (faire) un discours
    \item pour expliquer ce qu'il \underline{\uncover<3->{souhaitait}} (souhaiter) accomplir durant la dernière année de son mandat.
    \item D'abord, il \underline{\uncover<4->{a dit}} (dire)
    \item qu'il \underline{\uncover<5->{voulait}} (vouloir) réduire la dette
    \item qui \underline{\uncover<6->{avait beaucoup augmenté}} (beaucoup augmenter) ces dernières années.
    \item Ensuite, il \underline{\uncover<7->{a précisé}} (préciser) que
    \item la crise \underline{\uncover<8->{n'était pas}} (ne pas être) finie
    \item mais que, l'année précédente, il \underline{\uncover<9->{avait tout fait}} (tout faire) pour y remédier.
    \item Il \underline{\uncover<10->{a parlé}} (parler) du chômage
    \item qui \underline{\uncover<11->{constituait}} (constituer) une préoccupation majeure des électeurs.
    \item Il \underline{\uncover<12->{a indiqué}} (indiquer) que les travailleurs
    \item qui \underline{\uncover<13->{avaient récemment perdu}} (perdre récemment) leur emploi ne seraient pas oubliés.
    \item Finalement, il \underline{\uncover<14->{a évoqué}} (évoquer) les élections mais sans insister.
  \end{enumerate}
\end{frame}