\begin{frame}{Les traitements}
  \small
  En groupes de 3 ou 4, comparez les vues de la société envers les hommes et les femmes dans les paires suivantes.
  Est-ce qu'on traite de la même façon ou différemment?
  Est-ce que les deux vivent \gloss{experience} les mêmes difficultés?
  Est-ce que les deux bénéficent des mêmes avantages?
  Pourquoi?
  \begin{columns}
    \small
    \column{0.4\textwidth}
      \begin{enumerate}
        \item avocat.e
        \item marchand.e
        \item étranger/étrangère
        \item champion.ne
        \item musicien.ne
        \item directeur/directrice
        \item veuf/veuve
        \item chirurgien.ne
        \item étudiant.e
      \end{enumerate}
    \column{0.6\textwidth}
      \begin{description}
        \item[] \textbf{Modèle:} \emph{écrivain.e}
        \item[E1:] Moi, je pense que les écrivaines ont plus de difficultés que les écrivains.
        \item[E2:] Je suis d'accord, parce que les écrivaines utilisent un nom de plume d'un homme quelquefois.
        \item[E3:] C'est vrai, mais il y a des écrivaines aussi fameuses que les écrivains: JK Rowling, par exemple.
      \end{description}
  \end{columns}
\end{frame}