\begin{frame}{La boxeuse Imane Khelif}
  \begin{columns}[t]
    \scriptsize
    \column{0.5\textwidth}
      BBC, 2 août 2024 - L'Algérienne Imane Khelif, l'une des deux boxeuses impliquées dans une controverse sur le genre aux Jeux olympiques de Paris 2024, a obtenu l'or après avoir battu en finale la Chinoise Yang Liu.
      Khelif est, avec la Taïwanaise Lin Yu-ting, l'une des deux boxeuses qui ont fait l'objet d'un examen approfondi pour participer à la boxe féminine à Paris.
      Il ne s'agit pas de leur première participation aux Jeux olympiques, puisqu'elles ont participé à Tokyo 2020, où elles ont été battues.
      Mais une controverse sur son sexe fait rage depuis l'année dernière et est aujourd'hui ravivée par sa participation aux Jeux Olympiques de Paris.
      Une grande partie de la controverse qui a suivi la nouvelle de la participation de Khelif et Lin à Paris 2024 découle d'une disqualification qu'elles ont subie lors des championnats du monde féminins de l'année dernière en Inde.
      L'Association internationale de boxe (IBA), qui n'est pas reconnue par le CIO depuis 2019, les a retirées de l'événement après que des tests ont
    \column{0.5\textwidth}
      déterminé qu'elles <<ne répondaient pas aux critères d'éligibilité>>.
      Le CIO a déclaré qu'il s'agissait de <<niveaux élevés de testostérone>>.
      Mais des commentaires ultérieurs du Président de l'IBA auraient insinué que Khelif et Lin n'avaient pas les chromosomes XX du sexe biologique féminin, mais XY du sexe masculin.
      Après le combat de jeudi, l'IBA a publié un communiqué indiquant que les boxeuses <<n'ont pas subi de test de testostérone, mais un test indépendant et reconnu, dont les détails sont confidentiels>>.
      \begin{itemize}
        \small
        \item Est-ce qu'il est juste que Imane a participé aux Jeux olympiques?
        \item Quel est l'importance des chromosomes et de la testostérone dans les sports féminins, à votre avis?
      \end{itemize}
  \end{columns}
\end{frame}