\begin{frame}{Homme ou femme}
  Avec des partenaires, prononcez les mots suivants à tour de rôle \gloss{taking turns}.
  Vous pouvez les prononcez au masculin (avec \lexi{le}) ou au féminin (avec \lexi{la}).
  La tâche de vos partenaires est de déterminer si la personne est un homme ou une femme.
  \begin{columns}
    \column{0.46\textwidth}
      \begin{description}
        \item[] \textbf{Modèle:} \emph{docteur(e)}
        \item[E1:] La docteure.
        \item[E2:] Est-ce que c'est une femme?
        \item[E1:] Oui, c'est une femme!
      \end{description}
    \column{0.26\textwidth}
      \begin{enumerate}
        \item diplomate
        \item riche
        \item fonctionnaire
        \item professeur.e
        \item retraité.e
        \item journaliste
        \item psychologue
      \end{enumerate}
    \column{0.23\textwidth}
      \begin{enumerate}
        \setcounter{enumi}{7}
        \item docteur.e
        \item marié.e
        \item cycliste
        \item réfugié.e
        \item ministre
        \item patriote
        \item fiancé.e
      \end{enumerate}
  \end{columns}
\end{frame}