\begin{frame}{Sur le métro}
  \begin{columns}
    \column{0.5\textwidth}
      Deux étudiantes, Caroline et Véronique, rentrent en métro après une longue journée à la fac à l'Université de Paris VII (3-3, p.~72).
      Écoutons la conversation pour répondre aux questions avec <<vrai>> ou <<faux>>.
    \column{0.5\textwidth}
      \begin{center}
        \includegraphics[scale=0.05]{métro.jpg}
      \end{center}
  \end{columns}
  \vspace{0.25cm}
  \begin{enumerate}
    \item Caroline préfère les grandes villes à la campagne.\hfill\underline{\uncover<2->{\ vrai}}
    \item Le métro est confortable.\hfill\underline{\uncover<3->{faux}}
    \item Il y a des pickpockets dans le métro.\hfill\underline{\uncover<4->{\ vrai}}
    \item Véronique veut que Caroline l'invite chez ses parents.\hfill\underline{\uncover<5->{\ vrai}}
    \item Il y a autant de monde à la campagne qu'en ville.\hfill\underline{\uncover<6->{faux}}
  \end{enumerate}
\end{frame}