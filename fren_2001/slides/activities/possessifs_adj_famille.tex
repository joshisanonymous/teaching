\begin{frame}{L'origine sociale de ta famille}
  Avec un/e partenaire, discute de l'origine sociale de \alert{ta} famille.
  Est-ce que \alert{tes} parents ou \alert{tes} grands-parents faisaient partie de la même classe sociale que toi?
  Sinon, quelles étaient les classes sociales de \alert{ta} mère, \alert{ton} père, \alert{tes} grands-mères, \alert{tes} grands-pères, etc.
  Est-ce que tu as des oncles ou des tantes qui ont changé de classe sociale?
  Utilise des \emph{adjectifs possessifs} comme \alert{mon}, \alert{ton}, \alert{son}, \alert{leur}, etc.
  \begin{itemize}
    \item[] \textbf{Modèle:}
    \item[E1:] \alert{Mon} grand-père vient d'un milieu social modeste. \alert{Ses} parents à lui étaient des immigrants allemands. \alert{Mon} grand-père voulait réussir et alors il a passé \alert{sa}
    vie à travailler dur. Il a rencontré \alert{sa} future épouse, \alert{ma} grand-mère, à l'université. Ils voulaient que \alert{leurs} enfants aient un niveau d'instruction élevé.
    \item[E2:] C'est intéressant! Quels cours est-ce que \alert{ton} grand-père et \alert{ta} grand-mère a suivis ensemble?
  \end{itemize}
\end{frame}