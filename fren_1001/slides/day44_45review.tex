%%%%%%%%%%%%%%%%%%%%%%%%%%%%%%%%%%%%%
%                                   %
% Compile with XeLaTeX and biber    %
%                                   %
% Questions or comments:            %
%                                   %
% joshua dot mcneill at uga dot edu %
%                                   %
%%%%%%%%%%%%%%%%%%%%%%%%%%%%%%%%%%%%%

\documentclass{beamer}
  % Read in standard preamble (cosmetic stuff)
  %%%%%%%%%%%%%%%%%%%%%%%%%%%%%%%%%%%%%%%%%%%%%%%%%%%%%%%%%%%%%%%%
% This is a standard preamble used in for all slide documents. %
% It basically contains cosmetic settings.                     %
%                                                              %
% Joshua McNeill                                               %
% joshua dot mcneill at uga dot edu                            %
%%%%%%%%%%%%%%%%%%%%%%%%%%%%%%%%%%%%%%%%%%%%%%%%%%%%%%%%%%%%%%%%

% Beamer settings
% \usetheme{Berkeley}
\usetheme{CambridgeUS}
% \usecolortheme{dove}
% \usecolortheme{rose}
\usecolortheme{seagull}
\usefonttheme{professionalfonts}
\usefonttheme{serif}
\setbeamertemplate{bibliography item}{}

% Packages and settings
\usepackage{fontspec}
  \setmainfont{Charis SIL}
\usepackage{hyperref}
  \hypersetup{colorlinks=true,
              allcolors=blue}
\usepackage{graphicx}
  \graphicspath{{../../figures/}}
\usepackage{soul}
  \setstcolor{red}
\usepackage[normalem]{ulem}
\usepackage{enumerate}
\usepackage{tikz}
  \usetikzlibrary{trees}

% Document information
\author{M. McNeill}
\title[FREN1001]{Français 1001}
\institute{\url{joshua.mcneill@uga.edu}}
\date{}

%% Custom commands
% Lexical items
\newcommand{\lexi}[1]{\textit{#1}}
% Gloss
\newcommand{\gloss}[1]{`#1'}
\newcommand{\tinygloss}[1]{{\tiny`#1'}}
% Orthographic representations
\newcommand{\orth}[1]{$\langle$#1$\rangle$}
% Utterances (pragmatics)
\newcommand{\uttr}[1]{`#1'}
% Sentences (pragmatics)
\newcommand{\sent}[1]{\textit{#1}}
% Fixed length underlines
\newcommand{\funderline}[2][4cm]{
  \underline{\makebox[\ifdim\width>#1\width\else#1\fi]{#2}}
}
% Base dir for definitions
\newcommand{\defs}{../definitions}
\newcommand{\activity}[1]{
  \input{./activities/#1.tex}
}


  % Packages and settings

  % Document information
  \subtitle[Révision: Examen final]{La révision pour l'examen final}

\begin{document}
  % Read in the standard intro slides (title page and table of contents)
  \begin{frame}
    \titlepage
    \tiny{Office: % Basically a variable for office hours location
Zoom (ID 978 2791 8221)
\\
          Office hours: % Basically a variable for office hours
 mercredi 10h15--13h15
}
  \end{frame}

  \begin{frame}{Renseignements}
    \begin{itemize}
      \item Le 14 décembre, 7h du soir.
      \item[] \tinygloss{December 14th, 7pm.}
    \end{itemize}
  \end{frame}

  \begin{frame}
    \tableofcontents[hideallsubsections]
  \end{frame}

  \section{Les quantités}
    \begin{frame}{Les quantités}
      Comment pouvons-nous quantifier ces aliments?
      \begin{description}
        \item[] \textbf{Modèle:} \emph{des baguettes}
        \item[E1:] beaucoup de baguettes
        \item[E2:] quinze baguettes
      \end{description}
      \begin{columns}
        \column{0.5\textwidth}
          \begin{enumerate}
            \item des croissants
            \item des œufs
            \item du chocolat chaud
            \item du beurre
          \end{enumerate}
          \column{0.5\textwidth}
          \begin{enumerate}
            \setcounter{enumi}{4}
            \item du thon
            \item de l'oignon
            \item du vin
            \item des glaçons
          \end{enumerate}
      \end{columns}
    \end{frame}

  \section{Les verbes au présent}
    \begin{frame}{Les verbes au présent}
      Comment conjuguons-nous ces verbes au présent?
      \begin{description}
        \item[] \textbf{Modèle:} \emph{tu \underline{donnes} (to give)}
      \end{description}
      \begin{columns}
        \column{0.5\textwidth}
          \begin{enumerate}
            \item tu \underline{\uncover<2->{sers}} (to serve)
            \item elles \underline{\uncover<3->{prennent}} (to take)
            \item nous \underline{\uncover<4->{buvons}} (to drink)
            \item j' \underline{\uncover<5->{attends}} (to wait)
          \end{enumerate}
          \column{0.5\textwidth}
          \begin{enumerate}
            \setcounter{enumi}{4}
            \item vous \underline{\uncover<6->{devez}} (to have to)
            \item ils \underline{\uncover<7->{veulent}} (to want)
            \item il \underline{\uncover<8->{met}} (to put)
            \item nous \underline{\uncover<9->{nous levons}} (to get up)
          \end{enumerate}
      \end{columns}
    \end{frame}

  \section{Les verbes au passé composé}
    \begin{frame}{Les verbes au passé composé}
      Comment conjuguons-nous ces verbes au passé composé?
      \begin{description}
        \item[] \textbf{Modèle:} \emph{tu \underline{as donné} (to give)}
      \end{description}
      \begin{columns}[t]
        \column{0.5\textwidth}
          \begin{enumerate}
            \item tu \underline{\uncover<2->{as travaillé}} (to work)
            \item elles \underline{\uncover<3->{ont acheté}} (to buy)
            \item nous \underline{\uncover<4->{sommes allé(e)s}} (to go)
            \item je \underline{\uncover<5->{suis devenu(e)}} (to become)
          \end{enumerate}
          \column{0.5\textwidth}
          \begin{enumerate}
            \setcounter{enumi}{4}
            \item vous \underline{\uncover<6->{avez jeté}} (to throw)
            \item ils \underline{\uncover<7->{se sont reposés}} (to relax)
            \item il \underline{\uncover<8->{se sont rentré}} (to return home)
            \item nous \underline{\uncover<9->{avons commandé}} (to order)
          \end{enumerate}
      \end{columns}
    \end{frame}

  \section{Les articles}
    \begin{frame}{Les articles}
      Quel article faut-il?
      \begin{description}
        \item[] \textbf{Modèle:} \emph{Je préfère \underline{les} pommes.}
      \end{description}
      \begin{columns}
        \column{0.5\textwidth}
          \begin{enumerate}
            \item Pourquoi est-ce que tu n'aimes pas \underline{\uncover<2->{le}} thé?
            \item Elle bois \underline{\uncover<3->{du}} café.
            \item Quand est-ce que je prends beaucoup \underline{\uncover<4->{de}} café?
            \item Nous détestons \underline{\uncover<5->{les}} épinards.
          \end{enumerate}
          \column{0.5\textwidth}
          \begin{enumerate}
            \setcounter{enumi}{4}
            \item Où est-ce qu'il achète \underline{\uncover<6->{des}} épinards?
            \item Vous mangez \underline{\uncover<7->{une}} saucisse.
            \item Ils adorent \underline{\uncover<8->{la}} viande.
            \item Nous voulons \underline{\uncover<9->{un}} morceau \underline{\uncover<10->{de}} viande.
          \end{enumerate}
      \end{columns}
    \end{frame}

  \section{Le pronom \lexi{en}}
    \begin{frame}{Le pronom \lexi{en}}
      Où faut-il mettre le pronom \lexi{en}?
      \begin{description}
        \item[] \textbf{Modèle:} \emph{Je \underline{ en } mange \underline{\hspace{0.5cm}} tous les jours.}
      \end{description}
      \begin{columns}[t]
        \scriptsize
        \column{0.5\textwidth}
          \begin{enumerate}
            \item Tu \underline{\hspace{0.35cm}} vas \underline{ \uncover<2->{en} } prendre \underline{\hspace{0.35cm}} au restaurant.
            \item Elle \underline{ \uncover<3->{en} } bois \underline{\hspace{0.35cm}} une tasse.
            \item Nous \underline{ \uncover<4->{en} } prenons \underline{\hspace{0.35cm}} beaucoup.
            \item Nous \underline{\hspace{0.35cm}} n'\underline{ \uncover<5->{en} } avons \underline{\hspace{0.35cm}} pas \underline{\hspace{0.35cm}} bu \underline{\hspace{0.35cm}} assez.
          \end{enumerate}
          \column{0.5\textwidth}
          \begin{enumerate}
            \setcounter{enumi}{4}
            \item Ils \underline{\hspace{0.35cm}} peuvent \underline{ \uncover<6->{en} } commander \underline{\hspace{0.35cm}} avant 9h du soir.
            \item Vous \underline{ \uncover<7->{en} } avez \underline{\hspace{0.35cm}} mangé \underline{\hspace{0.35cm}} dix.
            \item Je \underline{\hspace{0.35cm}} ne \underline{\hspace{0.35cm}} vais \underline{\hspace{0.35cm}} pas \underline{ \uncover<8->{en} } servir \underline{\hspace{0.35cm}} quinze.
            \item Nous \underline{ \uncover<8->{en} } servons \underline{\hspace{0.35cm}} trois \underline{\hspace{0.35cm}} assiettes \underline{\hspace{0.35cm}} le matin.
          \end{enumerate}
      \end{columns}
    \end{frame}
\end{document}
