%%%%%%%%%%%%%%%%%%%%%%%%%%%%%%%%%%%%%
%                                   %
% Compile with XeLaTeX and biber    %
%                                   %
% Questions or comments:            %
%                                   %
% joshua dot mcneill at uga dot edu %
%                                   %
%%%%%%%%%%%%%%%%%%%%%%%%%%%%%%%%%%%%%

\documentclass{beamer}
  % Read in standard preamble (cosmetic stuff)
  %%%%%%%%%%%%%%%%%%%%%%%%%%%%%%%%%%%%%%%%%%%%%%%%%%%%%%%%%%%%%%%%
% This is a standard preamble used in for all slide documents. %
% It basically contains cosmetic settings.                     %
%                                                              %
% Joshua McNeill                                               %
% joshua dot mcneill at uga dot edu                            %
%%%%%%%%%%%%%%%%%%%%%%%%%%%%%%%%%%%%%%%%%%%%%%%%%%%%%%%%%%%%%%%%

% Beamer settings
% \usetheme{Berkeley}
\usetheme{CambridgeUS}
% \usecolortheme{dove}
% \usecolortheme{rose}
\usecolortheme{seagull}
\usefonttheme{professionalfonts}
\usefonttheme{serif}
\setbeamertemplate{bibliography item}{}

% Packages and settings
\usepackage{fontspec}
  \setmainfont{Charis SIL}
\usepackage{hyperref}
  \hypersetup{colorlinks=true,
              allcolors=blue}
\usepackage{graphicx}
  \graphicspath{{../../figures/}}
\usepackage[normalem]{ulem}
\usepackage{enumerate}

% Document information
\author{M. McNeill}
\title[FREN2001]{Français 2001}
\institute{\url{joshua.mcneill@uga.edu}}
\date{}

%% Custom commands
% Lexical items
\newcommand{\lexi}[1]{\textit{#1}}
% Gloss
\newcommand{\gloss}[1]{`#1'}
\newcommand{\tinygloss}[1]{{\tiny`#1'}}
% Orthographic representations
\newcommand{\orth}[1]{$\langle$#1$\rangle$}
% Utterances (pragmatics)
\newcommand{\uttr}[1]{`#1'}
% Sentences (pragmatics)
\newcommand{\sent}[1]{\textit{#1}}
% Base dir for definitions
\newcommand{\defs}{../definitions}


  % Packages and settings

  % Document information
  \subtitle[Dates]{Les dates importantes}

\begin{document}
  % Read in the standard intro slides (title page and table of contents)
  \begin{frame}
    \titlepage
    \tiny{Office: % Basically a variable for office hours location
Gilbert 121\\
          Office hours: % Basically a variable for office hours
 lundi, mercredi, vendredi 10:10--11:10
}
  \end{frame}

  \begin{frame}{Annonces}
    \begin{itemize}
      \item Department French tables have begun (check eLC)
      \item Be careful to distinguish between \orth{e}, \orth{è}, and \orth{é}
      \item Again, please don't cheat
    \end{itemize}
  \end{frame}

  \begin{frame}{Les mois}
    \begin{columns}
      \column{0.5\textwidth}
        \begin{enumerate}
          \item janvier
          \item février
          \item mars
          \item avril
          \item mai
          \item juin
        \end{enumerate}
      \column{0.5\textwidth}
        \begin{enumerate}
          \setcounter{enumi}{6}
          \item juillet
          \item août
          \item septembre
          \item octobre
          \item novembre
          \item décembre
        \end{enumerate}
    \end{columns}
    \vspace{1cm}
    Quelle est date? \underline{\uncover<2->{le 31 août}}
  \end{frame}

  \begin{frame}{Les associations}
    \begin{enumerate}
      \item une paire $\to$ \underline{\uncover<2->{2}}
      \item l'alphabet $\to$ \underline{\uncover<3->{26}}
      \item le premier $\to$ \underline{\uncover<4->{1}}
      \item les mois $\to$ \underline{\uncover<5->{12}}
      \item l'indépendance américaine $\to$ \underline{\uncover<6->{4}}
    \end{enumerate}
  \end{frame}

  \begin{frame}{}
    \begin{center}
      \Large Quiz
    \end{center}
  \end{frame}

  \begin{frame}{Les anniversaires}
    \gloss{In groups of 3 or 4, take turns asking each person when their birthday is.
    For example:}
    \begin{description}
      \item[E1:] Ton anniversaire, c'est quel jour?
      \item[E2:] C'est le 30 septembre.
    \end{description}
  \end{frame}
  % Follow by asking who has a group member with birthdays on certain days or in
  % certain months.

  \begin{frame}{Cours de mathématiques}
    \gloss{Work with a partner and take turns giving each other simple math problems using addition (\lexi{plus}) or subtraction (\lexi{moins}).
    For example:}
    \begin{description}
      \item[E1:] 10 plus 2, ça fait combien?
      \item[E2:] Ça fait 12.
    \end{description}
  \end{frame}

  \begin{frame}{}
    \begin{center}
      \Large Questions?
    \end{center}
  \end{frame}
\end{document}
