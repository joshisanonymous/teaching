%%%%%%%%%%%%%%%%%%%%%%%%%%%%%%%%%%%%%
%                                   %
% Compile with XeLaTeX and biber    %
%                                   %
% Questions or comments:            %
%                                   %
% joshua dot mcneill at uga dot edu %
%                                   %
%%%%%%%%%%%%%%%%%%%%%%%%%%%%%%%%%%%%%

\documentclass{beamer}
  % Read in standard preamble (cosmetic stuff)
  %%%%%%%%%%%%%%%%%%%%%%%%%%%%%%%%%%%%%%%%%%%%%%%%%%%%%%%%%%%%%%%%
% This is a standard preamble used in for all slide documents. %
% It basically contains cosmetic settings.                     %
%                                                              %
% Joshua McNeill                                               %
% joshua dot mcneill at uga dot edu                            %
%%%%%%%%%%%%%%%%%%%%%%%%%%%%%%%%%%%%%%%%%%%%%%%%%%%%%%%%%%%%%%%%

% Beamer settings
% \usetheme{Berkeley}
\usetheme{CambridgeUS}
% \usecolortheme{dove}
% \usecolortheme{rose}
\usecolortheme{seagull}
\usefonttheme{professionalfonts}
\usefonttheme{serif}
\setbeamertemplate{bibliography item}{}

% Packages and settings
\usepackage{fontspec}
  \setmainfont{Charis SIL}
\usepackage{hyperref}
  \hypersetup{colorlinks=true,
              allcolors=blue}
\usepackage{graphicx}
  \graphicspath{{../../figures/}}
\usepackage[normalem]{ulem}
\usepackage{enumerate}

% Document information
\author{M. McNeill}
\title[FREN2001]{Français 2001}
\institute{\url{joshua.mcneill@uga.edu}}
\date{}

%% Custom commands
% Lexical items
\newcommand{\lexi}[1]{\textit{#1}}
% Gloss
\newcommand{\gloss}[1]{`#1'}
\newcommand{\tinygloss}[1]{{\tiny`#1'}}
% Orthographic representations
\newcommand{\orth}[1]{$\langle$#1$\rangle$}
% Utterances (pragmatics)
\newcommand{\uttr}[1]{`#1'}
% Sentences (pragmatics)
\newcommand{\sent}[1]{\textit{#1}}
% Base dir for definitions
\newcommand{\defs}{../definitions}


  % Packages and settings

  % Document information
  \subtitle[Petit déjeuner et passé composé (\lexi{avoir})]{Le petit-déjeuner et le passé composé avec \lexi{avoir}}

\begin{document}
  % Read in the standard intro slides (title page and table of contents)
  \begin{frame}
    \titlepage
    \tiny{Office: % Basically a variable for office hours location
Gilbert 121\\
          Office hours: % Basically a variable for office hours
 lundi, mercredi, vendredi 10:10--11:10
}
  \end{frame}

  \begin{frame}{Annonces}
    \begin{itemize}
      \item Le devoir 5 est à rendre le 15 (mercredi)
      \item[] \tinygloss{Homework 5 is due the 15th (Wednesday)}
    \end{itemize}
  \end{frame}

  \begin{frame}{}
    \begin{center}
      \Large Quiz
    \end{center}
  \end{frame}

  \begin{frame}{Plus de conjugaisons!}
    Quelle est la bonne conjugaison au passé composé? \\
    \tinygloss{What is the correct conjugation in the passé composé?}
    \begin{enumerate}
      \item Nous \underline{\uncover<2->{avons mangé}} (manger) du pain et du bacon il y a longtemps.
      \item Elle \underline{\uncover<3->{a pris}} (prendre) un pain au chocolat à ce moment-là.
      \item J\underline{\uncover<4->{'ai bu}} (boire) du café au lait avec ma tartine.
      \item Est-ce que vous \underline{\uncover<5->{avez goûté}} (goûter) cette bonne confiture?
    \end{enumerate}
  \end{frame}

  \begin{frame}{Normalement, mais...}
    Raconte avec un/e partenaire tes habitudes et les exceptions.
    Utilise \alert{le présent} pour tes habitudes et \alert{le passé composé} pour les exceptions. \\
    \tinygloss{Talk about your habits but also the exceptions.
    Use \alert{the present} for your habits and \alert{the passé composé} for the exceptions.}
    \begin{description}
      \item[] \textbf{Modèle:} \emph{dormir}
      \item[E1:] Normalement, je \alert{dors} jusqu'à sept heures, mais samedi dernier j'\alert{ai dormi} jusqu'à dix heures.
    \end{description}
    \begin{columns}[t]
      \column{0.5\textwidth}
        \begin{enumerate}
          \item dormir
          \item manger
          \item quitter la maison
          \item travailler
        \end{enumerate}
      \column{0.5\textwidth}
        \begin{enumerate}
          \setcounter{enumi}{4}
          \item jouer
          \item regarder à la télé
          \item boire
          \item prendre le soir
        \end{enumerate}
    \end{columns}
  \end{frame}

  \begin{frame}{Passez l'histoire!}
    Imaginons une histoire ensemble pour chaque scénario.
    Chaque personne donne une phrase \emph{au passé composé} à partir de celle de la personne précédente. \\
    \tinygloss{Let's imagine a story together for each scenario.
    Each person gives a sentence in \emph{passé composé} building off the previous person's.}
    \begin{columns}
      \column{0.5\textwidth}
        \begin{description}
          \item[] \textbf{Modèle:}
          \item[] \emph{Jean et Luke au stade hier}
          \item[E1:] Hier, Jean et Luke ont regardé le match.
          \item[E2:] Leur équipe a gagné.
          \item[E3:] Après, ils ont pris des cocas, mais Luke n'aime pas le coca.
        \end{description}
      \column{0.5\textwidth}
        \begin{enumerate}
          \item Jean et Luke au stade hier
          \item Kim dans un café avec sa mère et sa sœur la semaine dernière
        \end{enumerate}
    \end{columns}
  \end{frame}

  \begin{frame}{Qu'est-ce qu'ils ont fait?}
    Avec un/e partenaire, imagine autant d'activités que possible que ces personnes ont faites aux endroits mentionnés. \\
    \tinygloss{With a partner, imagine as many activities as possible that these people did at the places mentioned.}
    \begin{description}
      \item[] \textbf{Modèle:} \emph{Qu'est-ce que Julie a fait dans le magasin hier?}
      \item[E1:] Elle a travaillé. (Elle est vendeuse.)
      \item[E2:] Elle a mis une robe.
    \end{description}
    \begin{columns}[t]
      \column{0.5\textwidth}
        Qu'est-ce que...
        \begin{enumerate}
          \item ... vous avez fait au restaurant hier?
          \item ... les Aubert ont fait à la piscine l'été dernier?
          \item ... tu as fait au supermarché hier?
          \item ... on a fait en cours il y a deux jours?
        \end{enumerate}
      \column{0.5\textwidth}
        \begin{enumerate}
          \setcounter{enumi}{4}
          \item ... tu as fait chez toi hier soir?
          \item ... Clément a fait au café avant-hier?
          \item ... vos camarades ont fait chez eux le week-end dernier?
          \item ... le prof a fait dans son burean ce matin?
        \end{enumerate}
    \end{columns}
  \end{frame}

  \begin{frame}{}
    \begin{center}
      \Large Questions?
    \end{center}
  \end{frame}
\end{document}
