%%%%%%%%%%%%%%%%%%%%%%%%%%%%%%%%%%%%%
%                                   %
% Compile with XeLaTeX and biber    %
%                                   %
% Questions or comments:            %
%                                   %
% joshua dot mcneill at uga dot edu %
%                                   %
%%%%%%%%%%%%%%%%%%%%%%%%%%%%%%%%%%%%%

\documentclass{beamer}
  % Read in standard preamble (cosmetic stuff)
  %%%%%%%%%%%%%%%%%%%%%%%%%%%%%%%%%%%%%%%%%%%%%%%%%%%%%%%%%%%%%%%%
% This is a standard preamble used in for all slide documents. %
% It basically contains cosmetic settings.                     %
%                                                              %
% Joshua McNeill                                               %
% joshua dot mcneill at uga dot edu                            %
%%%%%%%%%%%%%%%%%%%%%%%%%%%%%%%%%%%%%%%%%%%%%%%%%%%%%%%%%%%%%%%%

% Beamer settings
% \usetheme{Berkeley}
\usetheme{CambridgeUS}
% \usecolortheme{dove}
% \usecolortheme{rose}
\usecolortheme{seagull}
\usefonttheme{professionalfonts}
\usefonttheme{serif}
\setbeamertemplate{bibliography item}{}

% Packages and settings
\usepackage{fontspec}
  \setmainfont{Charis SIL}
\usepackage{hyperref}
  \hypersetup{colorlinks=true,
              allcolors=blue}
\usepackage{graphicx}
  \graphicspath{{../../figures/}}
\usepackage{soul}
  \setstcolor{red}
\usepackage[normalem]{ulem}
\usepackage{enumerate}
\usepackage{tikz}
  \usetikzlibrary{trees}

% Document information
\author{M. McNeill}
\title[FREN1001]{Français 1001}
\institute{\url{joshua.mcneill@uga.edu}}
\date{}

%% Custom commands
% Lexical items
\newcommand{\lexi}[1]{\textit{#1}}
% Gloss
\newcommand{\gloss}[1]{`#1'}
\newcommand{\tinygloss}[1]{{\tiny`#1'}}
% Orthographic representations
\newcommand{\orth}[1]{$\langle$#1$\rangle$}
% Utterances (pragmatics)
\newcommand{\uttr}[1]{`#1'}
% Sentences (pragmatics)
\newcommand{\sent}[1]{\textit{#1}}
% Fixed length underlines
\newcommand{\funderline}[2][4cm]{
  \underline{\makebox[\ifdim\width>#1\width\else#1\fi]{#2}}
}
% Base dir for definitions
\newcommand{\defs}{../definitions}
\newcommand{\activity}[1]{
  \input{./activities/#1.tex}
}


  % Packages and settings

  % Document information
  \subtitle[Articles et adverbes]{Les articles de toilette et les adverbes}

\begin{document}
  % Read in the standard intro slides (title page and table of contents)
  \begin{frame}
    \titlepage
    \tiny{Office: % Basically a variable for office hours location
Zoom (ID 978 2791 8221)
\\
          Office hours: % Basically a variable for office hours
 mercredi 10h15--13h15
}
  \end{frame}

  \begin{frame}{}
    \begin{center}
      \Large Quiz
    \end{center}
  \end{frame}

  \begin{frame}{Combien? \gloss{How many?}}
    Avec un/e partenaire, dis combien de choses suivantes que tu as en utilisant les adverbes de quantité (\lexi{trop de}, \lexi{beaucoup de}, \lexi{assez de}, \lexi{peu de}, \lexi{ne...pas de}). \\
    \tinygloss{With a partner, say how many of the following things you have using quantity adverbs (\lexi{trop de}, \lexi{beaucoup de}, \lexi{assez de}, \lexi{peu de}, \lexi{ne...pas de}).}
    \begin{columns}[t]
      \column{0.38\textwidth}
        \begin{description}
          \item[] \textbf{Modèle:}
          \item[] \emph{des livres}
          \item[E1:] J'ai beaucoup de livres.
          \item[E2:] Moi, j'ai peu de livres.
        \end{description}
      \column{0.31\textwidth}
        \begin{enumerate}
          \item des livres
          \item des cours
          \item des rasoirs
          \item des serviettes
          \item des peignes
        \end{enumerate}
      \column{0.31\textwidth}
        \begin{enumerate}
          \setcounter{enumi}{5}
          \item du maquillage
          \item des amis
          \item de l'argent
          \item des problèmes
        \end{enumerate}
    \end{columns}
  \end{frame}

  \begin{frame}{La matinée}
    \small
    Avec un/e partenaire, répondez aux questions suivantes en utilisant les adverbes de fréquences (\lexi{tous les ...}, \lexi{souvent}, \lexi{quelquesfois}, \lexi{ne...jamais}). \\
    \tinygloss{With a partner, answer the following questions using frequency adverbs (\lexi{tous les ...}, \lexi{souvent}, \lexi{quelquesfois}, \lexi{ne...jamais}).}
    \begin{columns}
      \scriptsize
      \column{0.4\textwidth}
        \begin{description}
          \item[] \textbf{Modèle:}
          \item[] \emph{Numéro 1}
          \item[E1:] Oui, je prends deux douches tous les jours.
          \item[] \tinygloss{Yes, I take two showers every day.}
          \item[E2:] Non, je ne prends pas deux douches tous les jours. Je prends rarement une douche.
          \item[] \tinygloss{Non, I don't take two showers every day. I rarely take a shower.}
        \end{description}
      \column{0.6\textwidth}
        Est-ce que ...
        \begin{enumerate}
          \item ... tu prends deux douches ou deux bains tous les jours?
          \item ... tu te laves les cheveux tous les jours?
          \item ... tu te brosses les dents après chaque repas \gloss{meal}?
          \item ... tu te coiffes trois ou quatre fois par jour?
          \item ... tu t'habilles différemment chaque jour?
          \item ... tu te maquilles ou tu te rases tous les jours?
          \item ... tu te mets du parfum ou de l'eau de Cologne tous les jours?
          \item ... tu te mets des bijoux \gloss{jewelry} tous les jours?
        \end{enumerate}
    \end{columns}
  \end{frame}

  \begin{frame}{Les stéréotypes \gloss{Stereotypes}}
    Avec un/e partenaire, discutez si les stéréotypes suivants sont \emph{vrais} \gloss{true} ou \emph{faux} \gloss{false}.
    Utilisez les adverbes de fréquence ou d'intensité (\lexi{trop}, \lexi{beaucoup}, \lexi{assez}, \lexi{un peu}, \lexi{ne..pas}). \\
    \tinygloss{With a partner, discuss if the following stereotypes are true or false.
    Use frequency or intensity adverbs (\lexi{trop}, \lexi{beaucoup}, \lexi{assez}, \lexi{un peu}, \lexi{ne..pas}).}
    \begin{columns}[t]
      \column{0.5\textwidth}
        \scriptsize
        \begin{description}
          \item[] \textbf{Modèle:}
          \item[] \emph{les jeunes / aime trop McDo}
          \item[E1:] C'est vrai! Les jeunes aiment trop McDo.
          \item[] \tinygloss{That's true! Young people like McDonalds too much.}
          \item[E2:] Non, c'est faux! Les jeunes n'aiment pas McDo.
          \item[] \tinygloss{No, that's false! Young people don't like McDonalds.}
        \end{description}
      \column{0.5\textwidth}
        \scriptsize
        \begin{enumerate}
          \item les jeunes / manger beaucoup de fast-foods
          \item les grands-parents / s'endormir toujours devant la télé
          \item les parents / se coucher trop tôt
          \item les musiciens / se coucher trop tard
          \item les sportifs / ne jamais étudier
          \item les étudiants / ne pas avoir d'argent
          \item les animaux de compagnie / se lever trop tôt
          \item les professeurs / donner trop de devoirs
        \end{enumerate}
    \end{columns}
  \end{frame}

  \begin{frame}[t]{Il m'en faut un! \gloss{I need one of them!}}
    Tu as deux cartes: une carte avec une activité et une carte avec un objet que tu tiens \gloss{that you hold}.
    Circule la salle pour trouver l'objet dont tu as besoin \gloss{that you need} pour faire l'activité sur ta carte. \\
    \tinygloss{You have two cards: a card with an activity and a card with an object that you hold.
    Go around the room to find the object that you need to do the activity on your card.} \\
    \only<1>{
      \begin{description}
        \item[] \textbf{Modèle:} \emph{Tu te coiffes.}
        \item[E1:] Bonjour! Je me coiffe. Est-ce que tu as un peigne?
        \item[] \tinygloss{Hello! I'm doing my hair. Do you have a comb?}
        \item[E2:] Oui, j'ai un peigne. Tiens!
        \item[] \tinygloss{Yes, I have a comb. Here you go!}
        \item[] OU
        \item[E2:] Non, désolé. Je n'ai pas de peigne.
        \item[] \tinygloss{No, sorry. I don't have a comb.}
      \end{description}
    }
    \only<2>{
      \vspace{0.5cm}
      \begin{columns}
        \column{0.6\textwidth}
          Qu'est-ce que tu fais et avec quoi? \\
          \tinygloss{What are you doing and with what?}
          \begin{description}
            \item[] \textbf{Modèle:}
            \item[] \emph{Tu te coiffes. / un peigne}
            \item[E1:] Je me coiffe avec un peigne.
          \end{description}
        \column{0.4\textwidth}
          \begin{center}
          \includegraphics[scale=0.5]{peigne.jpg}
          \end{center}
      \end{columns}
    }
  \end{frame}

  \begin{frame}{}
    \begin{center}
      \Large Questions?
    \end{center}
  \end{frame}
\end{document}
