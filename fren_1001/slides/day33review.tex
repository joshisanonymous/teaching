%%%%%%%%%%%%%%%%%%%%%%%%%%%%%%%%%%%%%
%                                   %
% Compile with XeLaTeX and biber    %
%                                   %
% Questions or comments:            %
%                                   %
% joshua dot mcneill at uga dot edu %
%                                   %
%%%%%%%%%%%%%%%%%%%%%%%%%%%%%%%%%%%%%

\documentclass{beamer}
  % Read in standard preamble (cosmetic stuff)
  %%%%%%%%%%%%%%%%%%%%%%%%%%%%%%%%%%%%%%%%%%%%%%%%%%%%%%%%%%%%%%%%
% This is a standard preamble used in for all slide documents. %
% It basically contains cosmetic settings.                     %
%                                                              %
% Joshua McNeill                                               %
% joshua dot mcneill at uga dot edu                            %
%%%%%%%%%%%%%%%%%%%%%%%%%%%%%%%%%%%%%%%%%%%%%%%%%%%%%%%%%%%%%%%%

% Beamer settings
% \usetheme{Berkeley}
\usetheme{CambridgeUS}
% \usecolortheme{dove}
% \usecolortheme{rose}
\usecolortheme{seagull}
\usefonttheme{professionalfonts}
\usefonttheme{serif}
\setbeamertemplate{bibliography item}{}

% Packages and settings
\usepackage{fontspec}
  \setmainfont{Charis SIL}
\usepackage{hyperref}
  \hypersetup{colorlinks=true,
              allcolors=blue}
\usepackage{graphicx}
  \graphicspath{{../../figures/}}
\usepackage{soul}
  \setstcolor{red}
\usepackage[normalem]{ulem}
\usepackage{enumerate}
\usepackage{tikz}
  \usetikzlibrary{trees}

% Document information
\author{M. McNeill}
\title[FREN1001]{Français 1001}
\institute{\url{joshua.mcneill@uga.edu}}
\date{}

%% Custom commands
% Lexical items
\newcommand{\lexi}[1]{\textit{#1}}
% Gloss
\newcommand{\gloss}[1]{`#1'}
\newcommand{\tinygloss}[1]{{\tiny`#1'}}
% Orthographic representations
\newcommand{\orth}[1]{$\langle$#1$\rangle$}
% Utterances (pragmatics)
\newcommand{\uttr}[1]{`#1'}
% Sentences (pragmatics)
\newcommand{\sent}[1]{\textit{#1}}
% Fixed length underlines
\newcommand{\funderline}[2][4cm]{
  \underline{\makebox[\ifdim\width>#1\width\else#1\fi]{#2}}
}
% Base dir for definitions
\newcommand{\defs}{../definitions}
\newcommand{\activity}[1]{
  \input{./activities/#1.tex}
}


  % Packages and settings

  % Document information
  \subtitle[Révision: Examen 2]{Révision de l'examen 2}

\begin{document}
  % Read in the standard intro slides (title page and table of contents)
  \begin{frame}
    \titlepage
    \tiny{Office: % Basically a variable for office hours location
Zoom (ID 978 2791 8221)
\\
          Office hours: % Basically a variable for office hours
 mercredi 10h15--13h15
}
  \end{frame}

  \begin{frame}{Choisis ta propre aventure \\ \gloss{Choose Your Own Adventure}}
    \hypertarget{début}{}
    \begin{columns}
      \column{0.5\textwidth}
        \begin{center}
          \includegraphics[scale=0.3]{aventure.jpg} \\
          La Caverne du temps
        \end{center}
      \column{0.5\textwidth}
        \begin{enumerate}
          % \item \hyperlink{orale}{Compréhension orale} % remove this section
          \item \hyperlink{prépositions}{Prépositions}
          \item \hyperlink{sujets}{Sujets et métiers}
          
          \item \hyperlink{mort}{Passé composé}
          \item \hyperlink{mort}{Subjonctif}
          \item \hyperlink{adjectifs}{Adjectifs et démonstratifs}
          \item \hyperlink{mort}{Nourriture et boissons}
          \item \hyperlink{comparatifs}{Comparatifs et superlatifs}
        \end{enumerate}
    \end{columns}
  \end{frame}

  % Verbes (re, ir, pronominal, préférer, -oir)
  % Adverbes (intensity, frequency, quantity)
    % Inject time into this

  % \begin{frame}{Compréhension orale}
  %   \hypertarget{orale}{}
  %   Un gobelin commence à parler...
  %   \begin{columns}[t]
  %     \column{0.5\textwidth}
  %       \begin{enumerate}
  %         \item Benoît va...
  %         \begin{enumerate}
  %           \item faire du jogging...
  %           \item \alert<2->{faire du ski...}
  %           \item dîner...
  %         \end{enumerate}
  %         \item[] à \underline{\uncover<3->{2pm}}
  %         \item Claire va...
  %         \begin{enumerate}
  %           \item \alert<4->{aller au restaurant cher...}
  %           \item assister au match de football...
  %           \item nager...
  %         \end{enumerate}
  %         \item[] à \underline{\uncover<5->{6:30pm}}
  %       \end{enumerate}
  %     \column{0.5\textwidth}
  %       \begin{enumerate}
  %         \setcounter{enumi}{2}
  %         \item Ève va...
  %         \begin{enumerate}
  %           \item \alert<6->{se brosser les dents...}
  %           \item sortir à une fête avec ses amies...
  %           \item jouer au basket...
  %         \end{enumerate}
  %         \item[] à \underline{\uncover<7->{10am}}
  %       \end{enumerate}
  %       \begin{center}
  %         \includegraphics[scale=0.4]{gobelin.jpg}
  %       \end{center}
  %   \end{columns}
  %   Retourne au \hyperlink{début}{début}...
  % \end{frame}

  \begin{frame}{prépositions}
    \hypertarget{prépositions}{}
    \begin{columns}
      \column{0.7\textwidth}
        Ce vieux gobelin est plutôt poli.
        Il veut te indiquer le chemin vers un trésor!
      \column{0.3\textwidth}
        \begin{center}
          \includegraphics[scale=0.4]{gobelin.jpg}
        \end{center}
    \end{columns}

    \vspace{0.25cm}
    <<Le trésor est \underline{\uncover<2->{loin d'}} (far from) ici!
    D'abord, tu dois passer \underline{\uncover<3->{derrière}} (behind) moi vers un lac souterrain.
    Quand tu arrives \underline{\uncover<4->{devant}} (in front of) le lac, tu vas trouver un tunnel \underline{\uncover<5->{à gauche de}} (to the left of) toi.
    Suis-le à la fin!
    Tu vas te trouver \underline{\uncover<6->{entre}} (between) deux rochers \gloss{rocks}.
    Tu va pouvoir voir le trésor \underline{\uncover<7->{dans}} (in) le rocher, mais fais attention!
    Il y a un monstre \underline{\uncover<8->{tout près du}} (very close to) rocher.>>

    \vspace{0.25cm}
    Tu suis le chemin indiqué, mais ce n'est pas le bon chemin, et tu arrives au \hyperlink{début}{début}...
  \end{frame}

  \begin{frame}{Sujets et métiers}
    \hypertarget{sujets}{}
    Un mineur arrive.
    Il te demande quels sujets qu'il doit étudier pour ne plus être mineur...

    \vspace{0.25cm}
    \begin{columns}
      \column{0.6\textwidth}
        \small
        En tant que mineur, j'étudie les rochers, mais si je veux être écrivain, j'étudie \underline{\uncover<2->{les lettres, la littérature}}?
        D'accord, et je peux faire ça dans une \underline{\uncover<3->{bibliothèque}} ou un \underline{\uncover<3->{amphithéâtre}}?
        Parfait.
        Ma sœur est aussi mineure, mais elle étudie les mathématiques sur les ordinateurs pour être \underline{\uncover<4->{informaticienne}}.
        Est-ce une bonne idée?
        D'accord.
        Enfin, si je veux être assistant social, j'étudie \underline{\uncover<5->{l'anthropologie}}, \underline{\uncover<5->{la psychologie}}, \underline{\uncover<5->{la sociologie}}, ou \underline{\uncover<5->{le droit}}?
        Je ne sais pas si je veux travailler dans un \underline{\uncover<6->{bureau}}, mais j'aime aider les gens.

      \column{0.4\textwidth}
        \begin{center}
          \includegraphics[scale=0.75]{mineur.jpg}
        \end{center}
    \end{columns}
    \vspace{0.25cm}
    En tout cas, tu peux revenir au \hyperlink{début}{début}...
  \end{frame}

  \begin{frame}{Adjectifs et démonstratifs}
    \hypertarget{adjectifs}{}
    Il y a une grande orque âgée dans ce petit corridor gris.
    Elle ne sait pas bien ses adjectifs, alors tu dois les lui donner pour survivre...

    \vspace{0.25cm}
    \begin{columns}
      \column{0.6\textwidth}
        \small
        \underline{\uncover<2->{Cet}} (this) endroit est trop \underline{\uncover<3->{noir}} (black), mais je peux voir les gens.
        Par exemple, je regarde une sorcière, et \underline{\uncover<4->{cette}} (this) sorcière a un \underline{\uncover<5->{méchant}} (mean) mari.
        Et ce \underline{\uncover<6->{laid}} (ugly) gobelin, il pense qu'il est \underline{\uncover<7->{beau}} (handsome), mais moi je préfère regarder \underline{\uncover<8->{ces}} (these) baskets \underline{\uncover<9->{blanches}} (white), \underline{\uncover<10->{violettes}} (purple) et \underline{\uncover<11->{roses}} (pink).
        Je déteste les baskets.

      \column{0.4\textwidth}
        \begin{center}
          \includegraphics[scale=0.17]{orque.jpg}
        \end{center}
    \end{columns}
    \vspace{0.25cm}
    Mais tu es sympa.
    Tu peux revenir au \hyperlink{début}{début}...
  \end{frame}

  \begin{frame}{Comparatifs et superlatifs}
    \hypertarget{comparatifs}{}
    Un sorcier plus petit que Gandalf apparaît pour te faire faire des comparaisons...

    \vspace{0.25cm}
    \begin{columns}
      \column{0.6\textwidth}
        \scriptsize
        \begin{enumerate}
          \item David Attenborough est \underline{\uncover<2->{plus vieux que}} (older than) Mel Brooks par un mois.
          \item La cafétéria Snelling est \underline{\uncover<3->{meilleure que}} (better than) Bolton?
          \item La chimie est \underline{\uncover<4->{moins ennuyeuse que}} (less boring than) la comptabilité?
          \item Jordan Peele est \underline{\uncover<5->{le plus drôle}} (the funniest)?
          \item Les sandales sont \underline{\uncover<6->{moins chic que}} (less stylish than) les mocassins?
          \item Le rugby et le football américain sont \underline{\uncover<7->{les mêmes}} (the same).
        \end{enumerate}

      \column{0.4\textwidth}
        \begin{center}
          \includegraphics[scale=0.066]{willow.jpg}
        \end{center}
    \end{columns}
    \vspace{0.25cm}
    Tu peux revenir au \hyperlink{début}{début}...
  \end{frame}

  \begin{frame}{}
    \hypertarget{mort}{}
    Un fantôme maléfique \gloss{evil} apparaît!
    \begin{columns}
      \column{0.4\textwidth}
        \begin{center}
          \includegraphics[scale=0.25]{phantom.jpg}
        \end{center}
      \column{0.6\textwidth}
        \begin{center}
          \Large{Tu meurs.} \\
          (c'est-à-dire que ça ne va pas être sur l'examen) \\
          \tinygloss{You die. (aka that won't be on the exam)}
        \end{center}

        \vspace{0.25cm}
        Retourne au \hyperlink{début}{début}...
    \end{columns}
  \end{frame}
\end{document}
