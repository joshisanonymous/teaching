%%%%%%%%%%%%%%%%%%%%%%%%%%%%%%%%%%%%%
%                                   %
% Compile with XeLaTeX and biber    %
%                                   %
% Questions or comments:            %
%                                   %
% joshua dot mcneill at uga dot edu %
%                                   %
%%%%%%%%%%%%%%%%%%%%%%%%%%%%%%%%%%%%%

\documentclass{beamer}
  % Read in standard preamble (cosmetic stuff)
  %%%%%%%%%%%%%%%%%%%%%%%%%%%%%%%%%%%%%%%%%%%%%%%%%%%%%%%%%%%%%%%%
% This is a standard preamble used in for all slide documents. %
% It basically contains cosmetic settings.                     %
%                                                              %
% Joshua McNeill                                               %
% joshua dot mcneill at uga dot edu                            %
%%%%%%%%%%%%%%%%%%%%%%%%%%%%%%%%%%%%%%%%%%%%%%%%%%%%%%%%%%%%%%%%

% Beamer settings
% \usetheme{Berkeley}
\usetheme{CambridgeUS}
% \usecolortheme{dove}
% \usecolortheme{rose}
\usecolortheme{seagull}
\usefonttheme{professionalfonts}
\usefonttheme{serif}
\setbeamertemplate{bibliography item}{}

% Packages and settings
\usepackage{fontspec}
  \setmainfont{Charis SIL}
\usepackage{hyperref}
  \hypersetup{colorlinks=true,
              allcolors=blue}
\usepackage{graphicx}
  \graphicspath{{../../figures/}}
\usepackage[normalem]{ulem}
\usepackage{enumerate}

% Document information
\author{M. McNeill}
\title[FREN2001]{Français 2001}
\institute{\url{joshua.mcneill@uga.edu}}
\date{}

%% Custom commands
% Lexical items
\newcommand{\lexi}[1]{\textit{#1}}
% Gloss
\newcommand{\gloss}[1]{`#1'}
\newcommand{\tinygloss}[1]{{\tiny`#1'}}
% Orthographic representations
\newcommand{\orth}[1]{$\langle$#1$\rangle$}
% Utterances (pragmatics)
\newcommand{\uttr}[1]{`#1'}
% Sentences (pragmatics)
\newcommand{\sent}[1]{\textit{#1}}
% Base dir for definitions
\newcommand{\defs}{../definitions}


  % Packages and settings

  % Document information
  \subtitle[Loisirs (jouer) et prépositions]{Les loisirs que nous jouons et \lexi{à} et \lexi{de}}

\begin{document}
  % Read in the standard intro slides (title page and table of contents)
  \begin{frame}
    \titlepage
    \tiny{Office: % Basically a variable for office hours location
Gilbert 121\\
          Office hours: % Basically a variable for office hours
 lundi, mercredi, vendredi 10:10--11:10
}
  \end{frame}

  \begin{frame}{Annonces \gloss{Announcements}}
    \begin{itemize}
      \item Le devoir 2 est à rendre le 20 septembre (vendredi).
      \item[] \gloss{Homework 2 is due September 20th (Friday).}
    \end{itemize}
  \end{frame}

  \begin{frame}{Révision \gloss{Review}}
    Adjectives agree with the nouns that they modify
    \begin{itemize}
      \item Usually this means the noun they're next to:
      \item[] la femme (f./sg.) intelligente (f./sg.)
      \item[] les cheveux (m./pl.) blonds (m./pl.)
      \item With a verb like \lexi{être}, the adjective agrees with the subject
      \item[] la femme (f./sg.) est intelligente (f./sg.)
      \item A verb like \lexi{avoir} doesn't work this way
      \item[] la femme (f./sg.) a les cheveux (\alert{m./pl.}) blonds (\alert{m./pl.})
    \end{itemize}
  \end{frame}

  \activity{verbe_jouer_prononce}
  \activity{verbe_jouer_ensemble}
  \activity{prepositions_contractions}
  \activity{deux_verites}
  \activity{liaison}
  \activity{jouer_questions}
  \activity{activites_ensemble}
\end{document}
