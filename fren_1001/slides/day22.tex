%%%%%%%%%%%%%%%%%%%%%%%%%%%%%%%%%%%%%
%                                   %
% Compile with XeLaTeX and biber    %
%                                   %
% Questions or comments:            %
%                                   %
% joshua dot mcneill at uga dot edu %
%                                   %
%%%%%%%%%%%%%%%%%%%%%%%%%%%%%%%%%%%%%

\documentclass{beamer}
  % Read in standard preamble (cosmetic stuff)
  %%%%%%%%%%%%%%%%%%%%%%%%%%%%%%%%%%%%%%%%%%%%%%%%%%%%%%%%%%%%%%%%
% This is a standard preamble used in for all slide documents. %
% It basically contains cosmetic settings.                     %
%                                                              %
% Joshua McNeill                                               %
% joshua dot mcneill at uga dot edu                            %
%%%%%%%%%%%%%%%%%%%%%%%%%%%%%%%%%%%%%%%%%%%%%%%%%%%%%%%%%%%%%%%%

% Beamer settings
% \usetheme{Berkeley}
\usetheme{CambridgeUS}
% \usecolortheme{dove}
% \usecolortheme{rose}
\usecolortheme{seagull}
\usefonttheme{professionalfonts}
\usefonttheme{serif}
\setbeamertemplate{bibliography item}{}

% Packages and settings
\usepackage{fontspec}
  \setmainfont{Charis SIL}
\usepackage{hyperref}
  \hypersetup{colorlinks=true,
              allcolors=blue}
\usepackage{graphicx}
  \graphicspath{{../../figures/}}
\usepackage{soul}
  \setstcolor{red}
\usepackage[normalem]{ulem}
\usepackage{enumerate}
\usepackage{tikz}
  \usetikzlibrary{trees}

% Document information
\author{M. McNeill}
\title[FREN1001]{Français 1001}
\institute{\url{joshua.mcneill@uga.edu}}
\date{}

%% Custom commands
% Lexical items
\newcommand{\lexi}[1]{\textit{#1}}
% Gloss
\newcommand{\gloss}[1]{`#1'}
\newcommand{\tinygloss}[1]{{\tiny`#1'}}
% Orthographic representations
\newcommand{\orth}[1]{$\langle$#1$\rangle$}
% Utterances (pragmatics)
\newcommand{\uttr}[1]{`#1'}
% Sentences (pragmatics)
\newcommand{\sent}[1]{\textit{#1}}
% Fixed length underlines
\newcommand{\funderline}[2][4cm]{
  \underline{\makebox[\ifdim\width>#1\width\else#1\fi]{#2}}
}
% Base dir for definitions
\newcommand{\defs}{../definitions}
\newcommand{\activity}[1]{
  \input{./activities/#1.tex}
}


  % Packages and settings

  % Document information
  \subtitle[Études et verbes \lexi{préférer}]{Vos études et les verbes comme \lexi{préférer}}

\begin{document}
  % Read in the standard intro slides (title page and table of contents)
  \begin{frame}
    \titlepage
    \tiny{Office: % Basically a variable for office hours location
Zoom (ID 978 2791 8221)
\\
          Office hours: % Basically a variable for office hours
 mercredi 10h15--13h15
}
  \end{frame}

  \begin{frame}{Les verbes comme \lexi{préférer}}
    \begin{center}
      \begin{tabular}{l | l l | l l}
  \multicolumn{5}{c}{préférer \gloss{to prefer}} \\
      & \multicolumn{2}{l |}{singulier} & \multicolumn{2}{l}{pluriel} \\
  \hline
  1re & je         & préfère            & nous        & préférons \\
  2e  & tu         & préfères           & vous        & préférez \\
  \hline
  3e  & il (masc)  &                    & ils (masc)  & \\
      & elle (fem) & préfère            & elles (fem) & préfèrent \\
      & on         &                    &             & \\
\end{tabular}

    \end{center}
  \end{frame}

  \begin{frame}{Les majeures}
    Quelles sont leurs majeures?
    \begin{enumerate}
      \item Cécile préfère suivre les cours l'Europe moderne, le Canada (son histoire et son peuple) et l'histoire de la civilisation occidentale.
      \item[$\to$]<2-> Sa majeure est l'histoire.
      \item<3-> Arnaud préfère suivre les cours la civilisation allemande, l'allemand écrit 1 et la pratique d'allemand parlé.
      \item[$\to$]<4-> Sa majeure est l'allemand.
      \item<5-> Romain préfère suivre les cours l'introduction des concepts sociologiques, la communication et l'organisation et la psychologie sociale.
      \item[$\to$]<6-> Sa majeure est la sociologie.
      \item<7-> Anne-Marie préfère suivre les cours la biologie expérimentale, les principes d'écologie et l'introduction à la génétique.
      \item[$\to$]<8-> Sa majeure est biologie.
    \end{enumerate}
  \end{frame}

  \begin{frame}{}
    \begin{center}
      \Large Quiz
    \end{center}
  \end{frame}

  \begin{frame}{Le /e/ contre le /ɛ/}
    \begin{enumerate}
      \item[/e/] $\to$ Vous pr\textbf{é}f\textbf{é}r\textbf{e}z l\textbf{e}s expos\textbf{é}s ou l\textbf{e}s proj\textbf{e}ts?
      \item[/e/] $\to$ Vous \textbf{é}criv\textbf{e}z un \textbf{e}ssai en \textbf{é}t\textbf{é}?
      \item[/ɛ/] $\to$ Les fr\textbf{è}res rép\textbf{è}tent apr\textbf{è}s \textbf{e}lle.
      \item[/ɛ/] $\to$ \textbf{E}lle \textbf{e}st pr\textbf{è}s de la biblioth\textbf{è}que.
    \end{enumerate}
  \end{frame}

  \begin{frame}{Une préférence entre deux choix}
    Avec un/e partenaire, indique quel choix que tu préfères et pourquoi. \\
    \tinygloss{With a partner, indicate which choice you prefer and why.} \\
    \vspace{1cm}
    Tu préfères ...
    \begin{enumerate}
      \item ... écrire un essai ou faire un exposé oral?
      \item ... suivre un cours de littérature ou de sciences naturelles?
      \item ... faire du sport ou participer à une association étudiante?
      \item ... travailler en groupe ou travailler individuellement?
      \item ... les professeurs exigeants \gloss{demanding} ou indulgents?
      \item ... assister à un cours au laboratoire ou à l'amphithéâtre?
    \end{enumerate}
  \end{frame}

  \begin{frame}{J'aime et je déteste}
    Parle à cinq camarades de classe au moins, et échangez ce que vous aimez faire et ce que vous détestez faire.
    Note bien la personne avec qui tu as le plus en commun. \\
    \tinygloss{Speak to at least 5 classmates, and share what you like to do and hate to do.
    Take note of the person with whom you have the most in common.}
    \begin{center}
      \begin{description}
        \item[] \textbf{Modèle:}
        \item[E1:] J'adore travailler dans le jardin, mais je déteste faire des courses le week-end. Et toi?
        \item[E2:] Moi, je ne travaille pas dans le jardin, mais je déteste aussi faire des courses.
      \end{description}
    \end{center}
  \end{frame}

  \begin{frame}{}
    \begin{center}
      \Large Questions?
    \end{center}
  \end{frame}
\end{document}
