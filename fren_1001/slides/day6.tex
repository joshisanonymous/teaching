%%%%%%%%%%%%%%%%%%%%%%%%%%%%%%%%%%%%%
%                                   %
% Compile with XeLaTeX and biber    %
%                                   %
% Questions or comments:            %
%                                   %
% joshua dot mcneill at uga dot edu %
%                                   %
%%%%%%%%%%%%%%%%%%%%%%%%%%%%%%%%%%%%%

\documentclass{beamer}
  % Read in standard preamble (cosmetic stuff)
  %%%%%%%%%%%%%%%%%%%%%%%%%%%%%%%%%%%%%%%%%%%%%%%%%%%%%%%%%%%%%%%%
% This is a standard preamble used in for all slide documents. %
% It basically contains cosmetic settings.                     %
%                                                              %
% Joshua McNeill                                               %
% joshua dot mcneill at uga dot edu                            %
%%%%%%%%%%%%%%%%%%%%%%%%%%%%%%%%%%%%%%%%%%%%%%%%%%%%%%%%%%%%%%%%

% Beamer settings
% \usetheme{Berkeley}
\usetheme{CambridgeUS}
% \usecolortheme{dove}
% \usecolortheme{rose}
\usecolortheme{seagull}
\usefonttheme{professionalfonts}
\usefonttheme{serif}
\setbeamertemplate{bibliography item}{}

% Packages and settings
\usepackage{fontspec}
  \setmainfont{Charis SIL}
\usepackage{hyperref}
  \hypersetup{colorlinks=true,
              allcolors=blue}
\usepackage{graphicx}
  \graphicspath{{../../figures/}}
\usepackage{soul}
  \setstcolor{red}
\usepackage[normalem]{ulem}
\usepackage{enumerate}
\usepackage{tikz}
  \usetikzlibrary{trees}

% Document information
\author{M. McNeill}
\title[FREN1001]{Français 1001}
\institute{\url{joshua.mcneill@uga.edu}}
\date{}

%% Custom commands
% Lexical items
\newcommand{\lexi}[1]{\textit{#1}}
% Gloss
\newcommand{\gloss}[1]{`#1'}
\newcommand{\tinygloss}[1]{{\tiny`#1'}}
% Orthographic representations
\newcommand{\orth}[1]{$\langle$#1$\rangle$}
% Utterances (pragmatics)
\newcommand{\uttr}[1]{`#1'}
% Sentences (pragmatics)
\newcommand{\sent}[1]{\textit{#1}}
% Fixed length underlines
\newcommand{\funderline}[2][4cm]{
  \underline{\makebox[\ifdim\width>#1\width\else#1\fi]{#2}}
}
% Base dir for definitions
\newcommand{\defs}{../definitions}
\newcommand{\activity}[1]{
  \input{./activities/#1.tex}
}


  % Packages and settings

  % Document information
  \subtitle[Famille et verbe \lexi{avoir}]{La famille et le verb \lexi{avoir}}

\begin{document}
  % Read in the standard intro slides (title page and table of contents)
  \begin{frame}
    \titlepage
    \tiny{Office: % Basically a variable for office hours location
Zoom (ID 978 2791 8221)
\\
          Office hours: % Basically a variable for office hours
 mercredi 10h15--13h15
}
  \end{frame}

  \begin{frame}{}
    \begin{center}
      \Large Quiz
    \end{center}
  \end{frame}

  % This is primarily just an opportunity to practice pronunciation.
  \begin{frame}{Le verbe \lexi{avoir}}
    \begin{center}
      \begin{tabular}{l | l l | l l}
  \multicolumn{5}{c}{avoir \gloss{to have}} \\
      & \multicolumn{2}{l |}{singulier} & \multicolumn{2}{l}{pluriel} \\
  \hline
  1re & je         & ai               & nous        & avons \\
  2e  & tu         & as               & vous        & avez \\
  \hline
  3e  & il (masc)  &                  & ils (masc)  & \\
      & elle (fem) & a                & elles (fem) & ont \\
      & on         &                  &             & \\
\end{tabular}

    \end{center}
  \end{frame}

  \begin{frame}{Les animaux de compagnie \gloss{Pets}}
    Quelle est la bonne conjugaison du verbe \lexi{avoir}? \\
    \tinygloss{What is the correct conjugation for the verb \lexi{avoir}?}
    \begin{columns}
      \column{0.6\textwidth}
        \begin{enumerate}
          \item Vous \underline{\uncover<2->{avez}} un chien.
          \item On \underline{\uncover<4->{\ a\ \ }} des poissons.
          \item J'\underline{\uncover<6->{\ ai\ }} des oiseaux.
          \item Elles \underline{\uncover<8->{ont\ }} un chat.
          \item M. McNeill \underline{\uncover<10->{\ a\ \ }} un lézard.
        \end{enumerate}
      \column{0.4\textwidth}
        \begin{minipage}[c][0.6\textheight]{\linewidth}
          \begin{center}
            \only<1-2>{
              \includegraphics[scale=0.11]{chien.jpg}
            }
            \only<3-4>{
              \includegraphics[scale=0.1]{aquaman.jpg}
            }
            \only<5-6>{
              \includegraphics[scale=0.18]{oiseaux.jpg}
            }
            \only<7-8>{
              \includegraphics[scale=0.5]{puss_in_boots.png}
            }
            \only<9-10>{
              \includegraphics[scale=0.35]{milton.jpg}
            }
          \end{center}
        \end{minipage}
    \end{columns}
  \end{frame}

  \begin{frame}{Arbre généalogique (encore) \gloss{Family tree (cont.)}}
    \begin{columns}[t]
      \column{0.5\textwidth}
        La famille de Jacques
        \includegraphics[scale=0.4]{famille_de_jacques.png}
      \column{0.5\textwidth}
        \small
        \gloss{In groups of 3 or 4, each person choose a family member in the tree to be, and take turns asking each other if they have a particular family member and, if so, that family member's name.
        For example:}
        \begin{itemize}
          \item[E1:] Est-ce que tu as une sœur?
          \item[] \tinygloss{Do you have a sister?}
          \item[E2:] Oui, j'ai une sœur.
          \item[] \tinygloss{Yes, I have a sister.}
          \item[E1:] Comment s'appelle ta sœur?
          \item[] \tinygloss{What is your sister's name?}
          \item[E2:] Ma sœur s'appelle ...
          \item[] \tinygloss{My sister's name is ...}
        \end{itemize}
    \end{columns}
  \end{frame}

  % Preface this by describing what you have with you today.
  \begin{frame}{Qu'est-ce que vous avez? \gloss{What do y'all have?}}
    \small
    En groupes de 3 ou 4, demandez à votre tour ce que les autres ont aujourd'hui, puis comparez si vous avez les mêmes choses ou non. \\
    \tinygloss{In groups of 3 or 4, take turns asking each other what each of you have with you today, then compare if you have the same things or not.
    For example:}
    \begin{columns}
      \column{0.6\textwidth}
        \begin{itemize}
          \item[E1:] (to E2) Qu'est-ce que tu as?
          \item[] \tinygloss{What do you have?}
          \item[E2:] J'ai un \alert{stylo}, un ordinateur et des feuilles de papier.
          \item[] \tinygloss{I have a \alert{pen}, a computer et some pieces of paper.}
          \item[E2:] (to E3) Qu'est-ce que tu as?
          \item[] \tinygloss{What do you have?}
          \item[E3:] J'ai un cahier, un crayon et un \alert{stylo}. Nous avons \alert{des stylos}.
          \item[] \tinygloss{I have a notebook, a pencil and a \alert{pen.} We have \alert{pens}.}
        \end{itemize}
      \column{0.4\textwidth}
        \begin{center}
          \includegraphics[scale=0.15]{sacs.jpg}
        \end{center}
    \end{columns}
  \end{frame}

  \begin{frame}{}
    \begin{center}
      \Large Questions?
    \end{center}
  \end{frame}
\end{document}
