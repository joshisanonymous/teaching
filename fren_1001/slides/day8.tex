%%%%%%%%%%%%%%%%%%%%%%%%%%%%%%%%%%%%%
%                                   %
% Compile with XeLaTeX and biber    %
%                                   %
% Questions or comments:            %
%                                   %
% joshua dot mcneill at uga dot edu %
%                                   %
%%%%%%%%%%%%%%%%%%%%%%%%%%%%%%%%%%%%%

\documentclass{beamer}
  % Read in standard preamble (cosmetic stuff)
  %%%%%%%%%%%%%%%%%%%%%%%%%%%%%%%%%%%%%%%%%%%%%%%%%%%%%%%%%%%%%%%%
% This is a standard preamble used in for all slide documents. %
% It basically contains cosmetic settings.                     %
%                                                              %
% Joshua McNeill                                               %
% joshua dot mcneill at uga dot edu                            %
%%%%%%%%%%%%%%%%%%%%%%%%%%%%%%%%%%%%%%%%%%%%%%%%%%%%%%%%%%%%%%%%

% Beamer settings
% \usetheme{Berkeley}
\usetheme{CambridgeUS}
% \usecolortheme{dove}
% \usecolortheme{rose}
\usecolortheme{seagull}
\usefonttheme{professionalfonts}
\usefonttheme{serif}
\setbeamertemplate{bibliography item}{}

% Packages and settings
\usepackage{fontspec}
  \setmainfont{Charis SIL}
\usepackage{hyperref}
  \hypersetup{colorlinks=true,
              allcolors=blue}
\usepackage{graphicx}
  \graphicspath{{../../figures/}}
\usepackage{soul}
  \setstcolor{red}
\usepackage[normalem]{ulem}
\usepackage{enumerate}
\usepackage{tikz}
  \usetikzlibrary{trees}

% Document information
\author{M. McNeill}
\title[FREN1001]{Français 1001}
\institute{\url{joshua.mcneill@uga.edu}}
\date{}

%% Custom commands
% Lexical items
\newcommand{\lexi}[1]{\textit{#1}}
% Gloss
\newcommand{\gloss}[1]{`#1'}
\newcommand{\tinygloss}[1]{{\tiny`#1'}}
% Orthographic representations
\newcommand{\orth}[1]{$\langle$#1$\rangle$}
% Utterances (pragmatics)
\newcommand{\uttr}[1]{`#1'}
% Sentences (pragmatics)
\newcommand{\sent}[1]{\textit{#1}}
% Fixed length underlines
\newcommand{\funderline}[2][4cm]{
  \underline{\makebox[\ifdim\width>#1\width\else#1\fi]{#2}}
}
% Base dir for definitions
\newcommand{\defs}{../definitions}
\newcommand{\activity}[1]{
  \input{./activities/#1.tex}
}


  % Packages and settings

  % Document information
  \subtitle[Grands nombres et adjectifs invariables]{Les grands nombres et les adjectifs invariables}

\begin{document}
  % Read in the standard intro slides (title page and table of contents)
  \begin{frame}
    \titlepage
    \tiny{Office: % Basically a variable for office hours location
Zoom (ID 978 2791 8221)
\\
          Office hours: % Basically a variable for office hours
 mercredi 10h15--13h15
}
  \end{frame}

  \begin{frame}{}
    \begin{center}
      \Large Quiz
    \end{center}
  \end{frame}

  \begin{frame}{Annonces}
    \begin{itemize}
      \item Pas de cours lundi (la fête du travail)!
      \item \tinygloss{No class Monday (Labor Day)!}
    \end{itemize}
  \end{frame}

  \begin{frame}{Les maths avec de grands nombres}
    Résolvons les équations ci-dessous ensemble! \\
    \tinygloss{Let's solve the equations below together!}
    \begin{enumerate}
      \item 24 + 18 = \uncover<2->{42}
      \item 67 - 16 = \uncover<3->{51}
      \item 49 + 36 = \uncover<4->{85}
      \item 100 - 70 = \uncover<5->{30}
      \item 53 + 20 = \uncover<6->{73}
    \end{enumerate}
  \end{frame}

  \begin{frame}{Quel âge?}
    \begin{center}
      \includegraphics[scale=0.2]{beyonce.jpg}

      Elle a quel âge? \underline{\uncover<2->{Elle a 41 ans.}} \\
      \uncover<2->{Et elle est comment?}
    \end{center}
  \end{frame}

  \begin{frame}{Quel âge?}
    \begin{center}
      \includegraphics[scale=0.35]{emmanuel_macron.jpg}

      Il a quel âge? \underline{\uncover<2->{Il a 45 ans.}} \\
      \uncover<2->{Et il est comment?}
    \end{center}
  \end{frame}

  \begin{frame}{Quel âge?}
    \begin{center}
      \includegraphics{emeril_lagasse.jpg}

      Il a quel âge? \underline{\uncover<2->{Il a 63 ans.}}
    \end{center}
  \end{frame}

  \begin{frame}{Quel âge?}
    \begin{center}
      \includegraphics[scale=0.35]{morgan_freeman.jpg}

      Il a quel âge? \underline{\uncover<2->{Il a 86 ans.}} \\
      \uncover<2->{Et il est comment?}
    \end{center}
  \end{frame}

  \begin{frame}{Quel âge?}
    \begin{center}
      \includegraphics[scale=0.45]{lea_seydoux.jpg}

      Elle a quel âge? \underline{\uncover<2->{Elle a 38 ans.}}
    \end{center}
  \end{frame}

  \begin{frame}{Quel âge?}
    \begin{center}
      \includegraphics[scale=0.25]{paul_rudd.jpg}

      Il a quel âge? \underline{\uncover<2->{Il a 54 ans.}} \\
      \uncover<2->{Et il est comment?}
    \end{center}
  \end{frame}

  \begin{frame}{Les âges de ta famille?}
    Avec un/e partenaire, demande-lui l'âge de différents membres de sa famille ainsi que leurs caractères.
    Par exemple: \\
    \tinygloss{With a partner, ask them how old different members of their family are as well as what they're like.
    For example:}
    \begin{description}
      \item[E1:] Tu as un oncle? Il a quel âge?
      \item[] \tinygloss{Do you have an uncle? How old is he?}
      \item[E2:] Mon oncle a 42 ans.
      \item[] \tinygloss{My uncle is 42 years old.}
      \item[E1:] Il est comment?
      \item[] \tinygloss{What is he like?}
      \item[E2:] Il est pessimiste et sympathique.
      \item[] \tinygloss{He's pessimistic and nice.}
    \end{description}
  \end{frame}

  \begin{frame}{Des personnes idéales}
    Avec un/e partenaire, utilise des adjectifs pour décrire ta version idéale des suivants.
    Par exemple:
    \emph{Mon ami idéal est sympa et un peu indiscipliné.} \\
    \tinygloss{With a partner, use adjectives to describe your ideal version of the following. For example: \emph{Mon ami idéal est sympa et un peu indiscipliné.}}
    \begin{enumerate}
      \item ton camarade de classe ou ta camarade de classe
      \item ton professeur ou ta professeure
      \item ton ami(e)
      \item ton animal de compagnie
    \end{enumerate}
  \end{frame}

  \begin{frame}{}
    \begin{center}
      \Large Questions?
    \end{center}
  \end{frame}
\end{document}
