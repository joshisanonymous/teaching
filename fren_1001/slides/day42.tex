%%%%%%%%%%%%%%%%%%%%%%%%%%%%%%%%%%%%%
%                                   %
% Compile with XeLaTeX and biber    %
%                                   %
% Questions or comments:            %
%                                   %
% joshua dot mcneill at uga dot edu %
%                                   %
%%%%%%%%%%%%%%%%%%%%%%%%%%%%%%%%%%%%%

\documentclass{beamer}
  % Read in standard preamble (cosmetic stuff)
  %%%%%%%%%%%%%%%%%%%%%%%%%%%%%%%%%%%%%%%%%%%%%%%%%%%%%%%%%%%%%%%%
% This is a standard preamble used in for all slide documents. %
% It basically contains cosmetic settings.                     %
%                                                              %
% Joshua McNeill                                               %
% joshua dot mcneill at uga dot edu                            %
%%%%%%%%%%%%%%%%%%%%%%%%%%%%%%%%%%%%%%%%%%%%%%%%%%%%%%%%%%%%%%%%

% Beamer settings
% \usetheme{Berkeley}
\usetheme{CambridgeUS}
% \usecolortheme{dove}
% \usecolortheme{rose}
\usecolortheme{seagull}
\usefonttheme{professionalfonts}
\usefonttheme{serif}
\setbeamertemplate{bibliography item}{}

% Packages and settings
\usepackage{fontspec}
  \setmainfont{Charis SIL}
\usepackage{hyperref}
  \hypersetup{colorlinks=true,
              allcolors=blue}
\usepackage{graphicx}
  \graphicspath{{../../figures/}}
\usepackage{soul}
  \setstcolor{red}
\usepackage[normalem]{ulem}
\usepackage{enumerate}
\usepackage{tikz}
  \usetikzlibrary{trees}

% Document information
\author{M. McNeill}
\title[FREN1001]{Français 1001}
\institute{\url{joshua.mcneill@uga.edu}}
\date{}

%% Custom commands
% Lexical items
\newcommand{\lexi}[1]{\textit{#1}}
% Gloss
\newcommand{\gloss}[1]{`#1'}
\newcommand{\tinygloss}[1]{{\tiny`#1'}}
% Orthographic representations
\newcommand{\orth}[1]{$\langle$#1$\rangle$}
% Utterances (pragmatics)
\newcommand{\uttr}[1]{`#1'}
% Sentences (pragmatics)
\newcommand{\sent}[1]{\textit{#1}}
% Fixed length underlines
\newcommand{\funderline}[2][4cm]{
  \underline{\makebox[\ifdim\width>#1\width\else#1\fi]{#2}}
}
% Base dir for definitions
\newcommand{\defs}{../definitions}
\newcommand{\activity}[1]{
  \input{./activities/#1.tex}
}


  % Packages and settings

  % Document information
  \subtitle[\lexi{En} et passé composé]{Le pronom \lexi{en} et plus de passé composé}

\begin{document}
  % Read in the standard intro slides (title page and table of contents)
  \begin{frame}
    \titlepage
    \tiny{Office: % Basically a variable for office hours location
Zoom (ID 978 2791 8221)
\\
          Office hours: % Basically a variable for office hours
 mercredi 10h15--13h15
}
  \end{frame}

  \begin{frame}{}
    \begin{center}
      \Large Quiz
    \end{center}
  \end{frame}

  \begin{frame}{Est-on d'accord?}
    \begin{columns}
      \column{0.4\textwidth}
        \begin{itemize}
          \item[E1:] Il y a combien de \underline{\uncover<2>{boîtes de pizza}}?
          \item[E2:] Il y a \underline{\uncover<2>{quatre boîtes de pizza}}.
          \item[E3:] Oui, il y en a \underline{\uncover<2>{quatre}}.
        \end{itemize}
      \column{0.6\textwidth}
        \begin{minipage}[c][0.6\textheight]{\linewidth}
          \begin{center}
            \only<2>{
              \includegraphics[scale=0.25]{pizza_boxes.jpg}
            }
            \only<3>{
              \includegraphics[scale=0.15]{crevettes.jpg}
            }
            \only<4>{
              \includegraphics[scale=0.28]{melons.jpg}
            }
            \only<5>{
              \includegraphics{tranches.jpg}
            }
            \only<6>{
              \includegraphics[scale=0.33]{bols.jpg}
            }
            \only<7>{
              \includegraphics[scale=0.28]{bouteilles.jpg}
            }
          \end{center}
        \end{minipage}
    \end{columns}
  \end{frame}

  \begin{frame}{Écrivons les conjugaisons}
    Allez au lien suivant et écrivez vos réponses en donnant les bonnes conjugaisons \emph{au passé composé}. \\
    \tinygloss{Go to the following link and write your answers by giving the correct conjugations in \emph{passé composé}.}
    \begin{center}
      \url{PollEv.com/joshuamcneill919}
    \end{center}
    Ou envoyez le texto \alert{joshuamcneill919} et puis vos réponses à: \\
    \tinygloss{Or send the text message \alert{joshuamcneill919} and then your answers to:}
    \begin{center}
      22333
    \end{center}
  \end{frame}

  \begin{frame}{Une autre histoire ensemble}
    Imaginons une histoire ensemble pour le scénario.
    Chaque personne donne une phrase \emph{au passé composé} à partir de celle de la personne précédente. \\
    \tinygloss{Let's imagine a story together for the scenario.
    Each person gives a sentence in \emph{passé composé} building off the previous person's.}
    \begin{columns}
      \column{0.45\textwidth}
        \small
        \begin{description}
          \item[] \textbf{Modèle:} \emph{Jean et Luke au stade hier}
          \item[E1:] Hier, Jean et Luke ont regardé le match.
          \item[E2:] Leur équipe a gagné.
          \item[E3:] Après, ils ont pris des cocas, mais Luke n'aime pas le coca.
        \end{description}
      \column{0.55\textwidth}
        \alert{Le scénario:}
        \begin{itemize}
          \item Avant-hier, notre classe au centre-ville
        \end{itemize}
    \end{columns}
  \end{frame}

  \begin{frame}{}
    \begin{center}
      \Large Questions?
    \end{center}
  \end{frame}
\end{document}