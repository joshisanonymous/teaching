%%%%%%%%%%%%%%%%%%%%%%%%%%%%%%%%%%%%%
%                                   %
% Compile with XeLaTeX and biber    %
%                                   %
% Questions or comments:            %
%                                   %
% joshua dot mcneill at uga dot edu %
%                                   %
%%%%%%%%%%%%%%%%%%%%%%%%%%%%%%%%%%%%%

\documentclass{beamer}
  % Read in standard preamble (cosmetic stuff)
  %%%%%%%%%%%%%%%%%%%%%%%%%%%%%%%%%%%%%%%%%%%%%%%%%%%%%%%%%%%%%%%%
% This is a standard preamble used in for all slide documents. %
% It basically contains cosmetic settings.                     %
%                                                              %
% Joshua McNeill                                               %
% joshua dot mcneill at uga dot edu                            %
%%%%%%%%%%%%%%%%%%%%%%%%%%%%%%%%%%%%%%%%%%%%%%%%%%%%%%%%%%%%%%%%

% Beamer settings
% \usetheme{Berkeley}
\usetheme{CambridgeUS}
% \usecolortheme{dove}
% \usecolortheme{rose}
\usecolortheme{seagull}
\usefonttheme{professionalfonts}
\usefonttheme{serif}
\setbeamertemplate{bibliography item}{}

% Packages and settings
\usepackage{fontspec}
  \setmainfont{Charis SIL}
\usepackage{hyperref}
  \hypersetup{colorlinks=true,
              allcolors=blue}
\usepackage{graphicx}
  \graphicspath{{../../figures/}}
\usepackage[normalem]{ulem}
\usepackage{enumerate}

% Document information
\author{M. McNeill}
\title[FREN2001]{Français 2001}
\institute{\url{joshua.mcneill@uga.edu}}
\date{}

%% Custom commands
% Lexical items
\newcommand{\lexi}[1]{\textit{#1}}
% Gloss
\newcommand{\gloss}[1]{`#1'}
\newcommand{\tinygloss}[1]{{\tiny`#1'}}
% Orthographic representations
\newcommand{\orth}[1]{$\langle$#1$\rangle$}
% Utterances (pragmatics)
\newcommand{\uttr}[1]{`#1'}
% Sentences (pragmatics)
\newcommand{\sent}[1]{\textit{#1}}
% Base dir for definitions
\newcommand{\defs}{../definitions}


  % Packages and settings

  % Document information
  \subtitle[À la fac]{À la fac}

\begin{document}
  % Read in the standard intro slides (title page and table of contents)
  \begin{frame}
    \titlepage
    \tiny{Office: % Basically a variable for office hours location
Gilbert 121\\
          Office hours: % Basically a variable for office hours
 lundi, mercredi, vendredi 10:10--11:10
}
  \end{frame}

  \begin{frame}{}
    \begin{center}
      \Large Quiz
    \end{center}
  \end{frame}

  % Begin by asking people to raise their hands if they have certain objects.

  \begin{frame}{Le genre \gloss{Gender}}
    Quel est le genre du mot? \\
    \tinygloss{What is the gender of the word?}
    \begin{center}
      \begin{tabular}{l c c}
        \hline
        Mot \gloss{Word} & Masculin        & Féminin \\
        \hline
        le tableau       & \uncover<2->{X} & \\
        une étudiante    &                 & \uncover<3->{X} \\
        la tablette      &                 & \uncover<4->{X} \\
        une carte        &                 & \uncover<5->{X} \\
        un téléphone     & \uncover<6->{X} & \\
        l'étudiant       & \uncover<7->{X} & \\
      \end{tabular}
    \end{center}
  \end{frame}

  \begin{frame}{Singulier ou pluriel}
    Pour chaque numéro, y a-t-il un seul objet ou plusieurs? \\
    \tinygloss{For each number, is there one object or several?}
    \begin{center}
      \begin{tabular}{l c c}
        \hline
        Mot \gloss{Word} & Singulier       & Pluriel \\
        \hline
        les ordinateurs  &                 & \uncover<2->{X} \\
        des brosses      &                 & \uncover<3->{X} \\
        le crayon        & \uncover<4->{X} & \\
        la chaise        & \uncover<5->{X} & \\
        une gomme        & \uncover<6->{X} & \\
        les affiches     &                 & \uncover<7->{X}
      \end{tabular}
    \end{center}
  \end{frame}

  \begin{frame}{Associations \gloss{Associations}}
    Travaille avec un/e partenaire, et énumérez tous les objets dont vous pouvez penser qui sont associés avec les objets suivants. \\
    \tinygloss{Work with a partner, and list all the items you can think of that are associated with the following items.} \\
    Par exemple \gloss{For example}: \lexi{un tableau} $\to$ une craie, une brosse, des devoirs
    \begin{enumerate}
      \item un feutre
      \item une porte
      \item une professeure
      \item un bureau
      \item un étudiant
      \item un cahier
      \item une feuille de papier
    \end{enumerate}
  \end{frame}

  \begin{frame}{Ordres \gloss{Commands}}
    Avec un/e partenaire, complète les ordres suivants avec du vocabulaire pour les choses dans la salle de classe. \\
    \tinygloss{With a partner, complete the following commands using vocabulary for the things in the classroom.} \\
    Par exemple \gloss{For example}: \lexi{Regardez... le tableau.}
    \begin{columns}
      \column{0.5\textwidth}
        \begin{enumerate}
          \item Ouvrez...
          \item Écoutez...
          \item Rendez-moi...
          \item Montrez-moi...
          \item Fermez...
        \end{enumerate}
      \column{0.5\textwidth}
        \begin{enumerate}
          \setcounter{enumi}{5}
          \item Effacez...
          \item Répondez...
          \item Allez...
          \item Écrivez...
          \item Prenez...
        \end{enumerate}
      % Have some students provide examples of commands, perform the command yourself
      % first as a demonstration, then try to get other students to perform them
      % after.
    \end{columns}
  \end{frame}

  \begin{frame}{}
    \begin{center}
      \Large Questions?
    \end{center}
  \end{frame}
\end{document}
