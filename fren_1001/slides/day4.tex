%%%%%%%%%%%%%%%%%%%%%%%%%%%%%%%%%%%%%
%                                   %
% Compile with XeLaTeX and biber    %
%                                   %
% Questions or comments:            %
%                                   %
% joshua dot mcneill at uga dot edu %
%                                   %
%%%%%%%%%%%%%%%%%%%%%%%%%%%%%%%%%%%%%

\documentclass{beamer}
  % Read in standard preamble (cosmetic stuff)
  %%%%%%%%%%%%%%%%%%%%%%%%%%%%%%%%%%%%%%%%%%%%%%%%%%%%%%%%%%%%%%%%
% This is a standard preamble used in for all slide documents. %
% It basically contains cosmetic settings.                     %
%                                                              %
% Joshua McNeill                                               %
% joshua dot mcneill at uga dot edu                            %
%%%%%%%%%%%%%%%%%%%%%%%%%%%%%%%%%%%%%%%%%%%%%%%%%%%%%%%%%%%%%%%%

% Beamer settings
% \usetheme{Berkeley}
\usetheme{CambridgeUS}
% \usecolortheme{dove}
% \usecolortheme{rose}
\usecolortheme{seagull}
\usefonttheme{professionalfonts}
\usefonttheme{serif}
\setbeamertemplate{bibliography item}{}

% Packages and settings
\usepackage{fontspec}
  \setmainfont{Charis SIL}
\usepackage{hyperref}
  \hypersetup{colorlinks=true,
              allcolors=blue}
\usepackage{graphicx}
  \graphicspath{{../../figures/}}
\usepackage{soul}
  \setstcolor{red}
\usepackage[normalem]{ulem}
\usepackage{enumerate}
\usepackage{tikz}
  \usetikzlibrary{trees}

% Document information
\author{M. McNeill}
\title[FREN1001]{Français 1001}
\institute{\url{joshua.mcneill@uga.edu}}
\date{}

%% Custom commands
% Lexical items
\newcommand{\lexi}[1]{\textit{#1}}
% Gloss
\newcommand{\gloss}[1]{`#1'}
\newcommand{\tinygloss}[1]{{\tiny`#1'}}
% Orthographic representations
\newcommand{\orth}[1]{$\langle$#1$\rangle$}
% Utterances (pragmatics)
\newcommand{\uttr}[1]{`#1'}
% Sentences (pragmatics)
\newcommand{\sent}[1]{\textit{#1}}
% Fixed length underlines
\newcommand{\funderline}[2][4cm]{
  \underline{\makebox[\ifdim\width>#1\width\else#1\fi]{#2}}
}
% Base dir for definitions
\newcommand{\defs}{../definitions}
\newcommand{\activity}[1]{
  \input{./activities/#1.tex}
}


  % Packages and settings

  % Document information
  \subtitle[Il y a, c'est, ce sont]{Il y a, c'est, ce sont}

\begin{document}
  % Read in the standard intro slides (title page and table of contents)
  \begin{frame}
    \titlepage
    \tiny{Office: % Basically a variable for office hours location
Zoom (ID 978 2791 8221)
\\
          Office hours: % Basically a variable for office hours
 mercredi 10h15--13h15
}
  \end{frame}

  \begin{frame}{Annonces \gloss{Announcements}}
    \begin{enumerate}
      \item Il y a des leçons en vidéo sur eLC.
      \item[] \gloss{There are video lessons on eLC.}
      \item Vérifie les corrections sur tes exercices sur MFL.
      \item[] \gloss{Check the correction on your exercises on MFL.}
    \end{enumerate}
  \end{frame}

  \begin{frame}{}
    \begin{center}
      \Large Quiz
    \end{center}
  \end{frame}

  \begin{frame}[t]{Jacques a dit \gloss{Simon Says}}
    \parbox[t]{\linewidth}{
      Faites ce que Jacques vous a dit. \\
      \tinygloss{Do what James told you.}
    }
    \vspace{1.5cm}
    \begin{center}
      \only<2>{
        Levez-vous.
      }
      \only<3>{
        Asseyez-vous.
      }
      \only<4>{
        Montrez-moi un stylo.
      }
      \only<5>{
        Donnez le stylo à un/e étudiant/e.
      }
      \only<6>{
        Prenez une feuille de papier.
      }
      \only<7>{
        Levez-vous.
      }
      \only<8>{
        Écrivez en français: His/Her name is \underline{\hspace{2cm}}.
      }
      \only<9>{
        Rendez-moi les papiers.
      }
      \only<10>{
        Allez aux fenêtres.
      }
      \only<11>{
        Asseyez-vous.
      }
      \only<12>{
        Levez-vous et allez à vos chaises.
      }
    \end{center}
  \end{frame}

  \begin{frame}{\lexi{C'est} ou \lexi{ce sont}}
    Est-ce qu'on dit \lexi{c'est ...} ou \lexi{ce sont ...}?
    \begin{enumerate}
      \item \underline{\uncover<2->{\ \ C'est\ \ }} un crayon.
      \item \underline{\uncover<3->{\ \ C'est\ \ }} un tableau.
      \item \underline{\uncover<4->{Ce sont}} les papiers.
      \item \underline{\uncover<5->{Ce sont}} des bureaux.
      \item \underline{\uncover<6->{\ \ C'est\ \ }} l'écran.
      \item \underline{\uncover<7->{\ \ C'est\ \ }} \textbf<8->{le prof}.
      \item C'est \underline{\uncover<9->{\ \ lui\ \ }}.
    \end{enumerate}
  \end{frame}

  \begin{frame}{Sur mon bureau \gloss{On my desk}}
    \begin{itemize}
      \item En groupes de trois, dites à vos partenaires ce qui se trouve sur ton bureau chez toi.
        Par exemple: \emph{Sur mon bureau, il y a un écran, un ordinateur et des stylos.}
      \item[] \tinygloss{In groups of three, tell your partners what is on your desk at home.
        For example: \emph{Sur mon bureau, il y a un écran, un ordinateur et des stylos.}}
      \item Si tu ne parles pas, dessine ce que ton/ta partenaire a décrit pour voir si tu l'as bien compris/e.
      \item[] \tinygloss{If you're not speaking, draw what your paretner described to see if you correctly understood him or her.}
    \end{itemize}
  \end{frame}

  \begin{frame}{}
    \begin{center}
      \Large Questions?
    \end{center}
  \end{frame}
\end{document}
