%%%%%%%%%%%%%%%%%%%%%%%%%%%%%%%%%%%%%
%                                   %
% Compile with XeLaTeX and biber    %
%                                   %
% Questions or comments:            %
%                                   %
% joshua dot mcneill at uga dot edu %
%                                   %
%%%%%%%%%%%%%%%%%%%%%%%%%%%%%%%%%%%%%

\documentclass{beamer}
  % Read in standard preamble (cosmetic stuff)
  %%%%%%%%%%%%%%%%%%%%%%%%%%%%%%%%%%%%%%%%%%%%%%%%%%%%%%%%%%%%%%%%
% This is a standard preamble used in for all slide documents. %
% It basically contains cosmetic settings.                     %
%                                                              %
% Joshua McNeill                                               %
% joshua dot mcneill at uga dot edu                            %
%%%%%%%%%%%%%%%%%%%%%%%%%%%%%%%%%%%%%%%%%%%%%%%%%%%%%%%%%%%%%%%%

% Beamer settings
% \usetheme{Berkeley}
\usetheme{CambridgeUS}
% \usecolortheme{dove}
% \usecolortheme{rose}
\usecolortheme{seagull}
\usefonttheme{professionalfonts}
\usefonttheme{serif}
\setbeamertemplate{bibliography item}{}

% Packages and settings
\usepackage{fontspec}
  \setmainfont{Charis SIL}
\usepackage{hyperref}
  \hypersetup{colorlinks=true,
              allcolors=blue}
\usepackage{graphicx}
  \graphicspath{{../../figures/}}
\usepackage[normalem]{ulem}
\usepackage{enumerate}

% Document information
\author{M. McNeill}
\title[FREN2001]{Français 2001}
\institute{\url{joshua.mcneill@uga.edu}}
\date{}

%% Custom commands
% Lexical items
\newcommand{\lexi}[1]{\textit{#1}}
% Gloss
\newcommand{\gloss}[1]{`#1'}
\newcommand{\tinygloss}[1]{{\tiny`#1'}}
% Orthographic representations
\newcommand{\orth}[1]{$\langle$#1$\rangle$}
% Utterances (pragmatics)
\newcommand{\uttr}[1]{`#1'}
% Sentences (pragmatics)
\newcommand{\sent}[1]{\textit{#1}}
% Base dir for definitions
\newcommand{\defs}{../definitions}


  % Packages and settings

  % Document information
  \subtitle[Il y a, c'est, ce sont]{Il y a, c'est, ce sont}

\begin{document}
  % Read in the standard intro slides (title page and table of contents)
  \begin{frame}
    \titlepage
    \tiny{Office: % Basically a variable for office hours location
Gilbert 121\\
          Office hours: % Basically a variable for office hours
 lundi, mercredi, vendredi 10:10--11:10
}
  \end{frame}

  % Start by pointing to things and people and asking qu'est-ce que c'est
  % and c'est qui.

  \begin{frame}{Sur mon bureau \gloss{On my desk}}
    \gloss{In groups of three, tell your partners what is on your desk at home using the following model:}
    \begin{itemize}
      \item Sur mon bureau, il y a un écran, un ordinateur et des stylos.
    \end{itemize}
  \end{frame}

  \begin{frame}{}
    \begin{center}
      \Large Quiz
    \end{center}
  \end{frame}

  \begin{frame}{Sac à dos}
    \gloss{Your card represents what's in your imaginary \lexi{sac à dos}.
    With a partner, try to guess what is in each others bags, like so:}
    \begin{itemize}
      \item[E1] Est-ce qu'il y a ... dans ton sac?
      \item[E2] Non, il n'y a pas de ...
      \item[] OU
      \item[E2] Oui, il y a ...
    \end{itemize}
    % Have some students tell you the first thing in their bag, ask if other students
    % also have it, then follow with il y a DES ...
  \end{frame}

  \begin{frame}{Comment ça s'écrit? \gloss{How is that spelled?}}
    \gloss{Ask a partner how to spell their name and write it down. You can follow this model:}
    \begin{itemize}
      \item[E1] Comment tu t'appelles?
      \item[E2] Je m'appelle ...
      \item[E1] Comment ça s'écrit?
      \item[E2] Ça s'écrit ...
    \end{itemize}
  \end{frame}

  \begin{frame}{}
    \begin{center}
      \Large Questions?
    \end{center}
  \end{frame}
\end{document}
