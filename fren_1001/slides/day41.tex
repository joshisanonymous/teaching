%%%%%%%%%%%%%%%%%%%%%%%%%%%%%%%%%%%%%
%                                   %
% Compile with XeLaTeX and biber    %
%                                   %
% Questions or comments:            %
%                                   %
% joshua dot mcneill at uga dot edu %
%                                   %
%%%%%%%%%%%%%%%%%%%%%%%%%%%%%%%%%%%%%

\documentclass{beamer}
  % Read in standard preamble (cosmetic stuff)
  %%%%%%%%%%%%%%%%%%%%%%%%%%%%%%%%%%%%%%%%%%%%%%%%%%%%%%%%%%%%%%%%
% This is a standard preamble used in for all slide documents. %
% It basically contains cosmetic settings.                     %
%                                                              %
% Joshua McNeill                                               %
% joshua dot mcneill at uga dot edu                            %
%%%%%%%%%%%%%%%%%%%%%%%%%%%%%%%%%%%%%%%%%%%%%%%%%%%%%%%%%%%%%%%%

% Beamer settings
% \usetheme{Berkeley}
\usetheme{CambridgeUS}
% \usecolortheme{dove}
% \usecolortheme{rose}
\usecolortheme{seagull}
\usefonttheme{professionalfonts}
\usefonttheme{serif}
\setbeamertemplate{bibliography item}{}

% Packages and settings
\usepackage{fontspec}
  \setmainfont{Charis SIL}
\usepackage{hyperref}
  \hypersetup{colorlinks=true,
              allcolors=blue}
\usepackage{graphicx}
  \graphicspath{{../../figures/}}
\usepackage[normalem]{ulem}
\usepackage{enumerate}

% Document information
\author{M. McNeill}
\title[FREN2001]{Français 2001}
\institute{\url{joshua.mcneill@uga.edu}}
\date{}

%% Custom commands
% Lexical items
\newcommand{\lexi}[1]{\textit{#1}}
% Gloss
\newcommand{\gloss}[1]{`#1'}
\newcommand{\tinygloss}[1]{{\tiny`#1'}}
% Orthographic representations
\newcommand{\orth}[1]{$\langle$#1$\rangle$}
% Utterances (pragmatics)
\newcommand{\uttr}[1]{`#1'}
% Sentences (pragmatics)
\newcommand{\sent}[1]{\textit{#1}}
% Base dir for definitions
\newcommand{\defs}{../definitions}


  % Packages and settings

  % Document information
  \subtitle[Fruits et légumes, quantités et \lexi{en})]{Les fruits et légumes, les quantités et le pronom \lexi{en}}

\begin{document}
  % Read in the standard intro slides (title page and table of contents)
  \begin{frame}
    \titlepage
    \tiny{Office: % Basically a variable for office hours location
Gilbert 121\\
          Office hours: % Basically a variable for office hours
 lundi, mercredi, vendredi 10:10--11:10
}
  \end{frame}

  \begin{frame}{Annonces}
    \begin{itemize}
      \item L'atelier d'écriture vendredi ainsi que la rédaction
      \item[] \tinygloss{Writing workshop Friday as well as the composition}
    \end{itemize}
  \end{frame}

  \begin{frame}{}
    \begin{center}
      \Large Quiz
    \end{center}
  \end{frame}

  \begin{frame}{Tu en as combien?}
    Avec un/e partenaire, dis-lui combien tu as pour chaque numéro en utilisant le pronom \lexi{en}. \\
    \tinygloss{With a partner, tell them how many you have for each number using the pronoun \lexi{en}.}
    \begin{description}
      \item[] \textbf{Modèle:} \emph{des sœurs}
      \item[E1:] J'en ai deux. Elles s'appellent Holly et Amy.
      \item[E2:] Je n'en ai pas.
    \end{description}
    \begin{columns}
      \column{0.5\textwidth}
        \begin{enumerate}
          \item des sœurs
          \item des frères
          \item des amis
          \item des problèmes
        \end{enumerate}
        \column{0.5\textwidth}
        \begin{enumerate}
          \setcounter{enumi}{4}
          \item de l'argent
          \item des devoirs
          \item des responsabilités
          \item des vacances
        \end{enumerate}
    \end{columns}
  \end{frame}

  \begin{frame}{Qu'est-ce qu'il a acheté?}
    Avec un/e partenaire, imaginez tout ce que Philippe aurait pu acheter.
    Ensuite, écris un aliment chacun pour trois numéros sur le tableau. 
    \alert{Écris un aliment que personne n'a pas encore écrit.}\\
    \tinygloss{With a partner, imagine all the things that Philippe might have bought.
    Then, write one food each for three numbers on the board.
    \alert{Write a food that hasn't been written yet.}}
    \begin{description}
      \item[] \textbf{Modèle:} \emph{Il en a acheté une douzaine.}
      \item[E1:] Il a acheté une douzaine d'œufs.
      \item[E2:] Il a acheté une douzaine de citrons.
      \item[] \emph{On écrit <<une douzaine d'œufs>>}
    \end{description}
    \begin{columns}
      \column{0.5\textwidth}
        \begin{enumerate}
          \item Il en a pris un pot.
          \item Il en a acheté un morceau.
          \item Il en a pris une douzaine.
          \item Il en a acheté une bouteille.
        \end{enumerate}
        \column{0.5\textwidth}
        \begin{enumerate}
          \setcounter{enumi}{4}
          \item Il en a pris deux paquets.
          \item Il en a acheté un kilo.
          \item Il en a demandé dix tranches.
          \item Il en a acheté une boîte.
        \end{enumerate}
    \end{columns}
  \end{frame}

  \begin{frame}{Trouve tes fruits et légumes}
    Tu es à un marché fermier \gloss{farmer's market}, mais les marchands (tes camarades de classe) n'ont pas assez de fruits et légumes chacun.
    Trouve ceux dont tu as besoin en les demandant à tes camarades de classe.
    Une de tes cartes représente ce que tu vends et une carte ce dont tu as besoin. \\
    \tinygloss{You are at a farmer's market, but the merchants (your classmates) don't each have enough fruits and vegetables.
    Find those that you need by asking your classmates for them.
    One of your cards represents what you're selling and the other what you need.}
    \begin{columns}
      \column{0.6\textwidth}
        \small
        \begin{description}
          \item[E1:] Est-ce que tu as des pommes?
          \item[E2:] Non, j'\alert{en} ai pas.
          \item[] OU
          \item[E2:] Oui, j'\alert{en} ai \alert{deux}.
          \item[E1:] Est-ce que tu peux m'\alert{en} vendre?
          \item[E2:] Oui, tiens!
        \end{description}
      \column{0.4\textwidth}
        \begin{center}
          \includegraphics[scale=0.18]{marché_fermier.jpg}
        \end{center}
    \end{columns}
  \end{frame}

  \begin{frame}{}
    \begin{center}
      \Large Questions?
    \end{center}
  \end{frame}
\end{document}
