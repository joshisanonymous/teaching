%%%%%%%%%%%%%%%%%%%%%%%%%%%%%%%%%%%%%
%                                   %
% Compile with XeLaTeX and biber    %
%                                   %
% Questions or comments:            %
%                                   %
% joshua dot mcneill at uga dot edu %
%                                   %
%%%%%%%%%%%%%%%%%%%%%%%%%%%%%%%%%%%%%

\documentclass{beamer}
  % Read in standard preamble (cosmetic stuff)
  %%%%%%%%%%%%%%%%%%%%%%%%%%%%%%%%%%%%%%%%%%%%%%%%%%%%%%%%%%%%%%%%
% This is a standard preamble used in for all slide documents. %
% It basically contains cosmetic settings.                     %
%                                                              %
% Joshua McNeill                                               %
% joshua dot mcneill at uga dot edu                            %
%%%%%%%%%%%%%%%%%%%%%%%%%%%%%%%%%%%%%%%%%%%%%%%%%%%%%%%%%%%%%%%%

% Beamer settings
% \usetheme{Berkeley}
\usetheme{CambridgeUS}
% \usecolortheme{dove}
% \usecolortheme{rose}
\usecolortheme{seagull}
\usefonttheme{professionalfonts}
\usefonttheme{serif}
\setbeamertemplate{bibliography item}{}

% Packages and settings
\usepackage{fontspec}
  \setmainfont{Charis SIL}
\usepackage{hyperref}
  \hypersetup{colorlinks=true,
              allcolors=blue}
\usepackage{graphicx}
  \graphicspath{{../../figures/}}
\usepackage{soul}
  \setstcolor{red}
\usepackage[normalem]{ulem}
\usepackage{enumerate}
\usepackage{tikz}
  \usetikzlibrary{trees}

% Document information
\author{M. McNeill}
\title[FREN1001]{Français 1001}
\institute{\url{joshua.mcneill@uga.edu}}
\date{}

%% Custom commands
% Lexical items
\newcommand{\lexi}[1]{\textit{#1}}
% Gloss
\newcommand{\gloss}[1]{`#1'}
\newcommand{\tinygloss}[1]{{\tiny`#1'}}
% Orthographic representations
\newcommand{\orth}[1]{$\langle$#1$\rangle$}
% Utterances (pragmatics)
\newcommand{\uttr}[1]{`#1'}
% Sentences (pragmatics)
\newcommand{\sent}[1]{\textit{#1}}
% Fixed length underlines
\newcommand{\funderline}[2][4cm]{
  \underline{\makebox[\ifdim\width>#1\width\else#1\fi]{#2}}
}
% Base dir for definitions
\newcommand{\defs}{../definitions}
\newcommand{\activity}[1]{
  \input{./activities/#1.tex}
}


  % Packages and settings

  % Document information
  \subtitle[\lexi{Mettre} et adjectifs démonstratifs]{Le verbe \lexi{mettre} et les adjectifs démonstratifs}

\begin{document}
  % Read in the standard intro slides (title page and table of contents)
  \begin{frame}
    \titlepage
    \tiny{Office: % Basically a variable for office hours location
Zoom (ID 978 2791 8221)
\\
          Office hours: % Basically a variable for office hours
 mercredi 10h15--13h15
}
  \end{frame}

  \begin{frame}{Annonces}
    \begin{itemize}
      \item La révision de l'examen 2 va avoir lieu vendredi.
      \item[] \tinygloss{The exam 2 review will take place Friday.}
      \item L'examen 2 va avoir lieu lundi!
      \item[] \tinygloss{Exam 2 will take place Monday!}
    \end{itemize}
  \end{frame}

  \begin{frame}{}
    \begin{center}
      \Large Quiz
    \end{center}
  \end{frame}

  \begin{frame}{Le verbe}
    \begin{center}
      \begin{tabular}{l | l l | l l}
  \multicolumn{5}{c}{mettre \gloss{to put (on)}} \\
      & \multicolumn{2}{l |}{singulier} & \multicolumn{2}{l}{pluriel} \\
  \hline
  1re & je         & mets               & nous        & mettons \\
  2e  & tu         & mets               & vous        & mettez \\
  \hline
  3e  & il (masc)  &                    & ils (masc)  & \\
      & elle (fem) & met                & elles (fem) & mettent \\
      & on         &                    &             & \\
\end{tabular}

    \end{center}
  \end{frame}

  \begin{frame}[t]{Les vêtements}
    Qu'est-ce qu'on met?
    \vspace{0.5cm}
    \begin{columns}
      \column{0.5\textwidth}
        \scriptsize
        \begin{enumerate}
          \item Pour aller en classe, mes amis \underline{\uncover<2->{mettent}} ...
          \item Pour manger au resto U, nous \underline{\uncover<4->{mettons}} ...
          \item Pour courir dans un marathon, tu \underline{\uncover<6->{mets}} ...
          \item Pour faire des courses, ma mère \underline{\uncover<8->{met}} ...
          \item Pour travailler dans le jardin, mon père \underline{\uncover<10->{met}} ...
          \item Pour faire du ski, je \underline{\uncover<12->{mets}} ...
          \item Pour nager, elles \underline{\uncover<14->{mettent}} ...
          \item Pour sortir avec des amis, vous \underline{\uncover<16->{mettez}} ...
        \end{enumerate}
      \column{0.5\textwidth}
        \begin{minipage}[c][0.6\textheight]{\linewidth}
          \begin{center}
            \only<1-2>{
              \includegraphics[scale=0.125]{classe.png}
            }
            \only<3-4>{
              \includegraphics[scale=0.18]{restou.jpg}
            }
            \only<5-6>{
              \includegraphics[scale=0.09]{marathon.jpg}
            }
            \only<7-8>{
              \includegraphics[scale=0.18]{courses.jpg}
            }
            \only<9-10>{
              \includegraphics[scale=0.07]{giverny.jpg} \\
              Le jardin de Monet à Giverny
            }
            \only<11-12>{
              \includegraphics[scale=0.125]{kingsbury.jpg} \\
              Mikaël Kingsbury du Québec
            }
            \only<13-14>{
              \includegraphics[scale=0.2]{balmy.jpg} \\
              Coralie Balmy de la Martinique
            }
            \only<15-16>{
              \includegraphics[scale=0.14]{bar_halloween.jpg}
            }
          \end{center}
        \end{minipage}
    \end{columns}
  \end{frame}

  \begin{frame}{Les voyelles /ø/ et /œ/}
    Quelle voyelle ai-je prononcé dans le mot?
    \begin{columns}
      \column{0.5\textwidth}
        \begin{enumerate}
          \item peu \only<-1>{/ø/}\only<2->{\underline{/ø/}} /œ/
          \item veulent /ø/ \only<-2>{/œ/}\only<3->{\underline{/œ/}}
          \item tailleur /ø/ \only<-3>{/œ/}\only<4->{\underline{/œ/}}
          \item bleu \only<-4>{/ø/}\only<5->{\underline{/ø/}} /œ/
          \item jeudi \only<-5>{/ø/}\only<6->{\underline{/ø/}} /œ/
        \end{enumerate}
      \column{0.5\textwidth}
        \begin{enumerate}
          \setcounter{enumi}{5}
          \item couleur /ø/ \only<-6>{/œ/}\only<7->{\underline{/œ/}}
          \item monsieur \only<-7>{/ø/}\only<8->{\underline{/ø/}} /œ/
          \item vendeuse \only<-8>{/ø/}\only<9->{\underline{/ø/}} /œ/
          \item sœur /ø/ \only<-9>{/œ/}\only<10->{\underline{/œ/}}
          \item neuf /ø/ \only<-10>{/œ/}\only<11->{\underline{/œ/}}
        \end{enumerate}
    \end{columns}
  \end{frame}

  \begin{frame}{Mettez ou jetez?}
    Avec un/e partenaire, décidez si vous allez \textbf{mettre} les articles suivants ou les \textbf{jeter}. \\
    \tinygloss{With a partner, decide you will \textbf{put/try on} the following articles or \textbf{throw them out}.} \\
    \begin{description}
      \item[\textbf{Modèle:}] \emph{un teeshirt blanc de l'université}
      \item[E1:] Tu mets ce teeshirt.
      \item[] \emph{une robe démodée}
      \item[E2:] Tu dois jeter cette robe.
    \end{description}
    \begin{columns}
      \column{0.5\textwidth}
        \begin{enumerate}
          \item des bottes chic
          \item un teeshirt serré
          \item des gants roses, violets et noirs
          \item un manteau trop large
          \item un pantalon trop court
        \end{enumerate}
      \column{0.5\textwidth}
        \begin{enumerate}
          \setcounter{enumi}{5}
          \item une belle écharpe
          \item des baskets à la mode
          \item un blouson démodé
          \item un vieux pull
          \item des mocassins démodés
        \end{enumerate}
    \end{columns}
  \end{frame}

  \begin{frame}{Marier let vêtements}
    Avec un/e partenaire, décide ce qui va bien avec chaque article. \\
    \tinygloss{With a partner, decide what goes well with each article.}
    \begin{description}
      \item[] \textbf{Modèle:} \lexi{avec une robe bleue en soie}
      \item[E1:] Avec une robe bleue en soie, on peut mettre un foulard bleu et vert.
      \item[E2:] Et des chausseurs à talons.
    \end{description}
    \begin{columns}
      \column{0.45\textwidth}
        \begin{enumerate}
          \item avec une mini-jupe rouge
          \item avec un costume bleu
          \item avec un pantalon bleu
          \item avec une veste noire
        \end{enumerate}
      \column{0.55\textwidth}
        \begin{enumerate}
          \setcounter{enumi}{4}
          \item avec une belle jupe multicolore
          \item avec un tailleur pantalon marron
          \item avec un jean
          \item avec un pantalon noir en cuir
        \end{enumerate}
    \end{columns}
  \end{frame}

  \begin{frame}{}
    \begin{center}
      \Large Questions?
    \end{center}
  \end{frame}
\end{document}
