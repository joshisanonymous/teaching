%%%%%%%%%%%%%%%%%%%%%%%%%%%%%%%%%%%%%
%                                   %
% Compile with XeLaTeX and biber    %
%                                   %
% Questions or comments:            %
%                                   %
% joshua dot mcneill at uga dot edu %
%                                   %
%%%%%%%%%%%%%%%%%%%%%%%%%%%%%%%%%%%%%

\documentclass{beamer}
  % Read in standard preamble (cosmetic stuff)
  %%%%%%%%%%%%%%%%%%%%%%%%%%%%%%%%%%%%%%%%%%%%%%%%%%%%%%%%%%%%%%%%
% This is a standard preamble used in for all slide documents. %
% It basically contains cosmetic settings.                     %
%                                                              %
% Joshua McNeill                                               %
% joshua dot mcneill at uga dot edu                            %
%%%%%%%%%%%%%%%%%%%%%%%%%%%%%%%%%%%%%%%%%%%%%%%%%%%%%%%%%%%%%%%%

% Beamer settings
% \usetheme{Berkeley}
\usetheme{CambridgeUS}
% \usecolortheme{dove}
% \usecolortheme{rose}
\usecolortheme{seagull}
\usefonttheme{professionalfonts}
\usefonttheme{serif}
\setbeamertemplate{bibliography item}{}

% Packages and settings
\usepackage{fontspec}
  \setmainfont{Charis SIL}
\usepackage{hyperref}
  \hypersetup{colorlinks=true,
              allcolors=blue}
\usepackage{graphicx}
  \graphicspath{{../../figures/}}
\usepackage[normalem]{ulem}
\usepackage{enumerate}

% Document information
\author{M. McNeill}
\title[FREN2001]{Français 2001}
\institute{\url{joshua.mcneill@uga.edu}}
\date{}

%% Custom commands
% Lexical items
\newcommand{\lexi}[1]{\textit{#1}}
% Gloss
\newcommand{\gloss}[1]{`#1'}
\newcommand{\tinygloss}[1]{{\tiny`#1'}}
% Orthographic representations
\newcommand{\orth}[1]{$\langle$#1$\rangle$}
% Utterances (pragmatics)
\newcommand{\uttr}[1]{`#1'}
% Sentences (pragmatics)
\newcommand{\sent}[1]{\textit{#1}}
% Base dir for definitions
\newcommand{\defs}{../definitions}


  % Packages and settings

  % Document information
  \subtitle[Carrières et verbes]{Les carrières et les verbes \lexi{pouvoir}, \lexi{devoir}, \lexi{vouloir}}

\begin{document}
  % Read in the standard intro slides (title page and table of contents)
  \begin{frame}
    \titlepage
    \tiny{Office: % Basically a variable for office hours location
Gilbert 121\\
          Office hours: % Basically a variable for office hours
 lundi, mercredi, vendredi 10:10--11:10
}
  \end{frame}

  \begin{frame}{Annonces}
    \begin{itemize}
      \item L'atelier de lecture lundi.
      \item[] \tinygloss{Reading workshop Monday.}
      \item Le devoir 3 à rendre le 17 octobre (lundi).
      \item[] \tinygloss{Homework 3 due October 17th (Monday).}
      \item Le devoir 4 à rendre le 26 octobre.
      \item[] \tinygloss{Homework 4 due October 26th.}
    \end{itemize}
  \end{frame}

  \begin{frame}{Les verbes \lexi{pouvoir} et \lexi{vouloir}}
    \scriptsize
        \begin{center}
          \noindent\begin{tabular}{l | l l | l l}
  \multicolumn{5}{c}{pouvoir \gloss{to be able to/to can}} \\
      & \multicolumn{2}{l |}{singulier} & \multicolumn{2}{l}{pluriel} \\
  \hline
  1re & je         & peux               & nous        & pouvons \\
  2e  & tu         & peux               & vous        & pouvez \\
  \hline
  3e  & il (masc)  &                    & ils (masc)  & \\
      & elle (fem) & peut               & elles (fem) & peuvent \\
      & on         &                    &             & \\
\end{tabular}

          \noindent\begin{tabular}{l | l l | l l}
  \multicolumn{5}{c}{vouloir \gloss{to want}} \\
      & \multicolumn{2}{l |}{singulier} & \multicolumn{2}{l}{pluriel} \\
  \hline
  1re & je         & veux               & nous        & voulons \\
  2e  & tu         & veux               & vous        & voulez \\
  \hline
  3e  & il (masc)  &                    & ils (masc)  & \\
      & elle (fem) & veut               & elles (fem) & veulent \\
      & on         &                    &             & \\
\end{tabular}

        \end{center}
  \end{frame}

  \begin{frame}{Le verbe \lexi{devoir}}
    \begin{center}
      \begin{tabular}{l | l l | l l}
  \multicolumn{5}{c}{devoir \gloss{to have to/to must/to owe}} \\
      & \multicolumn{2}{l |}{singulier} & \multicolumn{2}{l}{pluriel} \\
  \hline
  1re & je         & dois               & nous        & devons \\
  2e  & tu         & dois               & vous        & devez \\
  \hline
  3e  & il (masc)  &                    & ils (masc)  & \\
      & elle (fem) & doit               & elles (fem) & doivent \\
      & on         &                    &             & \\
\end{tabular}

    \end{center}
  \end{frame}

  \begin{frame}{La politesse}
    Est-ce que l'ordre de ton patron \gloss{boss} est \emph{ poli } ou \emph{impoli}?
    \begin{center}
      \begin{enumerate}
        \item Apportez-moi les rapports! \underline{\uncover<2->{impoli}}
        \item Vous voulez bien préparer un mémorandum? \underline{\uncover<3->{ poli  }}
        \item Vous pouvez attendre un instant? \underline{\uncover<4->{ poli }}
        \item Vous voulez bien lire ce message? \underline{\uncover<5->{ poli }}
        \item Téléphonez à ces clients! \underline{\uncover<6->{impoli}}
        \item Vous pouvez répondre à ces questions? \underline{\uncover<7->{ poli }}
        \item Vous voulez téléphoner au directeur? \underline{\uncover<8->{ poli }}
        \item Fermez la porte du bureau! \underline{\uncover<9->{impoli}}
      \end{enumerate}
    \end{center}
  \end{frame}

  \begin{frame}{}
    \begin{center}
      \Large Quiz
    \end{center}
  \end{frame}

  \begin{frame}{Les voyelles /o/ et /ɔ/}
    \begin{columns}
      \column{0.5\textwidth}
        \begin{enumerate}
          \item fort $\to$ /o/ \only<-1>{/ɔ/}\only<2->{\underline{/ɔ/}}
          \item mot $\to$ \only<-2>{/o/}\only<3->{\underline{/o/}} /ɔ/
          \item bureau$\to$ \only<-3>{/o/}\only<4->{\underline{/o/}} /ɔ/
          \item bonne $\to$ \only<-4>{/o/}\only<5->{\underline{/o/}} /ɔ/
        \end{enumerate}
      \column{0.5\textwidth}
        \begin{enumerate}
          \setcounter{enumi}{4}
          \item trop $\to$ \only<-5>{/o/}\only<6->{\underline{/o/}} /ɔ/
          \item diplôme $\to$ \only<-6>{/o/}\only<7->{\underline{/o/}} /ɔ/
          \item golf $\to$ /o/ \only<-7>{/ɔ/}\only<8->{\underline{/ɔ/}}
          \item prof $\to$ \only<-8>{/o/}\only<9->{\underline{/o/}} /ɔ/
        \end{enumerate}
    \end{columns}
  \end{frame}

  \begin{frame}{Les nécessités}
    En groupes de 3 ou 4, parlez de ce qu'on doit faire pour devenir les professions suivantes.\\
    \tinygloss{In groups of 3 or 4, talk about what one must do in order to become the following professions.}
    \begin{columns}
      \column{0.5\textwidth}
        \begin{description}
          \item[] \textbf{Modèle:}
          \item[] \emph{assistant/e social/e}
          \item[E1:] Pour être assistant social, on doit assister à l'université.
          \item[E2:] On doit aussi étudier la sociologie.
          \item[E3:] Pour faire le travail, on doit aider les gens.
        \end{description}
      \column{0.5\textwidth}
        \begin{enumerate}
          \item avocat/e
          \item médecin/e
          \item musicien/ne
          \item informaticien/ne
          \item professeur/e
          \item comptable
        \end{enumerate}
    \end{columns}
  \end{frame}

  \begin{frame}{Vos plans futurs}
    Encore en groupes de 3 ou 4, parlez de vos plans futurs par rapport aux sujets suivants. \\
    \tinygloss{In groups of 3 or 4, talk about your future plans with respect to the following subjects}
    \begin{columns}
      \column{0.6\textwidth}
        \begin{description}
          \item[] \textbf{Modèle:}
          \item[] \emph{faire comme travail}
          \item[E1:] Qu'est-ce que vous voulez faire comme travail? (to the group)
          \item[E2:] Je veux aider les gens. Je veux être médecin ou dentiste.
          \item[E3:] Moi, je veux être architecte. J'aime le dessin.
        \end{description}
      \column{0.4\textwidth}
        \begin{enumerate}
          \item faire comme travail
          \item habiter
          \item voyager
          \item avoir des enfants
          \item gagner de l'argent
        \end{enumerate}
    \end{columns}
  \end{frame}

  \begin{frame}{}
    \begin{center}
      \Large Questions?
    \end{center}
  \end{frame}
\end{document}
