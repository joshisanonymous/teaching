\begin{frame}{Les stéréotypes \gloss{Stereotypes}}
  Avec un/e partenaire, discutez si les stéréotypes suivants sont \emph{vrais} \gloss{true} ou \emph{faux} \gloss{false}.
  Utilisez les adverbes de fréquence ou d'intensité (\lexi{trop}, \lexi{beaucoup}, \lexi{assez}, \lexi{un peu}, \lexi{ne..pas}). \\
  \tinygloss{With a partner, discuss if the following stereotypes are true or false.
  Use frequency or intensity adverbs (\lexi{trop}, \lexi{beaucoup}, \lexi{assez}, \lexi{un peu}, \lexi{ne..pas}).}
  \begin{columns}[t]
    \column{0.5\textwidth}
      \scriptsize
      \begin{description}
        \item[] \textbf{Modèle:}
        \item[] \emph{les jeunes / aime trop McDo}
        \item[E1:] C'est vrai! Les jeunes aiment trop McDo.
        \item[] \tinygloss{That's true! Young people like McDonalds too much.}
        \item[E2:] Non, c'est faux! Les jeunes n'aiment pas McDo.
        \item[] \tinygloss{No, that's false! Young people don't like McDonalds.}
      \end{description}
    \column{0.5\textwidth}
      \scriptsize
      \begin{enumerate}
        \item les jeunes / manger beaucoup de fast-foods
        \item les grands-parents / s'endormir toujours devant la télé
        \item les parents / se coucher trop tôt
        \item les musiciens / se coucher trop tard
        \item les sportifs / ne jamais étudier
        \item les étudiants / ne pas avoir d'argent
        \item les animaux de compagnie / se lever trop tôt
        \item les professeurs / donner trop de devoirs
      \end{enumerate}
  \end{columns}
\end{frame}