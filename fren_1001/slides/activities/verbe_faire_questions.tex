\begin{frame}{Faisons des questions}
  Pour chaque mot interrogatif au-dessous, écris sur une feuille de papier une phrase complète en utilisant des expressions avec \lexi{faire}.
  N'utilise pas une expression plus qu'une fois! Par exemple:\\
  \tinygloss{
    For each question word below, write on a piece of paper one complete sentence using expressions with \lexi{faire}.
    Don't use an expression more than one time! For example:
  } \\
  \vspace{0.25cm}
  \begin{columns}
    \column{0.5\textwidth}
      \begin{itemize}
        \item[] \lexi{faire du sport} + \lexi{pourquoi}
        \item[E1:] Pourquoi fais-tu du sport le matin?
      \end{itemize}
    \column{0.5\textwidth}
      Les mots interrogatifs:
      \begin{enumerate}
        \item qui
        \item quand
        \item où
        \item pourquoi
        \item comment
        \item combien de
      \end{enumerate}
  \end{columns}
\end{frame}