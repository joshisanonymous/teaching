\begin{frame}{Dessinons la famille}
  \begin{columns}
    \column{0.5\textwidth}
      Décris une femme dans ta famille pour qu'un/e partenaire la dessine.
      Utilise le nouveau vocabulaire, et sois aussi descriptif que possible. \\
      \tinygloss{Describe a woman in your family for a partner to draw her.
      Use the new vocabulary, and be as descriptive as possible.}
    \column{0.25\textwidth}
      \tiny
      Elle est...
      \begin{enumerate}
        \item âgée
        \item belle
        \item blonde
        \item brune
        \item châtain
        \item de taille moyenne
        \item d'un certain âge
        \item élégante
        \item forte
        \item grande
        \item grosse
        \item jeune
        \item jolie
        \item laide
        \item maigre
        \item mal habillée
        \item mince
        \item petite
        \item rousse
      \end{enumerate}
    \column{0.25\textwidth}
      \tiny
      % \begin{enumerate}
        
      % \end{enumerate}
      Elle a les yeux...
      \begin{enumerate}
        \setcounter{enumi}{19}
        \item bleus
        \item marron
        \item noirs
        \item noisette
        \item verts
      \end{enumerate}
      Elle a les cheveux...
      \begin{enumerate}
        \setcounter{enumi}{24}
        \item blonds
        \item bouclés
        \item bruns
        \item châtains
        \item courts
        \item frisés
        \item gris
        \item longs
        \item noirs
        \item raides
        \item roux
      \end{enumerate}
  \end{columns}
\end{frame}

\begin{frame}{Écrivons}
  \begin{itemize}
    \item Rendez-moi les dessins.
    \item[] \tinygloss{Turn in your drawings.}
    \item<2-> Écrivez une description du dessin que vous avez reçu par des phrases complètes.
    \item<2->[] \tinygloss{Write a description of the drawing you received using complete sentences.}
  \end{itemize}
\end{frame}