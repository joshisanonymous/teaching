\begin{frame}{Les mots interrogatifs}
  Remplaçons la phrase en \emph{italique} avec le mot interrogatif le plus logique. \\
  \tinygloss{Let's replace the phrase in \emph{italics} with the most logical question word.}
  \begin{columns}[t]
    \column{0.5\textwidth}
      \begin{enumerate}
        \item Nous travaillons \emph{dans la salle de classe}.
        \item<2->[$\to$] \emph{Où} est-ce que nous travaillons?
        \item Il y a un examen \emph{mardi}.
        \item<3->[$\to$] \emph{Quand} est-ce qu'il y a un examen?
        \item Elle a \emph{deux} sœurs.
        \item<4->[$\to$] \emph{Combien de} sœurs est-ce qu'elle a?
      \end{enumerate}
    \column{0.5\textwidth}
      \begin{enumerate}
        \setcounter{enumi}{3}
        \item Yannick est absent \emph{parce qu'il est malade}.
        \item<5->[$\to$] \emph{Pourquoi} est-ce que Yannick est absent?
        \item \emph{Jacky} est généreuse.
        \item<6->[$\to$] \emph{Qui} est généreuse?
        \item Elle s'appelle \emph{Chloé Dupont}.
        \item<7->[$\to$] \emph{Comment} est-ce qu'elle s'appelle?
      \end{enumerate}
  \end{columns}
\end{frame}