\begin{frame}{Qu'est-ce qu'il a acheté?}
  Avec un/e partenaire, imaginez tout ce que Philippe aurait pu acheter.
  Ensuite, écris un aliment chacun pour trois numéros sur le tableau. 
  \alert{Écris un aliment que personne n'a pas encore écrit.}\\
  \tinygloss{With a partner, imagine all the things that Philippe might have bought.
  Then, write one food each for three numbers on the board.
  \alert{Write a food that hasn't been written yet.}}
  \begin{description}
    \item[] \textbf{Modèle:} \emph{Il en a acheté une douzaine.}
    \item[E1:] Il a acheté une douzaine d'œufs.
    \item[E2:] Il a acheté une douzaine de citrons.
    \item[] \emph{On écrit <<une douzaine d'œufs>>}
  \end{description}
  \begin{columns}
    \column{0.5\textwidth}
      \begin{enumerate}
        \item Il en a pris un pot.
        \item Il en a acheté un morceau.
        \item Il en a pris une douzaine.
        \item Il en a acheté une bouteille.
      \end{enumerate}
      \column{0.5\textwidth}
      \begin{enumerate}
        \setcounter{enumi}{4}
        \item Il en a pris deux paquets.
        \item Il en a acheté un kilo.
        \item Il en a demandé dix tranches.
        \item Il en a acheté une boîte.
      \end{enumerate}
  \end{columns}
\end{frame}