\begin{frame}{Trouvez une personne...}
  \begin{columns}
    \column{0.5\textwidth}
      Pour chaque description, trouve un/e camarade de classe pour qui elle est vraie, puis écris le nom de ce/tte camarade. \\
      \tinygloss{For each description, find a classmate for whom it is true, then write down that classmate's name.}
    \column{0.5\textwidth}
      {\scriptsize
      \begin{description}
        \item[] \textbf{Modèle:}
        \item[] \emph{... qui a un bon prof de maths.}
        \item[E1:] Est-ce que tu as un bon prof de maths?
        \item[] \tinygloss{Do you have a good math professor?}
        \item[E2:] Non, je n'ai pas un bon prof de maths. (\emph{ask another classmate})
        \item[] \tinygloss{No, I don't have a good math professor.}
        \item[E3:] Oui, j'ai un bon prof de maths.
        \item[] \tinygloss{Yes, I have a good math professor.}
      \end{description}
      }
  \end{columns}
  \vspace{0.5cm}
  {\scriptsize
  \begin{columns}[t]
    \column{0.5\textwidth}
      \begin{enumerate}
        \item ... qui habite une vieille résidence.
        \item ... qui habite un bel appartement.
        \item ... qui a un nouvel ordinateur.
        \item ... qui a une petite voiture.
        \item ... qui a son premier cours à huit heures du matin.
      \end{enumerate}
    \column{0.5\textwidth}
      \begin{enumerate}
        \setcounter{enumi}{5}
        \item ... qui prépare un grand examen.
        \item ... qui est en première année de fac.
        \item ... qui est en dernière année de fac.
        \item ... qui a un bon prof de maths.
        \item ... qui a un vieil ami sur le campus.
      \end{enumerate}
  \end{columns}
  }
\end{frame}