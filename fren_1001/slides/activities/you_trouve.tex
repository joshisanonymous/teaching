\begin{frame}{Toi ou vous?}
  Trouve un/e partenaire, et imagine que ce partenaire est la personne sur la carte.
  Présente-toi, et demandez l'un à l'autre comment ça va.
  Par exemple:\\
  \tinygloss{Find a partner, and imagine that they are the person on their card.
  Introduce yourself, and ask each other how you are doing.
  For example:}

  \begin{columns}
    \column{0.46\textwidth}
      <E1 $\to$ teacher> \\
      <E2 $\to$ student>
      \begin{itemize}
        \item[E1:] Bonjour, je m'appelle Monsieur McNeill. Comment \alert{tu t'appelles}?
        \item[E2:] Bonjour, je m'appelle ...
        \item[E1:] Comment ça va?
        \item[E2:] Je suis fatigué/e, et \alert{vous}?
        \item[E1:] Je vais bien.
      \end{itemize}
    \column{0.54\textwidth}
      \small
      \begin{center}
        \begin{tabular}{| l l}
          ça va         & \gloss{fine} \\
          je vais bien  & \gloss{I'm fine} \\
          très bien     & \gloss{very well} \\
          pas mal       & \gloss{not bad} \\
          ça peut aller & \gloss{I'm getting by} \\
          ça va mal     & \gloss{things are going badly} \\
          en forme      & \gloss{in shape} \\
          fatigué/e     & \gloss{tired} \\
          malade        & \gloss{sick} \\
          occupé/e      & \gloss{busy} \\
          stressé/e     & \gloss{stressed out} \\
        \end{tabular}
      \end{center}
  \end{columns}
\end{frame}