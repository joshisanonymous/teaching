\begin{frame}{Tu en as combien?}
  \only<1>{
    Avec un/e partenaire, dis-lui combien tu as pour chaque numéro en utilisant le pronom \lexi{en}. \\
    \tinygloss{With a partner, tell them how many you have for each number using the pronoun \lexi{en}.}
    \begin{description}
      \item[] \textbf{Modèle:} \emph{des sœurs}
      \item[E1:] J'en ai deux. Elles s'appellent Holly et Amy.
      \item[E2:] Je n'en ai pas.
    \end{description}
  }
  \only<2->{
    Maintenant, sur une feuille de papier, écris tes réponses pour chaque numéro. \\
    \vspace{2.42cm}
  }
  \begin{columns}
    \column{0.5\textwidth}
      \begin{enumerate}
        \item des sœurs
        \item des frères
        \item des amis
        \item des problèmes
      \end{enumerate}
      \column{0.5\textwidth}
      \begin{enumerate}
        \setcounter{enumi}{4}
        \item de l'argent
        \item des devoirs
        \item des responsabilités
        \item des vacances
      \end{enumerate}
  \end{columns}
\end{frame}