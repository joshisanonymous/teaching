%%%%%%%%%%%%%%%%%%%%%%%%%%%%%%%%%%%%%
%                                   %
% Compile with XeLaTeX and biber    %
%                                   %
% Questions or comments:            %
%                                   %
% joshua dot mcneill at uga dot edu %
%                                   %
%%%%%%%%%%%%%%%%%%%%%%%%%%%%%%%%%%%%%

\documentclass{beamer}
  % Read in standard preamble (cosmetic stuff)
  %%%%%%%%%%%%%%%%%%%%%%%%%%%%%%%%%%%%%%%%%%%%%%%%%%%%%%%%%%%%%%%%
% This is a standard preamble used in for all slide documents. %
% It basically contains cosmetic settings.                     %
%                                                              %
% Joshua McNeill                                               %
% joshua dot mcneill at uga dot edu                            %
%%%%%%%%%%%%%%%%%%%%%%%%%%%%%%%%%%%%%%%%%%%%%%%%%%%%%%%%%%%%%%%%

% Beamer settings
% \usetheme{Berkeley}
\usetheme{CambridgeUS}
% \usecolortheme{dove}
% \usecolortheme{rose}
\usecolortheme{seagull}
\usefonttheme{professionalfonts}
\usefonttheme{serif}
\setbeamertemplate{bibliography item}{}

% Packages and settings
\usepackage{fontspec}
  \setmainfont{Charis SIL}
\usepackage{hyperref}
  \hypersetup{colorlinks=true,
              allcolors=blue}
\usepackage{graphicx}
  \graphicspath{{../../figures/}}
\usepackage[normalem]{ulem}
\usepackage{enumerate}

% Document information
\author{M. McNeill}
\title[FREN2001]{Français 2001}
\institute{\url{joshua.mcneill@uga.edu}}
\date{}

%% Custom commands
% Lexical items
\newcommand{\lexi}[1]{\textit{#1}}
% Gloss
\newcommand{\gloss}[1]{`#1'}
\newcommand{\tinygloss}[1]{{\tiny`#1'}}
% Orthographic representations
\newcommand{\orth}[1]{$\langle$#1$\rangle$}
% Utterances (pragmatics)
\newcommand{\uttr}[1]{`#1'}
% Sentences (pragmatics)
\newcommand{\sent}[1]{\textit{#1}}
% Base dir for definitions
\newcommand{\defs}{../definitions}


  % Bibliography
  \usepackage[backend=biber,style=apa]{biblatex}
    \addbibresource{../References.bib}

  % Document information
  \subtitle[Welcome]{Welcome}

  %% Custom commands
  % Subsection/frame titles
  \newcommand{\subtwoone}{What is linguistics?}
  \newcommand{\subtwotwo}{What is language?}
  \newcommand{\subtwothree}{\citeauthor{hockett_origin_1960}'s (\citeyear{hockett_origin_1960}) design features of language}
  \newcommand{\subtwofour}{What you know about your language}
  \newcommand{\subtwofive}{What else?}
  \newcommand{\subtwosix}{How we study this stuff}
  \newcommand{\subtwoseven}{What about writing?}

\begin{document}
  % Read in the standard intro slides (title page and table of contents)
  %%%%%%%%%%%%%%%%%%%%%%%%%%%%%%%%%%%%%%%%%%%%%%%%%%%%%%%%%%%%%%%%
% This is a standard set of intro slides used in for all slide %
% documents. It basically contains the title page and table of %
% contents.                                                    %
%                                                              %
% Joshua McNeill                                               %
% joshua dot mcneill at uga dot edu                            %
%%%%%%%%%%%%%%%%%%%%%%%%%%%%%%%%%%%%%%%%%%%%%%%%%%%%%%%%%%%%%%%%

\begin{frame}
  \titlepage
  \tiny{Office: % Basically a variable for office hours location
Gilbert 121\\
        Office hours: % Basically a variable for office hours
 lundi, mercredi, vendredi 10:10--11:10
}
\end{frame}

\begin{frame}
  \tableofcontents[hideallsubsections]
\end{frame}

\AtBeginSection[]{
  \begin{frame}
    \tableofcontents[currentsection,
                     hideallsubsections]
  \end{frame}
}


  \section{Course Information}
    \begin{frame}{Description}
      \begin{block}{}
        The \emph{scientific} study of language, emphasizing such topics as the organization of grammar, language in space and time, and the relationship between the study of language and other disciplines.
      \end{block}
    \end{frame}

    \begin{frame}{Objectives}
      \begin{block}{}
        By the end of this course, you will be able to:
        \begin{itemize}
          \item Use linguistic terminology accurately
          \item Discuss basic linguistic facts, theories, and methodologies
          \item Analyze samples of written or spoken language from a variety of world languages
          \item Understand both language change and how language supports all learning and communication
        \end{itemize}
      \end{block}
    \end{frame}

    \begin{frame}{Materials}
      \begin{block}{}
        \fullcitebib{dawson_language_2016}
      \end{block}
    \end{frame}

    \begin{frame}{Grading}
      \begin{block}{}
        \begin{tabular}{r l}
          Participation & 11\%\\
          Homework      & 45\% (4x11\% + 1x1\%)\\
          Exams         & 44\% (4x11\%)
        \end{tabular}
      \end{block}

      \begin{alertblock}<2->{}
        \begin{itemize}
          \item Everything is 11\% of your grade
          \item There will be 4 exams (3 regular + final)
        \end{itemize}
      \end{alertblock}
    \end{frame}

    \begin{frame}{Attendance}
      \begin{block}{}
        Basically your participation grade:
        \begin{itemize}
          \item Allowed 3 absences, \emph{no excuses required}
          \item Each absence $>$3 reduces your participation grade by 1\% (out of the 11\%)
        \end{itemize}
      \end{block}
      \begin{alertblock}<2>{That said... please participate}
        \begin{itemize}
          \item I'll try not to ask dumb questions
          \item You try to to stop me to ask questions when you don't understand
        \end{itemize}
      \end{alertblock}
    \end{frame}

    \begin{frame}{Make-up Work}
      \begin{block}{}
        Not accepted unless you say something \emph{before the due date} so that we can work something out
      \end{block}
    \end{frame}

    \begin{frame}{Schedule}
      \begin{block}{}
        By the beginning of each class, you will be expected to have:
        \begin{itemize}
          \item Read the relevant section(s) of the book
          \item Turned in any assignment(s) that are due
        \end{itemize}
      \end{block}
    \end{frame}

  \section{Introduction}
    \subsection{\subtwoone}
      \begin{frame}{\subtwoone}
        \begin{definition}<2->
          The \emph{scientific} study of how language works
        \end{definition}
        \begin{block}<3->{A subfield of}
          \begin{itemize}
            \item Psychology and cognitive science when we're talking about the structure of a language
            \item Sociology and anthropology when we're talking about use, variation, and change
          \end{itemize}
        \end{block}
      \end{frame}

    \subsection{\subtwotwo}
      \begin{frame}{\subtwotwo}
        \begin{definition}<2->{}
          \begin{itemize}
            \item ``Language'': Human communication
            \item ``\emph{A} language'': A systematic set of words and rules that can be used for human communication \emph{and that is unique to each individual}
          \end{itemize}
        \end{definition}
        \begin{alertblock}<3->{}
          In this class, references to things like ``English'' are shorthand for ``the set of features that happen to be shared in the languages of some group of people''
        \end{alertblock}
      \end{frame}

    \subsection{\subtwothree}
      \begin{frame}{\subtwothree}
        \only<1-2>{
          \begin{block}<1-2>{Mode of communication}
            Languages can be communicated in different ways
            \begin{itemize}
              \item speech, gesture, writing
            \end{itemize}
          \end{block}
          \begin{block}<2>{Semanticity}
            Language that is communicated has some meaning or function
            \begin{itemize}
              \item ``pizza'' means something, and you can assume that ``alveolar'' means something
            \end{itemize}
          \end{block}
        }
        \only<3-4>{
          \begin{block}<3-4>{Pragmatic function}
            Language that is communicated has some purpose
            \begin{itemize}
              \item I didn't tell you about ``alveolar'' for nothing
            \end{itemize}
          \end{block}
          \begin{block}<4>{Interchangeability}
            Language can be transmitted and interpreted
            \begin{itemize}
              \item This is happening right now
            \end{itemize}
          \end{block}
        }
        \only<5-6>{
          \begin{block}<5-6>{Cultural transmission}
            Features of languages can be passed from person to person through interaction
            \begin{itemize}
              \item You will (hopefully) know what ``alveolar'' means by the end of this course
            \end{itemize}
          \end{block}
          \begin{block}<6>{Arbitrariness} % Book introduces symbols and icons here
            The connection between the forms of a language's words and their meanings is arbitrary
            \begin{itemize}
              \item A strawberry is called ``strawberry'' in English but ``fraise'' in French
            \end{itemize}
          \end{block}
        }
        \only<7-8>{
          \begin{block}<7-8>{Discreteness}
            The elements of a language are made up of smaller elements
            \begin{itemize}
              \item Consonants and vowels make up words, words make up sentences
            \end{itemize}
          \end{block}
          \begin{block}<8>{Displacement}
            Language can be used to communicate about things that are not present
            \begin{itemize}
              \item ``Last week, I went to New Orleans.''
            \end{itemize}
          \end{block}
        }
        \only<9>{
          \begin{block}{Productivity}
            The elements of a language can be combined and recombined in novel ways
            \begin{itemize}
              \item You've probably never heard the sentence: ``Strawberry farmers throw pizza in New Orleans''
            \end{itemize}
          \end{block}
        }
      \end{frame}

    \subsection{\subtwofour}
      \begin{frame}{\subtwofour}
        \only<1>{
          \begin{block}{Competence vs performance \parencite{chomsky_syntactic_2002}}
            \begin{itemize}
              \item \alert{Competence}: What you know about your language
              \item \alert{Performance}: What you actually produce when you use your language
            \end{itemize}
          \end{block}
        }
        \only<2>{
          \begin{block}{Aspects of your language}
            \begin{itemize}
              \item Phonetics and Phonology
              \item Morphology
              \item Syntax
              \item Semantics and Pragmatics
            \end{itemize}
          \end{block}
        }
        \only<3-4>{
          \begin{block}<3-4>{Phonetics}
            You know what is an is not a speech sound
            \begin{itemize}
              \item A dog's bark is not a speech sound
            \end{itemize}
          \end{block}
          \begin{block}<4>{Phonology}
            You know what combination of speech sounds is possible
            \begin{itemize}
              \item /pt/ at the beginning of a word is not possible in your language
            \end{itemize}
          \end{block}
        }
        \only<5-6>{
          \begin{block}<5-6>{Morphology}
            You know where word boundaries are and the boundaries of the elements of words
            \begin{itemize}
              \item thedogisplayinginthebackyard
              % Might want to add
                % unbelievable has how many parts?
                % ungiraffelike makes sense? How?
            \end{itemize}
          \end{block}
          \begin{block}<6>{Syntax}
            You know what is and is not a grammatical (i.e., acceptable as possible) sentence
            \begin{itemize}
              \item ``Colorless green ideas sleep furiously'' vs
              \item ``Furiously sleep ideas green colorless'' \parencite{chomsky_syntactic_2002}
            \end{itemize}
          \end{block}
        }
        \only<7-8>{
          \begin{block}<7-8>{Semantics}
            You know the meanings of the word forms in your lexicon
            \begin{itemize}
              \item What does ``duck'' mean?
            \end{itemize}
          \end{block}
          \begin{block}<8>{Pragmatics}
            You know what the purpose is behind an utterance you make
            \begin{itemize}
              \item What do you mean if you yell "duck"?
            \end{itemize}
          \end{block}
        }
        \only<9>{
          \begin{block}{}
            All the aspects of your language are stored in your \alert{mental lexicon} and \alert{your mental grammar}
            \begin{itemize}
              \item Mental lexicon: All the words you know, their meanings, functions, pronunciations, and relations to other words
              \item Mental grammar: The set of rules that you use to combine sounds, parts of words, and words
            \end{itemize}
          \end{block}
        }
      \end{frame}

    \subsection{\subtwofive}
      \begin{frame}{\subtwofive}
        \begin{block}<1-2>{Language acquisition}
          \begin{itemize}
            \item The process of learning your first language
            \item Studied in psycholinguistics and neurolinguistics
          \end{itemize}
        \end{block}
        \begin{block}<2>{Language variation and change}
          \begin{itemize}
            \item How the languages of individuals within and between communities differ and how the languages of individuals of different time periods differ
            \item Studied in sociolinguistics
          \end{itemize}
        \end{block}
      \end{frame}

    \subsection{\subtwosix}
      \begin{frame}{\subtwosix}
        \only<1-3>{
          \begin{block}<1-3>{Descriptive grammar}
            The set of rules that you deduce from observing a speaker's linguistic performance
          \end{block}
          \begin{block}<2-3>{Mental grammar}
            The set of rules that you use to combine sounds, parts of words, and words
          \end{block}
          \begin{block}<3>{Prescriptive grammar}
            The set of rules that are prescribed for some context
          \end{block}
        }
        \only<4>{
          \begin{example}
            A speaker says, ``Where do you come \emph{from}?'', which contains a stranded preposition
            \begin{itemize}
              \item Descriptive grammar: This speaker's language seems to allow stranded prepositions
              \item Mental grammar: There is definitely some rule about this in this speaker's head, but we can only assume that it's the one we just described
              \item Prescriptive grammar: Stranded prepositions are inappropriate when writing essays
            \end{itemize}
          \end{example}
        }
      \end{frame}

    \subsection{\subtwoseven}
      \begin{frame}{\subtwoseven}
        \begin{block}{}
          Writing differs from speech in one important way:
          \begin{itemize}
            \item Writing can be edited
          \end{itemize}
        \end{block}
        \begin{block}<2->{}
          We want to study spontaneous speech so as to avoid the influence of prescriptive grammars on performance
        \end{block}
        \begin{alertblock}<3->{}
          Writing is not \emph{the} language nor a more perfect form of language nor primary over speech
        \end{alertblock}
      \end{frame}

  \begin{frame}[allowframebreaks]{References}
    \printbibliography
  \end{frame}
\end{document}
