%%%%%%%%%%%%%%%%%%%%%%%%%%%%%%%%%%%%%
%                                   %
% Compile with XeLaTeX and biber    %
%                                   %
% Questions or comments:            %
%                                   %
% joshua dot mcneill at uga dot edu %
%                                   %
%%%%%%%%%%%%%%%%%%%%%%%%%%%%%%%%%%%%%

\documentclass{beamer}
  % Read in standard preamble (cosmetic stuff)
  %%%%%%%%%%%%%%%%%%%%%%%%%%%%%%%%%%%%%%%%%%%%%%%%%%%%%%%%%%%%%%%%
% This is a standard preamble used in for all slide documents. %
% It basically contains cosmetic settings.                     %
%                                                              %
% Joshua McNeill                                               %
% joshua dot mcneill at uga dot edu                            %
%%%%%%%%%%%%%%%%%%%%%%%%%%%%%%%%%%%%%%%%%%%%%%%%%%%%%%%%%%%%%%%%

% Beamer settings
% \usetheme{Berkeley}
\usetheme{CambridgeUS}
% \usecolortheme{dove}
% \usecolortheme{rose}
\usecolortheme{seagull}
\usefonttheme{professionalfonts}
\usefonttheme{serif}
\setbeamertemplate{bibliography item}{}

% Packages and settings
\usepackage{fontspec}
  \setmainfont{Charis SIL}
\usepackage{hyperref}
  \hypersetup{colorlinks=true,
              allcolors=blue}
\usepackage{graphicx}
  \graphicspath{{../../figures/}}
\usepackage[normalem]{ulem}
\usepackage{enumerate}

% Document information
\author{M. McNeill}
\title[FREN2001]{Français 2001}
\institute{\url{joshua.mcneill@uga.edu}}
\date{}

%% Custom commands
% Lexical items
\newcommand{\lexi}[1]{\textit{#1}}
% Gloss
\newcommand{\gloss}[1]{`#1'}
\newcommand{\tinygloss}[1]{{\tiny`#1'}}
% Orthographic representations
\newcommand{\orth}[1]{$\langle$#1$\rangle$}
% Utterances (pragmatics)
\newcommand{\uttr}[1]{`#1'}
% Sentences (pragmatics)
\newcommand{\sent}[1]{\textit{#1}}
% Base dir for definitions
\newcommand{\defs}{../definitions}


  % Packages and settings
  \usepackage[backend=biber, style=apa]{biblatex}
    \addbibresource{../references/References.bib}

  % Document information
  \subtitle[Phonotactics]{Phonology and Phonotactics}

  %% Custom commands
  % Subsection/frame titles
  \newcommand{\suboneone}{Speech sounds continued}
  \newcommand{\subtwoone}{All the combinations}
  \newcommand{\subtwotwo}{Practice}

\begin{document}
  % Read in the standard intro slides (title page and table of contents)
  %%%%%%%%%%%%%%%%%%%%%%%%%%%%%%%%%%%%%%%%%%%%%%%%%%%%%%%%%%%%%%%%
% This is a standard set of intro slides used in for all slide %
% documents. It basically contains the title page and table of %
% contents.                                                    %
%                                                              %
% Joshua McNeill                                               %
% joshua dot mcneill at uga dot edu                            %
%%%%%%%%%%%%%%%%%%%%%%%%%%%%%%%%%%%%%%%%%%%%%%%%%%%%%%%%%%%%%%%%

\begin{frame}
  \titlepage
  \tiny{Office: % Basically a variable for office hours location
Gilbert 121\\
        Office hours: % Basically a variable for office hours
 lundi, mercredi, vendredi 10:10--11:10
}
\end{frame}

\begin{frame}
  \tableofcontents[hideallsubsections]
\end{frame}

\AtBeginSection[]{
  \begin{frame}
    \tableofcontents[currentsection,
                     hideallsubsections]
  \end{frame}
}


  \section{What is phonology?}
    \subsection{\suboneone}
      \begin{frame}{\suboneone}
        \only<1-3>{
          \begin{block}{It's a lot like phonetics}
            \begin{itemize}
              \item Phonetics: % Phonetics
The study of the minimal units that make up a language

              \item \alert{Phonology}: % Phonology
The study of the mental organization of the minimal sound units of a language and how they interact

            \end{itemize}
          \end{block}
          \begin{example}<2->
            \begin{itemize}
              \item Which features of sounds are meaningful?
              \item How does the context of the segments around a sound affect its pronunciation?
              \item \alert<3->{What kind of sound combinations are possible?}
            \end{itemize}
          \end{example}
        }
        \only<4>{
          \begin{alertblock}{}
            We're still talking about \emph{a} language: i.e., % A language
A systematic set of words and rules that can be used for human communication \emph{and that is unique to each individual}

            \begin{itemize}
              \item We'll use ``English'' as shorthand for ``the set of features that happen to be shared in the languages of some group of people''
            \end{itemize}
          \end{alertblock}
        }
      \end{frame}

  \section{Phonotactics}
    \subsection{\subtwoone}
      \begin{frame}[t]{\subtwoone}
        \only<1-4>{
          \begin{block}{Why do languages sound different?}
            \begin{itemize}
              \item<2-> Different inventory of speech sounds
              \item<3-> Same inventory of speech sounds but \emph{combined differently}
            \end{itemize}
          \end{block}
          \begin{alertblock}<4->{Phonotactics}
            % Phonotactics
The study of what is permissible in languages in terms of syllable structures and sound combinations

            \begin{itemize}
              \item Thus, \alert{phonotactic contraints}
            \end{itemize}
          \end{alertblock}
        }
        \only<5-7>{
          \begin{block}{Word-initial}
            \begin{itemize}
              \item English does not allow word-initial [ŋ] or [ʒ],
              \item<7-> but it does allow many two consonant \alert{clusters} (i.e., % Cluster
A series of adjacent consonants in a single syllable
)
            \end{itemize}
          \end{block}
          \only<5-6>{
            \begin{example}
              \begin{itemize}
                \item Vietnamese \lexi{ngà} `ivory' [ŋaː˨˩] is difficult for English-speakers
                \item<6> French \lexi{genre} [ʒɑ̃ʀ] becomes [ˈdʒɑn.ɹə] in English... or does it?
              \end{itemize}
            \end{example}
          }
          \only<7>{
            \begin{example}
              \parbox{0.48\linewidth}{
                \begin{itemize}
                  \item \lexi{three} [ˈθɹi]
                  \item \lexi{glean} [ˈglin]
                  \item \lexi{stay} [ˈsteɪ]
                \end{itemize}
              }
              \parbox{0.48\linewidth}{
                \begin{itemize}
                  \item \lexi{humor} [ˈhju.mɹ̩]
                  \item \lexi{quick} [ˈkwɪk]
                  \item \lexi{shrink} [ˈʃɹɪŋk]
                \end{itemize}
              }
            \end{example}
          }
        }
        \only<8-10>{
          \begin{block}{Syllable structures}
            Notation: We use C and V to represent consonant-vowel syllable structures
            \begin{itemize}
              \item<9-> The most widespread syllable structure in the world's languages is CV,
              \item<10> but English allows many, many syllable structures, mostly by allowing many consonant clusters
            \end{itemize}
          \end{block}
          \only<9>{
            \begin{example}
              \parbox{0.48\linewidth}{
                \begin{itemize}
                  \item \lexi{no} [ˈnoʊ]
                  \item \lexi{toe} [ˈtoʊ]
                \end{itemize}
              }
              \parbox{0.48\linewidth}{
                \begin{itemize}
                  \item \lexi{fee} [ˈfi]
                  \item \lexi{tie} [ˈtaɪ]
                \end{itemize}
              }
            \end{example}
          }
        }
        \only<11>{
          \begin{block}{A brief sample}
            \parbox{0.48\linewidth}{
            \begin{itemize}
              \item \lexi{a} V
              \item \lexi{at} VC
              \item \lexi{ask} VCC
              \item \lexi{asked} VCCC
              \item \lexi{crafts} CCVCCC
            \end{itemize}
            }
            \parbox{0.48\linewidth}{
            \begin{itemize}
              \item \lexi{not} CVC
              \item \lexi{flute} CCVC
              \item \lexi{spree} CCCV
              \item \lexi{ramp} CVCC
              \item \lexi{strengths} CCCVCCC
            \end{itemize}
            }
          \end{block}
        }
        \only<12-13>{
          \begin{block}{Syllable structures for borrowed words}
            Borrowed words must be made to fit the phonotactics of the borrowing language
          \end{block}
          \begin{example}<13>
            Japanese only allows V, CV, and CVC syllables, but CVC can only end in [n]. As a result:
            \begin{itemize}
              \item English \lexi{cake} [ˈkeɪk] becomes [keː.ki]
            \end{itemize}
          \end{example}
        }
        \only<14-15>{
          \begin{block}{Foreign accents}
            Differing phonotactic constraints can be behind foreign language learners' accents
          \end{block}
          \begin{example}<15>
            Spanish doesn't allow word-initial [st] clusters. As a result:
            \begin{itemize}
              \item Some English learners will add [ɛ] to the beginning of \lexi{student}: [ɛˈstu.dn̩t]
            \end{itemize}
          \end{example}
        }
      \end{frame}

    \subsection{\subtwotwo}
      \begin{frame}{\subtwotwo}
        \begin{block}{Try these}
          \textcite{dawson_language_2016}, chapter 3 exercises 6 and 7
        \end{block}
      \end{frame}
\end{document}
