%%%%%%%%%%%%%%%%%%%%%%%%%%%%%%%%%%%%%
%                                   %
% Compile with XeLaTeX and biber    %
%                                   %
% Questions or comments:            %
%                                   %
% joshua dot mcneill at uga dot edu %
%                                   %
%%%%%%%%%%%%%%%%%%%%%%%%%%%%%%%%%%%%%

\documentclass{beamer}
  % Read in standard preamble (cosmetic stuff)
  %%%%%%%%%%%%%%%%%%%%%%%%%%%%%%%%%%%%%%%%%%%%%%%%%%%%%%%%%%%%%%%%
% This is a standard preamble used in for all slide documents. %
% It basically contains cosmetic settings.                     %
%                                                              %
% Joshua McNeill                                               %
% joshua dot mcneill at uga dot edu                            %
%%%%%%%%%%%%%%%%%%%%%%%%%%%%%%%%%%%%%%%%%%%%%%%%%%%%%%%%%%%%%%%%

% Beamer settings
% \usetheme{Berkeley}
\usetheme{CambridgeUS}
% \usecolortheme{dove}
% \usecolortheme{rose}
\usecolortheme{seagull}
\usefonttheme{professionalfonts}
\usefonttheme{serif}
\setbeamertemplate{bibliography item}{}

% Packages and settings
\usepackage{fontspec}
  \setmainfont{Charis SIL}
\usepackage{hyperref}
  \hypersetup{colorlinks=true,
              allcolors=blue}
\usepackage{graphicx}
  \graphicspath{{../../figures/}}
\usepackage[normalem]{ulem}
\usepackage{enumerate}

% Document information
\author{M. McNeill}
\title[FREN2001]{Français 2001}
\institute{\url{joshua.mcneill@uga.edu}}
\date{}

%% Custom commands
% Lexical items
\newcommand{\lexi}[1]{\textit{#1}}
% Gloss
\newcommand{\gloss}[1]{`#1'}
\newcommand{\tinygloss}[1]{{\tiny`#1'}}
% Orthographic representations
\newcommand{\orth}[1]{$\langle$#1$\rangle$}
% Utterances (pragmatics)
\newcommand{\uttr}[1]{`#1'}
% Sentences (pragmatics)
\newcommand{\sent}[1]{\textit{#1}}
% Base dir for definitions
\newcommand{\defs}{../definitions}


  % Packages and settings
  \usepackage[backend=biber, style=apa]{biblatex}
    \addbibresource{../references/References.bib}

  % Document information
  \subtitle[Morphological Processes]{Morphological Processes}

  %% Custom commands
  % Subsection/frame titles
  \newcommand{\suboneone}{The processes}
  \newcommand{\subonetwo}{Affixation}
  \newcommand{\subonethree}{Compounding}
  \newcommand{\subonefour}{Reduplication}
  \newcommand{\subonefive}{Apophony (a type of alternation)}
  \newcommand{\subonesix}{Suppletion}
  \newcommand{\suboneseven}{Practice}

\begin{document}
  % Read in the standard intro slides (title page and table of contents)
  %%%%%%%%%%%%%%%%%%%%%%%%%%%%%%%%%%%%%%%%%%%%%%%%%%%%%%%%%%%%%%%%
% This is a standard set of intro slides used in for all slide %
% documents. It basically contains the title page and table of %
% contents.                                                    %
%                                                              %
% Joshua McNeill                                               %
% joshua dot mcneill at uga dot edu                            %
%%%%%%%%%%%%%%%%%%%%%%%%%%%%%%%%%%%%%%%%%%%%%%%%%%%%%%%%%%%%%%%%

\begin{frame}
  \titlepage
  \tiny{Office: % Basically a variable for office hours location
Gilbert 121\\
        Office hours: % Basically a variable for office hours
 lundi, mercredi, vendredi 10:10--11:10
}
\end{frame}

\begin{frame}
  \tableofcontents[hideallsubsections]
\end{frame}

\AtBeginSection[]{
  \begin{frame}
    \tableofcontents[currentsection,
                     hideallsubsections]
  \end{frame}
}


  \section{Morphological processes}
    \subsection{\suboneone}
      \begin{frame}{\suboneone}
        \begin{block}{We'll look at five}
          \begin{itemize}
            \item Affixation
            \item Compounding
            \item Reduplication
            \item Alternation
            \item Suppletion
          \end{itemize}
        \end{block}
      \end{frame}

    \subsection{\subonetwo}
      \begin{frame}[t]{\subonetwo}
        \begin{example}
          \begin{itemize}
            \item \lexi{un-} + \lexi{wind} $=$ \lexi{unwind}
            \item \lexi{re-} + \lexi{wind} $=$ \lexi{rewind}
            \item \lexi{swamp} + \lexi{-s} $=$ \lexi{swamps}
          \end{itemize}
        \end{example}
        \only<1>{
          \begin{block}{}
            This is almost all we've seen so far
          \end{block}
        }
        \only<2->{
          \begin{alertblock}{Affixation}
            % Affixation
The process of forming new words by attaching affixes to words

          \end{alertblock}
        }
        \only<3-4>{
          \begin{block}{What kind of affixes are these?}
            \uncover<4->{
              \begin{itemize}
                \item Prefixes: \lexi{un-} and \lexi{re-}
                \item Suffixes: \lexi{-s}
              \end{itemize}
            }
          \end{block}
        }
        \only<5-6>{
          \begin{block}{Are these derivations or inflections?}
            \uncover<6->{
              \begin{itemize}
                \item Inflections: \lexi{swamps}
                \item Derivations: \lexi{unwind} and \lexi{rewind}
              \end{itemize}
            }
          \end{block}
        }
    \end{frame}

  \subsection{\subonethree}
    \begin{frame}[t]{\subonethree}
      \begin{example}
        \begin{itemize}
          \item \lexi{swamp} + \lexi{thing} $=$ \only<-2>{\lexi{Swamp Thing}}\only<3->{\orth{Swamp Thing}}
          \item<2-> \lexi{black} + \lexi{bird} $=$ \only<-2>{\lexi{blackbird}}\only<3->{\orth{blackbird}}
          \item<2-> \lexi{self} + \lexi{evident} $=$ \only<-2>{\lexi{self-evident}}\only<3->{\orth{self-evident}}
        \end{itemize}
      \end{example}
      \only<1>{
        \begin{block}{}
          We've also seen this
        \end{block}
      }
      \only<2-3>{
        \begin{block}{}
          But there's more!
        \end{block}
        \begin{block}<3->{Spelling doesn't matter}
          Compounds can be spelled with spaces, no spaces, or hyphens, but they're always single words
        \end{block}
      }
      \only<4>{
        \begin{alertblock}{Compound}
          % Compound
A word formed from combining multiple complete words

        \end{alertblock}
        \begin{alertblock}{Compounding}
          % Compounding
The process of forming new words by combining multiple complete words

        \end{alertblock}
      }
      \only<5>{
        \begin{block}{Compounds or phrases?}
          \begin{itemize}
            \item If (in English) the stress is on the first syllable
            \begin{itemize}
              \item[$\rightarrow$] Compound
            \end{itemize}
            \item If you can't insert other material between components without changing the meaning
            \begin{itemize}
              \item[$\rightarrow$] Compound
            \end{itemize}
          \end{itemize}
        \end{block}
      }
      \only<6-7>{
        \begin{block}{Compounding is highly productive in English}
          \begin{itemize}
            \item \begin{tabular}{l l l l}
                    \lexi{income}      & \lexi{tax}         & \lexi{preparation} & \lexi{fees} \\
                    \uncover<7->{noun} & \uncover<7->{noun} & \uncover<7->{noun} & \uncover<7->{noun}
                  \end{tabular}
          \end{itemize}
        \end{block}
      }
    \end{frame}

  \subsection{\subonefour}
    \begin{frame}[t]{\subonefour}
      \only<-2>{
        \begin{example}
          \begin{itemize}
            \item \lexi{like-like}
            \item \lexi{green-green}
            \item \lexi{date-date}
          \end{itemize}
        \end{example}
        \begin{block}<2->{Only marginal examples in English}
          These are very context dependent
        \end{block}
      }
      \only<3-8>{
        \begin{example}
          \begin{itemize}
            \item Indonesian \lexi{rumah} `house' $\rightarrow$ \lexi{rumahrumah} `houses'
            \item Indonesian \lexi{ibu} `mother' $\rightarrow$ \lexi{ibuibu} `mothers'
            \item Tagalog \lexi{kain} `eat' $\rightarrow$ \lexi{kakain} `will eat'
            \item Tagalog [ʔisda] `fish' $\rightarrow$ [maŋʔiʔisda] `fisherman'
          \end{itemize}
        \end{example}
        \only<4>{
          \begin{alertblock}{Reduplication}
            % Reduplication
The process of forming new words by doubling all or part of a word

          \end{alertblock}
          \begin{alertblock}{Reduplicant}
            % Reduplicant
The part of a word that is doubled in reduplication

          \end{alertblock}
        }
        \only<5>{
          \begin{block}{Two types of reduplication}
            \begin{itemize}
              \item Total reduplication
              \item Partial reduplication
            \end{itemize}
          \end{block}
        }
        \only<6>{
          \begin{alertblock}{Total reduplication}
            % Total reduplication
A type of reduplication in which an entire word is doubled

          \end{alertblock}
          \begin{alertblock}{Partial reduplication}
            % Partial reduplication
A type of reduplication in which only part of a word is doubled

          \end{alertblock}
        }
        \only<7->{
          \begin{block}{Which are derivations and which inflections?}
            \uncover<8->{
              \begin{itemize}
                \item Inflections: \lexi{rumahrumah}, \lexi{ibuibu}, and \lexi{kakain}
                \item Derivations: [maŋʔiʔisda]
              \end{itemize}
            }
          \end{block}
        }
      }
    \end{frame}

  \subsection{\subonefive}
    \begin{frame}[t]{\subonefive}
      \begin{example}
        \begin{itemize}
          \item {[}ˈgus] $\rightarrow$ [ˈgis]
          \item {[}ˈfʊt] $\rightarrow$ [ˈfit]
          \item {[}ˈdɹɪŋk] $\rightarrow$ [ˈdɹæŋk] or $\rightarrow$ [ˈdɹʌŋk]
          \item {[}ˈstɹaɪf] $\rightarrow$ [ˈstɹaɪv]
          \item {[}ˈjus] $\rightarrow$ [ˈjuz]
        \end{itemize}
      \end{example}
      \only<2>{
        \begin{alertblock}{Apophony}
          % Apophony
The process of forming new words by changing the quality of a vowel or consonant of a word

        \end{alertblock}
      }
      \only<3-4>{
        \begin{block}{Which are derivations and which inflections?}
          \uncover<4->{
            \begin{itemize}
              \item Inflections: [ˈgis], [ˈfit], and [ˈdɹæŋk]/[ˈdɹʌŋk]
              \item Derivations: [ˈstɹaɪv] and [ˈjuz]
            \end{itemize}
          }
        \end{block}
      }
    \end{frame}

  \subsection{\subonesix}
    \begin{frame}[t]{\subonesix}
      \begin{example}
        \begin{itemize}
          \item {[}ˈiz] ̩$\rightarrow$ [ˈwʌz]
          \item {[}ˈgoʊ] $\rightarrow$ [ˈwɛnt]
          \item {[}ˈgʊd] $\rightarrow$ [ˈbɛ.ɾɹ̩]
          \item {[}ˈbæd] $\rightarrow$ [ˈwɹ̩s]
        \end{itemize}
      \end{example}
      \only<1>{
        \begin{block}{}
          But we can change more than just a vowel or consonant
        \end{block}
      }
      \only<2>{
        \begin{alertblock}{Suppletion}
          % Suppletion
The process of inflecting a word by changing its entire phonological form

        \end{alertblock}
      }
      \only<3-4>{
        \begin{block}{Which are derivations and which inflections?}
          \begin{itemize}
            \item Inflections: [ˈwʌz], [ˈwɛnt], [ˈbɛ.ɾɹ̩], and [ˈwɹ̩s]
            \item Derivations: None. This doesn't exist
          \end{itemize}
        \end{block}
        \begin{alertblock}<4->{}
          Suppletion requires paradigms
        \end{alertblock}
      }
    \end{frame}

  \subsection{\suboneseven}
    \begin{frame}{\suboneseven}
      \begin{block}{Try these}
        \textcite{dawson_language_2016}, chapter 4 exercises 10, 11, and 12
      \end{block}
    \end{frame}
\end{document}
