%%%%%%%%%%%%%%%%%%%%%%%%%%%%%%%%%%%%%
%                                   %
% Compile with XeLaTeX and biber    %
%                                   %
% Questions or comments:            %
%                                   %
% joshua dot mcneill at uga dot edu %
%                                   %
%%%%%%%%%%%%%%%%%%%%%%%%%%%%%%%%%%%%%

\documentclass{beamer}
  % Read in standard preamble (cosmetic stuff)
  %%%%%%%%%%%%%%%%%%%%%%%%%%%%%%%%%%%%%%%%%%%%%%%%%%%%%%%%%%%%%%%%
% This is a standard preamble used in for all slide documents. %
% It basically contains cosmetic settings.                     %
%                                                              %
% Joshua McNeill                                               %
% joshua dot mcneill at uga dot edu                            %
%%%%%%%%%%%%%%%%%%%%%%%%%%%%%%%%%%%%%%%%%%%%%%%%%%%%%%%%%%%%%%%%

% Beamer settings
% \usetheme{Berkeley}
\usetheme{CambridgeUS}
% \usecolortheme{dove}
% \usecolortheme{rose}
\usecolortheme{seagull}
\usefonttheme{professionalfonts}
\usefonttheme{serif}
\setbeamertemplate{bibliography item}{}

% Packages and settings
\usepackage{fontspec}
  \setmainfont{Charis SIL}
\usepackage{hyperref}
  \hypersetup{colorlinks=true,
              allcolors=blue}
\usepackage{graphicx}
  \graphicspath{{../../figures/}}
\usepackage{soul}
  \setstcolor{red}
\usepackage[normalem]{ulem}
\usepackage{enumerate}
\usepackage{tikz}
  \usetikzlibrary{trees}

% Document information
\author{M. McNeill}
\title[FREN1001]{Français 1001}
\institute{\url{joshua.mcneill@uga.edu}}
\date{}

%% Custom commands
% Lexical items
\newcommand{\lexi}[1]{\textit{#1}}
% Gloss
\newcommand{\gloss}[1]{`#1'}
\newcommand{\tinygloss}[1]{{\tiny`#1'}}
% Orthographic representations
\newcommand{\orth}[1]{$\langle$#1$\rangle$}
% Utterances (pragmatics)
\newcommand{\uttr}[1]{`#1'}
% Sentences (pragmatics)
\newcommand{\sent}[1]{\textit{#1}}
% Fixed length underlines
\newcommand{\funderline}[2][4cm]{
  \underline{\makebox[\ifdim\width>#1\width\else#1\fi]{#2}}
}
% Base dir for definitions
\newcommand{\defs}{../definitions}
\newcommand{\activity}[1]{
  \input{./activities/#1.tex}
}


  % Packages and settings
  \usepackage[backend=biber, style=apa]{biblatex}
    \addbibresource{../references/References.bib}

  % Document information
  \subtitle[Pragmatics Intro]{Introduction to Pragmatics}

  %% Custom commands
  % Subsection/frame titles
  \newcommand{\suboneone}{What is it?}
  \newcommand{\subonetwo}{Sentences and utterances}
  \newcommand{\subonethree}{Context types}
  \newcommand{\subonefour}{Felicity}
  \newcommand{\subonefive}{Practice}

\begin{document}
  % Read in the standard intro slides (title page and table of contents)
  %%%%%%%%%%%%%%%%%%%%%%%%%%%%%%%%%%%%%%%%%%%%%%%%%%%%%%%%%%%%%%%%
% This is a standard set of intro slides used in for all slide %
% documents. It basically contains the title page and table of %
% contents.                                                    %
%                                                              %
% Joshua McNeill                                               %
% joshua dot mcneill at uga dot edu                            %
%%%%%%%%%%%%%%%%%%%%%%%%%%%%%%%%%%%%%%%%%%%%%%%%%%%%%%%%%%%%%%%%

\begin{frame}
  \titlepage
  \tiny{Office: % Basically a variable for office hours location
T Gilbert 141/W Library 4th Fl
\\
        Office hours: % Basically a variable for office hours
T 11-12/W 11-12:30
}
\end{frame}

\begin{frame}
  \tableofcontents[hideallsubsections]
\end{frame}

\AtBeginSection[]{
  \begin{frame}
    \tableofcontents[currentsection,
                     hideallsubsections]
  \end{frame}
}


  \section{Pragmatics Intro}
    \subsection{\suboneone}
      \begin{frame}{\suboneone}
        \includegraphics[scale=0.24]{garfield.jpg}
        \begin{block}{In what two ways can Jon's statement be understood?}
          \begin{enumerate}
            \item<2-> It's nice to lie around all day. \hfill (Semantics)
            \item<2-> Get up!                          \hfill (Pragmatics)
          \end{enumerate}
        \end{block}
        \begin{alertblock}<2->{Pragmatics}
          % Pragmatics
The study of the meaning of language in context

        \end{alertblock}
      \end{frame}

    \subsection{\subonetwo}
      \begin{frame}[t]{\subonetwo}
        \begin{example}
          \begin{enumerate}
            \item \uttr{Lying around all day... must be nice.}  \hfill (Utterance)
            \item \sent{It must be nice to lie around all day.} \hfill (Sentence)
            \item Get up!                                       \hfill (Meaning)
          \end{enumerate}
        \end{example}
        \only<1>{
          \begin{alertblock}{Utterance}
            % Utterance
The use by a particular speaker, on a particular occasion, of a sentence

          \end{alertblock}
          \begin{alertblock}{Sentence}
            % Sentence
An ideal string of words underlying an utterance as they exist in a speaker's mind

          \end{alertblock}
        }
        \only<2>{
          \begin{block}{Notation}
            \begin{itemize}
              \item Utterances are in single quotes
              \item Sentences are in italics
            \end{itemize}
          \end{block}
        }
      \end{frame}

      \begin{frame}{\subonetwo}
        \begin{block}{A clearer example}
          \begin{enumerate}
            \item \uttr{It must be nice, I mean, all day, lying around.}
            \item \sent{It must be nice to lie around all day.}
          \end{enumerate}
        \end{block}
      \end{frame}

      \begin{frame}[t]{\subonetwo}
        \begin{block}{Why do we need this distinction?}
          Because context matters, and sentences don't have any
        \end{block}
        \only<1-2>{
          \begin{example}
            \sent{He is there now.}
            \begin{itemize}
              \item What does this mean?
            \end{itemize}
          \end{example}
          \begin{alertblock}<2->{Deictic word}
            % Deictic word
An word that takes some element of its meaning from the context in which it was uttered

          \end{alertblock}
        }
        \only<3->{
          \begin{block}{Even besides deictics}
            \sent{Can you take the trash out?}
          \end{block}
          \begin{block}{What does this mean?}
            \begin{itemize}
              \item<4-> A parent requesting that you do something?
              \item<4-> A therapist asking if you're physically capable of lifting trash after an accident?
              \item<4-> Someone trying to get rid of you be reminding you of your chores?
            \end{itemize}
          \end{block}
        }
      \end{frame}

    \subsection{\subonethree}
      \begin{frame}{\subonethree}
        \begin{block}{Three types of contexts}
          \begin{itemize}
            \item Linguistic context
            \item Situational context
            \item Social context
          \end{itemize}
        \end{block}
      \end{frame}

      \begin{frame}{\subonethree}
        \begin{alertblock}{Linguistic context}
          % Linguistic context
Context in the sense of what else has been uttered before the utterance in question

        \end{alertblock}
        \begin{block}{Does \uttr{yes} mean the same thing in each?}
          After someone asks:
          \begin{enumerate}
            \item \uttr{Do you like green beans?}
            \item \uttr{Is there a computer available in the lab?}
            \item \uttr{Does Santa Claus exist?}
          \end{enumerate}
        \end{block}
      \end{frame}

      \begin{frame}{\subonethree}
        \begin{alertblock}{Situational context}
          % Situational context
Context in the sense of the time, place, and activities occuring during an utterance

        \end{alertblock}
        \begin{block}{Who are we talking about here?}
          \uttr{The governor was on TV last night.}
        \end{block}
        \begin{block}<2->{How tall should we expect Rachel to be?}
          When stating that \uttr{Rachel is very tall} and
          \begin{enumerate}
            \item Rachel is 5 years old
            \item Rachel is 20 years old
          \end{enumerate}
        \end{block}
      \end{frame}

      \begin{frame}{\subonethree}
        \begin{alertblock}{Social context}
          % Social context
Context in the sense of the relationships between the people talking during an utterance

        \end{alertblock}
        \begin{block}{How should we take this?}
          Someone says \uttr{yes ma'am} to:
          \begin{enumerate}
            \item A younger person
            \item An older person
            \item An employee of theirs
          \end{enumerate}
        \end{block}
      \end{frame}

    \subsection{\subonefour}
      \begin{frame}{\subonefour}
        \begin{alertblock}{Felicitous utterances}
          % Felicitous utterance
An utterance that is appropriate for the context

        \end{alertblock}
        \begin{block}{Are these each appropriate responses?}
          Someone asks, \uttr{What do you do for a living?}, and you respond:
          \begin{enumerate}
            \item \uttr{I'm a student at UGA, and I wait tables on the weekends.}
            \item \only<2->{\#}\uttr{I have a job.}
            \item \only<2->{\#}\uttr{My favorite color is purple.}
          \end{enumerate}
        \end{block}
      \end{frame}

    \subsection{\subonefive}
      \begin{frame}{\subonefive}
        \begin{block}{Try these}
          \textcite{dawson_language_2016}, chapter 7 exercises 1i, 5, and 6
        \end{block}
      \end{frame}
\end{document}
