\documentclass[addpoints]{exam}
  % Read in shared preamble for all homeworks
  %%%%%%%%%%%%%%%%%%%%%%%%%%%%%%%%%%%%%%%%%%%%%%%%%%%%%%%%%%%%%%%%%%%%
% This is the standard preamble for homework assignments and exams %
%                                                                  %
% -Joshua McNeill (joshua dot mcneill at uga dot edu)              %
%%%%%%%%%%%%%%%%%%%%%%%%%%%%%%%%%%%%%%%%%%%%%%%%%%%%%%%%%%%%%%%%%%%%
% Exam settings
\pointsinmargin
\pointformat{}

% Packages and settings
\usepackage{fontspec}
  \setmainfont{Charis SIL}
\usepackage{tikz}

%% Custom commands
% Instructions for a section
\newcommand{\instr}[1]{
  \begin{center}
    \fbox{
      \parbox{0.85\textwidth}
             {#1}
    }
  \end{center}
}
\newcommand{\lexi}[1]{\textit{#1}}
\newcommand{\gloss}[1]{`#1'}


  % Document information
  \title{Homework 1: Phonetics}
  \date{}

\begin{document}
  \maketitle

  % Header
  %%%%%%%%%%%%%%%%%%%%%%%%%%%%%%%%%%%%%%%%%%%%%%%%%%%%%%%%%%%%%%%%%%%%%%%
% This is the the header that all homework assignments and exams use. %
%                                                                     %
% -Joshua McNeill (joshua dot mcneill at uga dot edu)                 %
%%%%%%%%%%%%%%%%%%%%%%%%%%%%%%%%%%%%%%%%%%%%%%%%%%%%%%%%%%%%%%%%%%%%%%%
\noindent\makebox[0.5\textwidth][l]{Name:} \makebox[0.5\textwidth][r]{Course: LING2100, The Study of Language}\\
\makebox[0.5\textwidth][l]{Date:} \makebox[0.5\textwidth][r]{Instructor: Joshua McNeill}


  % Questions
    \instr{Give the IPA symbol for the \emph{consonant} that corresponds to the description. (1 point each)}

  \begin{questions}

      \parbox[t]{0.45\linewidth}{
        \question[1] Voiced alveolar approximant: \hrulefill
        \question[1] Voiceless alveolar fricative: \hrulefill
        \question[1] Voiced dental fricative: \hrulefill
        \question[1] Voiceless labiodental fricative: \hrulefill
        \question[1] Voiceless velar stop: \hrulefill
        \question[1] Voiced labial-velar approximant: \hrulefill
      }
      \hspace{0.1\linewidth}
      \parbox[t]{0.45\linewidth}{
        \question[1] Voiced alveolar fricative: \hrulefill
        \question[1] Voiced alveolar nasal: \hrulefill
        \question[1] Voiced postalveolar fricative: \hrulefill
        \question[1] Voiced affricate: \hrulefill
        \question[1] Voiced palatal approximant: \hrulefill
        \question[1] Voiceless dental fricative: \hrulefill
      }

    \instr{Give the IPA symbol for the \emph{vowel} that corresponds to the description. (1 point each)}

      \parbox[t]{0.45\linewidth}{
        \question[1] High-mid back rounded: \hrulefill
        \question[1] Low-mid front unrounded: \hrulefill
        \question[1] High front unrounded: \hrulefill
      }
      \hspace{0.1\linewidth}
      \parbox[t]{0.45\linewidth}{
        \question[1] Near-low front unrounded: \hrulefill
        \question[1] Low-mid back rounded: \hrulefill
        \question[1] Near-high back rounded: \hrulefill
      }

    \instr{Give the description -- meaning the voicing, place of articulation, and manner of articulation -- that corresponds to the following \emph{consonant} IPA symbols. (1 point each)}

      \parbox[t]{0.45\linewidth}{
        \question[1] [d]: \hrulefill
        \question[1] [l]: \hrulefill
        \question[1] [h]: \hrulefill
        \question[1] [m]: \hrulefill
        \question[1] [ʃ]: \hrulefill
        \question[1] [t]: \hrulefill
      }
      \hspace{0.1\linewidth}
      \parbox[t]{0.45\linewidth}{
        \question[1] [g]: \hrulefill
        \question[1] [ʔ]: \hrulefill
        \question[1] [p]: \hrulefill
        \question[1] [tʃ]: \hrulefill
        \question[1] [ɾ]: \hrulefill
        \question[1] [b]: \hrulefill
      }

    \instr{Give the description -- meaning the height, advancement, and lip rounding -- that corresponds to the following \emph{vowel} IPA symbols. (1 point each)}

      \parbox[t]{0.45\linewidth}{
        \question[1] [u]: \hrulefill
        \question[1] [a]: \hrulefill
        \question[1] [ɑ]: \hrulefill
      }
      \hspace{0.1\linewidth}
      \parbox[t]{0.45\linewidth}{
        \question[1] [ɪ]: \hrulefill
        \question[1] [ʌ]: \hrulefill
        \question[1] [e]: \hrulefill
      }

    \newpage

    \instr{Give the word, in standard English spelling, that corresponds to the IPA transcription. (1 point each)}

      \parbox[t]{0.45\linewidth}{
        \question[1] [ˈfaɪ.næn.sɪŋ]: \hrulefill
        \question[1] [ɹəˈtɹækt]: \hrulefill
        \question[1] [ˈɹeɪl.ɹoʊd]: \hrulefill
        \question[1] [ˈæ.dʒɪ.teɪt]: \hrulefill
        \question[1] [ˈtɔl]: \hrulefill
        \question[1] [kæ.lɪˈbɹeɪ.ʃn̩]: \hrulefill
        \question[1] [ˈʃaʊ.təd]: \hrulefill
        \question[1] [ˈdʒæm]: \hrulefill
        \question[1] [əˈmæs]: \hrulefill
        \question[1] [bɑmˈbɑɹ.dəd]: \hrulefill
      }
      \hspace{0.1\linewidth}
      \parbox[t]{0.45\linewidth}{
        \question[1] [gəˈzɛl]: \hrulefill
        \question[1] [ˈdʒæz.mɪn]: \hrulefill
        \question[1] [ˈɔ.di.əns]: \hrulefill
        \question[1] [ˈmeɪ.ɹ̩]: \hrulefill
        \question[1] [dəˈɹɛk.tɪv]: \hrulefill
        \question[1] [ˈpɹoʊ.faɪl]: \hrulefill
        \question[1] [ˈwɹ̩st]: \hrulefill
        \question[1] [ˈtʃɑ.pi]: \hrulefill
        \question[1] [əˈpɹeɪ.zɹ̩]: \hrulefill
        \question[1] [ʌnˈhɪn.dɹ̩d]: \hrulefill
      }

    \instr{Give the syllable structure, as a tree diagram, for the following one syllable words. (1 point per label)}

      \question[1] [ˈðɛɹ]

      \vspace{\stretch{1}}

      \question[1] [ˈwaɪld]

      \vspace{\stretch{1}}

      \question[1] [ˈkæt]

      \vspace{\stretch{1}}

      \newpage

      \question[1] [ˈbɑɹk]

      \vspace{\stretch{1}}

      \question[1] [ˈkloʊz]

      \vspace{\stretch{1}}

      \question[1] [ˈnæt]

      \vspace{\stretch{1}}

      % Show 2-3 spectrograms for vowels and have them identify F1 and F2 in each

  \end{questions}

  \vspace{1.25cm}

  % Grade
  \begin{center}
    \gradetable[v][pages]
  \end{center}
\end{document}
