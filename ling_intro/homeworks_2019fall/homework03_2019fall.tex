\documentclass[addpoints]{exam}
  % Read in shared preamble for all homeworks
  %%%%%%%%%%%%%%%%%%%%%%%%%%%%%%%%%%%%%%%%%%%%%%%%%%%%%%%%%%%%%%%%%%%%
% This is the standard preamble for homework assignments and exams %
%                                                                  %
% -Joshua McNeill (joshua dot mcneill at uga dot edu)              %
%%%%%%%%%%%%%%%%%%%%%%%%%%%%%%%%%%%%%%%%%%%%%%%%%%%%%%%%%%%%%%%%%%%%
% Exam settings
\pointsinmargin
\pointformat{}

% Packages and settings
\usepackage{fontspec}
  \setmainfont{Charis SIL}
\usepackage{tikz}

%% Custom commands
% Instructions for a section
\newcommand{\instr}[1]{
  \begin{center}
    \fbox{
      \parbox{0.85\textwidth}
             {#1}
    }
  \end{center}
}
\newcommand{\lexi}[1]{\textit{#1}}
\newcommand{\gloss}[1]{`#1'}


  % Packages and settings
  \usepackage{phonrule}

  % Document information
  \title{Homework 3: Syntax, Semantics, \& Pragmatics}
  \date{}

\begin{document}
  \maketitle

  % Header
  %%%%%%%%%%%%%%%%%%%%%%%%%%%%%%%%%%%%%%%%%%%%%%%%%%%%%%%%%%%%%%%%%%%%%%%
% This is the the header that all homework assignments and exams use. %
%                                                                     %
% -Joshua McNeill (joshua dot mcneill at uga dot edu)                 %
%%%%%%%%%%%%%%%%%%%%%%%%%%%%%%%%%%%%%%%%%%%%%%%%%%%%%%%%%%%%%%%%%%%%%%%
\noindent\makebox[0.5\textwidth][l]{Name:} \makebox[0.5\textwidth][r]{Course: LING2100, The Study of Language}\\
\makebox[0.5\textwidth][l]{Date:} \makebox[0.5\textwidth][r]{Instructor: Joshua McNeill}


    \section{Syntax}

      \instr{Model after (4), word order problem or co-occurrence problem?. Each of the following sentences is ungrammatical due to violating the syntactic requirements of the linguistic expressions involved. For each, identify whether \emph{word order} or \emph{co-occurrence} requirements or both was violated. (1 point each)}

  \begin{questions}

      \instr{Maybe do (5). For each pair of sentences, the underlined expression is an \emph{argument} in one and an \emph{adjunct} in the other. Identify which is which. (1 point each)}

      \instr{Model after (16). For each of the following sentences, identify \emph{all} of the syntactic constituents that are present (e.g., \textit{A boy kicked the ball} contains the syntactic constituents \textit{a boy}, \textit{kicked the ball}, and \textit{the ball}). (\emph{Hint}: Use the three tests constituent tests: answers to question, clefting, and pro-form substitution.) (1 point each)}

      \instr{Model after (17) combined with (18). For each pair of expressions, say whether they have the \emph{same} or \emph{different} syntactic distributions \textbf{and} give their syntactic \emph{or} lexical categories (e.g., \textit{the dog} and \textit{a ball}: same, NP and NP). (2 points each)}

      \instr{Model after (27). For each sentence, give the phrase structure tree using the basic grammar of English syntax that we've constructed in class (i.e., the set of phrase structure rewrite rules that we've talked about). (3 points each)}

    \section{Semantics \& Pragmatics}

      \instr{Model after (8). For each pair of lexical expressions, indicate their sense relation, be it synonymy, antonymy, or hyponymy. Where the sense relation is antonymy, specify which of the four types it is, and where the sense relation is hyponymy, specify which is the hypernym and which the hyponym (e.g., \textit{dog} and \textit{poodle}: Hyponymy, \textit{poodle} is the hyponym and \textit{dog} the hypernym). (2 points each)}

      \instr{Model after (17) combined with (18). For each expression, indicate whether it is a \emph{proposition} or \emph{not a proposition}, \textbf{and} if it is a proposition, specify its truth condition(s). (2 points each)}

      \instr{Model after (23). For each pair of propositions, indicate whether \emph{one entails the other}, whether they are \emph{mutually entailing}, or whether they are \emph{incompatible}. (1 point each)}

      \instr{identify the deictic words or which maxim is being flouted or provide a scenario in which the maxim is being flouted or list all the existence presuppositions in the sentences}

  \end{questions}

  \vspace{1.25cm}

  % Grade
  \begin{center}
    \gradetable[v][pages]
  \end{center}
\end{document}
