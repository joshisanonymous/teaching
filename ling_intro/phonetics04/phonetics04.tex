%%%%%%%%%%%%%%%%%%%%%%%%%%%%%%%%%%%%%
%                                   %
% Compile with XeLaTeX and biber    %
%                                   %
% Questions or comments:            %
%                                   %
% joshua dot mcneill at uga dot edu %
%                                   %
%%%%%%%%%%%%%%%%%%%%%%%%%%%%%%%%%%%%%

\documentclass{beamer}
  % Read in standard preamble (cosmetic stuff)
  %%%%%%%%%%%%%%%%%%%%%%%%%%%%%%%%%%%%%%%%%%%%%%%%%%%%%%%%%%%%%%%%
% This is a standard preamble used in for all slide documents. %
% It basically contains cosmetic settings.                     %
%                                                              %
% Joshua McNeill                                               %
% joshua dot mcneill at uga dot edu                            %
%%%%%%%%%%%%%%%%%%%%%%%%%%%%%%%%%%%%%%%%%%%%%%%%%%%%%%%%%%%%%%%%

% Beamer settings
% \usetheme{Berkeley}
\usetheme{CambridgeUS}
% \usecolortheme{dove}
% \usecolortheme{rose}
\usecolortheme{seagull}
\usefonttheme{professionalfonts}
\usefonttheme{serif}
\setbeamertemplate{bibliography item}{}

% Packages and settings
\usepackage{fontspec}
  \setmainfont{Charis SIL}
\usepackage{hyperref}
  \hypersetup{colorlinks=true,
              allcolors=blue}
\usepackage{graphicx}
  \graphicspath{{../../figures/}}
\usepackage[normalem]{ulem}
\usepackage{enumerate}

% Document information
\author{M. McNeill}
\title[FREN2001]{Français 2001}
\institute{\url{joshua.mcneill@uga.edu}}
\date{}

%% Custom commands
% Lexical items
\newcommand{\lexi}[1]{\textit{#1}}
% Gloss
\newcommand{\gloss}[1]{`#1'}
\newcommand{\tinygloss}[1]{{\tiny`#1'}}
% Orthographic representations
\newcommand{\orth}[1]{$\langle$#1$\rangle$}
% Utterances (pragmatics)
\newcommand{\uttr}[1]{`#1'}
% Sentences (pragmatics)
\newcommand{\sent}[1]{\textit{#1}}
% Base dir for definitions
\newcommand{\defs}{../definitions}


  % Packages and settings
  \usepackage{tikz}
  \usepackage{adjustbox}
  \usepackage[backend=biber, style=apa]{biblatex}
    \addbibresource{../references/References.bib}

  % Document information
  \subtitle[Non-English Sounds]{Articulation of non-English Sounds}

  %% Custom commands
  % Subsection/frame titles
  \newcommand{\suboneone}{Definition}
  \newcommand{\subonetwo}{Non-English vowels}
  \newcommand{\subonethree}{Non-English consonants}
  \newcommand{\subonefour}{Resources and practice}

\begin{document}
  % Read in the standard intro slides (title page and table of contents)
  %%%%%%%%%%%%%%%%%%%%%%%%%%%%%%%%%%%%%%%%%%%%%%%%%%%%%%%%%%%%%%%%
% This is a standard set of intro slides used in for all slide %
% documents. It basically contains the title page and table of %
% contents.                                                    %
%                                                              %
% Joshua McNeill                                               %
% joshua dot mcneill at uga dot edu                            %
%%%%%%%%%%%%%%%%%%%%%%%%%%%%%%%%%%%%%%%%%%%%%%%%%%%%%%%%%%%%%%%%

\begin{frame}
  \titlepage
  \tiny{Office: % Basically a variable for office hours location
Gilbert 121\\
        Office hours: % Basically a variable for office hours
 lundi, mercredi, vendredi 10:10--11:10
}
\end{frame}

\begin{frame}
  \tableofcontents[hideallsubsections]
\end{frame}

\AtBeginSection[]{
  \begin{frame}
    \tableofcontents[currentsection,
                     hideallsubsections]
  \end{frame}
}


  \section{What else can the IPA do}
    \subsection{\suboneone}
      \begin{frame}{\suboneone}
        \begin{alertblock}{}
          The IPA is a \emph{language independent} alphabet
        \end{alertblock}
      \end{frame}

    \subsection{\subonetwo}
      \begin{frame}{\subonetwo}
        \begin{columns}
          \column{0.5\textwidth}
            \begin{minipage}[c][0.6\textheight]{\linewidth}
              \only<1>{
                \begin{block}{}
                  These are the vowels in English
                \end{block}
              }
              \only<2>{
                \begin{block}{}
                  These are \emph{all} the vowels
                \end{block}
              }
              \only<3>{
                \begin{block}{Front rounded vowels}
                  French has [y ø] and [œ]
                  \begin{itemize}
                    \item \orth{vue} `view' [vy] vs \orth{vous} `you' [vu]
                    \item \orth{feu} `fire' [fø] vs \orth{faux} `false' [fo]
                  \end{itemize}
                  German has [y] and [ø]
                  \begin{itemize}
                    \item \orth{Güte} `kindness' [ɡytə] vs \orth{guter} `good' [ɡutə]
                    \item \orth{schön} `beautiful' [ʃøn] vs \orth{schon} `already' [ʃon]
                  \end{itemize}
                \end{block}
              }
              \only<4>{
                \begin{block}{Nasal vowels}
                  Produced by lowering the velum to redirect the airstream through the nasal cavity
                \end{block}
                \begin{example}
                  French has [ɛ̃ ɔ̃] and [ɑ̃]
                  \begin{itemize}
                    \item \orth{mais} `but' [mɛ] vs \orth{main} `hand' [mɛ̃]
                    \item \orth{chasse} `hunt' [ʃɑs] vs \orth{chance} `luck' [ʃɑ̃s]
                  \end{itemize}
                \end{example}
              }
            \end{minipage}
          \column{0.5\textwidth}
            \only<1>{
              \begin{adjustbox}{height=4.5cm}
                % Read in English vowel chart
                %%%%%%%%%%%%%%%%%%%%%%%%%%%%%%%%%%%%%%%%%%%%%%%%%%%%%%%%%%%%%%%%%%%%%%%%%%%%%%%%%%%%%
% This creates an English IPA chart for vowels                                      %
%                                                                                   %
% Compiled from material_IPA_en_chart.tex when a                                    %
% standalone document is needed                                                     %
%                                                                                   %
% Code is only slightly modified from:                                              %
%   https://tex.stackexchange.com/questions/156955/tikz-pgf-linguistics-vowel-chart %
%                                                                                   %
% -Joshua McNeill (joshua dot mcneill at uga dot edu)                               %
%%%%%%%%%%%%%%%%%%%%%%%%%%%%%%%%%%%%%%%%%%%%%%%%%%%%%%%%%%%%%%%%%%%%%%%%%%%%%%%%%%%%%

% Custom command
\def\V(#1,#2){barycentric cs:hf={(3-#1)*(2-#2)},hb={(3-#1)*#2},lf={#1*(2-#2)},lb={#1*#2}}

% Chart
\begin{tikzpicture}[scale=3]
  \large
  \tikzset{
    vowel/.style={fill=white, anchor=mid, text depth=0ex, text height=1ex},
    dot/.style={circle,fill=black,minimum size=0.4ex,inner sep=0pt,outer sep=-1pt},
  }
  \coordinate (hf) at (0,2); % high front
  \coordinate (hb) at (2,2); % high back
  \coordinate (lf) at (1,0); % low front
  \coordinate (lb) at (2,0); % low back
  \def\V(#1,#2){barycentric cs:hf={(3-#1)*(2-#2)},hb={(3-#1)*#2},lf={#1*(2-#2)},lb={#1*#2}}

  % Draw the horizontal lines first.
  \draw (\V(0,0)) -- (\V(0,2));
  \draw (\V(1,0)) -- (\V(1,2));
  \draw (\V(2,0)) -- (\V(2,2));
  \draw (\V(3,0)) -- (\V(3,2));

  % Place all the unrounded-rounded pairs next, on top of the horizontal lines.
  \path (\V(0,0))     node[vowel, left] {i}          node[vowel, right] { }          node[dot] {};
  \path (\V(0,1))     node[vowel, left] { }          node[vowel, right] { }          node[dot] {};
  \path (\V(0,2))     node[vowel, left] { }          node[vowel, right] {u}          node[dot] {};
  \path (\V(0.5,0.4)) node[vowel, left] {\textbf{ɪ}} node[vowel, right] { }          node[   ] {};
  \path (\V(0.5,1.6)) node[vowel, left] { }          node[vowel, right] {\textbf{ʊ}} node[   ] {};
  \path (\V(1,0))     node[vowel, left] {e}          node[vowel, right] { }          node[dot] {};
  \path (\V(1,1))     node[vowel, left] { }          node[vowel, right] { }          node[dot] {};
  \path (\V(1,2))     node[vowel, left] { }          node[vowel, right] {o}          node[dot] {};
  \path (\V(2,0))     node[vowel, left] {\textbf{ɛ}} node[vowel, right] { }          node[dot] {};
  \path (\V(2,1))     node[vowel, left] { }          node[vowel, right] { }          node[dot] {};
  \path (\V(2,2))     node[vowel, left] {\textbf{ʌ}} node[vowel, right] {ɔ}          node[dot] {};
  \path (\V(2.5,0))   node[vowel, left] {\textbf{æ}} node[vowel, right] { }          node[   ] {};
  \path (\V(3,0))     node[vowel, left] {a}          node[vowel, right] { }          node[dot] {};
  \path (\V(3,2))     node[vowel, left] {ɑ}          node[vowel, right] { }          node[dot] {};

  % Draw the vertical lines.
  \draw (\V(0,0)) -- (\V(3,0));
  \draw (\V(0,1)) -- (\V(3,1));
  \draw (\V(0,2)) -- (\V(3,2));

  % Place the unpaired symbols last, on top of the vertical lines.
  \path (\V(1.5,1))   node[vowel]       {\textbf{ə}};
  \path (\V(-0.25,0)) node[vowel]       {front};
  \path (\V(-0.25,1)) node[vowel]       {central};
  \path (\V(-0.25,2)) node[vowel]       {back};
  \path (\V(0,-0.5))  node[vowel]       {high};
  \path (\V(1,-0.8))  node[vowel]       {high-mid};
  \path (\V(2,-1.55)) node[vowel]       {low-mid};
  \path (\V(3,-2.95)) node[vowel]       {low};
\end{tikzpicture}

              \end{adjustbox}
            }
            \only<2->{
              \begin{adjustbox}{height=4.5cm}
                % Read in full vowel chart
                %%%%%%%%%%%%%%%%%%%%%%%%%%%%%%%%%%%%%%%%%%%%%%%%%%%%%%%%%%%%%%%%%%%%%%%%%%%%%%%%%%%%%
% This creates an English IPA chart for vowels                                      %
%                                                                                   %
% Compiled from material_IPA_en_chart.tex when a                                    %
% standalone document is needed                                                     %
%                                                                                   %
% Code is only slightly from:                                                       %
%   https://tex.stackexchange.com/questions/156955/tikz-pgf-linguistics-vowel-chart %
%                                                                                   %
% -Joshua McNeill (joshua dot mcneill at uga dot edu)                               %
%%%%%%%%%%%%%%%%%%%%%%%%%%%%%%%%%%%%%%%%%%%%%%%%%%%%%%%%%%%%%%%%%%%%%%%%%%%%%%%%%%%%%

% Custom command
\def\V(#1,#2){barycentric cs:hf={(3-#1)*(2-#2)},hb={(3-#1)*#2},lf={#1*(2-#2)},lb={#1*#2}}

% Chart
\begin{tikzpicture}[scale=3]
  \large
  \tikzset{
    vowel/.style={fill=white, anchor=mid, text depth=0ex, text height=1ex},
    dot/.style={circle,fill=black,minimum size=0.4ex,inner sep=0pt,outer sep=-1pt},
  }
  \coordinate (hf) at (0,2); % high front
  \coordinate (hb) at (2,2); % high back
  \coordinate (lf) at (1,0); % low front
  \coordinate (lb) at (2,0); % low back
  \def\V(#1,#2){barycentric cs:hf={(3-#1)*(2-#2)},hb={(3-#1)*#2},lf={#1*(2-#2)},lb={#1*#2}}

  % Draw the horizontal lines first.
  \draw (\V(0,0)) -- (\V(0,2));
  \draw (\V(1,0)) -- (\V(1,2));
  \draw (\V(2,0)) -- (\V(2,2));
  \draw (\V(3,0)) -- (\V(3,2));

  % Place all the unrounded-rounded pairs next, on top of the horizontal lines.
  \path (\V(0,0))     node[vowel, left] {i} node[vowel, right] {y} node[dot] {};
  \path (\V(0,1))     node[vowel, left] {ɨ} node[vowel, right] {ʉ} node[dot] {};
  \path (\V(0,2))     node[vowel, left] {ɯ} node[vowel, right] {u} node[dot] {};
  \path (\V(0.5,0.4)) node[vowel, left] {ɪ} node[vowel, right] {ʏ} node[   ] {};
  \path (\V(0.5,1.6)) node[vowel, left] { } node[vowel, right] {ʊ} node[   ] {};
  \path (\V(1,0))     node[vowel, left] {e} node[vowel, right] {ø} node[dot] {};
  \path (\V(1,1))     node[vowel, left] {ɘ} node[vowel, right] {ɵ} node[dot] {};
  \path (\V(1,2))     node[vowel, left] {ɤ} node[vowel, right] {o} node[dot] {};
  \path (\V(2,0))     node[vowel, left] {ɛ} node[vowel, right] {œ} node[dot] {};
  \path (\V(2,1))     node[vowel, left] {ɜ} node[vowel, right] {ɞ} node[dot] {};
  \path (\V(2,2))     node[vowel, left] {ʌ} node[vowel, right] {ɔ} node[dot] {};
  \path (\V(2.5,0))   node[vowel, left] {æ} node[vowel, right] { } node[   ] {};
  \path (\V(3,0))     node[vowel, left] {a} node[vowel, right] {ɶ} node[dot] {};
  \path (\V(3,2))     node[vowel, left] {ɑ} node[vowel, right] {ɒ} node[dot] {};

  % Draw the vertical lines.
  \draw (\V(0,0)) -- (\V(3,0));
  \draw (\V(0,1)) -- (\V(3,1));
  \draw (\V(0,2)) -- (\V(3,2));

  % Place the unpaired symbols last, on top of the vertical lines.
  \path (\V(1.5,1))   node[vowel]       {ə};
  \path (\V(2.5,1))   node[vowel]       {ɐ};
  \path (\V(-0.25,0)) node[vowel]       {front};
  \path (\V(-0.25,1)) node[vowel]       {central};
  \path (\V(-0.25,2)) node[vowel]       {back};
  \path (\V(0,-0.5))  node[vowel]       {high};
  \path (\V(1,-0.8))  node[vowel]       {high-mid};
  \path (\V(2,-1.55)) node[vowel]       {low-mid};
  \path (\V(3,-2.95)) node[vowel]       {low};
\end{tikzpicture}

              \end{adjustbox}
            }
        \end{columns}
      \end{frame}

    \subsection{\subonethree}
      \begin{frame}[t]{\subonethree}
        \only<1>{
          \begin{adjustbox}{width=\textwidth}
            % Read in English IPA consonant chart
            %%%%%%%%%%%%%%%%%%%%%%%%%%%%%%%%%%%%%%%%%%%%%%%%%%%%%%%%%%
% This creates an English IPA chart for consonants       %
%                                                        %
% Compiled from material_IPA_en_chart.tex when a         %
% standalone document is needed                          %
%                                                        %
% -Joshua McNeill (joshua dot mcneill at uga dot edu)    %
%%%%%%%%%%%%%%%%%%%%%%%%%%%%%%%%%%%%%%%%%%%%%%%%%%%%%%%%%%

\begin{tabular}{| l | c c | c c | c c | c c | c c | c c | c c | c c | c c | c c | c c |}
  \hline
  & \multicolumn{2}{c}{Bilabial} & \multicolumn{2}{c}{Labiodental} & \multicolumn{2}{c}{Dental} & \multicolumn{2}{c}{Alveolar} & \multicolumn{2}{c}{Postalveolar} & \multicolumn{2}{c}{Retroflex} & \multicolumn{2}{c}{Palatal} & \multicolumn{2}{c}{Velar} & \multicolumn{2}{c}{Uvular} & \multicolumn{2}{c}{Pharyngeal} & \multicolumn{2}{c}{Glottal} \\
  \hline
  Plosive/Stop & p & b &   &   &   &   & t & d &   &   & & & &   & k & ɡ & & & & & ʔ & \\
  \hline
  Nasal        &   & m &   &   &   &   &   & n &   &   & & & &   &   & ŋ & & & & &   & \\
  \hline
  Trill        &   &   &   &   &   &   &   &   &   &   & & & &   &   &   & & & & &   & \\
  \hline
  Tap/Flap     &   &   &   &   &   &   &   & ɾ &   &   & & & &   &   &   & & & & &   & \\
  \hline
  Fricative    &   &   & f & v & θ & ð & s & z & ʃ & ʒ & & & &   &   &   & & & & & h & \\
  \hline
  \begin{tabular}{@{} l @{}}
    Lateral \\
    fricative
  \end{tabular}&   &   &   &   &   &   &   &   &   &   & & & &   &   &   & & & & &   & \\
  \hline
  Approximant  &   &   &   &   &   &   &   & ɹ &   &   & & & & j &   &   & & & & &   & \\
  \hline
  \begin{tabular}{@{} l @{}}
    Lateral\\
    approximant
  \end{tabular}&   &   &   &   &   &   &   & l &   &   & & & &   &   &   & & & & &   & \\
  \hline
\end{tabular}

          \end{adjustbox}
          \begin{block}{}
            These are the consonants in English
          \end{block}
        }
        \only<2->{
          \begin{adjustbox}{width=\textwidth}
            % Read in full IPA consonant chart
            %%%%%%%%%%%%%%%%%%%%%%%%%%%%%%%%%%%%%%%%%%%%%%%%%%%%%%%%%%
% This creates a full IPA chart for consonants           %
%                                                        %
% Compiled from material_IPA_full_chart.tex when a       %
% standalone document is needed                          %
%                                                        %
% -Joshua McNeill (joshua dot mcneill at uga dot edu)    %
%%%%%%%%%%%%%%%%%%%%%%%%%%%%%%%%%%%%%%%%%%%%%%%%%%%%%%%%%%

\begin{tabular}{| l | c c | c c | c c | c c | c c | c c | c c | c c | c c | c c | c c |}
  \hline
  & \multicolumn{2}{c}{Bilabial} & \multicolumn{2}{c}{Labiodental} & \multicolumn{2}{c}{Dental} & \multicolumn{2}{c}{Alveolar} & \multicolumn{2}{c}{Postalveolar} & \multicolumn{2}{c}{Retroflex} & \multicolumn{2}{c}{Palatal} & \multicolumn{2}{c}{Velar} & \multicolumn{2}{c}{Uvular} & \multicolumn{2}{c}{Pharyngeal} & \multicolumn{2}{c}{Glottal} \\
  \hline
  Plosive/Stop & p & b &   &   &   &   & t & d &   &   & ʈ & ɖ & c & ɟ & k & g & q & ɢ &   &   & ʔ & \\
  \hline
  Nasal        &   & m &   & ɱ &   &   &   & n &   &   &   & ɳ &   & ɲ &   & ŋ &   & ɴ &   &   &   & \\
  \hline
  Trill        &   & ʙ &   &   &   &   &   & r &   &   &   &   &   &   &   &   &   & ʀ &   &   &   & \\
  \hline
  Tap/Flap     &   &   &   & ⱱ &   &   &   & ɾ &   &   &   & ɽ &   &   &   &   &   &   &   &   &   & \\
  \hline
  Fricative    & ɸ & β & f & v & θ & ð & s & z & ʃ & ʒ & ʂ & ʐ & ç & ʝ & x & ɣ & χ & ʁ & ħ & ʕ & h & ɦ \\
  \hline
  \begin{tabular}{@{} l @{}}
    Lateral \\
    fricative
  \end{tabular}&   &   &   &   &   &   & ɬ & ɮ &   &   &   &   &   &   &   &   &   &   &   &   &   & \\
  \hline
  Approximant  &   &   &   & ʋ &   &   &   & ɹ &   &   &   & ɻ &   & j &   & ɰ &   &   &   &   &   & \\
  \hline
  \begin{tabular}{@{} l @{}}
    Lateral\\
    approximant
  \end{tabular}&   &   &   &   &   &   &   & l &   &   &   & ɭ &   & ʎ &   & ʟ &   &   &   &   &   & \\
  \hline
\end{tabular}

          \end{adjustbox}
        }
        \only<2>{
          \begin{block}{}
            These are \emph{all} the consonants
          \end{block}
        }
        \only<3>{
          \begin{block}{All the fricatives}
            Ewe has [ɸ]
            \begin{itemize}
              \item \orth{eya} `he' [́éɸá]
            \end{itemize}
            Dutch has [ʝ]
            \begin{itemize}
              \item \orth{goed} `good' [ʝut]
            \end{itemize}
            Greek has [x]
            \begin{itemize}
              \item \orth{χώμα} `soil' [xɔma]
            \end{itemize}
          \end{block}
        }
        \only<4-5>{
          \begin{block}{More affricates}
            \uncover<5>{
              Quebec French has [dz]
              \begin{itemize}
                \item \orth{dîtes} `say' [dzɪt]
              \end{itemize}
              German has [pf]
              \begin{itemize}
                \item \orth{Pfennig} `penny' [pfɛnɪk]
              \end{itemize}
            }
          \end{block}
        }
        \only<6>{
          \begin{block}{More nasals}
            Spanish has [ɲ]
            \begin{itemize}
              \item \orth{baño} `bathroom' [baɲo]
            \end{itemize}
            Italian, too
            \begin{itemize}
              \item \orth{gnocchi} `gnocchi' [ɲoki]
            \end{itemize}
          \end{block}
        }
        \only<7>{
          \begin{block}{Going to other places... of articulation}
            Farsi has [q]
            \begin{itemize}
              \item `little bit' [qædri]
            \end{itemize}
            Inuktitut adds [ɢ]
            \begin{itemize}
              \item `explore' [ihipɢeoqteq]
            \end{itemize}
          \end{block}
        }
        \only<8>{
          \begin{block}{And other manners}
            \begin{columns}
              \column{0.5\textwidth}
                Spanish has [r]
                \begin{itemize}
                  \item \orth{perro} `dog' [pero]
                \end{itemize}
                Russian has [pʲ]
                \begin{itemize}
                  \item \orth{пять} `five' [pʲatʲ]
                \end{itemize}
              \column{0.5\textwidth}
                Lakota has [\href{https://youtu.be/mfrAlv-5P1c?t=7}{pʼ}]
                \begin{itemize}
                  \item `foggy' [pʼo]
                \end{itemize}
                Macedonian has [ɫ]
                \begin{itemize}
                  \item \orth{бела} `white' [beɫa]
                \end{itemize}
            \end{columns}
          \end{block}
        }
      \end{frame}

    \subsection{\subonefour}
      \begin{frame}{\subonefour}
        \begin{block}{}
          % A set of links to useful resources when dealing articulatory phonetics
\begin{itemize}
  \item To hear these sounds: \url{http://web.uvic.ca/ling/resources/ipa/charts/IPAlab/IPAlab.htm} and \url{https://americanipachart.com/}
  \item To type these symbols: \url{https://ipa.typeit.org/}
\end{itemize}

        \end{block}
        \begin{block}{Try these}
          \textcite{dawson_language_2016}, chapter 2 exercises 22 and 23
        \end{block}
      \end{frame}
\end{document}
