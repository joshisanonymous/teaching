%%%%%%%%%%%%%%%%%%%%%%%%%%%%%%%%%%%%%
%                                   %
% Compile with XeLaTeX and biber    %
%                                   %
% Questions or comments:            %
%                                   %
% joshua dot mcneill at uga dot edu %
%                                   %
%%%%%%%%%%%%%%%%%%%%%%%%%%%%%%%%%%%%%

\documentclass{beamer}
  % Read in standard preamble (cosmetic stuff)
  %%%%%%%%%%%%%%%%%%%%%%%%%%%%%%%%%%%%%%%%%%%%%%%%%%%%%%%%%%%%%%%%
% This is a standard preamble used in for all slide documents. %
% It basically contains cosmetic settings.                     %
%                                                              %
% Joshua McNeill                                               %
% joshua dot mcneill at uga dot edu                            %
%%%%%%%%%%%%%%%%%%%%%%%%%%%%%%%%%%%%%%%%%%%%%%%%%%%%%%%%%%%%%%%%

% Beamer settings
% \usetheme{Berkeley}
\usetheme{CambridgeUS}
% \usecolortheme{dove}
% \usecolortheme{rose}
\usecolortheme{seagull}
\usefonttheme{professionalfonts}
\usefonttheme{serif}
\setbeamertemplate{bibliography item}{}

% Packages and settings
\usepackage{fontspec}
  \setmainfont{Charis SIL}
\usepackage{hyperref}
  \hypersetup{colorlinks=true,
              allcolors=blue}
\usepackage{graphicx}
  \graphicspath{{../../figures/}}
\usepackage[normalem]{ulem}
\usepackage{enumerate}

% Document information
\author{M. McNeill}
\title[FREN2001]{Français 2001}
\institute{\url{joshua.mcneill@uga.edu}}
\date{}

%% Custom commands
% Lexical items
\newcommand{\lexi}[1]{\textit{#1}}
% Gloss
\newcommand{\gloss}[1]{`#1'}
\newcommand{\tinygloss}[1]{{\tiny`#1'}}
% Orthographic representations
\newcommand{\orth}[1]{$\langle$#1$\rangle$}
% Utterances (pragmatics)
\newcommand{\uttr}[1]{`#1'}
% Sentences (pragmatics)
\newcommand{\sent}[1]{\textit{#1}}
% Base dir for definitions
\newcommand{\defs}{../definitions}


  % Packages and settings
  \usepackage[backend=biber, style=apa]{biblatex}
    \addbibresource{../references/References.bib}

  % Document information
  \subtitle[Compositional Semantics]{Compositional Semantics}

  %% Custom commands
  % Subsection/frame titles
  \newcommand{\suboneone}{Why we need it}
  \newcommand{\subonetwo}{Propositions}
  \newcommand{\subonethree}{Entailment}
  \newcommand{\subonefour}{Incompatibility}
  \newcommand{\subonefive}{Practice}
  % First example of a proposition
  \newcommand{\proptrue}{China is the most populous country in the world}

\begin{document}
  % Read in the standard intro slides (title page and table of contents)
  %%%%%%%%%%%%%%%%%%%%%%%%%%%%%%%%%%%%%%%%%%%%%%%%%%%%%%%%%%%%%%%%
% This is a standard set of intro slides used in for all slide %
% documents. It basically contains the title page and table of %
% contents.                                                    %
%                                                              %
% Joshua McNeill                                               %
% joshua dot mcneill at uga dot edu                            %
%%%%%%%%%%%%%%%%%%%%%%%%%%%%%%%%%%%%%%%%%%%%%%%%%%%%%%%%%%%%%%%%

\begin{frame}
  \titlepage
  \tiny{Office: % Basically a variable for office hours location
Gilbert 121\\
        Office hours: % Basically a variable for office hours
 lundi, mercredi, vendredi 10:10--11:10
}
\end{frame}

\begin{frame}
  \tableofcontents[hideallsubsections]
\end{frame}

\AtBeginSection[]{
  \begin{frame}
    \tableofcontents[currentsection,
                     hideallsubsections]
  \end{frame}
}


  \section{Compositional Semantics}
    \subsection{\suboneone}
      \begin{frame}{\suboneone}
        \begin{block}{What do these words express together?}
          \begin{enumerate}
            \item \lexi{world}, \lexi{country}, \lexi{China}, \lexi{populous}
          \end{enumerate}
          \uncover<2->{
            What about these?
            \begin{enumerate}
              \setcounter{enumi}{1}
              \item \proptrue
            \end{enumerate}
          }
        \end{block}
        \begin{block}<2->{}
          We cannot communicate complex meanings solely through word meanings
          \begin{itemize}
            \item (2) goes beyond (1); it makes a claim
          \end{itemize}
        \end{block}
      \end{frame}

      \begin{frame}{\suboneone}
        \begin{alertblock}{Compositional semantics}
          % Compositional semantics
The area of semantics concerned with the meanings created through the combination of multiple lexical expressions

          \begin{itemize}
            \item The basic unit of compositional semantics is the proposition
          \end{itemize}
        \end{alertblock}
      \end{frame}

    \subsection{\subonetwo}
      \begin{frame}[t]{\subonetwo}
        \begin{definition}
          % Proposition
A sentence that describes some state of affairs and therefore makes a claim about the world

          \begin{itemize}
            \item They have \alert{truth values}: % Truth value
The property of a proposition that indicates its relation to truth (i.e., whether the proposition is true or false)

          \end{itemize}
        \end{definition}
        \only<2>{
          \begin{example}
            This proposition is true according to the world
            \begin{enumerate}
              \setcounter{enumi}{1}
              \item \proptrue
            \end{enumerate}
          \end{example}
        }
        \only<3>{
          \begin{example}
            This proposition is false according to the world
            \begin{enumerate}
              \setcounter{enumi}{2}
              \item Luxembourg is the most populous country in the world
            \end{enumerate}
          \end{example}
        }
      \end{frame}

      \begin{frame}{\subonetwo}
        \begin{block}{Truth values are based on \alert{truth conditions}}
          % Truth condition
That which must be true about the world in order for a proposition to be true

        \end{block}
        \begin{block}{What is this proposition's truth value?}
          \begin{itemize}
            \item The Queen of England is sleeping.
          \end{itemize}
          \uncover<2->{It depends on its the truth conditions.}
        \end{block}
      \end{frame}

    \subsection{\subonethree}
      \begin{frame}[t]{\subonethree}
        \begin{definition}
          % Entailment
When the truth of one proposition necessarily follows from the truth of another proposition

        \end{definition}
        \only<-2>{
          \begin{example}
            \begin{enumerate}
              \item All dogs bark.
              \item[$\rightarrow$] Sally's dog barks.
              \item<2-> All cats bark.
              \item<2->[$\rightarrow$] Sally's cat barks.
            \end{enumerate}
            \uncover<2->{
              The first proposition becomes the truth condition for the second
            }
          \end{example}
        }
        \only<3->{
          \begin{block}{Does this work as entailment?}
            \begin{enumerate}
              \item Sally's dog barks.
              \item[$\rightarrow$] All dogs bark.
            \end{enumerate}
            \uncover<4->{
            No, this entailment relationship is asymmetric
            }
          \end{block}
        }
      \end{frame}

      \begin{frame}{\subonethree}
        \begin{alertblock}{Mutual entailment}
          % Mutual entailment
When an entailment between two propositions is symmetric (e.g., it can be reversed)

        \end{alertblock}
        \begin{example}
          \begin{enumerate}
            \item Ian has a female sibling.
            \item[$\rightarrow$] Ian has a sister.
            \item Ian has a sister.
            \item[$\rightarrow$] Ian has a female sibling.
          \end{enumerate}
        \end{example}
      \end{frame}

    \subsection{\subonefour}
      \begin{frame}{\subonefour}
        \begin{definition}
          % Incompatibility
When two propositions cannot be both be true at the same time

        \end{definition}
        \begin{example}
          \begin{enumerate}
            \item Abraham Lincoln is dead.
            \item Abraham Lincoln is alive.
          \end{enumerate}
        \end{example}
      \end{frame}

    \subsection{\subonefive}
      \begin{frame}{\subonefive}
        \begin{block}{Try these}
          \textcite{dawson_language_2016}, chapter 6 exercises 17, 18, 20, 22, and 23
        \end{block}
      \end{frame}
\end{document}
