\documentclass[addpoints]{exam}
  % Read in shared preamble for all homeworks
  %%%%%%%%%%%%%%%%%%%%%%%%%%%%%%%%%%%%%%%%%%%%%%%%%%%%%%%%%%%%%%%%%%%%
% This is the standard preamble for homework assignments and exams %
%                                                                  %
% -Joshua McNeill (joshua dot mcneill at uga dot edu)              %
%%%%%%%%%%%%%%%%%%%%%%%%%%%%%%%%%%%%%%%%%%%%%%%%%%%%%%%%%%%%%%%%%%%%
% Exam settings
\pointsinmargin
\pointformat{}

% Packages and settings
\usepackage{fontspec}
  \setmainfont{Charis SIL}
\usepackage{tikz}

%% Custom commands
% Instructions for a section
\newcommand{\instr}[1]{
  \begin{center}
    \fbox{
      \parbox{0.85\textwidth}
             {#1}
    }
  \end{center}
}
\newcommand{\lexi}[1]{\textit{#1}}
\newcommand{\gloss}[1]{`#1'}


  % Packages and settings
  \usepackage{phonrule}

  % Document information
  \title{Homework 3: Syntax, Semantics, \& Pragmatics}
  \date{}

\begin{document}
  \maketitle

  % Header
  %%%%%%%%%%%%%%%%%%%%%%%%%%%%%%%%%%%%%%%%%%%%%%%%%%%%%%%%%%%%%%%%%%%%%%%
% This is the the header that all homework assignments and exams use. %
%                                                                     %
% -Joshua McNeill (joshua dot mcneill at uga dot edu)                 %
%%%%%%%%%%%%%%%%%%%%%%%%%%%%%%%%%%%%%%%%%%%%%%%%%%%%%%%%%%%%%%%%%%%%%%%
\noindent\makebox[0.5\textwidth][l]{Name:} \makebox[0.5\textwidth][r]{Course: LING2100, The Study of Language}\\
\makebox[0.5\textwidth][l]{Date:} \makebox[0.5\textwidth][r]{Instructor: Joshua McNeill}


    \section{Syntax}

      \instr{Each of the following sentences is ungrammatical due to violating the syntactic requirements of the linguistic expressions involved. For each, identify whether \emph{word order} or \emph{co-occurrence} requirements were violated. (1 point each)}
  \begin{questions}
        \question[1] \rule{4cm}{0.4pt} Their fields were.
        \question[1] \rule{4cm}{0.4pt} Attended the concert the audience.
        \question[1] \rule{4cm}{0.4pt} The was folded slip of paper.
        \question[1] \rule{4cm}{0.4pt} The dark water was closer to.
        \question[1] \rule{4cm}{0.4pt} The dining table repaired he.
        \question[1] \rule{4cm}{0.4pt} Saw the look in her face.

      \instr{For each sentence, indicate whether the underlined expression is an \emph{argument} or an \emph{adjunct}. (1 point each)}
        \question[1] \rule{4cm}{0.4pt} His father manufactured \underline{clothing}.
        \question[1] \rule{4cm}{0.4pt} The \underline{big} guy started down the path.
        \question[1] \rule{4cm}{0.4pt} Lucy was eating salad with \underline{a fork}.
        \question[1] \rule{4cm}{0.4pt} Alice stood still \underline{for a minute}.
        \question[1] \rule{4cm}{0.4pt} The doctor asked the patient \underline{to take a deep breath}.
        \question[1] \rule{4cm}{0.4pt} David saw the \underline{federal} soldiers in the distance.

      \instr{For each of the following sentences, identify \emph{all} of the syntactic constituents that are present (e.g., \textit{A boy kicked the ball} contains the syntactic constituents \textit{a boy}, \textit{kicked the ball}, and \textit{the ball}). (\emph{Hint}: Use the three constituent tests: answers to question, clefting, and pro-form substitution.) (1 point each)}
        \question[1] Paul briefed the two lawyers.

                     \hrulefill
        \question[1] Harold demanded some water from the waitor.

                     \hrulefill
        \question[1] Annie wandered into the woods.

                     \hrulefill
        \question[1] The green cat walked on the ledge.

                     \hrulefill

      \instr{For each pair of expressions, say whether they have the \emph{same} or \emph{different} syntactic distributions \textbf{and} give their syntactic categories (e.g., \textit{the dog} and \textit{a ball}: same, NP and NP). (2 points each)}
        \question[2] \rule{4cm}{0.4pt} talk to Mrs. Catt, stumble down the road
        \question[2] \rule{4cm}{0.4pt} Watson, unusual
        \question[2] \rule{4cm}{0.4pt} new positions, heroes
        \question[2] \rule{4cm}{0.4pt} in the bedroom, a few hands
        \question[2] \rule{4cm}{0.4pt} reached out, their lives
        \question[2] \rule{4cm}{0.4pt} prominent, remorseful

      \instr{For each sentence, give the phrase structure tree using the basic grammar of English syntax that we've constructed in class (i.e., the set of phrase structure rewrite rules that we've talked about). (3 points each)}
        \question[3] The tourists grabbed their suitcases.
          \vspace{\stretch{1}}
          \newpage
        \question[3] His eyes scanned the dirty wooden structure.
          \vspace{\stretch{1}}
        \question[3] Jim skimmed the long shelves for the oldest book.
          \vspace{\stretch{1}}
          \newpage
        \question[3] The little melody danced in his mind.
          \vspace{\stretch{1}}

    \section{Semantics \& Pragmatics}

      \instr{For each pair of lexical expressions, indicate their sense relation, be it synonymy, antonymy, or hyponymy. Where the sense relation is antonymy, specify if they are complementary or gradable, and where the sense relation is hyponymy, specify which is the hypernym and which the hyponym (e.g., \textit{dog} and \textit{poodle}: Hyponymy, \textit{poodle} is the hyponym and \textit{dog} the hypernym). (2 points each)}
        \question[2] \rule{4cm}{0.4pt} top, bottom
        \question[2] \rule{4cm}{0.4pt} introverted, extroverted
        \question[2] \rule{4cm}{0.4pt} frantically, wildly
        \question[2] \rule{4cm}{0.4pt} cedar, tree
        \question[2] \rule{4cm}{0.4pt} nervous, uneasy
        \question[2] \rule{4cm}{0.4pt} car, honda

        % \newpage

      \instr{For each pair of propositions, indicate whether \emph{the first entails the second}, whether \emph{the second entails the first}, whether they are \emph{mutually entailing}, or whether they are \emph{incompatible}. (1 point each)}
        \question[1] \rule{4cm}{0.4pt} \parbox[t]{\linewidth}{
                                         His house is in the countryside.

                                         His house is in a rural area.
                                       }
          \newpage
        \question[1] \rule{4cm}{0.4pt} \parbox[t]{\linewidth}{
                                         Scotty's eyes fell to his grumbling stomach.

                                         Scotty stared down at his grumbling stomach.
                                       }
        \question[1] \rule{4cm}{0.4pt} \parbox[t]{\linewidth}{
                                         Joe lives in New Orleans.

                                         Joe lives in a city.
                                       }
        \question[1] \rule{4cm}{0.4pt} \parbox[t]{\linewidth}{
                                         Kathy visited someone.

                                         Kathy walked swiftly to her aunt's house.
                                       }
        \question[1] \rule{4cm}{0.4pt} \parbox[t]{\linewidth}{
                                         She greeted her colleague.

                                         She ignored her colleague.
                                       }
        \question[1] \rule{4cm}{0.4pt} \parbox[t]{\linewidth}{
                                         The blues was playing on the radio.

                                         Music was playing on the radio.
                                       }

      \instr{For each maxim of the cooperative principle, describe a scenario in which it is being flouted. (2 points each)}
        \question[2] The maxim of relevance: \hrulefill

          \hrulefill
        \question[2] The maxim of quantity: \hrulefill

          \hrulefill

      \instr{For each sentence, identify all the existence presuppositions (e.g., \textit{Santa Claus is riding on his sleigh} presupposes that \textit{Santa Claus} and \textit{his sleigh} exist). (1 point each)}
        \question[1] The cook threw a frying pan.

          \hrulefill
        \question[1] The trial cannot proceed until all the jurymen are back.

          \hrulefill
        \question[1] There's certainly too much pepper in that soup.

          \hrulefill
        \question[1] The king pointed to the tarts on the table.

          \hrulefill

  \end{questions}

  \vspace{1.25cm}

  % Grade
  \begin{center}
    \gradetable[v][pages]
  \end{center}
\end{document}
