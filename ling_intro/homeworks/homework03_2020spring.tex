\documentclass[addpoints]{exam}
  % Read in shared preamble for all homeworks
  %%%%%%%%%%%%%%%%%%%%%%%%%%%%%%%%%%%%%%%%%%%%%%%%%%%%%%%%%%%%%%%%%%%%
% This is the standard preamble for homework assignments and exams %
%                                                                  %
% -Joshua McNeill (joshua dot mcneill at uga dot edu)              %
%%%%%%%%%%%%%%%%%%%%%%%%%%%%%%%%%%%%%%%%%%%%%%%%%%%%%%%%%%%%%%%%%%%%
% Exam settings
\pointsinmargin
\pointformat{}

% Packages and settings
\usepackage{fontspec}
  \setmainfont{Charis SIL}
\usepackage{tikz}

%% Custom commands
% Instructions for a section
\newcommand{\instr}[1]{
  \begin{center}
    \fbox{
      \parbox{0.85\textwidth}
             {#1}
    }
  \end{center}
}
\newcommand{\lexi}[1]{\textit{#1}}
\newcommand{\gloss}[1]{`#1'}


  % Packages and settings
  \usepackage{phonrule}

  % Document information
  \title{Homework 3: Syntax, Semantics, \& Pragmatics}
  \date{}

\begin{document}
  \maketitle

  % Header
  %%%%%%%%%%%%%%%%%%%%%%%%%%%%%%%%%%%%%%%%%%%%%%%%%%%%%%%%%%%%%%%%%%%%%%%
% This is the the header that all homework assignments and exams use. %
%                                                                     %
% -Joshua McNeill (joshua dot mcneill at uga dot edu)                 %
%%%%%%%%%%%%%%%%%%%%%%%%%%%%%%%%%%%%%%%%%%%%%%%%%%%%%%%%%%%%%%%%%%%%%%%
\noindent\makebox[0.5\textwidth][l]{Name:} \makebox[0.5\textwidth][r]{Course: LING2100, The Study of Language}\\
\makebox[0.5\textwidth][l]{Date:} \makebox[0.5\textwidth][r]{Instructor: Joshua McNeill}


    \section{Syntax}

      \instr{Each of the following sentences is ungrammatical due to violating the syntactic requirements of the linguistic expressions involved. For each, identify whether \emph{word order} or \emph{co-occurrence} requirements were violated. (1 point each)}
  \begin{questions}
        \question[1] \rule{6cm}{0.4pt} Jonathan and Ben were.
        \question[1] \rule{6cm}{0.4pt} I guess rather than.
        \question[1] \rule{6cm}{0.4pt} It of the river was the bank wrong.
        \question[1] \rule{6cm}{0.4pt} Balls six were bouncing.
        \question[1] \rule{6cm}{0.4pt} Harry still had his memory of.
        \question[1] \rule{6cm}{0.4pt} Was not growing any larger.

      \instr{For each sentence, indicate whether the underlined expression is an \emph{argument} or an \emph{adjunct}. (1 point each)}
        \question[1] \rule{6cm}{0.4pt} The company called her on \underline{the phone}.
        \question[1] \rule{6cm}{0.4pt} \underline{Her uncle} is a surgeon.
        \question[1] \rule{6cm}{0.4pt} He had a \underline{good} reason to rejoice.
        \question[1] \rule{6cm}{0.4pt} Her optimism gave \underline{me} heart.
        \question[1] \rule{6cm}{0.4pt} The \underline{diligent} actor imitated the laborers.
        \question[1] \rule{6cm}{0.4pt} He could not picture the \underline{small} red sports car.

      \instr{For each of the following sentences, identify \emph{all} of the syntactic constituents that are present (e.g., \textit{A boy kicked the ball} contains the syntactic constituents \textit{a boy}, \textit{kicked the ball}, and \textit{the ball}). (\emph{Hint}: Use the three constituent tests: answers to question, clefting, and pro-form substitution.) (2 point each)}
        \question[2] His neighbors cheered for their team.

                     \hrulefill

                     \hrulefill
        \question[2] Fritz provded Claire with a good opportunity.

                     \hrulefill

                     \hrulefill
        \question[2] The authoritative words sounded harsh.

                     \hrulefill

                     \hrulefill
        \question[2] They approached the sea with caution.

                     \hrulefill

                     \hrulefill

      \instr{For each pair of expressions, say whether they have the \emph{same} or \emph{different} syntactic distributions \textbf{and} give their syntactic categories (e.g., \textit{the dog} and \textit{a ball}: same, NP and NP). (2 points each)}
        \question[2] \rule{6cm}{0.4pt} pointed toward the briefing room, the basket of apples
        \question[2] \rule{9cm}{0.4pt} the solemn occasion, jabbed at him
        \question[2] \rule{9cm}{0.4pt} yelled some gibberish, at the table
        \question[2] \rule{9cm}{0.4pt} very heroic, five steps
        \question[2] \rule{9cm}{0.4pt} underhanded, important
        \question[2] \rule{9cm}{0.4pt} she, the expansive bay

      \instr{For each sentence, give the phrase structure tree using the basic grammar of English syntax that we've constructed in class (i.e., the set of phrase structure rewrite rules that we've talked about). (3 points each)}
        \question[3] They stood at the end of the street.
          \vspace{\stretch{1}}
          \newpage
        \question[3] A dark foreboding premonition of danger surfaced.
          \vspace{\stretch{1}}
        \question[3] A group of dignitaries from Tokyo arrived in the morning.
          \vspace{\stretch{1}}
          \newpage
        \question[3] The gold record spun in the hot sand for many minutes.
          \vspace{\stretch{1}}

    \section{Semantics \& Pragmatics}

      \instr{For each pair of lexical expressions, indicate their sense relation, be it synonymy, antonymy, or hyponymy. Where the sense relation is antonymy, specify if they are complementary or gradable, and where the sense relation is hyponymy, specify which is the hypernym and which the hyponym (e.g., \textit{dog} and \textit{poodle}: Hyponymy, \textit{poodle} is the hyponym and \textit{dog} the hypernym). (2 points each)}
        \question[2] \rule{11cm}{0.4pt} flawed, faulty
        \question[2] \rule{11cm}{0.4pt} fragrance, aroma
        \question[2] \rule{11cm}{0.4pt} stable, instable
        \question[2] \rule{11cm}{0.4pt} sea vessel, submarine
        \question[2] \rule{11cm}{0.4pt} unicorn, animal
        \question[2] \rule{11cm}{0.4pt} charitable, selfish

      \instr{For each pair of propositions, indicate whether \emph{the first entails the second}, whether \emph{the second entails the first}, whether they are \emph{mutually entailing}, or whether they are \emph{incompatible}. (1 point each)}
        \question[1] \rule{4cm}{0.4pt} \parbox[t]{\linewidth}{
                                         Kathy visited someone.

                                         Kathy walked swiftly to her aunt's house to talk.
                                       }
          \newpage
        \question[1] \rule{4cm}{0.4pt} \parbox[t]{\linewidth}{
                                         Scotty's eyes slowly fell to his grumbling stomach.

                                         Scotty looked at his grumbling stomach.
                                       }
        \question[1] \rule{4cm}{0.4pt} \parbox[t]{\linewidth}{
                                         His house is in the countryside.

                                         His house is in a rural area.
                                       }
        \question[1] \rule{4cm}{0.4pt} \parbox[t]{\linewidth}{
                                         Joe lives in New Orleans.

                                         Joe lives in a city.
                                       }
        \question[1] \rule{4cm}{0.4pt} \parbox[t]{\linewidth}{
                                         The blues was playing on the radio.

                                         Music was playing on the radio.
                                       }
        \question[1] \rule{4cm}{0.4pt} \parbox[t]{\linewidth}{
                                         She greeted her colleague.

                                         She ignored her colleague.
                                       }

      \instr{For each maxim of the cooperative principle, describe a scenario in which it is being flouted. \emph{Hint}: Remember that flouting a maximum is not the same as simply violating a maxim. (2 points each)}
        \question[2] The maxim of relevance: \hrulefill

          \hrulefill

          \hrulefill

          \hrulefill
        \question[2] The maxim of quantity: \hrulefill

          \hrulefill

          \hrulefill

          \hrulefill

      \instr{For each sentence, identify all the existence presuppositions (e.g., \textit{Santa Claus is riding on his sleigh} presupposes that \textit{Santa Claus} and \textit{his sleigh} exist). (2 point each)}
        \question[2] The cook threw a frying pan.

          \hrulefill

          \hrulefill
        \question[2] The trial cannot proceed until all the jurymen are back.

          \hrulefill

          \hrulefill
        \question[2] There's certainly too much pepper in that soup.

          \hrulefill

          \hrulefill
        \question[2] The king pointed to the tarts on the table.

          \hrulefill

          \hrulefill

  \end{questions}

  \vspace{1.25cm}

  % Grade
  \begin{center}
    \gradetable[v][pages]
  \end{center}
\end{document}
