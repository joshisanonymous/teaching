%%%%%%%%%%%%%%%%%%%%%%%%%%%%%%%%%%%%%
%                                   %
% Compile with XeLaTeX and biber    %
%                                   %
% Questions or comments:            %
%                                   %
% joshua dot mcneill at uga dot edu %
%                                   %
%%%%%%%%%%%%%%%%%%%%%%%%%%%%%%%%%%%%%

\documentclass{beamer}
  % Read in standard preamble (cosmetic stuff)
  %%%%%%%%%%%%%%%%%%%%%%%%%%%%%%%%%%%%%%%%%%%%%%%%%%%%%%%%%%%%%%%%
% This is a standard preamble used in for all slide documents. %
% It basically contains cosmetic settings.                     %
%                                                              %
% Joshua McNeill                                               %
% joshua dot mcneill at uga dot edu                            %
%%%%%%%%%%%%%%%%%%%%%%%%%%%%%%%%%%%%%%%%%%%%%%%%%%%%%%%%%%%%%%%%

% Beamer settings
% \usetheme{Berkeley}
\usetheme{CambridgeUS}
% \usecolortheme{dove}
% \usecolortheme{rose}
\usecolortheme{seagull}
\usefonttheme{professionalfonts}
\usefonttheme{serif}
\setbeamertemplate{bibliography item}{}

% Packages and settings
\usepackage{fontspec}
  \setmainfont{Charis SIL}
\usepackage{hyperref}
  \hypersetup{colorlinks=true,
              allcolors=blue}
\usepackage{graphicx}
  \graphicspath{{../../figures/}}
\usepackage[normalem]{ulem}
\usepackage{enumerate}

% Document information
\author{M. McNeill}
\title[FREN2001]{Français 2001}
\institute{\url{joshua.mcneill@uga.edu}}
\date{}

%% Custom commands
% Lexical items
\newcommand{\lexi}[1]{\textit{#1}}
% Gloss
\newcommand{\gloss}[1]{`#1'}
\newcommand{\tinygloss}[1]{{\tiny`#1'}}
% Orthographic representations
\newcommand{\orth}[1]{$\langle$#1$\rangle$}
% Utterances (pragmatics)
\newcommand{\uttr}[1]{`#1'}
% Sentences (pragmatics)
\newcommand{\sent}[1]{\textit{#1}}
% Base dir for definitions
\newcommand{\defs}{../definitions}


  % Packages and settings
  \usepackage{phonrule}
  \usepackage[backend=biber, style=apa]{biblatex}
    \addbibresource{../references/References.bib}

  % Document information
  \subtitle[Morphological Analysis]{Morphological Analysis}

  %% Custom commands
  % Subsection/frame titles
  \newcommand{\suboneone}{Identifying morphemes}
  \newcommand{\subonetwo}{Identifying allomorphs}
  \newcommand{\subonethree}{Be careful}
  \newcommand{\subonefour}{Practice}

\begin{document}
  % Read in the standard intro slides (title page and table of contents)
  %%%%%%%%%%%%%%%%%%%%%%%%%%%%%%%%%%%%%%%%%%%%%%%%%%%%%%%%%%%%%%%%
% This is a standard set of intro slides used in for all slide %
% documents. It basically contains the title page and table of %
% contents.                                                    %
%                                                              %
% Joshua McNeill                                               %
% joshua dot mcneill at uga dot edu                            %
%%%%%%%%%%%%%%%%%%%%%%%%%%%%%%%%%%%%%%%%%%%%%%%%%%%%%%%%%%%%%%%%

\begin{frame}
  \titlepage
  \tiny{Office: % Basically a variable for office hours location
Gilbert 121\\
        Office hours: % Basically a variable for office hours
 lundi, mercredi, vendredi 10:10--11:10
}
\end{frame}

\begin{frame}
  \tableofcontents[hideallsubsections]
\end{frame}

\AtBeginSection[]{
  \begin{frame}
    \tableofcontents[currentsection,
                     hideallsubsections]
  \end{frame}
}


  \section{Morphological Analysis}
    \subsection{\suboneone}
      \begin{frame}[t]{\suboneone}
        \only<-7>{
          \begin{block}{A Classical Greek example}
            \begin{itemize}
              \item {[}ɡrapʰɔː] `I write'
              \item<2-> {[}pʰɛːmi] `to speak'
              \item<3-> {[}ɡrapʰɛː] `s/he writes'
            \end{itemize}
          \end{block}
          \only<-6>{
            \begin{block}{Which part means `I'?}
              \begin{itemize}
                \item<4-> {[}-ɔː] `I'
              \end{itemize}
            \end{block}
            \begin{block}<5->{What other morphemes can you identify?}
              \begin{itemize}
                \item<6-> {[}ɡrapʰ] `write'
                \item<6-> {[}-ɛː] `s/he'
              \end{itemize}
            \end{block}
          }
          \only<7->{
            \begin{alertblock}{}
              This is possible because we have similar forms and similar meanings
            \end{alertblock}
          }
        }
        \only<8->{
          \begin{block}{A Hungarian example}
            \begin{columns}
              \column{0.48\linewidth}
                \begin{itemize}
                  \item {[}hɔz] `house'
                  \item {[}ɛɟhɔz] `a house'
                  \item {[}hɔzɔ] `his/her house'
                \end{itemize}
              \column{0.48\linewidth}
                \begin{itemize}
                  \item {[}boɾ] `wine'
                  \item {[}ɛɟboɾ] `a wine'
                  \item {[}boɾɔ] `his/her wine'
                \end{itemize}
            \end{columns}
          \end{block}
          \begin{block}{What morphemes can you identify?}
            \begin{columns}
              \column{0.48\linewidth}
                \begin{itemize}
                  \item<9-> {[}hɔz] `house'
                  \item<9-> {[}boɾ] `wine'
                \end{itemize}
              \column{0.48\linewidth}
                \begin{itemize}
                  \item<9-> {[}ɛɟ-] `a'
                  \item<9-> {[}-ɔ] `his/her'
                \end{itemize}
            \end{columns}
          \end{block}
          \only<10-11>{
            \begin{block}{Prefixes or suffixes?}
              \begin{columns}
                \column{0.48\linewidth}
                  \begin{itemize}
                    \item<11-> Prefix: [ɛɟ-]
                  \end{itemize}
                \column{0.48\linewidth}
                  \begin{itemize}
                    \item<11-> Suffix: [-ɔ]
                  \end{itemize}
              \end{columns}
            \end{block}
          }
          \only<12-13>{
            \begin{block}{Derivational or inflectional?}
              \uncover<13->{
                Both inflectional
              }
            \end{block}
          }
        }
      \end{frame}

    \subsection{\subonetwo}
      \begin{frame}[t]{\subonetwo}
        \only<1-6>{
          \begin{example}
            \begin{tabular}{l l}
              [\alert<4->{ɪm}.pɹəˈsaɪs]     & [\alert<4->{ɪɹ}.ɹɪˈspɑn.sə.bl̩] \\
              {[}\alert<4->{ɪn}ˈæ.də.kwɪt]  & [\alert<4->{ɪl}ˈlɛ.dʒə.bl̩] \\
              {[}\alert<4->{ɪŋ}.kəmˈplit]
            \end{tabular}
          \end{example}
        }
        \only<1-5>{
          \begin{block}{What meaning do these all share?}
            \begin{itemize}
              \item<2-> Negation
            \end{itemize}
          \end{block}
          \begin{block}<3->{Which morpheme adds the sense of negation?}
            \uncover<5->{
              Each \alert{morph} here shares [ɪ-]
              \begin{itemize}
                \item[$\rightarrow$] We can write phonological rules
              \end{itemize}
            }
          \end{block}
        }
        \only<6>{
          \begin{block}{The rules}
            \begin{itemize}
              \item \phonr{/ɪn-/}{[ɪm-]}{\phonfeat{+labial}}
              \item \phonr{/ɪn-/}{[ɪŋ-]}{\phonfeat{+velar}}
              \item \phonr{/ɪn-/}{[ɪɹ-]}{[ɹ]}
              \item \phonr{/ɪn-/}{[ɪl-]}{[l]}
              \item \phonc{/ɪn-/}{[ɪn-]}{elsewhere}
            \end{itemize}
          \end{block}
        }
        \only<7->{
          \begin{block}{Some definitions}
            \begin{itemize}
              \item \alert{Morpheme}: % Morpheme
The smallest linguistic unit that has lexical or grammatical meaning

              \begin{itemize}
                \item /ɪn-/
              \end{itemize}
              \item \alert{Allomorph}: % Allomorph
One of a set of possible realizations of one morpheme

              \begin{itemize}
                \item {[}ɪm- ɪŋ- ɪɹ- ɪl-] and [ɪn-]
              \end{itemize}
              \item \alert{Morph}: % Morph
A general term referring to any realization of a morpheme

            \end{itemize}
          \end{block}
        }
      \end{frame}

    \subsection{\subonethree}
      \begin{frame}[t]{\subonethree}
        \begin{block}{}
          Don't assume all languages have the same types of inflections
        \end{block}
        \only<1>{
          \begin{block}{Tagalog doesn't inflect for plural}
            \begin{itemize}
              \item {[}aŋ bataʔ] `the child'
              \item {[}aŋ mɡa bataʔ] `the children'
            \end{itemize}
          \end{block}
        }
        \only<2>{
          \begin{block}{Sanskrit inflects for dual}
            \begin{itemize}
              \item \lexi{juhomi} `I sacrifice'
              \item \lexi{juhuvas} `we (two people) sacrifice'
              \item \lexi{juhumas} `we (more than two) sacrifice'
            \end{itemize}
          \end{block}
        }
        \only<3>{
          \begin{block}{Comanche inflects for inclusive/exclusive}
            \begin{itemize}
              \item {[}taa] `we (inclusive, i.e. including you)'
              \item {[}nɨnɨ] `we (exclusive, i.e. not including you)'
            \end{itemize}
          \end{block}
        }
      \end{frame}

    \subsection{\subonefour}
      \begin{frame}{\subonefour}
        \begin{block}{Try these}
          \textcite{dawson_language_2016}, chapter 4 exercises 29, 40, and 44
        \end{block}
      \end{frame}
\end{document}
