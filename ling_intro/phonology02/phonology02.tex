%%%%%%%%%%%%%%%%%%%%%%%%%%%%%%%%%%%%%
%                                   %
% Compile with XeLaTeX and biber    %
%                                   %
% Questions or comments:            %
%                                   %
% joshua dot mcneill at uga dot edu %
%                                   %
%%%%%%%%%%%%%%%%%%%%%%%%%%%%%%%%%%%%%

\documentclass{beamer}
  % Read in standard preamble (cosmetic stuff)
  %%%%%%%%%%%%%%%%%%%%%%%%%%%%%%%%%%%%%%%%%%%%%%%%%%%%%%%%%%%%%%%%
% This is a standard preamble used in for all slide documents. %
% It basically contains cosmetic settings.                     %
%                                                              %
% Joshua McNeill                                               %
% joshua dot mcneill at uga dot edu                            %
%%%%%%%%%%%%%%%%%%%%%%%%%%%%%%%%%%%%%%%%%%%%%%%%%%%%%%%%%%%%%%%%

% Beamer settings
% \usetheme{Berkeley}
\usetheme{CambridgeUS}
% \usecolortheme{dove}
% \usecolortheme{rose}
\usecolortheme{seagull}
\usefonttheme{professionalfonts}
\usefonttheme{serif}
\setbeamertemplate{bibliography item}{}

% Packages and settings
\usepackage{fontspec}
  \setmainfont{Charis SIL}
\usepackage{hyperref}
  \hypersetup{colorlinks=true,
              allcolors=blue}
\usepackage{graphicx}
  \graphicspath{{../../figures/}}
\usepackage{soul}
  \setstcolor{red}
\usepackage[normalem]{ulem}
\usepackage{enumerate}
\usepackage{tikz}
  \usetikzlibrary{trees}

% Document information
\author{M. McNeill}
\title[FREN1001]{Français 1001}
\institute{\url{joshua.mcneill@uga.edu}}
\date{}

%% Custom commands
% Lexical items
\newcommand{\lexi}[1]{\textit{#1}}
% Gloss
\newcommand{\gloss}[1]{`#1'}
\newcommand{\tinygloss}[1]{{\tiny`#1'}}
% Orthographic representations
\newcommand{\orth}[1]{$\langle$#1$\rangle$}
% Utterances (pragmatics)
\newcommand{\uttr}[1]{`#1'}
% Sentences (pragmatics)
\newcommand{\sent}[1]{\textit{#1}}
% Fixed length underlines
\newcommand{\funderline}[2][4cm]{
  \underline{\makebox[\ifdim\width>#1\width\else#1\fi]{#2}}
}
% Base dir for definitions
\newcommand{\defs}{../definitions}
\newcommand{\activity}[1]{
  \input{./activities/#1.tex}
}


  % Document information
  \subtitle[Phonemes and Allophones]{Phonemes and Allophones}

  %% Custom commands
  % Subsection/frame titles
  \newcommand{\suboneone}{What's going on?}
  \newcommand{\subonetwo}{A clearer example}
  \newcommand{\subonethree}{Some definitions}

\begin{document}
  % Read in the standard intro slides (title page and table of contents)
  %%%%%%%%%%%%%%%%%%%%%%%%%%%%%%%%%%%%%%%%%%%%%%%%%%%%%%%%%%%%%%%%
% This is a standard set of intro slides used in for all slide %
% documents. It basically contains the title page and table of %
% contents.                                                    %
%                                                              %
% Joshua McNeill                                               %
% joshua dot mcneill at uga dot edu                            %
%%%%%%%%%%%%%%%%%%%%%%%%%%%%%%%%%%%%%%%%%%%%%%%%%%%%%%%%%%%%%%%%

\begin{frame}
  \titlepage
  \tiny{Office: % Basically a variable for office hours location
T Gilbert 141/W Library 4th Fl
\\
        Office hours: % Basically a variable for office hours
T 11-12/W 11-12:30
}
\end{frame}

\begin{frame}
  \tableofcontents[hideallsubsections]
\end{frame}

\AtBeginSection[]{
  \begin{frame}
    \tableofcontents[currentsection,
                     hideallsubsections]
  \end{frame}
}


  \section{Phonemes and allophones}
    \subsection{\suboneone}
      \begin{frame}{\suboneone}
        \only<1-4>{
          \begin{block}{Are they different words?}
            \begin{itemize}
              \item {[}ˈ\alert{t}ɛɹ]
              \item {[}ˈ\alert{d}ɛɹ]
            \end{itemize}
            \uncover<2->{[t] and [d] are phonemes}
          \end{block}
          \begin{block}<3->{Why do we spell these with \orth{t}?}
            \begin{itemize}
              \item {[}ˈ\alert{t}ɛɹ]
              \item {[}ˈwɔ.\alert{ɾ}ɹ̩]
            \end{itemize}
            \uncover<4->{This [t] and this [ɾ] are allophones of one phoneme}
          \end{block}
        }
        \only<5-6>{
          \begin{block}{And what's up with these?}
            \begin{itemize}
              \item {[}ˈwɔ.\alert{ɾ}ɹ̩]
              \item {[}ˈwɔ.\alert{t}ə]
              \item {[}ˈwɔ.\alert{ʔ}ə]
            \end{itemize}
            \uncover<6->{This [ɾ], this [t], and this [ʔ] are in free variation}
          \end{block}
        }
      \end{frame}

    \subsection{\subonetwo}
      \begin{frame}{\subonetwo}
        \begin{block}{Are these [p]s the same?}
          \begin{itemize}
            \item {[}ˈ\alert{p\only<2>{ʰ}}ɪt]
            \item {[}ˈs\alert{p}ɪt]
          \end{itemize}
          \uncover<2>{Aspiration is lost when a stop is preceded by a fricative}
        \end{block}
      \end{frame}

    \subsection{\subonethree}
      \begin{frame}{\subonethree}
        \only<-3>{
          \begin{alertblock}{Allophones}
            % Allophones
A set of speech sounds that are perceived as being variants of one meaningful sound

            \begin{itemize}
              \item {[}pʰ] and [p] are allophones of the phoneme /p/
            \end{itemize}
          \end{alertblock}
          \begin{alertblock}<2->{Phoneme}
            % Phoneme
A mental representation of a meaningful speech sound, which may or may not be realized in multiple ways

            \begin{itemize}
              \item /p/ is a phoneme that can be realized as [pʰ] or [p]
            \end{itemize}
          \end{alertblock}
          \begin{alertblock}<3->{Phone}
            % Phone
The technical term for a speech sound

          \end{alertblock}
        }
        \only<4-5>{
          \begin{block}{Phonemes and allophones are language specific}
            \begin{itemize}
              \item {[}pʰ] and [p] are semantically \alert{non-constrastive} in English, but
              \item {[}pʰ] and [p] are semantically \alert{constrastive} in Hindi
            \end{itemize}
            i.e., They're allophones in English but phonemes in Hindi
          \end{block}
          \begin{example}<5->
            \begin{itemize}
              \item {[}pʰəl] `fruit'
              \item {[}pəl] `moment'
            \end{itemize}
          \end{example}
        }
        \only<6->{
          \begin{block}{Summarizing these examples}
            \begin{tabular}{l r}
            Phonemes                  & Allophones \\
            (mental representations)  & (actual acoustic sound) \\
            \hline \\
            \multicolumn{2}{c}{\emph{English}} \\
            /p/                       & [p, pʰ] \\
            /t/                       & [t, tʰ, ɾ, ʔ] \\
            \multicolumn{2}{c}{\emph{Hindi}} \\
            /t/                       & [t] \\
            /tʰ/                      & [tʰ]
            \end{tabular}
          \end{block}
        }
      \end{frame}

  \section{Identifying them}

    % Sounds' distributions
    % Phonemes and allophones
    % Identifying phonemes and allophones
\end{document}
