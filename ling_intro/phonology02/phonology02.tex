%%%%%%%%%%%%%%%%%%%%%%%%%%%%%%%%%%%%%
%                                   %
% Compile with XeLaTeX and biber    %
%                                   %
% Questions or comments:            %
%                                   %
% joshua dot mcneill at uga dot edu %
%                                   %
%%%%%%%%%%%%%%%%%%%%%%%%%%%%%%%%%%%%%

\documentclass{beamer}
  % Read in standard preamble (cosmetic stuff)
  %%%%%%%%%%%%%%%%%%%%%%%%%%%%%%%%%%%%%%%%%%%%%%%%%%%%%%%%%%%%%%%%
% This is a standard preamble used in for all slide documents. %
% It basically contains cosmetic settings.                     %
%                                                              %
% Joshua McNeill                                               %
% joshua dot mcneill at uga dot edu                            %
%%%%%%%%%%%%%%%%%%%%%%%%%%%%%%%%%%%%%%%%%%%%%%%%%%%%%%%%%%%%%%%%

% Beamer settings
% \usetheme{Berkeley}
\usetheme{CambridgeUS}
% \usecolortheme{dove}
% \usecolortheme{rose}
\usecolortheme{seagull}
\usefonttheme{professionalfonts}
\usefonttheme{serif}
\setbeamertemplate{bibliography item}{}

% Packages and settings
\usepackage{fontspec}
  \setmainfont{Charis SIL}
\usepackage{hyperref}
  \hypersetup{colorlinks=true,
              allcolors=blue}
\usepackage{graphicx}
  \graphicspath{{../../figures/}}
\usepackage[normalem]{ulem}
\usepackage{enumerate}

% Document information
\author{M. McNeill}
\title[FREN2001]{Français 2001}
\institute{\url{joshua.mcneill@uga.edu}}
\date{}

%% Custom commands
% Lexical items
\newcommand{\lexi}[1]{\textit{#1}}
% Gloss
\newcommand{\gloss}[1]{`#1'}
\newcommand{\tinygloss}[1]{{\tiny`#1'}}
% Orthographic representations
\newcommand{\orth}[1]{$\langle$#1$\rangle$}
% Utterances (pragmatics)
\newcommand{\uttr}[1]{`#1'}
% Sentences (pragmatics)
\newcommand{\sent}[1]{\textit{#1}}
% Base dir for definitions
\newcommand{\defs}{../definitions}


  % Packages and settings
  \usepackage[backend=biber, style=apa]{biblatex}
    \addbibresource{../references/References.bib}

  % Document information
  \subtitle[Phonemes and Allophones]{Phonemes and Allophones}

  %% Custom commands
  % Subsection/frame titles
  \newcommand{\suboneone}{What's going on?}
  \newcommand{\subonetwo}{A clearer example}
  \newcommand{\subonethree}{Some definitions}
  \newcommand{\subtwoone}{The steps}
  \newcommand{\subtwotwo}{Class practice}
  \newcommand{\subtwothree}{Free variation}
  \newcommand{\subtwofour}{Practice}

\begin{document}
  % Read in the standard intro slides (title page and table of contents)
  %%%%%%%%%%%%%%%%%%%%%%%%%%%%%%%%%%%%%%%%%%%%%%%%%%%%%%%%%%%%%%%%
% This is a standard set of intro slides used in for all slide %
% documents. It basically contains the title page and table of %
% contents.                                                    %
%                                                              %
% Joshua McNeill                                               %
% joshua dot mcneill at uga dot edu                            %
%%%%%%%%%%%%%%%%%%%%%%%%%%%%%%%%%%%%%%%%%%%%%%%%%%%%%%%%%%%%%%%%

\begin{frame}
  \titlepage
  \tiny{Office: % Basically a variable for office hours location
Gilbert 121\\
        Office hours: % Basically a variable for office hours
 lundi, mercredi, vendredi 10:10--11:10
}
\end{frame}

\begin{frame}
  \tableofcontents[hideallsubsections]
\end{frame}

\AtBeginSection[]{
  \begin{frame}
    \tableofcontents[currentsection,
                     hideallsubsections]
  \end{frame}
}


  \section{Phonemes and allophones}
    \subsection{\suboneone}
      \begin{frame}{\suboneone}
        \only<1-4>{
          \begin{block}{Are they different words?}
            \begin{itemize}
              \item {[}ˈ\alert{t}ɛɹ]
              \item {[}ˈ\alert{d}ɛɹ]
            \end{itemize}
            \uncover<2->{[t] and [d] are phonemes}
          \end{block}
          \begin{block}<3->{Why do we spell these with \orth{t}?}
            \begin{itemize}
              \item {[}ˈ\alert{t}ɛɹ]
              \item {[}ˈwɔ.\alert{ɾ}ɹ̩]
            \end{itemize}
            \uncover<4->{This [t] and this [ɾ] are allophones of one phoneme}
          \end{block}
        }
        \only<5-6>{
          \begin{block}{And what's up with these?}
            \begin{itemize}
              \item {[}ˈwɔ.\alert{ɾ}ɹ̩]
              \item {[}ˈwɔ.\alert{t}ə]
              \item {[}ˈwɔ.\alert{ʔ}ə]
            \end{itemize}
            \uncover<6->{This [ɾ], this [t], and this [ʔ] are in free variation}
          \end{block}
        }
      \end{frame}

    \subsection{\subonetwo}
      \begin{frame}{\subonetwo}
        \begin{block}{Are these [p]s the same?}
          \begin{itemize}
            \item {[}ˈ\alert{p\only<2>{ʰ}}ɪt]
            \item {[}ˈs\alert{p}ɪt]
          \end{itemize}
          \uncover<2>{Aspiration is lost when a stop is preceded by a fricative}
        \end{block}
      \end{frame}

    \subsection{\subonethree}
      \begin{frame}{\subonethree}
        \only<-3>{
          \begin{alertblock}{Allophones}
            % Allophones
A set of speech sounds that are perceived as being variants of one meaningful sound

            \begin{itemize}
              \item {[}pʰ] and [p] are allophones of the phoneme /p/
            \end{itemize}
          \end{alertblock}
          \begin{alertblock}<2->{Phoneme}
            % Phoneme
A mental representation of a meaningful speech sound, which may or may not be realized in multiple ways

            \begin{itemize}
              \item /p/ is a phoneme that can be realized as [pʰ] or [p]
            \end{itemize}
          \end{alertblock}
          \begin{alertblock}<3->{Phone}
            % Phone
The technical term for a speech sound

          \end{alertblock}
        }
        \only<4-5>{
          \begin{block}{Phonemes and allophones are language specific}
            \begin{itemize}
              \item {[}pʰ] and [p] are semantically \alert{non-constrastive} in English, but
              \item {[}pʰ] and [p] are semantically \alert{constrastive} in Hindi
            \end{itemize}
            i.e., They're allophones in English but phonemes in Hindi
          \end{block}
          \begin{example}<5->
            \begin{itemize}
              \item {[}pʰəl] `fruit'
              \item {[}pəl] `moment'
            \end{itemize}
          \end{example}
        }
        \only<6->{
          \begin{block}{Summarizing these examples}
            \begin{tabular}{l r}
            Phonemes                  & Allophones \\
            (mental representations)  & (actual acoustic sounds) \\
            \hline \\
            \multicolumn{2}{c}{\emph{English}} \\
            /p/                       & [p, pʰ] \\
            /t/                       & [t, tʰ, ɾ, ʔ] \\
            \multicolumn{2}{c}{\emph{Hindi}} \\
            /t/                       & [t] \\
            /tʰ/                      & [tʰ]
            \end{tabular}
          \end{block}
        }
      \end{frame}

  \section{Identifying them}
    \subsection{\subtwoone}
      \begin{frame}[t]{\subtwoone}
        \begin{block}{}
          \begin{tabular}{r c c}
            1)  & \multicolumn{2}{c}{Choose two phones} \\
            2)  & \multicolumn{2}{c}{Determine distributions} \\
            3)  & Contrastive?            & Complementary? \\
                & $\rightarrow$ Phonemes  & $\rightarrow$ Allophones
          \end{tabular}
        \end{block}
        \only<2>{
          \begin{block}{Distribution types}
            By distribution, we mean phonetic environment (e.g., [p] in [ˈspɪt] is between [s] and [ɪ])
            \begin{itemize}
              \item Contrastive distribution
              \item Complementary distribution
            \end{itemize}
          \end{block}
        }
        \only<3-5>{
          \begin{alertblock}{Contrastive distribution}
            \only<3>{
              % Contrastive distribution
When two speech sounds occur in the same phonetic environment and create a difference in meaning

            }
            \only<4->{
              The [t] and [d] in [ˈtɛɹ] and [ˈdɛɹ]
            }
          \end{alertblock}
          \only<3-4>{
            \begin{alertblock}{Complementary distribution}
                % Complementary distribution
When two speech sounds never occur in the same phonetic environment

              \only<4>{
                The [p] and [pʰ] in [ˈspɪt] and [ˈpʰɪt]
                \begin{itemize}
                  \item Their environments are [s\_ɪ] and [\#\_ɪ]
                \end{itemize}
              }
            \end{alertblock}
          }
          \only<5->{
            \begin{alertblock}{Minimal pair}
              % Minimal pair
A set of words (with different meanings) that differ in only one speech sound

            \end{alertblock}
          }
        }
      \end{frame}

    \subsection{\subtwotwo}
      \begin{frame}{\subtwotwo}
        \begin{columns}
          \column{0.48\textwidth}
            \only<1-2>{
              \begin{block}{Data}
                \begin{tabular}{r l}
                  \lexi{dean}    & [ˈdĩn] \\
                  \lexi{deed}    & [ˈdid] \\
                  \lexi{lean}    & [ˈlĩn] \\
                  \lexi{leap}    & [ˈlip] \\
                  \lexi{mean}    & [ˈmĩn] \\
                  \lexi{mere}    & [ˈmir] \\
                  \lexi{team}    & [ˈtĩm] \\
                  \lexi{seat}    & [ˈsit] \\
                  \lexi{scream}  & [ˈskɹĩm] \\
                  \lexi{see}     & [ˈsi]
                \end{tabular}
              \end{block}
            }
            \only<3-4>{
              \begin{block}{Data}
                \begin{tabular}{r l}
                  \lexi{stop}   & [ˈstɑp] \\
                  \lexi{steal}  & [ˈstil] \\
                  \lexi{stump}  & [ˈstʌmp] \\
                  \lexi{tap}    & [ˈtʰæp] \\
                  \lexi{tool}   & [ˈtʰul] \\
                  \lexi{teal}   & [ˈtʰil]
                \end{tabular}
              \end{block}
            }
            \only<5-6>{
              \begin{block}{Data}
                \begin{tabular}{r l}
                  \lexi{card}   & [ˈkɑɹd] \\
                  \lexi{garner}  & [ˈgɑɹ.nɹ̩] \\
                  \lexi{good}   & [ˈgʊd] \\
                  \lexi{could}  & [ˈkʊd] \\
                  \lexi{cave}   & [ˈkeɪv] \\
                  \lexi{gave}   & [ˈgeɪv] \\
                  \lexi{graze}  & [ˈgɹeɪz] \\
                  \lexi{credit} & [ˈkrɛ.dɪt]
                \end{tabular}
              \end{block}
            }
          \column{0.48\textwidth}
            \only<1-2>{
              \begin{block}{}
                Are [i] and [ĩ] phonemes or allophones?
              \end{block}
              \uncover<2->{
                \begin{block}{Allophones of /i/}
                  They're in complementary distribution
                  \begin{itemize}
                    \item {[}ĩ] before nasals
                    \item {[}i] elsewhere
                  \end{itemize}
                \end{block}
              }
            }
            \only<3-4>{
              \begin{block}{}
                Are [t] and [tʰ] phonemes or allophones?
              \end{block}
              \uncover<4->{
                \begin{block}{Allophones of /t/}
                    They're in complementary distribution
                    \begin{itemize}
                      \item {[}t] after fricatives
                      \item {[}tʰ] elsewhere
                    \end{itemize}
                \end{block}
              }
            }
            \only<5-6>{
              \begin{block}{}
                Are [k] and [g] phonemes or allophones?
              \end{block}
              \uncover<6->{
                \begin{block}{Phonemes /k/ and /g/}
                  They're in contrastive distribution, some of the words even being minimal pairs
                  \begin{itemize}
                    \item \lexi{could}-\lexi{good} and \lexi{cave}-\lexi{gave}
                  \end{itemize}
                \end{block}
              }
            }
        \end{columns}
      \end{frame}

    \subsection{\subtwothree}
      \begin{frame}{\subtwothree}
        \begin{columns}
          \column{0.48\textwidth}
            \begin{block}{Data}
              \begin{itemize}
                \item {[}ˈwɔ.ɾɹ̩]
                \item {[}ˈwʊ.ɾɹ̩]
              \end{itemize}
            \end{block}
            \begin{block}{}
              Are [ɔ] and [ʊ] phonemes or allophones?
            \end{block}
          \column{0.48\textwidth}
            \begin{block}<2->{Neither}
              They're in overlapping distribution, making this ``free variation''
            \end{block}
            \begin{alertblock}<3->{Overlapping distribution}
              % Overlapping distribution
When two speech sounds occur in the same phonetic environment but \emph{don't} create a difference in meaning

            \end{alertblock}
        \end{columns}
      \end{frame}

    \subsection{\subtwofour}
      \begin{frame}{\subtwofour}
        \begin{block}{Try these}
          \textcite{dawson_language_2016}, chapter 3 exercises 9
        \end{block}
      \end{frame}
\end{document}
