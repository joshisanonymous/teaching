%%%%%%%%%%%%%%%%%%%%%%%%%%%%%%%%%%%%%
%                                   %
% Compile with XeLaTeX and biber    %
%                                   %
% Questions or comments:            %
%                                   %
% joshua dot mcneill at uga dot edu %
%                                   %
%%%%%%%%%%%%%%%%%%%%%%%%%%%%%%%%%%%%%

\documentclass{beamer}
  % Read in standard preamble (cosmetic stuff)
  %%%%%%%%%%%%%%%%%%%%%%%%%%%%%%%%%%%%%%%%%%%%%%%%%%%%%%%%%%%%%%%%
% This is a standard preamble used in for all slide documents. %
% It basically contains cosmetic settings.                     %
%                                                              %
% Joshua McNeill                                               %
% joshua dot mcneill at uga dot edu                            %
%%%%%%%%%%%%%%%%%%%%%%%%%%%%%%%%%%%%%%%%%%%%%%%%%%%%%%%%%%%%%%%%

% Beamer settings
% \usetheme{Berkeley}
\usetheme{CambridgeUS}
% \usecolortheme{dove}
% \usecolortheme{rose}
\usecolortheme{seagull}
\usefonttheme{professionalfonts}
\usefonttheme{serif}
\setbeamertemplate{bibliography item}{}

% Packages and settings
\usepackage{fontspec}
  \setmainfont{Charis SIL}
\usepackage{hyperref}
  \hypersetup{colorlinks=true,
              allcolors=blue}
\usepackage{graphicx}
  \graphicspath{{../../figures/}}
\usepackage[normalem]{ulem}
\usepackage{enumerate}

% Document information
\author{M. McNeill}
\title[FREN2001]{Français 2001}
\institute{\url{joshua.mcneill@uga.edu}}
\date{}

%% Custom commands
% Lexical items
\newcommand{\lexi}[1]{\textit{#1}}
% Gloss
\newcommand{\gloss}[1]{`#1'}
\newcommand{\tinygloss}[1]{{\tiny`#1'}}
% Orthographic representations
\newcommand{\orth}[1]{$\langle$#1$\rangle$}
% Utterances (pragmatics)
\newcommand{\uttr}[1]{`#1'}
% Sentences (pragmatics)
\newcommand{\sent}[1]{\textit{#1}}
% Base dir for definitions
\newcommand{\defs}{../definitions}


  % Packages and settings
  \usepackage{extarrows}
  \usepackage[linguistics]{forest}
  \usepackage[backend=biber, style=apa]{biblatex}
    \addbibresource{../references/References.bib}

  % Document information
  \subtitle[Typologies \& Structures]{Morphological Typologies \& Hierarchical Word Structures}

  %% Custom commands
  % Subsection/frame titles
  \newcommand{\suboneone}{The types}
  \newcommand{\subonetwo}{Analytic languages}
  \newcommand{\subonethree}{Agglutinating languages}
  \newcommand{\subonefour}{Fusional languages}
  \newcommand{\subonefive}{Polysynthetic languages}
  \newcommand{\subonesix}{And English?}
  \newcommand{\subtwoone}{Let's talk about order}
  \newcommand{\subtwotwo}{Ambiguity}
  \newcommand{\subtwothree}{Practice}

\begin{document}
  % Read in the standard intro slides (title page and table of contents)
  %%%%%%%%%%%%%%%%%%%%%%%%%%%%%%%%%%%%%%%%%%%%%%%%%%%%%%%%%%%%%%%%
% This is a standard set of intro slides used in for all slide %
% documents. It basically contains the title page and table of %
% contents.                                                    %
%                                                              %
% Joshua McNeill                                               %
% joshua dot mcneill at uga dot edu                            %
%%%%%%%%%%%%%%%%%%%%%%%%%%%%%%%%%%%%%%%%%%%%%%%%%%%%%%%%%%%%%%%%

\begin{frame}
  \titlepage
  \tiny{Office: % Basically a variable for office hours location
Gilbert 121\\
        Office hours: % Basically a variable for office hours
 lundi, mercredi, vendredi 10:10--11:10
}
\end{frame}

\begin{frame}
  \tableofcontents[hideallsubsections]
\end{frame}

\AtBeginSection[]{
  \begin{frame}
    \tableofcontents[currentsection,
                     hideallsubsections]
  \end{frame}
}


  \section{Morphological Typology}
    \subsection{\suboneone}
      \begin{frame}{\suboneone}
        \begin{block}{}
          Languages are often classified into types by their word formation processes or lack thereof
        \end{block}
        \begin{alertblock}{This is merely a convention}
          Languages often do not fit neatly into any one type
        \end{alertblock}

        \vspace{1cm}
        \begin{tabular}{c | c c c}
          Analytic  &               & Synthetic & \\
          \multicolumn{4}{c}{$\xleftrightarrow{\hspace{0.9\linewidth}}$} \\
                    & Agglutinating & Fusional  & Polysynthetic
        \end{tabular}
      \end{frame}

    \subsection{\subonetwo}
      \begin{frame}{\subonetwo}
        \begin{alertblock}{Analytic languages}
          % Analytic language
A language type that rarely uses word formation processes and instead relies on word order

        \end{alertblock}
        \begin{example}
          Mandarin:
          \begin{itemize}
            \item \begin{tabular}{r @{} l l l l l @{} l}
                    [  & wɔ  & mən             & tan   & tçin  & lə             & ] \\
                       & I   & \textsc{plural} & play  & piano & \textsc{past}  & \\
                    \multicolumn{6}{l}{`We played the piano.'}
                  \end{tabular}
          \end{itemize}
        \end{example}
      \end{frame}

    \subsection{\subonethree}
      \begin{frame}{\subonethree}
        \begin{alertblock}{Agglutinating languages}
          % Agglutinating language
A language type that uses word formation processes and in which bound morphemes typically carry only one meaning

        \end{alertblock}
        \begin{example}
          Swahili:
          \begin{itemize}
            \item \begin{tabular}{r @{} l l l @{} l}
                    [  & ni-  & li-             & soma  & ] \\
                       & I-   & \textsc{past-}  & read  & \\
                    \multicolumn{5}{l}{`I was reading.'}
                  \end{tabular}
          \end{itemize}
        \end{example}
      \end{frame}

    \subsection{\subonefour}
      \begin{frame}{\subonefour}
        \begin{alertblock}{Fusional languages}
          % Fusional language
A language type that uses word formation processes and in which bound morphemes typically carry more than one meaning

        \end{alertblock}
        \begin{example}
          Spanish:
          \begin{itemize}
            \item \begin{tabular}{r @{} l l @{} l}
                    [  & abl    & -e  & ] \\
                       & speak  & \textsc{-past.1sg} & \\
                    \multicolumn{4}{l}{`I spoke.'}
                  \end{tabular}
          \end{itemize}
        \end{example}
      \end{frame}

    \subsection{\subonefive}
      \begin{frame}{\subonefive}
        \begin{alertblock}{Polysynthetic languages}
          % Polysynthetic language
A language type that uses word formation processes extensively, to the point at which whole sentences may be expressed with single words

        \end{alertblock}
        \begin{example}
          Sora:
          \begin{itemize}
            \item \begin{tabular}{r @{} l l l l l @{} l}
                    [  & pɔ   & -poʊŋ   & -koʊn   & -t                & -am & ] \\
                       & stab & -belly  & -knife  & \textsc{non-past} & you & \\
                    \multicolumn{7}{l}{`(Someone) will stab you with a knife in your belly.'}
                  \end{tabular}
          \end{itemize}
        \end{example}
      \end{frame}

    \subsection{\subonesix}
      \begin{frame}{\subonesix}
        \begin{block}{Which language type is English?}
          \uncover<2->{
            Kind of analytic, kind of agglutinating
          }
        \end{block}
      \end{frame}

  \section{Hierarchical Word Structure}
    \subsection{\subtwoone}
      \begin{frame}[t]{\subtwoone}
        \only<-8>{
          \begin{block}{}
            There are three possible ways to derive \lexi{un-us-able}:
            \begin{itemize}
              \item Attach \lexi{un-} and \lexi{-able} to \lexi{use} at the same time
              \item Attach \lexi{un-} first then \lexi{-able}
              \item Attach \lexi{-able} first then \lexi{un-}
            \end{itemize}
          \end{block}
        }
        \only<2-4>{
          \begin{block}{All at once then?}
            \uncover<3->{
              Possible in this case, but what about words like \lexi{in-depend-ent-ly}?
            }
          \end{block}
          \begin{alertblock}<4->{}
            In general, let's assume that affixes attach at different times
          \end{alertblock}
        }
        \only<5-6>{
          \begin{block}{\lexi{un-} before \lexi{-able}?}
            \uncover<6->{
              Probably not: \lexi{unuse} doesn't make much sense
            }
          \end{block}
        }
        \only<7-8>{
          \begin{block}{\lexi{-able} before \lexi{un-}?}
            \uncover<8->{
              Seems most likely
            }
          \end{block}
        }
        \only<9>{
          \begin{block}{}
            We can represent this derivation with a tree diagram
          \end{block}
          \begin{center}
            \begin{forest}
              for tree={calign=fixed angles, parent anchor=south}
              [Adj
                [un, tier=morph]
                [Adj
                  [use (V), tier=morph]
                  [able, tier=morph]
                ]
              ]
            \end{forest}
          \end{center}
          \begin{alertblock}{}
            The relationship between the affixes is therefore hierarchical
          \end{alertblock}
        }
        \only<10-11>{
          \begin{block}{Order is often not clear, though}
            Take \lexi{re-us-able}
            \begin{itemize}
              \item \lexi{Re-use} works
              \item \lexi{Us-able} works
            \end{itemize}
          \end{block}
          \begin{block}<11->{}
            In these cases, for this class, both orders are acceptable
          \end{block}
        }
      \end{frame}

    \subsection{\subtwotwo}
      \begin{frame}{\subtwotwo}
        \begin{block}{}
          These diagrams can sometimes shed light on ambiguous words: e.g., \lexi{unlockable}
          \begin{itemize}
            \item `Unable to be locked' or
            \item `Able to be unlocked'
          \end{itemize}
        \end{block}
        \begin{columns}
          \column{0.48\textwidth}
            \begin{forest}
              for tree={calign=fixed angles, parent anchor=south}
              [Adj
                [un, tier=morph]
                [Adj
                  [lock (V), tier=morph]
                  [able, tier=morph]
                ]
              ]
            \end{forest}
          \column{0.48\textwidth}
            \begin{forest}
              for tree={calign=fixed angles, parent anchor=south}
              [Adj
                [V
                  [un, tier=morph]
                  [lock (V), tier=morph]
                ]
                [able, tier=morph]
              ]
            \end{forest}
        \end{columns}
      \end{frame}

    \subsection{\subtwothree}
      \begin{frame}{\subtwothree}
        \begin{block}{Try these}
          \textcite{dawson_language_2016}, chapter 4 exercises 22, 23, and 24
        \end{block}
      \end{frame}
\end{document}
