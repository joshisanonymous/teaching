%%%%%%%%%%%%%%%%%%%%%%%%%%%%%%%%%%%%%
%                                   %
% Compile with XeLaTeX and biber    %
%                                   %
% Questions or comments:            %
%                                   %
% joshua dot mcneill at uga dot edu %
%                                   %
%%%%%%%%%%%%%%%%%%%%%%%%%%%%%%%%%%%%%

\documentclass{beamer}
  % Read in standard preamble (cosmetic stuff)
  %%%%%%%%%%%%%%%%%%%%%%%%%%%%%%%%%%%%%%%%%%%%%%%%%%%%%%%%%%%%%%%%
% This is a standard preamble used in for all slide documents. %
% It basically contains cosmetic settings.                     %
%                                                              %
% Joshua McNeill                                               %
% joshua dot mcneill at uga dot edu                            %
%%%%%%%%%%%%%%%%%%%%%%%%%%%%%%%%%%%%%%%%%%%%%%%%%%%%%%%%%%%%%%%%

% Beamer settings
% \usetheme{Berkeley}
\usetheme{CambridgeUS}
% \usecolortheme{dove}
% \usecolortheme{rose}
\usecolortheme{seagull}
\usefonttheme{professionalfonts}
\usefonttheme{serif}
\setbeamertemplate{bibliography item}{}

% Packages and settings
\usepackage{fontspec}
  \setmainfont{Charis SIL}
\usepackage{hyperref}
  \hypersetup{colorlinks=true,
              allcolors=blue}
\usepackage{graphicx}
  \graphicspath{{../../figures/}}
\usepackage[normalem]{ulem}
\usepackage{enumerate}

% Document information
\author{M. McNeill}
\title[FREN2001]{Français 2001}
\institute{\url{joshua.mcneill@uga.edu}}
\date{}

%% Custom commands
% Lexical items
\newcommand{\lexi}[1]{\textit{#1}}
% Gloss
\newcommand{\gloss}[1]{`#1'}
\newcommand{\tinygloss}[1]{{\tiny`#1'}}
% Orthographic representations
\newcommand{\orth}[1]{$\langle$#1$\rangle$}
% Utterances (pragmatics)
\newcommand{\uttr}[1]{`#1'}
% Sentences (pragmatics)
\newcommand{\sent}[1]{\textit{#1}}
% Base dir for definitions
\newcommand{\defs}{../definitions}


  % Packages and settings
  \usepackage{tikz}
  \usepackage{adjustbox}
  \usepackage[backend=biber, style=apa]{biblatex}
    \addbibresource{../references/References.bib}

  % Document information
  \subtitle[Vowel Articulation]{Articulation of English Vowels}

  %% Custom commands
  % Subsection/frame titles
  \newcommand{\suboneone}{Not consonants}
  \newcommand{\subtwoone}{Properties}
  \newcommand{\subtwotwo}{Resources and practice}

\begin{document}
  % Read in the standard intro slides (title page and table of contents)
  %%%%%%%%%%%%%%%%%%%%%%%%%%%%%%%%%%%%%%%%%%%%%%%%%%%%%%%%%%%%%%%%
% This is a standard set of intro slides used in for all slide %
% documents. It basically contains the title page and table of %
% contents.                                                    %
%                                                              %
% Joshua McNeill                                               %
% joshua dot mcneill at uga dot edu                            %
%%%%%%%%%%%%%%%%%%%%%%%%%%%%%%%%%%%%%%%%%%%%%%%%%%%%%%%%%%%%%%%%

\begin{frame}
  \titlepage
  \tiny{Office: % Basically a variable for office hours location
Gilbert 121\\
        Office hours: % Basically a variable for office hours
 lundi, mercredi, vendredi 10:10--11:10
}
\end{frame}

\begin{frame}
  \tableofcontents[hideallsubsections]
\end{frame}

\AtBeginSection[]{
  \begin{frame}
    \tableofcontents[currentsection,
                     hideallsubsections]
  \end{frame}
}


  \section{What are vowels}
    \subsection{\suboneone}
      \begin{frame}{\suboneone}
        \begin{block}{How do we describe consonants?}
          \begin{itemize}
            \item<2-3> Voicing \uncover<3->{$\times$ Always voiced}
            \item<2-3> Place of articulation \uncover<3->{$\times$ No place}
            \item<2-3> Manner of articulation \uncover<3->{$\times$ No manner}
          \end{itemize}
          \uncover<3>{These don't work for vowels}
        \end{block}
      \end{frame}

  \section{Articulation}
    \subsection{\subtwoone}
      \begin{frame}{\subtwoone}
        \begin{columns}
          \column{0.5\textwidth}
            \begin{minipage}[c][0.6\textheight]{\linewidth}
              \only<1>{
                \begin{block}{How to describe vowels}
                  \begin{itemize}
                    \item Tongue height
                    \item Tongue advancement
                    \item Lip rounding
                    \item Tenseness
                  \end{itemize}
                \end{block}
              }
              \only<2>{
                \begin{block}{Tongue height}
                  \begin{itemize}
                    \item high or close $\rightarrow$ [i ɪ u ʊ]
                    \item high-mid or close-mid $\rightarrow$ [e o]
                    \item mid $\rightarrow$ [ə]
                    \item low-mid or open-mid $\rightarrow$ [ɛ ʌ ɔ]
                    \item low or open $\rightarrow$ [æ a ɑ]
                  \end{itemize}
                \end{block}
              }
              \only<3>{
                \begin{block}<3->{Not quite there}
                  \begin{itemize}
                    \item \emph{near} high or \emph{near} close $\rightarrow$ [ɪ ʊ]
                    \item \emph{near} low or \emph{near} open $\rightarrow$ [æ]
                  \end{itemize}
                \end{block}
              }
              \only<4>{
                \begin{block}{Tongue advancement}
                  \begin{itemize}
                    \item front $\rightarrow$ [i ɪ e ɛ æ a]
                    \item central $\rightarrow$ [ə]
                    \item back $\rightarrow$ [u ʊ o ʌ ɔ ɑ]
                  \end{itemize}
                \end{block}
              }
              \only<5>{
                \begin{block}{Lip rounding}
                  \begin{itemize}
                    \item rounded $\rightarrow$ [u ʊ o ɔ]
                    \begin{itemize}
                      \item To the right of the dots
                    \end{itemize}
                    \item unrounded $\rightarrow$ [i ɪ e ɛ æ a ə ʌ ɑ]
                    \begin{itemize}
                      \item To the left of the dots
                    \end{itemize}
                  \end{itemize}
                \end{block}
              }
              \only<6>{
                \begin{block}{Tenseness}
                  \begin{itemize}
                    \item tense $\rightarrow$ [i e a u o ɔ ɑ]
                    \item lax $\rightarrow$ [ɪ ɛ æ ə ʊ ʌ]
                  \end{itemize}
                \end{block}
              }
              \only<7-8>{
                \begin{block}{Diphthongs}
                  \uncover<8>{
                    % Diphthong
More than one vowel in a syllable

                    \begin{itemize}
                      \item {[}eɪ oʊ ɔɪ aɪ aʊ{]}
                      \item {[}e o{]} and [a] only ever appear in diphthongs
                    \end{itemize}
                  }
                \end{block}
              }
              \only<9-10>{
                \begin{block}{Schwas everywhere!}
                  [ə] is found exclusively in \emph{unstressed} syllables and almost all unstressed syllables contain [ə]
                \end{block}
                \begin{alertblock}<10->{}
                  [ʌ] is the transcription when what otherwise sounds like a schwa is found in a \emph{stressed} syllable
                \end{alertblock}
              }
            \end{minipage}
          \column{0.5\textwidth}
            \begin{adjustbox}{height=4.5cm}
              % Read in vowel chart
              %%%%%%%%%%%%%%%%%%%%%%%%%%%%%%%%%%%%%%%%%%%%%%%%%%%%%%%%%%%%%%%%%%%%%%%%%%%%%%%%%%%%%
% This creates an English IPA chart for vowels                                      %
%                                                                                   %
% Compiled from material_IPA_en_chart.tex when a                                    %
% standalone document is needed                                                     %
%                                                                                   %
% Code is only slightly modified from:                                              %
%   https://tex.stackexchange.com/questions/156955/tikz-pgf-linguistics-vowel-chart %
%                                                                                   %
% -Joshua McNeill (joshua dot mcneill at uga dot edu)                               %
%%%%%%%%%%%%%%%%%%%%%%%%%%%%%%%%%%%%%%%%%%%%%%%%%%%%%%%%%%%%%%%%%%%%%%%%%%%%%%%%%%%%%

% Custom command
\def\V(#1,#2){barycentric cs:hf={(3-#1)*(2-#2)},hb={(3-#1)*#2},lf={#1*(2-#2)},lb={#1*#2}}

% Chart
\begin{tikzpicture}[scale=3]
  \large
  \tikzset{
    vowel/.style={fill=white, anchor=mid, text depth=0ex, text height=1ex},
    dot/.style={circle,fill=black,minimum size=0.4ex,inner sep=0pt,outer sep=-1pt},
  }
  \coordinate (hf) at (0,2); % high front
  \coordinate (hb) at (2,2); % high back
  \coordinate (lf) at (1,0); % low front
  \coordinate (lb) at (2,0); % low back
  \def\V(#1,#2){barycentric cs:hf={(3-#1)*(2-#2)},hb={(3-#1)*#2},lf={#1*(2-#2)},lb={#1*#2}}

  % Draw the horizontal lines first.
  \draw (\V(0,0)) -- (\V(0,2));
  \draw (\V(1,0)) -- (\V(1,2));
  \draw (\V(2,0)) -- (\V(2,2));
  \draw (\V(3,0)) -- (\V(3,2));

  % Place all the unrounded-rounded pairs next, on top of the horizontal lines.
  \path (\V(0,0))     node[vowel, left] {i}          node[vowel, right] { }          node[dot] {};
  \path (\V(0,1))     node[vowel, left] { }          node[vowel, right] { }          node[dot] {};
  \path (\V(0,2))     node[vowel, left] { }          node[vowel, right] {u}          node[dot] {};
  \path (\V(0.5,0.4)) node[vowel, left] {\textbf{ɪ}} node[vowel, right] { }          node[   ] {};
  \path (\V(0.5,1.6)) node[vowel, left] { }          node[vowel, right] {\textbf{ʊ}} node[   ] {};
  \path (\V(1,0))     node[vowel, left] {e}          node[vowel, right] { }          node[dot] {};
  \path (\V(1,1))     node[vowel, left] { }          node[vowel, right] { }          node[dot] {};
  \path (\V(1,2))     node[vowel, left] { }          node[vowel, right] {o}          node[dot] {};
  \path (\V(2,0))     node[vowel, left] {\textbf{ɛ}} node[vowel, right] { }          node[dot] {};
  \path (\V(2,1))     node[vowel, left] { }          node[vowel, right] { }          node[dot] {};
  \path (\V(2,2))     node[vowel, left] {\textbf{ʌ}} node[vowel, right] {ɔ}          node[dot] {};
  \path (\V(2.5,0))   node[vowel, left] {\textbf{æ}} node[vowel, right] { }          node[   ] {};
  \path (\V(3,0))     node[vowel, left] {a}          node[vowel, right] { }          node[dot] {};
  \path (\V(3,2))     node[vowel, left] {ɑ}          node[vowel, right] { }          node[dot] {};

  % Draw the vertical lines.
  \draw (\V(0,0)) -- (\V(3,0));
  \draw (\V(0,1)) -- (\V(3,1));
  \draw (\V(0,2)) -- (\V(3,2));

  % Place the unpaired symbols last, on top of the vertical lines.
  \path (\V(1.5,1))   node[vowel]       {\textbf{ə}};
  \path (\V(-0.25,0)) node[vowel]       {front};
  \path (\V(-0.25,1)) node[vowel]       {central};
  \path (\V(-0.25,2)) node[vowel]       {back};
  \path (\V(0,-0.5))  node[vowel]       {high};
  \path (\V(1,-0.8))  node[vowel]       {high-mid};
  \path (\V(2,-1.55)) node[vowel]       {low-mid};
  \path (\V(3,-2.95)) node[vowel]       {low};
\end{tikzpicture}

            \end{adjustbox}
        \end{columns}
      \end{frame}

    \subsection{\subtwotwo}
      \begin{frame}{\subtwotwo}
        \begin{block}{}
          % A set of links to useful resources when dealing articulatory phonetics
\begin{itemize}
  \item To hear these sounds: \url{http://web.uvic.ca/ling/resources/ipa/charts/IPAlab/IPAlab.htm} and \url{https://americanipachart.com/}
  \item To type these symbols: \url{https://ipa.typeit.org/}
\end{itemize}

        \end{block}
        \begin{block}{Try these}
          \textcite{dawson_language_2016}, chapter 2 exercises 13, 14, 15, 16, and 19
        \end{block}
      \end{frame}
\end{document}
