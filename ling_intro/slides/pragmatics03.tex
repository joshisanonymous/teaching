%%%%%%%%%%%%%%%%%%%%%%%%%%%%%%%%%%%%%
%                                   %
% Compile with XeLaTeX and biber    %
%                                   %
% Questions or comments:            %
%                                   %
% joshua dot mcneill at uga dot edu %
%                                   %
%%%%%%%%%%%%%%%%%%%%%%%%%%%%%%%%%%%%%

\documentclass{beamer}
  % Read in standard preamble (cosmetic stuff)
  %%%%%%%%%%%%%%%%%%%%%%%%%%%%%%%%%%%%%%%%%%%%%%%%%%%%%%%%%%%%%%%%
% This is a standard preamble used in for all slide documents. %
% It basically contains cosmetic settings.                     %
%                                                              %
% Joshua McNeill                                               %
% joshua dot mcneill at uga dot edu                            %
%%%%%%%%%%%%%%%%%%%%%%%%%%%%%%%%%%%%%%%%%%%%%%%%%%%%%%%%%%%%%%%%

% Beamer settings
% \usetheme{Berkeley}
\usetheme{CambridgeUS}
% \usecolortheme{dove}
% \usecolortheme{rose}
\usecolortheme{seagull}
\usefonttheme{professionalfonts}
\usefonttheme{serif}
\setbeamertemplate{bibliography item}{}

% Packages and settings
\usepackage{fontspec}
  \setmainfont{Charis SIL}
\usepackage{hyperref}
  \hypersetup{colorlinks=true,
              allcolors=blue}
\usepackage{graphicx}
  \graphicspath{{../../figures/}}
\usepackage{soul}
  \setstcolor{red}
\usepackage[normalem]{ulem}
\usepackage{enumerate}
\usepackage{tikz}
  \usetikzlibrary{trees}

% Document information
\author{M. McNeill}
\title[FREN1001]{Français 1001}
\institute{\url{joshua.mcneill@uga.edu}}
\date{}

%% Custom commands
% Lexical items
\newcommand{\lexi}[1]{\textit{#1}}
% Gloss
\newcommand{\gloss}[1]{`#1'}
\newcommand{\tinygloss}[1]{{\tiny`#1'}}
% Orthographic representations
\newcommand{\orth}[1]{$\langle$#1$\rangle$}
% Utterances (pragmatics)
\newcommand{\uttr}[1]{`#1'}
% Sentences (pragmatics)
\newcommand{\sent}[1]{\textit{#1}}
% Fixed length underlines
\newcommand{\funderline}[2][4cm]{
  \underline{\makebox[\ifdim\width>#1\width\else#1\fi]{#2}}
}
% Base dir for definitions
\newcommand{\defs}{../definitions}
\newcommand{\activity}[1]{
  \input{./activities/#1.tex}
}


  % Packages and settings
  \usepackage[backend=biber, style=apa]{biblatex}
    \addbibresource{../references/References.bib}

  % Document information
  \subtitle[Presupposition]{Presupposition}

  %% Custom commands
  % Subsection/frame titles
  \newcommand{\suboneone}{What is it?}
  \newcommand{\subonetwo}{Relation to truth values}
  \newcommand{\subonethree}{Presupposition triggers}
  \newcommand{\subonefour}{Presupposition accommodation}
  \newcommand{\subonefive}{Practice}
  % Existence presupposition examples
  \newcommand{\existsampone}{\uttr{The Bvryzax River runs through Europe.}}
  \newcommand{\existsamptwo}{\uttr{The monster under my bed has fangs.}}
  \newcommand{\existsampthree}{\uttr{The Niche has good pizza.}}

\begin{document}
  % Read in the standard intro slides (title page and table of contents)
  %%%%%%%%%%%%%%%%%%%%%%%%%%%%%%%%%%%%%%%%%%%%%%%%%%%%%%%%%%%%%%%%
% This is a standard set of intro slides used in for all slide %
% documents. It basically contains the title page and table of %
% contents.                                                    %
%                                                              %
% Joshua McNeill                                               %
% joshua dot mcneill at uga dot edu                            %
%%%%%%%%%%%%%%%%%%%%%%%%%%%%%%%%%%%%%%%%%%%%%%%%%%%%%%%%%%%%%%%%

\begin{frame}
  \titlepage
  \tiny{Office: % Basically a variable for office hours location
T Gilbert 141/W Library 4th Fl
\\
        Office hours: % Basically a variable for office hours
T 11-12/W 11-12:30
}
\end{frame}

\begin{frame}
  \tableofcontents[hideallsubsections]
\end{frame}

\AtBeginSection[]{
  \begin{frame}
    \tableofcontents[currentsection,
                     hideallsubsections]
  \end{frame}
}


  \section{Presupposition}
    \subsection{\suboneone}
      \begin{frame}{\suboneone}
        \begin{block}{If I say}
          \begin{enumerate}
            \item \existsampone
            \item \existsamptwo
          \end{enumerate}
        \end{block}
        \begin{block}{Would these be reasonable questions?}
          \begin{itemize}
            \item Where is the Bvryzax River actually located?
            \item Does the monster really have fangs?
          \end{itemize}
          \uncover<2->{
            Not really. The utterances presume facts that aren't accurate, and these responses accept those facts.
          }
        \end{block}
      \end{frame}

      \begin{frame}{\suboneone}
        \begin{alertblock}{Presupposition}
          % Presupposition
An underlying assumption about the world that must be satisfied in order for an utterance to make sense or be debatable

        \end{alertblock}
        \begin{block}{In order to be satisfied}
          All participants must believe or behave as if they believe that the presupposed information is true
        \end{block}
      \end{frame}

      \begin{frame}{\suboneone}
        \begin{block}{These were existence presuppositions}
          \begin{enumerate}
            \item \existsampone
            \item \existsamptwo
          \end{enumerate}
        \end{block}
        \begin{alertblock}{Existence presupposition}
          % Existence presupposition
A presupposition in which the assumption is that something in the utterance exists

        \end{alertblock}
      \end{frame}

      \begin{frame}{\suboneone}
        \begin{block}{}
          We're still talking about existence presuppositions even when a thing exist
        \end{block}
        \begin{example}
          \begin{enumerate}
            \item \#\existsampone
            \item \existsampthree
          \end{enumerate}
        \end{example}
      \end{frame}

    \subsection{\subonetwo}
      \begin{frame}{\subonetwo}
        \begin{block}{Test for an unsatisfied presupposition}
          If the truth value of the negation is still false
          \begin{itemize}
            \item[$\rightarrow$] Unsatisfied presupposition
          \end{itemize}
        \end{block}
        \begin{block}{What are the truth values here?}
          \begin{enumerate}
            \item \uttr{The Bvryzax River reaches a depth of over 25 meters.}
            \item \uttr{The Bvryzax River does not reach a depth of over 25 meters.}
            \item \uttr{The Niche does not have good pizza.}
            \item \existsampthree
          \end{enumerate}
        \end{block}
      \end{frame}

    \subsection{\subonethree}
      \begin{frame}[t]{\subonethree}
        \begin{alertblock}{Presupposition trigger}
          % Presupposition trigger
A word or phrase whose presence in an utterance often indicates the presence of a presupposition

        \end{alertblock}
        \only<-4>{
          \begin{block}{What are the main triggers in these utterances?}
            \begin{enumerate}
              \item \uttr{\alert<2->{The monster} under my bed has fangs.}
              \item \uttr{\alert<2->{The Niche} has good pizza.}
            \end{enumerate}
          \end{block}
          \begin{block}<3->{Triggers can also be Vs}
            \begin{enumerate}
              \setcounter{enumi}{2}
              \item \uttr{He \alert<4->{returned} to his hometown last year.}
              \item \uttr{She \alert<4->{stopped} writing books.}
            \end{enumerate}
          \end{block}
        }
        \only<5-8>{
          \begin{block}{As well as Ps}
            \begin{enumerate}
              \item \uttr{\alert<6->{After} the 54th US state was added, the flag was changed.}
            \end{enumerate}
          \end{block}
          \begin{block}<7->{And comparatives and even pitch accents}
            \begin{enumerate}
              \setcounter{enumi}{2}
              \item \uttr{Carol is a \alert<8->{better} linguist than Barbara.}
              \item \uttr{John didn't compete in the \alert<8->{↗Olympics}.}
            \end{enumerate}
          \end{block}
        }
      \end{frame}

    \subsection{\subonefour}
      \begin{frame}{\subonefour}
        \begin{block}{How would you react to the following utterances?}
          \begin{enumerate}
            \item \uttr{Sorry I'm late, my giraffe is sick.}
            \item \uttr{Sorry I'm late, my dog is sick.}
          \end{enumerate}
          \uncover<2->{
            What if I have neither a giraffe nor a dog?
          }
        \end{block}
        \begin{alertblock}<3->{Presupposition accommodation}
          % Presupposition accommodation
The process of assuming that a presupposition is true without knowing for sure

        \end{alertblock}
      \end{frame}

      \begin{frame}{\subonefour}
        \begin{block}{}
          We reserve accommodation for plausible and accessible presuppositions
        \end{block}
        \begin{block}{Is all relevant information accessible here?}
          \uttr{I went to the library, \alert<2->{too}.}
          \begin{itemize}
            \item<2-> Where else did I go?
            \item<2-> How can you accommodate the presupposition that I went there without knowing where there is?
          \end{itemize}
        \end{block}
      \end{frame}

    \subsection{\subonefive}
      \begin{frame}{\subonefive}
        \begin{block}{Try these}
          \textcite{dawson_language_2016}, chapter 7 exercises 41 and 43
        \end{block}
      \end{frame}
\end{document}
