%%%%%%%%%%%%%%%%%%%%%%%%%%%%%%%%%%%%%
%                                   %
% Compile with XeLaTeX and biber    %
%                                   %
% Questions or comments:            %
%                                   %
% joshua dot mcneill at uga dot edu %
%                                   %
%%%%%%%%%%%%%%%%%%%%%%%%%%%%%%%%%%%%%

\documentclass{beamer}
  % Read in standard preamble (cosmetic stuff)
  %%%%%%%%%%%%%%%%%%%%%%%%%%%%%%%%%%%%%%%%%%%%%%%%%%%%%%%%%%%%%%%%
% This is a standard preamble used in for all slide documents. %
% It basically contains cosmetic settings.                     %
%                                                              %
% Joshua McNeill                                               %
% joshua dot mcneill at uga dot edu                            %
%%%%%%%%%%%%%%%%%%%%%%%%%%%%%%%%%%%%%%%%%%%%%%%%%%%%%%%%%%%%%%%%

% Beamer settings
% \usetheme{Berkeley}
\usetheme{CambridgeUS}
% \usecolortheme{dove}
% \usecolortheme{rose}
\usecolortheme{seagull}
\usefonttheme{professionalfonts}
\usefonttheme{serif}
\setbeamertemplate{bibliography item}{}

% Packages and settings
\usepackage{fontspec}
  \setmainfont{Charis SIL}
\usepackage{hyperref}
  \hypersetup{colorlinks=true,
              allcolors=blue}
\usepackage{graphicx}
  \graphicspath{{../../figures/}}
\usepackage[normalem]{ulem}
\usepackage{enumerate}

% Document information
\author{M. McNeill}
\title[FREN2001]{Français 2001}
\institute{\url{joshua.mcneill@uga.edu}}
\date{}

%% Custom commands
% Lexical items
\newcommand{\lexi}[1]{\textit{#1}}
% Gloss
\newcommand{\gloss}[1]{`#1'}
\newcommand{\tinygloss}[1]{{\tiny`#1'}}
% Orthographic representations
\newcommand{\orth}[1]{$\langle$#1$\rangle$}
% Utterances (pragmatics)
\newcommand{\uttr}[1]{`#1'}
% Sentences (pragmatics)
\newcommand{\sent}[1]{\textit{#1}}
% Base dir for definitions
\newcommand{\defs}{../definitions}


  % Packages and settings
  \usepackage[backend=biber, style=apa]{biblatex}
    \addbibresource{../references/References.bib}

  % Document information
  \subtitle[Syntax Basics]{Some Basics of Syntax}

  %% Custom commands
  % Subsection/frame titles
  \newcommand{\suboneone}{It's this}
  \newcommand{\subtwoone}{Grammaticality}
  \newcommand{\subtwotwo}{Semantics in syntax}
  \newcommand{\subtwothree}{Practice}

\begin{document}
  % Read in the standard intro slides (title page and table of contents)
  %%%%%%%%%%%%%%%%%%%%%%%%%%%%%%%%%%%%%%%%%%%%%%%%%%%%%%%%%%%%%%%%
% This is a standard set of intro slides used in for all slide %
% documents. It basically contains the title page and table of %
% contents.                                                    %
%                                                              %
% Joshua McNeill                                               %
% joshua dot mcneill at uga dot edu                            %
%%%%%%%%%%%%%%%%%%%%%%%%%%%%%%%%%%%%%%%%%%%%%%%%%%%%%%%%%%%%%%%%

\begin{frame}
  \titlepage
  \tiny{Office: % Basically a variable for office hours location
Gilbert 121\\
        Office hours: % Basically a variable for office hours
 lundi, mercredi, vendredi 10:10--11:10
}
\end{frame}

\begin{frame}
  \tableofcontents[hideallsubsections]
\end{frame}

\AtBeginSection[]{
  \begin{frame}
    \tableofcontents[currentsection,
                     hideallsubsections]
  \end{frame}
}


  \section{What is syntax?}
    \subsection{\suboneone}
      \begin{frame}{\suboneone}
        \begin{definition}
          % Syntax
The study of sentence structure

        \end{definition}
        \begin{alertblock}<2->{}
          We're still talking about \emph{a} language: i.e., % A language
A systematic set of words and rules that can be used for human communication \emph{and that is unique to each individual}

          \begin{itemize}
            \item We'll use ``English'' as shorthand for ``the set of features that happen to be shared in the languages of some group of people''
          \end{itemize}
        \end{alertblock}
        \begin{alertblock}<3->{}
          We'll be learning a phrase structure \parencite{chomsky_syntactic_2002} theory of syntax
        \end{alertblock}
      \end{frame}

  \section{Basic ideas}
    \subsection{\subtwoone}
      \begin{frame}[t]{\subtwoone}
        \begin{example}
          \begin{enumerate}
            \item The dog likes to eat filet mignon.
            \item *Filet to dog eat mignon the likes.
          \end{enumerate}
        \end{example}
        \only<1-2>{
          \begin{block}{What's wrong with (2)?}
            \uncover<2->{
            The linguistic expressions are not combined in accordance with their syntactic properties
            }
          \end{block}
        }
        \only<3>{
          \begin{alertblock}{Linguistic expression}
            % Linguistic expression
A piece of language with a certain form, a certain meaning, and certain syntactic properties

          \end{alertblock}
        }
        \only<4>{
          \begin{block}{Two types of linguistic expressions}
            \begin{itemize}
              \item \alert{Lexical expression}: % Lexical expression
A linguistic expression that is part of one's mental lexicon

              \item \alert{Phrasal expression}: % Phrasal expression
A linguistic expression that is the combination of smaller linguistic expressions

            \end{itemize}
          \end{block}
        }
        \only<5-6>{
          \begin{alertblock}{A grammatical sentence}
            % Grammatical sentence
A sentence that conforms to the syntactic properties of its parts

          \end{alertblock}
          \begin{alertblock}<6->{Grammaticality judgment}
            % Grammaticality judgment
A speaker's intuitional decision about whether a sentence is grammatical in their language

          \end{alertblock}
        }
      \end{frame}

    \subsection{\subtwotwo}
      \begin{frame}[t]{\subtwotwo}
        \begin{block}{}
          Some links exist between syntax and meaning, but grammaticality is unrelated to meaning
        \end{block}
        \only<2-3>{
          \begin{example}
            \begin{enumerate}
              \item The dog likes cats.
              \item Cats like the dog.
            \end{enumerate}
          \end{example}
          \only<2>{
            \begin{block}{Syntax can affect meaning}
              Flipping the subject and object yields a different meaning
            \end{block}
          }
          \only<3>{
            \begin{alertblock}<2->{Principle of compositionality (Frege, late 19th century)}
              % Principle of compositionality
The meaning of an expression is a function of the meanings and syntactic properties of the expressions that make it up

            \end{alertblock}
          }
        }
        \only<4>{
          \begin{example}
            \begin{enumerate}
              \item Colorless green ideas sleep furiously.
              \item *Green sleep colorless furiously ideas. \parencite{chomsky_syntactic_2002}
            \end{enumerate}
          \end{example}
          \begin{block}{Grammaticality can be independent of meaning}
            \begin{itemize}
              \item (1) is nonsense but grammatical
              \item (2) is nonsense and ungrammatical
            \end{itemize}
          \end{block}
        }
        \only<5-6>{
          \begin{example}
            From a language learner:
            \begin{enumerate}
              \item (*)Me bought dog.
            \end{enumerate}
          \end{example}
          \begin{block}{}
            Understandable but not grammatical \emph{in your language}
          \end{block}
          \only<6>{
            \begin{alertblock}{However}
              This is grammatical \emph{in the learner's language}
            \end{alertblock}
          }
        }
        \only<7>{
          \begin{example}
            \begin{enumerate}
              \item The dog ate a filet mignon.
              \item The dog devoured a filet mignon.
              \item The dog ate.
              \item *The dog devoured.
            \end{enumerate}
          \end{example}
          \begin{block}{}
            Lexical meaning doesn't predict syntactic properties
          \end{block}
        }
      \end{frame}

    \subsection{\subtwothree}
      \begin{frame}{\subtwothree}
        \begin{block}{Try these}
          \textcite{dawson_language_2016}, chapter 5 exercise 3
        \end{block}
      \end{frame}

      \begin{frame}{References}
        \printbibliography
      \end{frame}
\end{document}
