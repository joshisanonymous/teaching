%%%%%%%%%%%%%%%%%%%%%%%%%%%%%%%%%%%%%
%                                   %
% Compile with XeLaTeX and biber    %
%                                   %
% Questions or comments:            %
%                                   %
% joshua dot mcneill at uga dot edu %
%                                   %
%%%%%%%%%%%%%%%%%%%%%%%%%%%%%%%%%%%%%

\documentclass{beamer}
  % Read in standard preamble (cosmetic stuff)
  %%%%%%%%%%%%%%%%%%%%%%%%%%%%%%%%%%%%%%%%%%%%%%%%%%%%%%%%%%%%%%%%
% This is a standard preamble used in for all slide documents. %
% It basically contains cosmetic settings.                     %
%                                                              %
% Joshua McNeill                                               %
% joshua dot mcneill at uga dot edu                            %
%%%%%%%%%%%%%%%%%%%%%%%%%%%%%%%%%%%%%%%%%%%%%%%%%%%%%%%%%%%%%%%%

% Beamer settings
% \usetheme{Berkeley}
\usetheme{CambridgeUS}
% \usecolortheme{dove}
% \usecolortheme{rose}
\usecolortheme{seagull}
\usefonttheme{professionalfonts}
\usefonttheme{serif}
\setbeamertemplate{bibliography item}{}

% Packages and settings
\usepackage{fontspec}
  \setmainfont{Charis SIL}
\usepackage{hyperref}
  \hypersetup{colorlinks=true,
              allcolors=blue}
\usepackage{graphicx}
  \graphicspath{{../../figures/}}
\usepackage{soul}
  \setstcolor{red}
\usepackage[normalem]{ulem}
\usepackage{enumerate}
\usepackage{tikz}
  \usetikzlibrary{trees}

% Document information
\author{M. McNeill}
\title[FREN1001]{Français 1001}
\institute{\url{joshua.mcneill@uga.edu}}
\date{}

%% Custom commands
% Lexical items
\newcommand{\lexi}[1]{\textit{#1}}
% Gloss
\newcommand{\gloss}[1]{`#1'}
\newcommand{\tinygloss}[1]{{\tiny`#1'}}
% Orthographic representations
\newcommand{\orth}[1]{$\langle$#1$\rangle$}
% Utterances (pragmatics)
\newcommand{\uttr}[1]{`#1'}
% Sentences (pragmatics)
\newcommand{\sent}[1]{\textit{#1}}
% Fixed length underlines
\newcommand{\funderline}[2][4cm]{
  \underline{\makebox[\ifdim\width>#1\width\else#1\fi]{#2}}
}
% Base dir for definitions
\newcommand{\defs}{../definitions}
\newcommand{\activity}[1]{
  \input{./activities/#1.tex}
}


  % Packages and settings
  \usepackage[backend=biber, style=apa]{biblatex}
    \addbibresource{../references/References.bib}

  % Document information
  \subtitle[Syntactic Constituency]{Syntactic Constituency}

  %% Custom commands
  % Subsection/frame titles
  \newcommand{\suboneone}{What are they?}
  \newcommand{\subonetwo}{Answers to questions}
  \newcommand{\subonethree}{Clefting}
  \newcommand{\subonefour}{Pro-form substitution}
  \newcommand{\subonefive}{Interpreting the results}
  \newcommand{\subonesix}{Practice}

\begin{document}
  % Read in the standard intro slides (title page and table of contents)
  %%%%%%%%%%%%%%%%%%%%%%%%%%%%%%%%%%%%%%%%%%%%%%%%%%%%%%%%%%%%%%%%
% This is a standard set of intro slides used in for all slide %
% documents. It basically contains the title page and table of %
% contents.                                                    %
%                                                              %
% Joshua McNeill                                               %
% joshua dot mcneill at uga dot edu                            %
%%%%%%%%%%%%%%%%%%%%%%%%%%%%%%%%%%%%%%%%%%%%%%%%%%%%%%%%%%%%%%%%

\begin{frame}
  \titlepage
  \tiny{Office: % Basically a variable for office hours location
T Gilbert 141/W Library 4th Fl
\\
        Office hours: % Basically a variable for office hours
T 11-12/W 11-12:30
}
\end{frame}

\begin{frame}
  \tableofcontents[hideallsubsections]
\end{frame}

\AtBeginSection[]{
  \begin{frame}
    \tableofcontents[currentsection,
                     hideallsubsections]
  \end{frame}
}


  \section{Syntactic Constituency}
    \subsection{\suboneone}
      \begin{frame}[t]{\suboneone}
        \begin{example}
          In `The tall man ate tasty gumbo':
          \begin{enumerate}
            \item the tall man
            \item ate tasty gumbo
            \item tasty gumbo
            \item \only<2->{*}ate tasty
          \end{enumerate}
        \end{example}
        \only<1-2>{
          \begin{block}{Do these form cohesive units?}
            \begin{itemize}
              \item<2> Yes: (1-3)
              \item<2> No: (4)
            \end{itemize}
          \end{block}
        }
        \only<3>{
          \begin{alertblock}{Syntactic constituent}
            % Syntactic constituent
A group of linguistic expressions that functions as a single unit within a larger expression

            \begin{itemize}
              \item Identifying these helps us tease out the structure of a sentence
            \end{itemize}
          \end{alertblock}
        }
      \end{frame}

      \begin{frame}{\suboneone}
        \begin{block}{Three tests for identifying constituents}
          \begin{itemize}
            \item Answers to questions
            \item Clefting
            \item Pro-form substitution
          \end{itemize}
        \end{block}
        \begin{alertblock}{}
          This is not an exhaustive list of tests
        \end{alertblock}
      \end{frame}

    \subsection{\subonetwo}
      \begin{frame}{\subonetwo}
        \begin{block}{How it works}
          Replace some part of the sentence with a \lexi{wh-} word and form a question
          \begin{itemize}
            \item If the result is grammatical: It's a constituent
            \item If the result is ungrammatical: It's not a constituent
          \end{itemize}
        \end{block}
        \begin{example}<2->
          `The tall man ate tasty gumbo':
          \begin{itemize}
            \item \emph{Who} ate tasty gumbo?
            \item \emph{Who} the tall ate tasty gumbo?
            \item \emph{What did} the tall man do?
            \item \emph{What did} the tall man do gumbo?
          \end{itemize}
        \end{example}
      \end{frame}

    \subsection{\subonethree}
      \begin{frame}{\subonethree}
        \begin{block}{How it works}
          Move some part $X$ of the sentence $Y$ to the beginning in the format `It was $X$ who/that $Y$'
          \begin{itemize}
            \item If the result is grammatical: It's a constituent
            \item If the result is ungrammatical: It's not a constituent
          \end{itemize}
        \end{block}
        \begin{example}<2->
          `The tall man ate tasty gumbo':
          \begin{itemize}
            \item It was \emph{the tall man} who ate tasty gumbo.
            \item It was \emph{man} that the tall ate tasty gumbo.
            \item It was \emph{eat tasty gumbo} that the tall man did.
            \item It was \emph{eat tasty} that the tall man did gumbo.
          \end{itemize}
        \end{example}
      \end{frame}

    \subsection{\subonefour}
      \begin{frame}[t]{\subonefour}
        \begin{block}{How it works}
          Replace some part of the sentence with a pro-form
          \begin{itemize}
            \item If the result is grammatical: It's a constituent
            \item If the result is ungrammatical: It's not a constituent
          \end{itemize}
        \end{block}
        \only<2>{
          \begin{alertblock}{Pro-form}
            % Pro-form
A relatively generic expression that can stand in for a more specific and often larger expression

          \end{alertblock}
          \begin{example}
            \begin{itemize}
              \item Pronouns: \emph{he}, \emph{it}, \emph{them}
              \item Pro-verbs: \emph{do}, \emph{be}, \emph{have}
            \end{itemize}
          \end{example}
        }
        \only<3>{
          \begin{example}
            `The tall man ate tasty gumbo':
            \begin{itemize}
              \item \emph{He} ate tasty gumbo.
              \item The tall \emph{he} ate tasty gumbo.
              \item The tall man \emph{did so}.
              \item The tall man \emph{did so} gumbo.
            \end{itemize}
          \end{example}
        }
      \end{frame}

    \subsection{\subonefive}
      \begin{frame}{\subonefive}
        \begin{block}{}
          \begin{itemize}
            \item Tests that are failed are not conclusive
            \item Tests that are passed are sufficient
          \end{itemize}
        \end{block}
        \begin{example}
          \begin{enumerate}
            \item *\emph{Who} the man ate tasty gumbo?
            \item<2-> \emph{Who} ate tasty gumbo?
          \end{enumerate}
          \only<1>{
            The test in (1) fails but more tests are needed to be sure
          }
          \only<2>{
            The test in (2) passes, so no more is needed
          }
        \end{example}
      \end{frame}

    \subsection{\subonesix}
      \begin{frame}{\subonesix}
        \begin{block}{Try these}
          \textcite{dawson_language_2016}, chapter 5 exercises 15 and 16
        \end{block}
      \end{frame}
\end{document}
