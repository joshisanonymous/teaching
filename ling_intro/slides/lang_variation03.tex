%%%%%%%%%%%%%%%%%%%%%%%%%%%%%%%%%%%%%
%                                   %
% Compile with XeLaTeX and biber    %
%                                   %
% Questions or comments:            %
%                                   %
% joshua dot mcneill at uga dot edu %
%                                   %
%%%%%%%%%%%%%%%%%%%%%%%%%%%%%%%%%%%%%

\documentclass{beamer}
  % Read in standard preamble (cosmetic stuff)
  %%%%%%%%%%%%%%%%%%%%%%%%%%%%%%%%%%%%%%%%%%%%%%%%%%%%%%%%%%%%%%%%
% This is a standard preamble used in for all slide documents. %
% It basically contains cosmetic settings.                     %
%                                                              %
% Joshua McNeill                                               %
% joshua dot mcneill at uga dot edu                            %
%%%%%%%%%%%%%%%%%%%%%%%%%%%%%%%%%%%%%%%%%%%%%%%%%%%%%%%%%%%%%%%%

% Beamer settings
% \usetheme{Berkeley}
\usetheme{CambridgeUS}
% \usecolortheme{dove}
% \usecolortheme{rose}
\usecolortheme{seagull}
\usefonttheme{professionalfonts}
\usefonttheme{serif}
\setbeamertemplate{bibliography item}{}

% Packages and settings
\usepackage{fontspec}
  \setmainfont{Charis SIL}
\usepackage{hyperref}
  \hypersetup{colorlinks=true,
              allcolors=blue}
\usepackage{graphicx}
  \graphicspath{{../../figures/}}
\usepackage{soul}
  \setstcolor{red}
\usepackage[normalem]{ulem}
\usepackage{enumerate}
\usepackage{tikz}
  \usetikzlibrary{trees}

% Document information
\author{M. McNeill}
\title[FREN1001]{Français 1001}
\institute{\url{joshua.mcneill@uga.edu}}
\date{}

%% Custom commands
% Lexical items
\newcommand{\lexi}[1]{\textit{#1}}
% Gloss
\newcommand{\gloss}[1]{`#1'}
\newcommand{\tinygloss}[1]{{\tiny`#1'}}
% Orthographic representations
\newcommand{\orth}[1]{$\langle$#1$\rangle$}
% Utterances (pragmatics)
\newcommand{\uttr}[1]{`#1'}
% Sentences (pragmatics)
\newcommand{\sent}[1]{\textit{#1}}
% Fixed length underlines
\newcommand{\funderline}[2][4cm]{
  \underline{\makebox[\ifdim\width>#1\width\else#1\fi]{#2}}
}
% Base dir for definitions
\newcommand{\defs}{../definitions}
\newcommand{\activity}[1]{
  \input{./activities/#1.tex}
}


  % Document information
  \subtitle[Dialectology]{Dialectology}

  %% Custom commands
  % Subsection/frame titles
  \newcommand{\suboneone}{What are we talking about?}
  \newcommand{\subonetwo}{What's so important about geography?}
  \newcommand{\subonethree}{Origin of American dialects}
  \newcommand{\subonefour}{}

\begin{document}
  % Read in the standard intro slides (title page and table of contents)
  %%%%%%%%%%%%%%%%%%%%%%%%%%%%%%%%%%%%%%%%%%%%%%%%%%%%%%%%%%%%%%%%
% This is a standard set of intro slides used in for all slide %
% documents. It basically contains the title page and table of %
% contents.                                                    %
%                                                              %
% Joshua McNeill                                               %
% joshua dot mcneill at uga dot edu                            %
%%%%%%%%%%%%%%%%%%%%%%%%%%%%%%%%%%%%%%%%%%%%%%%%%%%%%%%%%%%%%%%%

\begin{frame}
  \titlepage
  \tiny{Office: % Basically a variable for office hours location
T Gilbert 141/W Library 4th Fl
\\
        Office hours: % Basically a variable for office hours
T 11-12/W 11-12:30
}
\end{frame}

\begin{frame}
  \tableofcontents[hideallsubsections]
\end{frame}

\AtBeginSection[]{
  \begin{frame}
    \tableofcontents[currentsection,
                     hideallsubsections]
  \end{frame}
}


  \section{Dialectology}
    \subsection{\suboneone}
      \begin{frame}{\suboneone}
        \begin{block}{Would you expect these two people to speak the same?}
          \begin{enumerate}
            \item A 20 year old, white, working-class female from \alert<2->{Georgia} in a casual conversation
            \item A 20 year old, white, working-class female from \alert<2->{Wisconsin} in a casual conversation
          \end{enumerate}
        \end{block}
        \begin{alertblock}<2->{Dialectology}
          % Dialectology
The study of geographic language variation

        \end{alertblock}
      \end{frame}

    \subsection{\subonetwo}
      \begin{frame}{\subonetwo}
        \begin{block}{Face-to-face interaction is given precedence}
          Which person do you care more about?
          \begin{enumerate}
            \item @angrypoliticsguy from Idaho on Twitter
            \item Your friend that you hang out with every weekend
          \end{enumerate}
        \end{block}
      \end{frame}

      \begin{frame}{\subonetwo}
        \begin{alertblock}{Isogloss}
          % Isogloss
A geographic boundary distinguishing where different variants of a particular linguistic variable are used

        \end{alertblock}
      \end{frame}

      % "how do we end up with geographic variation"
        % geographic factors
          % geographical isolation can lead to varieties that develop in their own direction
            % physical boundaries often correlate with dialect boundaries: isoglosses
              % an isogloss is for one variable
              % Bundles exist also
              % bundles are used to define dialect boundaries
                % they use map of Pennsylvania as an example
        % how we ended up with American dialects
          % trace settlement patterns
          % talk about interactions with non-anglo communities
            % native american languages
            % other european languages
              % French in Louisiana
              % German in Pennsylvania
              % Spanish later in the southwest
            % African slaves in the southeast brought languages
              % later AA migrations to northern cities
          % Show graph (3) with the resulting broad dialect boundaries
          % As for those indirect contacts (e.g., tv, internet)
            % working-class dialects have been stable, but middle-class dialects have homogenized to a degree
            % in part, because the culture of initial settlers has undue influence on later settlers
        % Characteristics of the supra-regional dialects
          % The North
            % Northern Cities vowel shift
              % Show diagram and then video of Labov explaining (https://youtu.be/9UoJ1-ZGb1w?t=59)
            % Use of gerunds: the table needs cleaning where elsewhere it's the table needs to be cleaned
            % Lexical differences, too
              % sneaker, pop, roly-poly
            % maybe talk about "by" where others use "at": I was by Sarah's house yesterday
          % New England


    \subsection{}
      \begin{frame}{}
        \begin{block}{Try these}
          % \textcite{dawson_language_2016}, chapter 10 exercises
        \end{block}
      \end{frame}

      \begin{frame}{References}
        % \printbibliography
      \end{frame}
\end{document}
