%%%%%%%%%%%%%%%%%%%%%%%%%%%%%%%%%%%%%
%                                   %
% Compile with XeLaTeX and biber    %
%                                   %
% Questions or comments:            %
%                                   %
% joshua dot mcneill at uga dot edu %
%                                   %
%%%%%%%%%%%%%%%%%%%%%%%%%%%%%%%%%%%%%

\documentclass{beamer}
  % Read in standard preamble (cosmetic stuff)
  %%%%%%%%%%%%%%%%%%%%%%%%%%%%%%%%%%%%%%%%%%%%%%%%%%%%%%%%%%%%%%%%
% This is a standard preamble used in for all slide documents. %
% It basically contains cosmetic settings.                     %
%                                                              %
% Joshua McNeill                                               %
% joshua dot mcneill at uga dot edu                            %
%%%%%%%%%%%%%%%%%%%%%%%%%%%%%%%%%%%%%%%%%%%%%%%%%%%%%%%%%%%%%%%%

% Beamer settings
% \usetheme{Berkeley}
\usetheme{CambridgeUS}
% \usecolortheme{dove}
% \usecolortheme{rose}
\usecolortheme{seagull}
\usefonttheme{professionalfonts}
\usefonttheme{serif}
\setbeamertemplate{bibliography item}{}

% Packages and settings
\usepackage{fontspec}
  \setmainfont{Charis SIL}
\usepackage{hyperref}
  \hypersetup{colorlinks=true,
              allcolors=blue}
\usepackage{graphicx}
  \graphicspath{{../../figures/}}
\usepackage[normalem]{ulem}
\usepackage{enumerate}

% Document information
\author{M. McNeill}
\title[FREN2001]{Français 2001}
\institute{\url{joshua.mcneill@uga.edu}}
\date{}

%% Custom commands
% Lexical items
\newcommand{\lexi}[1]{\textit{#1}}
% Gloss
\newcommand{\gloss}[1]{`#1'}
\newcommand{\tinygloss}[1]{{\tiny`#1'}}
% Orthographic representations
\newcommand{\orth}[1]{$\langle$#1$\rangle$}
% Utterances (pragmatics)
\newcommand{\uttr}[1]{`#1'}
% Sentences (pragmatics)
\newcommand{\sent}[1]{\textit{#1}}
% Base dir for definitions
\newcommand{\defs}{../definitions}


  % Packages and settings
  \usepackage{forest}
  \usepackage[backend=biber, style=apa]{biblatex}
    \addbibresource{../references/References.bib}
  \usepackage{xecjk}
    \setCJKmainfont{SimSun}

  % Document information
  \subtitle[Suprasegmental Articulation]{Articulation of English Suprasegmentals}

  %% Custom commands
  % Subsection/frame titles
  \newcommand{\suboneone}{Definition}
  \newcommand{\subonetwo}{Length}
  \newcommand{\subonethree}{Intonation}
  \newcommand{\subonefour}{Tone}
  \newcommand{\subonefive}{Stress}
  \newcommand{\subonesix}{A note about syllabic consonants}
  \newcommand{\suboneseven}{Resources and Practice}

\begin{document}
  % Read in the standard intro slides (title page and table of contents)
  %%%%%%%%%%%%%%%%%%%%%%%%%%%%%%%%%%%%%%%%%%%%%%%%%%%%%%%%%%%%%%%%
% This is a standard set of intro slides used in for all slide %
% documents. It basically contains the title page and table of %
% contents.                                                    %
%                                                              %
% Joshua McNeill                                               %
% joshua dot mcneill at uga dot edu                            %
%%%%%%%%%%%%%%%%%%%%%%%%%%%%%%%%%%%%%%%%%%%%%%%%%%%%%%%%%%%%%%%%

\begin{frame}
  \titlepage
  \tiny{Office: % Basically a variable for office hours location
Gilbert 121\\
        Office hours: % Basically a variable for office hours
 lundi, mercredi, vendredi 10:10--11:10
}
\end{frame}

\begin{frame}
  \tableofcontents[hideallsubsections]
\end{frame}

\AtBeginSection[]{
  \begin{frame}
    \tableofcontents[currentsection,
                     hideallsubsections]
  \end{frame}
}


  \section{What are suprasegmentals?}
    \subsection{\suboneone}
      \begin{frame}{\suboneone}
        \begin{columns}
          \column{0.5\textwidth}
            \begin{minipage}[c][0.6\textheight]{\linewidth}
              \only<1-2>{
                \begin{block}{A segment}
                  % Segment
A consonant or a vowel

                \end{block}
                \begin{alertblock}<2>{A \emph{supra}segmental}
                  % Read in definition of a suprasegmental
                  % Suprasegmental
A sound feature that applies to a whole syllable and is best understood in relation to how it changes across syllables

                \end{alertblock}
              }
              \only<3>{
                \begin{block}{Examples of suprasegmentals}
                  \begin{itemize}
                    \item Length
                    \item Intonation
                    \item Tone
                    \item Stress
                  \end{itemize}
                \end{block}
              }
            \end{minipage}
          \column{0.5\textwidth}
            \uncover<2->{
              % Read in a tree graph of a syllable
              % This uses the forest package to print a basic syllable structure
\begin{forest}
  [ Syllable
    [ Onset
      [ p, tier=segment ]
    ]
    [ Rhyme
      [ Nucleus
        [ ɪ, tier=segment ]
      ]
      [ Coda
        [ n, tier=segment ]
      ]
    ]
  ]
\end{forest}

            }
        \end{columns}
      \end{frame}

    \subsection{\subonetwo}
      \begin{frame}[t]{\subonetwo}
        \begin{block}{}
          Notated using [ː] (e.g., [kɑː]\only<2->{ \lexi{car}})
        \end{block}
        \begin{example}<3->
          Finnish
          \begin{itemize}
            \item {[}tapan] `I kill' vs [tapaːn] `I meet'
            \item {[}muta] `mud' vs [mutːa] `but'
          \end{itemize}
        \end{example}
        \begin{alertblock}<4->{}
          Length can only be determined through comparing lengths in an utterance
        \end{alertblock}
      \end{frame}

    \subsection{\subonethree}
      \begin{frame}[t]{\subonethree}
        \only<1-2>{
        \begin{definition}
          % Intonation
The pattern of pitch changes across an utterance

        \end{definition}
          \begin{block}<2>{Notation}
            In \textcite{dawson_language_2016}:
            \begin{itemize}
              \item \uttr{Use CAPS to indicate words where the pitch is HIGHER}
            \end{itemize}
            In general:
            \begin{itemize}
              \item \uttr{Use ↗arrows ↘to indicate words where the pitch ↗changes.}
            \end{itemize}
          \end{block}
        }
        \only<3-6>{
          \begin{example}
            \uttr{Use ↗arrows ↘to indicate words where the pitch ↗changes.}
          \end{example}
          \only<4>{
            \begin{alertblock}{Pitch accent}
              % Pitch accent
A higher intonational pitch used on a word in the middle of a phrase to draw attention to it

              \begin{itemize}
                \item \lexi{Arrows} receives a pitch accent here
              \end{itemize}
            \end{alertblock}
          }
          \only<5-6>{
            \begin{alertblock}{Phrase tone}
              % Phrase tone
An intonational pitch used at the end of a phrase that indicates how the phrase should be understood

              \begin{itemize}
                \item The phrase reads like a question
              \end{itemize}
            \end{alertblock}
            \begin{block}<6>{Not limited to sentences}
              \uttr{You got an A on the ↘test, a C on the ↘homework, and a B on the ↘quiz.}
            \end{block}
          }
        }
      \end{frame}

    \subsection{\subonefour}
      \begin{frame}{\subonefour}
        \only<1-2>{
          \begin{definition}
            % Tone
A pitch change at the level of a syllable that creates a semantic distinction

            \begin{itemize}
              \item English does not have this
            \end{itemize}
          \end{definition}
          \begin{block}<2->{Two types}
            \begin{itemize}
              \item \alert{Level tone}: % Level tone
A tone that does not change over the course of a syllable

              \item \alert{Contour tone}: % Contour tone
A tone that changes over the course of a syllable

            \end{itemize}
          \end{block}
        }
        \only<3>{
          \begin{example}
            \href{https://youtu.be/f-ZDrmP-N1s?t=23}{Mandarin}

            \parbox{0.45\linewidth}{
              \begin{itemize}
                \item {[}ma55] or [má] or [ma˦] `mother' \lexi{妈}
                \item {[}ma35] or [ma᷄] or [ma˧˥] `hemp' \lexi{麻}
              \end{itemize}
            }
            \parbox{0.45\linewidth}{
              \begin{itemize}
                \item {[}ma214] or [ma᷉] or [ma˨˩˦] `horse' \lexi{马}
                \item {[}ma51] or [mâ] or [ma˥˩] `scold' \lexi{骂}
              \end{itemize}
            }
          \end{example}
        }
      \end{frame}

    \subsection{\subonefive}
      \begin{frame}[t]{\subonefive}
        \begin{definition}
          % Stress
A syllable with a full vowel that is longer, louder, and has a higher pitch than others around it

        \end{definition}
        \begin{block}<2->{Notation}
          \begin{itemize}
            \item {[}.] divides syllables
            \item {[}ˈ] indicates primary stress
            \item {[}ˌ] indicates secondary stress
          \end{itemize}
        \end{block}
        \only<3>{
          \begin{example}
            \lexi{photograph} [ˈfoʊ.də.gɹæf] vs \lexi{photography} [fəˈtɑ.gɹəˌfi]
          \end{example}
        }
        \only<4>{
          \begin{block}{Stress can be predictable}
            Almost always on the first syllable in Czech
            \begin{itemize}
              \item English is not so straight forward
            \end{itemize}
          \end{block}
        }
        \only<5>{
          \begin{block}{Stress can be semantically contrastive}
            \begin{itemize}
              \item {[}waɪtˈhaʊs] vs [ˈwaɪt.haʊs]
            \end{itemize}
          \end{block}
        }
      \end{frame}

    \subsection{\subonesix}
      \begin{frame}{\subonesix}
        \begin{columns}
          \column{0.5\textwidth}
            \begin{minipage}[c][0.6\textheight]{\linewidth}
              \begin{block}{Four syllabic consonants in English}
                [ɹ̩ l̩ n̩ m̩]
                \begin{itemize}
                  \item {[}ˈwɔ.ɾɹ̩]
                  \item {[}ˈteɪ.bl̩]
                  \item {[}ˈkɑʔ.n̩]
                  \item {[}ɹɪ.ðm̩]
                \end{itemize}
              \end{block}
            \end{minipage}
          \column{0.5\textwidth}
            \only<1-2>{
              \begin{forest}
                [ Syllable
                  [ Onset
                    [ \only<1>{ɾ}\only<2>{b}, tier=segment ]
                  ]
                  [ Rhyme
                    [ Nucleus
                      [ \only<1>{ɹ̩}\only<2>{l̩}, tier=segment ]
                    ]
                  ]
                ]
              \end{forest}
            }
            \only<3>{
              \begin{forest}
                [ Syllable
                  [ Rhyme
                    [ Nucleus
                      [ n̩, tier=segment ]
                    ]
                  ]
                ]
              \end{forest}
            }
            \only<4>{
              \begin{forest}
                [ Syllable
                  [ Onset
                    [ ð, tier=segment ]
                  ]
                  [ Rhyme
                    [ Nucleus
                      [ m̩, tier=segment ]
                    ]
                  ]
                ]
              \end{forest}
            }
        \end{columns}
      \end{frame}

    \subsection{\suboneseven}
      \begin{frame}{\suboneseven}
        \begin{block}{}
          % A set of links to useful resources when dealing articulatory phonetics
\begin{itemize}
  \item To hear these sounds: \url{http://web.uvic.ca/ling/resources/ipa/charts/IPAlab/IPAlab.htm} and \url{https://americanipachart.com/}
  \item To type these symbols: \url{https://ipa.typeit.org/}
\end{itemize}

        \end{block}
        \begin{block}{Try these}
          \textcite{dawson_language_2016}, chapter 2 exercises 25 and 27
        \end{block}
      \end{frame}
\end{document}
