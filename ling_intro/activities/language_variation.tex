\documentclass[addpoints]{exam}
  % Read in shared preamble for all homeworks
  %%%%%%%%%%%%%%%%%%%%%%%%%%%%%%%%%%%%%%%%%%%%%%%%%%%%%%%%%%%%%%%%%%%%
% This is the standard preamble for homework assignments and exams %
%                                                                  %
% -Joshua McNeill (joshua dot mcneill at uga dot edu)              %
%%%%%%%%%%%%%%%%%%%%%%%%%%%%%%%%%%%%%%%%%%%%%%%%%%%%%%%%%%%%%%%%%%%%
% Exam settings
\pointsinmargin
\pointformat{}

% Packages and settings
\usepackage{fontspec}
  \setmainfont{Charis SIL}
\usepackage{tikz}

%% Custom commands
% Instructions for a section
\newcommand{\instr}[1]{
  \begin{center}
    \fbox{
      \parbox{0.85\textwidth}
             {#1}
    }
  \end{center}
}
\newcommand{\lexi}[1]{\textit{#1}}
\newcommand{\gloss}[1]{`#1'}


  % Packages and settings
  \usepackage{graphicx}
    \graphicspath{{../figures/}}

  % Document information
  \title{Language Variation}
  \date{}

\begin{document}
  \maketitle

  % Header
  %%%%%%%%%%%%%%%%%%%%%%%%%%%%%%%%%%%%%%%%%%%%%%%%%%%%%%%%%%%%%%%%%%%%%%%
% This is the the header that all homework assignments and exams use. %
%                                                                     %
% -Joshua McNeill (joshua dot mcneill at uga dot edu)                 %
%%%%%%%%%%%%%%%%%%%%%%%%%%%%%%%%%%%%%%%%%%%%%%%%%%%%%%%%%%%%%%%%%%%%%%%
\noindent\makebox[0.5\textwidth][l]{Name:} \makebox[0.5\textwidth][r]{Course: LING2100, The Study of Language}\\
\makebox[0.5\textwidth][l]{Date:} \makebox[0.5\textwidth][r]{Instructor: Joshua McNeill}


  % Questions
    \instr{
      Answer the follow questions based on how you speak.
      Try not to dwell on any one question too long as you we don't want you to second guess yourself.
      All questions come from the Harvard Dialect Survey.
    }

  \begin{questions}
    \question What do you call the long cold sandwich that contains cold cuts, lettuce, and so on? \hrulefill
    \question What is your generic casual or informal term for a sweetened carbonated beverage? \hrulefill
    \question What is your general, informal term for the rubber-soled shoes worn in gym class, for athletic activities, etc.? \hrulefill
    \question What do you call the kind of crustacean that looks like a tiny lobster and lives in lakes and streams? \hrulefill
    \question What words do you use in casual speech to address a group of two or more people? \hrulefill
    \question What do you call the little gray (or black or brown) creature (that looks like an insect but is actually a crustacean) that rolls up into a ball when you touch it? \hrulefill
    \question What do you call the kind of rain that falls while the sun is shining? \hrulefill
    \question What do you call the gooey or dry matter that collects in the corners of your eyes, especially while you are sleeping? \hrulefill
    \question What is the kinship term for your parent's sister? Write it down in IPA based on how you say it. \hrulefill
    \question What is your preferred general and casual term for a sale of your unwanted items (which may be held on your porch, in your yard, garden, or house, from the back of your car, etc.)? \hrulefill
    \question What do you call the wheeled contraption in which you carry groceries at the grocery store or supermarket? \hrulefill
    \question What do you call a traffic intersection in which several roads meet in a circle and you have to get off at a certain point? \hrulefill
    \question How do you pronounce the first initial, shared part of each of the following words: car, cart, carton? Transcribe it in IPA. \hrulefill
    \question Do you pronounce <which> and <witch> the same? If not, transcribe the difference in IPA. \hrulefill
    \question What do you call the meal you eat in the evening, normally somewhere between 5 and 10 PM? \hrulefill
    \question What do you call an upholstered seat for more than one person? \hrulefill
    \question What do you a call a store that is devoted primarily to selling alcoholic beverages? \hrulefill
    \question What do you call a room equipped with toilets and lavatories for public use? \hrulefill
    \question What do you call the auxiliary brake that's attached to a rear wheel or the transmission and keeps the car from moving accidentally? \hrulefill
    \question What do you call an artificial nipple, usually made of plastic, which an infant can suck or chew on? \hrulefill
    \question What do you call food purchased at a restaurant to be eaten elsewhere? \hrulefill
    \question What do you call this large aquatic bug that skims along the surface of water? \hrulefill
    \question What do you call your fifth/smallest toe? \hrulefill

    \instr{
      We are now going to attempt to figure out which people in the class answered multiple questions the same and form into groups small groups based on that.
      Your instructor will ask the class which answers they gave for the first question in order to list out the possible answers.
      You will then physically sort yourselves so that you're standing near those who gave the same answer as you.
      Your instructor will then do the same for the second question, but now you will sort yourself only within the groups that you're already sorted into.
      You will repeat these steps until the class is made up of groups of roughly 4 each.
      Now discuss the following questions in your groups.
    }

    \question Which answers that you didn't give to the questions surprised you the most?
    \question Which answers that you didn't give surprised you the least?
    \question Which questions did you and your groupmates \emph{not} answer the same?
    \question What do you all have in common in your group in terms of your backgrounds? (Be accepting of your groupmates' responses and understanding if you or they wish to not discuss a question.)
    \begin{parts}
      \part Are you from the same place?
      \part Do you have the same gender identities?
      \part Do you identify the same ethnically?
      \part Do you identify the same racially?
      \part Do you follow the same religion?
      \part Are you part of the same socioeconomic class?
    \end{parts}
    \question What do you \emph{not} have in common in your group?
    \question What does this tell you about the relationship between speech and personal histories/backgrounds?

    \instr{
      You will now combine your group with several other groups to have a larger discussion about the results of the quertionnaire and your groupings.
      Your instructor and TAs will each lead one of these larger discussion groups.
      Consider questions such as the following:
    }
    
    \question For those questions for which your groupmates didn't have the same answer, did members of other groups give the same answers as you instead?
    \question If so, do you have anything in common with these classmates?
    \question What does this tell you about dialect/sociolect boundaries and geographic/social boundaries?
  \end{questions}
\end{document}
