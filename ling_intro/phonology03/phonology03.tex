%%%%%%%%%%%%%%%%%%%%%%%%%%%%%%%%%%%%%
%                                   %
% Compile with XeLaTeX and biber    %
%                                   %
% Questions or comments:            %
%                                   %
% joshua dot mcneill at uga dot edu %
%                                   %
%%%%%%%%%%%%%%%%%%%%%%%%%%%%%%%%%%%%%

\documentclass{beamer}
  % Read in standard preamble (cosmetic stuff)
  %%%%%%%%%%%%%%%%%%%%%%%%%%%%%%%%%%%%%%%%%%%%%%%%%%%%%%%%%%%%%%%%
% This is a standard preamble used in for all slide documents. %
% It basically contains cosmetic settings.                     %
%                                                              %
% Joshua McNeill                                               %
% joshua dot mcneill at uga dot edu                            %
%%%%%%%%%%%%%%%%%%%%%%%%%%%%%%%%%%%%%%%%%%%%%%%%%%%%%%%%%%%%%%%%

% Beamer settings
% \usetheme{Berkeley}
\usetheme{CambridgeUS}
% \usecolortheme{dove}
% \usecolortheme{rose}
\usecolortheme{seagull}
\usefonttheme{professionalfonts}
\usefonttheme{serif}
\setbeamertemplate{bibliography item}{}

% Packages and settings
\usepackage{fontspec}
  \setmainfont{Charis SIL}
\usepackage{hyperref}
  \hypersetup{colorlinks=true,
              allcolors=blue}
\usepackage{graphicx}
  \graphicspath{{../../figures/}}
\usepackage[normalem]{ulem}
\usepackage{enumerate}

% Document information
\author{M. McNeill}
\title[FREN2001]{Français 2001}
\institute{\url{joshua.mcneill@uga.edu}}
\date{}

%% Custom commands
% Lexical items
\newcommand{\lexi}[1]{\textit{#1}}
% Gloss
\newcommand{\gloss}[1]{`#1'}
\newcommand{\tinygloss}[1]{{\tiny`#1'}}
% Orthographic representations
\newcommand{\orth}[1]{$\langle$#1$\rangle$}
% Utterances (pragmatics)
\newcommand{\uttr}[1]{`#1'}
% Sentences (pragmatics)
\newcommand{\sent}[1]{\textit{#1}}
% Base dir for definitions
\newcommand{\defs}{../definitions}


  % Packages and settings
  \usepackage{tikz}
    \usetikzlibrary{shapes.geometric, arrows}
    \tikzstyle{process} = [rectangle,
                           text centered,
                           draw=black,
                           align=center]
    \tikzstyle{arrow} = [thick,->]
  \usepackage{phonrule}
  \usepackage[backend=biber, style=apa]{biblatex}
    \addbibresource{../references/References.bib}

  % Document information
  \subtitle[Phonological Rules]{Phonological Rules}

  %% Custom commands
  % Subsection/frame titles
  \newcommand{\suboneone}{What are rules}
  \newcommand{\subonetwo}{Natural classes}
  \newcommand{\subonethree}{Types of rules}
  \newcommand{\subonefour}{Combining rules}
  \newcommand{\subonefive}{Obligatory vs optional}
  \newcommand{\subonesix}{Practice}

\begin{document}
  % Read in the standard intro slides (title page and table of contents)
  %%%%%%%%%%%%%%%%%%%%%%%%%%%%%%%%%%%%%%%%%%%%%%%%%%%%%%%%%%%%%%%%
% This is a standard set of intro slides used in for all slide %
% documents. It basically contains the title page and table of %
% contents.                                                    %
%                                                              %
% Joshua McNeill                                               %
% joshua dot mcneill at uga dot edu                            %
%%%%%%%%%%%%%%%%%%%%%%%%%%%%%%%%%%%%%%%%%%%%%%%%%%%%%%%%%%%%%%%%

\begin{frame}
  \titlepage
  \tiny{Office: % Basically a variable for office hours location
Gilbert 121\\
        Office hours: % Basically a variable for office hours
 lundi, mercredi, vendredi 10:10--11:10
}
\end{frame}

\begin{frame}
  \tableofcontents[hideallsubsections]
\end{frame}

\AtBeginSection[]{
  \begin{frame}
    \tableofcontents[currentsection,
                     hideallsubsections]
  \end{frame}
}


  \section{Phonological rules}
    \subsection{\suboneone}
      \begin{frame}[t]{\suboneone}
        \begin{columns}
          \column{0.48\textwidth}
            \begin{block}{Data}
              \begin{itemize}
                \item \lexi{leap} [ˈlip]
                \item \lexi{lean} [ˈlĩn]
              \end{itemize}
            \end{block}
          \column{0.48\textwidth}
            \begin{block}{Allophones of /i/}
              \begin{itemize}
                \item {[}ĩ] before a nasal consonant
                \item {[}i] elsewhere
              \end{itemize}
            \end{block}
        \end{columns}
        \only<2>{
          \begin{block}{General process}
            \begin{tikzpicture}[node distance=3.5cm]
              \node (underlying) [process] {Underlying form \\
                                            Phoneme \\
                                            /i/};
              \node (rules) [process, right of=underlying] {Rules};
              \node (surface) [process, right of=rules] {Surface form \\
                                                         Phone \\
                                                         {[}ĩ]};
              \draw [arrow] (underlying) -- (rules);
              \draw [arrow] (rules) -- (surface);
            \end{tikzpicture}
          \end{block}
        }
        \only<3>{
          \begin{alertblock}{Underlying form}
            % Underlying form
The phonemic representation of a sound before any phonological rules have been applied

          \end{alertblock}
          \begin{alertblock}{Surface form}
            % Surface form
The phonetic representation of a sound after any phonological rules have been applied

          \end{alertblock}
        }
        \only<4>{
          \begin{alertblock}{Phonological rule}
            % Phonological rule
A description of the relationship between a phoneme and its allophones and the conditioning environment in which the allophone appears

          \end{alertblock}
          \begin{alertblock}{Conditioning environment}
            % Conditioning environment
The context in which a phonological rule applies

          \end{alertblock}
        }
        \only<5>{
          \begin{block}{Rule notation}
            \begin{center}
              \phonb{phoneme}{phone}{left context}{right context}
            \end{center}
          \end{block}
        }
        \only<6>{
          \begin{example}
            \begin{center}
              \phonc{/i/}{[ĩ]}{\oneof{\phold n \\
                                      \phold m \\
                                      \phold ŋ}}
            \end{center}
          \end{example}
        }
      \end{frame}

    \subsection{\subonetwo}
      \begin{frame}[t]{\subonetwo}
        \only<4-7>{
          \begin{columns}
            \column{0.48\textwidth}
              \begin{block}{Data}
                \begin{itemize}
                  \item {[}ˈ\alert{t}ɛɹ]
                  \item {[}ˈwɔ.\alert{ɾ}ɹ̩]
                  \item<2-> {[}ˈsi\alert{d}]
                  \item<2-> {[}ˈsi.\alert{ɾ}əd]
                \end{itemize}
              \end{block}
            \column{0.48\textwidth}
              \begin{block}{Allophones of /t/}
                \begin{itemize}
                  \item {[}ɾ] between two vowels (or vocalic sounds)
                  \item {[}t] elsewhere
                \end{itemize}
                \uncover<3->{This applies to /d/, too}
              \end{block}
          \end{columns}
          \begin{block}{The rule}
            \begin{center}
              \phonb{/t, d/}{[ɾ]}{V}{V}
            \end{center}
          \end{block}
          \only<5-6>{
            \begin{block}{What do /t/ and /d/ have in common?}
              \uncover<6->{They're both alveolar stops, a natural class}
            \end{block}
          }
          \only<7>{
            \begin{alertblock}{Natural class}
              % Natural class
A group of sounds that share one or more features

            \end{alertblock}
          }
        }
        \only<8-11>{
          \begin{block}{Some new natural classes}
            \begin{itemize}
              \item \alert{Sibilants}: % Sibilant`
A fricative that is high amplitude and high pitch, characterized by a hissing sound

              \begin{itemize}
                \item Includes [s, z, ʃ, ʒ, tʃ, dʒ]
              \end{itemize}
              \item<9-> \alert{Labials}: % Labial
Any sound created using the lips

              \begin{itemize}
                \item Includes labiodentals and bilabials
              \end{itemize}
              \item<10-> \alert{Obstruents}: % Obstruent
Any sound created by restricting airflow

              \begin{itemize}
                \item Includes stops, fricatives, and affricates
              \end{itemize}
              \item<11-> \alert{Sonorants}: % Sonorant
Any sound created without restricting airflow

              \begin{itemize}
                \item Includes vowels, approximants, and nasals
              \end{itemize}
            \end{itemize}
          \end{block}
        }
      \end{frame}

    \subsection{\subonethree}
      \begin{frame}{\subonethree}
        \only<1>{
          \begin{block}{We have seven}
            \begin{itemize}
              \item Assimilation
              \item Dissimilation
              \item Insertion (or epenthesis)
              \item Deletion (or elision)
              \item Metathesis
              \item Strengthening (or fortition)
              \item Weakening (or lenition)
            \end{itemize}
          \end{block}
        }
        \only<2-4>{
          \begin{alertblock}{Assimilation}
            % Assimilation
When adjacent sounds take on characteristics of each other

          \end{alertblock}
          \begin{example}<3->
            \begin{tabular}{c c c}
              [ˈlʌ\alert{m}p] & [ˈtɪ\alert{n}t] & [ˈθɪ\alert{ŋ}k]
            \end{tabular}
          \end{example}
          \begin{block}<4->{The rules}
            \begin{itemize}
              \item \phonr{/N/}{[m]}{\phonfeat{+bilabial}}
              \item \phonr{/N/}{[n]}{\phonfeat{+alveolar}}
              \item \phonr{/N/}{[ŋ]}{\phonfeat{+velar}}
            \end{itemize}
          \end{block}
        }
        \only<5-6>{
          \begin{block}{A special case}
            \begin{itemize}
              \item \alert{Vowel harmony}: % Vowel harmony
A special type of assimilation in which non-adjacent vowels in a word become more similar

            \end{itemize}
          \end{block}
          \begin{example}<6->
            In Finnish:
            \begin{itemize}
              \item All vowels in a word are either front vowels or back vowels
              \item /metsæ/ `forest' + /ss\alert{ɑ}/ `in' = [metsæss\alert{æ}] `in the forest'
            \end{itemize}
          \end{example}
        }
        \only<7-10>{
          \begin{alertblock}{Dissimilation}
            % Dissimilation
When similar adjacent sounds change to be more distinctly pronounced

          \end{alertblock}
          \begin{example}<8->
            \begin{tabular}{c c}
              \lexi{sixth}                & \\
              \uncover<9->{/ˈsɪks\alert{θ}/}  & \uncover<9->{[ˈsiks\alert{t}]}
            \end{tabular}
          \end{example}
          \begin{block}<10->{A possible rule}
            \begin{itemize}
              \item \phonb{/θ/}{[t]}{s}{\#}
            \end{itemize}
          \end{block}
        }
        \only<11-14>{
          \begin{alertblock}{Insertion}
            % Insertion
When a sound is added between two similar sounds to create a distinction

          \end{alertblock}
          \begin{example}<12->
            \begin{tabular}{c c}
              \lexi{dance}            & \lexi{hamster} \\
              \uncover<13->{/ˈdæns/}  & \uncover<13->{/ˈhæm.stɹ̩/} \\
              \uncover<13->{[ˈdæn\alert{t}s]} & \uncover<13->{[ˈhæm\alert{p}.stɹ̩]}
            \end{tabular}
          \end{example}
          \begin{block}<14->{Some possible rules}
            \begin{itemize}
              \item \phonb{//}{[t]}{n}{s}
              \item \phonb{//}{[p]}{m}{s}
            \end{itemize}
          \end{block}
        }
        \only<15-17>{
          \begin{alertblock}{Deletion}
            % Deletion
When a unit has been removed from an utterance

          \end{alertblock}
          \begin{example}<16->
            \begin{tabular}{c c c}
              \lexi{govern}       & \lexi{governer} & \lexi{government} \\
              /ˈgʌ.və\alert{ɹn}/  & [ˈgʌ.və.nɹ̩]     & [ˈgʌ.vəɹ.mn̩t] \\
            \end{tabular}
          \end{example}
          \begin{block}<17->{Some possible rules}
            \begin{itemize}
              \item \phonb{/ɹ/}{[]}{ə}{nɹ̩}
              \item \phonb{/n/}{[]}{ɹ}{m}
            \end{itemize}
          \end{block}
        }
        \only<18-21>{
          \begin{alertblock}{Metathesis}
            % Metathesis
When two sounds in a word are interchanged

          \end{alertblock}
          \begin{example}<19->
            \begin{tabular}{c c}
              \href{https://youtu.be/h0JI33xyT8Q?t=30}{\lexi{ask}}  & \\
              \uncover<20->{[ˈask]}                                 & \uncover<20->{[ˈaks]}
            \end{tabular}
          \end{example}
          \begin{block}<21->{Two possible rules}
            \begin{itemize}
              \item \phonb{/sk/}{[ks]}{æ}{\#}
              \item \phonb{/ks/}{[sk]}{æ}{\#}
            \end{itemize}
          \end{block}
        }
        \only<22-24>{
          \begin{alertblock}{Strengthening}
            % Strengthening
When a sound becomes more anunciated in some way, essentially becoming more obstruent

          \end{alertblock}
          \begin{example}<23->
            \begin{tabular}{c c c}
              \lexi{the}  & \lexi{they} & \lexi{their} \\
              /ðə/       & /ˈðeɪ/      & /ˈðɛɹ/ \\
              {[}də]     & [ˈdeɪ]      & [ˈdɛɹ]
            \end{tabular}
          \end{example}
          \begin{block}<24->{A possible rule}
            \phonb{/ð/}{[d]}{\#}{V}
          \end{block}
        }
        \only<25-27>{
          \begin{alertblock}{Weakening}
            % Weakening
When a sound becomes less anunciated in some way, essentially becoming more sonorant

          \end{alertblock}
          \begin{example}<26->
            \begin{tabular}{c c c}
              \lexi{the}  & \lexi{they} & \lexi{their} \\
              /də/       & /ˈdeɪ/      & /ˈdɛɹ/ \\
              {[}ðə]     & [ˈðeɪ]      & [ˈðɛɹ]
            \end{tabular}
          \end{example}
          \begin{block}<27->{A possible rule}
            \phonb{/d/}{[ð]}{\#}{V}
          \end{block}
        }
      \end{frame}

    \subsection{\subonefour}
      \begin{frame}{\subonefour}
        \only<1-2>{
          \begin{block}{}
            Multiple rules are often applied, in which case \emph{order matters}
          \end{block}
          \begin{example}<2->
            \begin{tabular}{c c c}
              \lexi{writer} & /ˈɹaɪ.tɹ̩/  & [ˈɹəɪ.ɾɹ̩]
            \end{tabular}

            \vspace{0.3cm}
            Two rules apply:
            \begin{itemize}
              \item \phonb{/t/}{[ɾ]}{V}{V} (consonant flapping)
              \item \phonr{/aɪ/}{[əɪ]}{\phonfeat{+voiceless}} (diphthong raising)
            \end{itemize}
          \end{example}
        }
        \only<3-4>{
          \begin{block}{Two orders}
            \begin{tabular}{r | l | l}
              Underlying  & /ˈɹaɪ.tɹ̩/  & /ˈɹaɪ.tɹ̩/ \\
                          & ˈɹaɪ.ɾɹ̩    & ˈɹəɪ.tɹ̩ \\
                          & ---        & ˈɹəɪ.ɾɹ̩ \\
              Surface     & */ˈɹaɪ.ɾɹ̩/ & /ˈɹəɪ.ɾɹ̩/
            \end{tabular}
          \end{block}
          \begin{block}<4->{Correct order}
            raising $\rightarrow$ flapping
          \end{block}
        }
        \only<5-7>{
          \begin{block}{Two more rule types}
            \begin{itemize}
              \item \alert{Obligatory rules}
              \begin{itemize}
                \item<6-> % Obligatory rule
A phonological rule that always applies for a speaker

              \end{itemize}
              \item \alert{Optional rules}
              \begin{itemize}
                \item<6-> % Optional rule
A phonological rule that a speaker consciously decides to apply or not to apply

              \end{itemize}
            \end{itemize}
          \end{block}
          \begin{alertblock}<7>{}
            So called ``optional rules'' are part of language variation, studied in sociolinguistics
          \end{alertblock}
        }
      \end{frame}

    \subsection{\subonesix}
      \begin{frame}{\subonesix}
        \begin{block}{Try these}
          \textcite{dawson_language_2016}, chapter 3 exercises 13, 14, 15, and 16
        \end{block}
      \end{frame}
  \end{document}
