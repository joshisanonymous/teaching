%%%%%%%%%%%%%%%%%%%%%%%%%%%%%%%%%%%%%
%                                   %
% Compile with XeLaTeX              %
%                                   %
% Questions or comments:            %
%                                   %
% joshua dot mcneill at uga dot edu %
%                                   %
%%%%%%%%%%%%%%%%%%%%%%%%%%%%%%%%%%%%%

\documentclass{beamer}
  % Read in standard preamble (cosmetic stuff)
  %%%%%%%%%%%%%%%%%%%%%%%%%%%%%%%%%%%%%%%%%%%%%%%%%%%%%%%%%%%%%%%%
% This is a standard preamble used in for all slide documents. %
% It basically contains cosmetic settings.                     %
%                                                              %
% Joshua McNeill                                               %
% joshua dot mcneill at uga dot edu                            %
%%%%%%%%%%%%%%%%%%%%%%%%%%%%%%%%%%%%%%%%%%%%%%%%%%%%%%%%%%%%%%%%

% Beamer settings
% \usetheme{Berkeley}
\usetheme{CambridgeUS}
% \usecolortheme{dove}
% \usecolortheme{rose}
\usecolortheme{seagull}
\usefonttheme{professionalfonts}
\usefonttheme{serif}
\setbeamertemplate{bibliography item}{}

% Packages and settings
\usepackage{fontspec}
  \setmainfont{Charis SIL}
\usepackage{hyperref}
  \hypersetup{colorlinks=true,
              allcolors=blue}
\usepackage{graphicx}
  \graphicspath{{../../figures/}}
\usepackage{soul}
  \setstcolor{red}
\usepackage[normalem]{ulem}
\usepackage{enumerate}
\usepackage{tikz}
  \usetikzlibrary{trees}

% Document information
\author{M. McNeill}
\title[FREN1001]{Français 1001}
\institute{\url{joshua.mcneill@uga.edu}}
\date{}

%% Custom commands
% Lexical items
\newcommand{\lexi}[1]{\textit{#1}}
% Gloss
\newcommand{\gloss}[1]{`#1'}
\newcommand{\tinygloss}[1]{{\tiny`#1'}}
% Orthographic representations
\newcommand{\orth}[1]{$\langle$#1$\rangle$}
% Utterances (pragmatics)
\newcommand{\uttr}[1]{`#1'}
% Sentences (pragmatics)
\newcommand{\sent}[1]{\textit{#1}}
% Fixed length underlines
\newcommand{\funderline}[2][4cm]{
  \underline{\makebox[\ifdim\width>#1\width\else#1\fi]{#2}}
}
% Base dir for definitions
\newcommand{\defs}{../definitions}
\newcommand{\activity}[1]{
  \input{./activities/#1.tex}
}


  % Document information
  \title[Phonetics 1]{Phonetics, Part 1}

  %% Custom commands
  % Subsection/frame titles
  \newcommand{\suboneone}{Definition}
  \newcommand{\subtwoone}{Studying speech sounds}
  \newcommand{\subtwotwo}{The ``right'' phonetic alphabet}
  \newcommand{\subtwothree}{Types of speech sounds}
  \newcommand{\subtwofour}{Phonetic symbols for English}

\begin{document}
  % Read in the standard intro slides (title page and table of contents)
  %%%%%%%%%%%%%%%%%%%%%%%%%%%%%%%%%%%%%%%%%%%%%%%%%%%%%%%%%%%%%%%%
% This is a standard set of intro slides used in for all slide %
% documents. It basically contains the title page and table of %
% contents.                                                    %
%                                                              %
% Joshua McNeill                                               %
% joshua dot mcneill at uga dot edu                            %
%%%%%%%%%%%%%%%%%%%%%%%%%%%%%%%%%%%%%%%%%%%%%%%%%%%%%%%%%%%%%%%%

\begin{frame}
  \titlepage
  \tiny{Office: % Basically a variable for office hours location
T Gilbert 141/W Library 4th Fl
\\
        Office hours: % Basically a variable for office hours
T 11-12/W 11-12:30
}
\end{frame}

\begin{frame}
  \tableofcontents[hideallsubsections]
\end{frame}

\AtBeginSection[]{
  \begin{frame}
    \tableofcontents[currentsection,
                     hideallsubsections]
  \end{frame}
}


  \section{What is phonetics?}
    \subsection{\suboneone}
      \begin{frame}{\suboneone}
        \only<1-3>{
          \begin{block}<1-3>{}
            The study of the minimal units that make up a language
            \begin{itemize}
              \item In other words, consonants, vowels, intonation, and stress
            \end{itemize}
          \end{block}
          \begin{block}<2-3>{}
            The three broad areas of phonetics:
            \begin{itemize}
              \item \alert<3>{Articulatory phonetics}
                \begin{itemize}
                  \item The study of the production of these minimal units
                \end{itemize}
              \item \alert<3>{Acoustic phonetics}
                \begin{itemize}
                  \item The study of the acoustic properties of these minimal units and their transmission
                \end{itemize}
              \item Auditory phonetics
                \begin{itemize}
                  \item The study of the perception of these minimal units
                \end{itemize}
            \end{itemize}
          \end{block}
        }
        \only<4>{
          \begin{alertblock}{}
            This is an aspect of \emph{your} language (i.e., your personal mental lexicon and mental grammar)
          \end{alertblock}
        }
      \end{frame}

  \section{Representing speech sounds}
    \subsection{\subtwoone}
      \begin{frame}{\subtwoone}
        \only<1>{
          \begin{block}{Articulatory phonetics}
            Several tools are used:
            \begin{itemize}
              \item \href{https://www.youtube.com/watch?v=DcNMCB-Gsn8}{X-ray}
              \item Ultrasound
              \item Palatography
                \begin{itemize}
                  \item Dye is added to the tongue and roof of the mouth and a picture is taken after pronouncing a sound
                \end{itemize}
            \end{itemize}
          \end{block}
        }
        \only<2>{

        }
      \end{frame}

    \subsection{\subtwotwo}

    \subsection{\subtwothree}

    \subsection{\subtwofour}

\end{document}
