%%%%%%%%%%%%%%%%%%%%%%%%%%%%%%%%%%%%%
%                                   %
% Compile with XeLaTeX and biber    %
%                                   %
% Questions or comments:            %
%                                   %
% joshua dot mcneill at uga dot edu %
%                                   %
%%%%%%%%%%%%%%%%%%%%%%%%%%%%%%%%%%%%%

\documentclass{beamer}
  % Read in standard preamble (cosmetic stuff)
  %%%%%%%%%%%%%%%%%%%%%%%%%%%%%%%%%%%%%%%%%%%%%%%%%%%%%%%%%%%%%%%%
% This is a standard preamble used in for all slide documents. %
% It basically contains cosmetic settings.                     %
%                                                              %
% Joshua McNeill                                               %
% joshua dot mcneill at uga dot edu                            %
%%%%%%%%%%%%%%%%%%%%%%%%%%%%%%%%%%%%%%%%%%%%%%%%%%%%%%%%%%%%%%%%

% Beamer settings
% \usetheme{Berkeley}
\usetheme{CambridgeUS}
% \usecolortheme{dove}
% \usecolortheme{rose}
\usecolortheme{seagull}
\usefonttheme{professionalfonts}
\usefonttheme{serif}
\setbeamertemplate{bibliography item}{}

% Packages and settings
\usepackage{fontspec}
  \setmainfont{Charis SIL}
\usepackage{hyperref}
  \hypersetup{colorlinks=true,
              allcolors=blue}
\usepackage{graphicx}
  \graphicspath{{../../figures/}}
\usepackage[normalem]{ulem}
\usepackage{enumerate}

% Document information
\author{M. McNeill}
\title[FREN2001]{Français 2001}
\institute{\url{joshua.mcneill@uga.edu}}
\date{}

%% Custom commands
% Lexical items
\newcommand{\lexi}[1]{\textit{#1}}
% Gloss
\newcommand{\gloss}[1]{`#1'}
\newcommand{\tinygloss}[1]{{\tiny`#1'}}
% Orthographic representations
\newcommand{\orth}[1]{$\langle$#1$\rangle$}
% Utterances (pragmatics)
\newcommand{\uttr}[1]{`#1'}
% Sentences (pragmatics)
\newcommand{\sent}[1]{\textit{#1}}
% Base dir for definitions
\newcommand{\defs}{../definitions}


  % Packages and settings
  \usepackage[backend=biber, style=apa]{biblatex}
    \addbibresource{../references/References.bib}

  % Document information
  \subtitle[Syntactic Properties]{Syntactic Properties}

  %% Custom commands
  % Subsection/frame titles
  \newcommand{\suboneone}{What are they?}
  \newcommand{\subonetwo}{Word order}
  \newcommand{\subonethree}{Co-occurrence}
  \newcommand{\subonefour}{Arguments}
  \newcommand{\subonefive}{Adjunct}
  \newcommand{\subonesix}{Practice}

\begin{document}
  % Read in the standard intro slides (title page and table of contents)
  %%%%%%%%%%%%%%%%%%%%%%%%%%%%%%%%%%%%%%%%%%%%%%%%%%%%%%%%%%%%%%%%
% This is a standard set of intro slides used in for all slide %
% documents. It basically contains the title page and table of %
% contents.                                                    %
%                                                              %
% Joshua McNeill                                               %
% joshua dot mcneill at uga dot edu                            %
%%%%%%%%%%%%%%%%%%%%%%%%%%%%%%%%%%%%%%%%%%%%%%%%%%%%%%%%%%%%%%%%

\begin{frame}
  \titlepage
  \tiny{Office: % Basically a variable for office hours location
Gilbert 121\\
        Office hours: % Basically a variable for office hours
 lundi, mercredi, vendredi 10:10--11:10
}
\end{frame}

\begin{frame}
  \tableofcontents[hideallsubsections]
\end{frame}

\AtBeginSection[]{
  \begin{frame}
    \tableofcontents[currentsection,
                     hideallsubsections]
  \end{frame}
}


  \section{Syntactic Properties}
    \subsection{\suboneone}
      \begin{frame}{\suboneone}
        \begin{definition}
          % Syntactic property
A property of a linguistic expression that dictates how it can be combined with other linguistic expressions

        \end{definition}
        \begin{alertblock}{}
          Both lexical expressions and phrasal expressions have syntactic properties
        \end{alertblock}
        \begin{block}{Two types}
          \begin{itemize}
            \item Word order
            \item Co-occurrence
          \end{itemize}
        \end{block}
      \end{frame}

    \subsection{\subonetwo}
      \begin{frame}[t]{\subonetwo}
        \begin{example}
          \begin{enumerate}
            \item The dog ate filet mignon.
            \item *The dog filet mignon ate.
            \item *Ate the dog filet mignon.
            \item *Ate filet mignon the dog.
          \end{enumerate}
        \end{example}
        \only<1-2>{
          \begin{block}{What's wrong with (2-4)?}
            \uncover<2->{
            They violate the word order dictated by English sentences: S(ubject)-V(erb)-O(object)
            }
          \end{block}
        }
        \only<3>{
          \begin{block}{Word orders in other languages}
            \begin{itemize}
              \item SOV (2), more common than SVO (e.g., Korean)
              \item VSO (3), rarer (e.g., Arabic)
              \item VOS (4), OVS, OSV, all very rare
            \end{itemize}
          \end{block}
        }
      \end{frame}

      \begin{frame}[t]{\subonetwo}
        \begin{alertblock}{}
          Expressions have set word orders, not whole languages
        \end{alertblock}
        \only<2>{
          \begin{example}
            \begin{enumerate}
              \item Is that dog a bulldog?
              \item Now filet mignon, that bulldog likes.
            \end{enumerate}
          \end{example}
          \begin{block}{}
            \begin{itemize}
              \item Interrogatives like (1) are VSO
              \item Topicalized sentences like (2) are OSV
            \end{itemize}
          \end{block}
        }
        \only<3->{
          \begin{block}{More exceptions}
            Some languages have no fixed word order for sentence-level expressions
            \begin{itemize}
              \item Dyirbal allows SVO, SOV, VSO, VOS, OVS, and OSV for declarative sentences
            \end{itemize}
          \end{block}
        }
      \end{frame}

      \begin{frame}[t]{\subonetwo}
        \begin{block}{Other expressions that dictate word order}
          \begin{itemize}
            \item Nouns
            \item Prepositions
          \end{itemize}
        \end{block}
        \only<2>{
          \begin{example}
            \begin{tabular}{@{} l l l l l @{}}
              1)  & these & books         &               & (English) \\
              2)  & liv   & -sa           & -ye           & (Louisiana Creole) \\
                  & book  & -\textsc{dem} & -\textsc{pl}  &
            \end{tabular}
          \end{example}
          \begin{block}{}
            \begin{itemize}
              \item Demonstrative $\rightarrow$ Noun (English)
              \item Noun $\rightarrow$ Demonstrative (LC)
            \end{itemize}
          \end{block}
        }
        \only<3>{
          \begin{example}
            \begin{tabular}{@{} l l l l l @{}}
              3)  & with          & this    & child & (English)\\
              4)  & kono          & kodomo  & to    & (Japanese) \\
                  & \textsc{dem}  & child   & with  &
            \end{tabular}
          \end{example}
          \begin{block}{}
            \begin{itemize}
              \item Preposition $\rightarrow$ Noun (English)
              \item Noun $\rightarrow$ Preposition (Japanese)
            \end{itemize}
          \end{block}
        }
      \end{frame}

    \subsection{\subonethree}
      \begin{frame}{\subonethree}
        \begin{block}{}
          Essentially, which expressions are required by other expressions
        \end{block}
        \begin{block}{Two classifications for relationships between expressions}
          \begin{itemize}
            \item Arguments
            \item Adjuncts
          \end{itemize}
        \end{block}
      \end{frame}

    \subsection{\subonefour}
      \begin{frame}[t]{\subonefour}
        \only<1-2>{
          \begin{alertblock}{Arguments}
            % Argument
An expression that is required by another expression

          \end{alertblock}
        }
        \only<3->{
          \begin{alertblock}{Complements}
            % Complement
Any non-subject argument

          \end{alertblock}
        }
        \begin{example}<2->
          \begin{itemize}
            \item The dog devoured a filet mignon.
            \item *The dog devoured.
          \end{itemize}
        \end{example}
        \only<2>{
          \begin{block}{}
            \lexi{Devour} requires an object noun, so \lexi{a filet mignon} is an argument of \lexi{devour}
          \end{block}
        }
        \only<4->{
          \begin{block}{}
            \lexi{The dog} and \lexi{a filet mignon} are arguments, but only \lexi{a filet mignon} is a complement
          \end{block}
        }
      \end{frame}

      \begin{frame}{\subonefour}
        \begin{block}{}
          Word order requirements have no bearing on argument requirements
        \end{block}
        \begin{block}{Serbo-Croatian}
          \begin{tabular}{@{} l l l l l @{}}
            1)  & a)  & Marija  & voli    & muziku \\
                &     & Marija  & likes   & music \\
                &     & \multicolumn{3}{l}{`Marija likes music.'} \\
                & b)  & Marija  & muziku  & voli \\
                & c)  & voli    & muziku  & Marija \\
                & d)  & voli    & Marija  & muziku \\
                & e)  & muziku  & voli    & Marija \\
                & f)  & muziku  & Marija  & voli \\
            2)  &     & *Marija & voli    &
          \end{tabular}
        \end{block}
      \end{frame}

      \begin{frame}{\subonefour}
        \begin{block}{}
          Arguments are \emph{not} always nouns
        \end{block}
        \begin{example}
          \begin{enumerate}
            \item *The dog wondered filet mignon.
            \item The dog wondered about filet mignon.
          \end{enumerate}
        \end{example}
        \begin{block}{}
          \lexi{Wonder} requires a preposition, not a noun
        \end{block}
      \end{frame}

      \begin{frame}{\subonefour}
        \begin{block}{}
          An expression can require more than one complement
        \end{block}
        \begin{example}
          \begin{enumerate}
            \item The man put the filet mignon in the trash.
          \end{enumerate}
        \end{example}
        \begin{block}{Two complements}
          \begin{itemize}
            \item the filet mignon
            \item in the trash
          \end{itemize}
        \end{block}
      \end{frame}

      \begin{frame}{\subonefour}
        \begin{block}{}
          Some expressions require arguments to take very specific forms
        \end{block}
        \begin{example}
          \begin{itemize}
            \item It rained.
            \item It's raining.
            \item (?)The sky rains.
            \item (?)The clouds rain.
          \end{itemize}
        \end{example}
        \begin{block}{}
          \lexi{Rain} seems to require \lexi{it} except in more poetic expressions
        \end{block}
      \end{frame}

      \begin{frame}{\subonefour}
        \begin{block}{}
          It's not just verbs that take arguments
        \end{block}
        \begin{example}
          \begin{enumerate}
            \item *The dog wondered about
            \item *The dog is fond
          \end{enumerate}
        \end{example}
        \begin{block}{}
          \begin{itemize}
            \item Prepositions like \lexi{about} take arguments
            \item Adjectives like \lexi{fond} take arguments
          \end{itemize}
        \end{block}
      \end{frame}

      \begin{frame}{\subonefour}
        \begin{alertblock}{}
          These features do not necessarily hold true for other languages
        \end{alertblock}
        \begin{block}{Italian}
          \begin{tabular}{@{} l l l l l l @{}}
            1)  & (io)  & ho                & comprato  & un  & libro \\
                & (I)   & have-\textsc{1sg} & bought    & a   & book \\
                & \multicolumn{5}{l}{`I bought a book.'}
          \end{tabular}
        \end{block}
      \end{frame}

    \subsection{\subonefive}
      \begin{frame}[t]{\subonefive}
        \begin{definition}
          % Adjunct
An expression that is optional (i.e., it's not required by any other expression in the sentence)

        \end{definition}
        \only<2>{
          \begin{example}
            \begin{itemize}
              \item Louis likes music.
              \item Louis likes loud music.
              \item Louis likes loud fast music.
              \item Louis likes loud fast swinging music.
            \end{itemize}
          \end{example}
          \begin{block}{}
            \lexi{Loud}, \lexi{fast}, and \lexi{swinging} are all adjuncts
          \end{block}
        }
        \only<3>{
          \begin{block}{Other defining features}
            \begin{itemize}
              \item You can add as many as you want (e.g., \lexi{loud fast swinging})
              \item Sometimes called modifiers because that's what they typically do
            \end{itemize}
          \end{block}
        }
      \end{frame}

      \begin{frame}{\subonefive}
        \begin{block}{}
          Other types of expressions can be adjuncts, too
        \end{block}
        \begin{example}<2->
          \begin{enumerate}
            \item Louis liked jazz last year.
            \item Louis listens to death metal in January.
          \end{enumerate}
        \end{example}
        \begin{block}<2->{}
          \begin{itemize}
            \item \lexi{Last year} is an adverb
            \item \lexi{In January} is a preposition
          \end{itemize}
        \end{block}
      \end{frame}

      \begin{frame}{\subonefive}
        \begin{alertblock}{}
          Just because it can be an adjunct, doesn't mean it is
        \end{alertblock}
        \begin{example}<2->
          \begin{itemize}
            \item Last year was a swinging year.
          \end{itemize}
        \end{example}
        \begin{block}<2->{}
          \lexi{Last year} is a subject noun here (i.e., an argument)
        \end{block}
      \end{frame}

      % Possibly add something from the agreement section here, but it seems off topic and doesn't come up again

    \subsection{\subonesix}
      \begin{frame}{\subonesix}
        \begin{block}{Try these}
          \textcite{dawson_language_2016}, chapter 5 exercise 4, 5, 7, 8, and 9
        \end{block}
      \end{frame}
\end{document}
