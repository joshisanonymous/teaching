%%%%%%%%%%%%%%%%%%%%%%%%%%%%%%%%%%%%%
%                                   %
% Compile with XeLaTeX and biber    %
%                                   %
% Questions or comments:            %
%                                   %
% joshua dot mcneill at uga dot edu %
%                                   %
%%%%%%%%%%%%%%%%%%%%%%%%%%%%%%%%%%%%%

\documentclass{beamer}
  % Read in standard preamble (cosmetic stuff)
  %%%%%%%%%%%%%%%%%%%%%%%%%%%%%%%%%%%%%%%%%%%%%%%%%%%%%%%%%%%%%%%%
% This is a standard preamble used in for all slide documents. %
% It basically contains cosmetic settings.                     %
%                                                              %
% Joshua McNeill                                               %
% joshua dot mcneill at uga dot edu                            %
%%%%%%%%%%%%%%%%%%%%%%%%%%%%%%%%%%%%%%%%%%%%%%%%%%%%%%%%%%%%%%%%

% Beamer settings
% \usetheme{Berkeley}
\usetheme{CambridgeUS}
% \usecolortheme{dove}
% \usecolortheme{rose}
\usecolortheme{seagull}
\usefonttheme{professionalfonts}
\usefonttheme{serif}
\setbeamertemplate{bibliography item}{}

% Packages and settings
\usepackage{fontspec}
  \setmainfont{Charis SIL}
\usepackage{hyperref}
  \hypersetup{colorlinks=true,
              allcolors=blue}
\usepackage{graphicx}
  \graphicspath{{../../figures/}}
\usepackage[normalem]{ulem}
\usepackage{enumerate}

% Document information
\author{M. McNeill}
\title[FREN2001]{Français 2001}
\institute{\url{joshua.mcneill@uga.edu}}
\date{}

%% Custom commands
% Lexical items
\newcommand{\lexi}[1]{\textit{#1}}
% Gloss
\newcommand{\gloss}[1]{`#1'}
\newcommand{\tinygloss}[1]{{\tiny`#1'}}
% Orthographic representations
\newcommand{\orth}[1]{$\langle$#1$\rangle$}
% Utterances (pragmatics)
\newcommand{\uttr}[1]{`#1'}
% Sentences (pragmatics)
\newcommand{\sent}[1]{\textit{#1}}
% Base dir for definitions
\newcommand{\defs}{../definitions}


  % Packages and settings
  \usepackage[backend=biber, style=apa]{biblatex}
    \addbibresource{../references/References.bib}

  % Document information
  \subtitle[Syntactic Categories]{Syntactic Categories}

  %% Custom commands
  % Subsection/frame titles
  \newcommand{\suboneone}{What are they?}
  \newcommand{\subonetwo}{Some major categories}
  \newcommand{\subonethree}{Summarizing}
  \newcommand{\subonefour}{Pratice}

\begin{document}
  % Read in the standard intro slides (title page and table of contents)
  %%%%%%%%%%%%%%%%%%%%%%%%%%%%%%%%%%%%%%%%%%%%%%%%%%%%%%%%%%%%%%%%
% This is a standard set of intro slides used in for all slide %
% documents. It basically contains the title page and table of %
% contents.                                                    %
%                                                              %
% Joshua McNeill                                               %
% joshua dot mcneill at uga dot edu                            %
%%%%%%%%%%%%%%%%%%%%%%%%%%%%%%%%%%%%%%%%%%%%%%%%%%%%%%%%%%%%%%%%

\begin{frame}
  \titlepage
  \tiny{Office: % Basically a variable for office hours location
Gilbert 121\\
        Office hours: % Basically a variable for office hours
 lundi, mercredi, vendredi 10:10--11:10
}
\end{frame}

\begin{frame}
  \tableofcontents[hideallsubsections]
\end{frame}

\AtBeginSection[]{
  \begin{frame}
    \tableofcontents[currentsection,
                     hideallsubsections]
  \end{frame}
}


  \section{Syntactic Categories}
    \subsection{\suboneone}
      \begin{frame}{\suboneone}
        \begin{definition}
          % Syntactic category
A group to which expressions that share the same syntactic distribution and syntactic properties belong

        \end{definition}
        \begin{example}<2->
          \begin{enumerate}
            \item \lexi{the dog} \hfill (noun phrase/NP)
            \item \lexi{in January} \hfill (prepositional phrase/PP)
            \item \lexi{The dog listens to death metal in January.} \hfill (sentence/S)
          \end{enumerate}
        \end{example}
        \begin{alertblock}<3->{}
          This does not include lexical categories (e.g., \lexi{dog}, an N)
        \end{alertblock}
      \end{frame}

      \begin{frame}{\suboneone}
        \begin{alertblock}{Syntactic distribution}
          % Syntactic distribution
The set of syntactic environments in which an expression can occur

        \end{alertblock}
        \begin{example}<2->
          \begin{enumerate}
            \item The dog listens to death metal
            \item The dog listens to jazz
            \item The dog listens to shoegaze
          \end{enumerate}
        \end{example}
        \begin{block}<2->{}
          \lexi{Death metal}, \lexi{jazz}, and \lexi{shoegaze} are all have the same distribution and are all NPs
        \end{block}
      \end{frame}

      \begin{frame}{\suboneone}
        \begin{block}{}
          Meaning is not so important
        \end{block}
        \begin{example}
          \begin{enumerate}
            \item Cooking is fun.
            \item Games are fun.
          \end{enumerate}
        \end{example}
        \begin{block}{}
          Both are NPs but \lexi{cooking} is an action and \lexi{games} a thing
        \end{block}
      \end{frame}

    \subsection{\subonetwo}
      \begin{frame}{\subonetwo}
        \begin{block}{S: Sentence}
          Made up of an NP and a VP
        \end{block}
        \begin{block}{To identify Ss}
          \begin{itemize}
            \item Sally thinks that \hrulefill
          \end{itemize}
        \end{block}
        \begin{example}<2->
          \begin{enumerate}
            \item \only<3->{*}Sally thinks that the dog.
            \item Sally thinks that the dog listens to death metal.
          \end{enumerate}
        \end{example}
      \end{frame}

      \begin{frame}[t]{\subonetwo}
        \begin{block}{NP: Noun Phrase}
          Made up of at least an N
        \end{block}
        \begin{block}{To identify NPs}
          Replace with a pronoun
        \end{block}
        \only<2-3>{
          \begin{example}
            \begin{enumerate}
              \item The dog listens to death metal.
              \item She listens to it.
              \item \only<3->{*}The she listens to death it.
            \end{enumerate}
          \end{example}
        }
        \only<4>{
          \begin{block}{Det: Determiner (a lexical category)}
            \begin{itemize}
              \item Articles: \lexi{the}, \lexi{a}, etc.
              \item Demonstratives: \lexi{this}, \lexi{those}, etc.
              \item Quantifiers: \lexi{some}, \lexi{many}, \lexi{few}, etc.
              \item Possessives: \lexi{my}, \lexi{your}, \lexi{their}, etc.
            \end{itemize}
            Can combine with Ns but not with NPs
          \end{block}
        }
        \only<5>{
          \begin{block}{NPs can be very long}
            \begin{enumerate}
              \item The tiny brown dog with a short tail from Athens listens to death metal.
              \item It listens to death metal.
            \end{enumerate}
            Contains APs and PPs
          \end{block}
        }
      \end{frame}

      \begin{frame}[t]{\subonetwo}
        \begin{block}{VP: Verb Phrase}
          Made up of at least a V
        \end{block}
        \begin{block}{To identify VPs}
          \begin{itemize}
            \item Add an NP to its left and you should have an S
            \item Replace with \lexi{does so} or \lexi{did so}
          \end{itemize}
        \end{block}
        \only<2>{
          \begin{example}
            \begin{enumerate}
              \item \rule{1cm}{0.15mm} ate gumbo. $\rightarrow$ Louis ate gumbo.
              \item Louis slept. $\rightarrow$ Louis did so.
            \end{enumerate}
          \end{example}
        }
        \only<3>{
          \begin{block}{}
            \begin{enumerate}
              \item Louis gave the tall slender man with a hat and cane the rest of his hot tasty gumbo
              \item[$\rightarrow$] Louis did so.
              \item Louis did so his gumbo.
            \end{enumerate}
          \end{block}
        }
      \end{frame}

      \begin{frame}{\subonetwo}
        \begin{block}{PP: Prepositional Phrase}
          Made up of at least a P and a NP
        \end{block}
        \begin{example}
          \begin{enumerate}
            \item \only<2->{[NP }A tall man \only<2->{[PP }with a hat\only<2->{]]}
            \item \only<3->{[VP }ate gumbo \only<3->{[PP }at the counter.\only<3->{]]}
          \end{enumerate}
        \end{example}
        \begin{block}<2->{Distribution}
          \begin{itemize}
            \item<2-> In NPs (1)
            \item<3-> In VPs (2)
          \end{itemize}
        \end{block}
      \end{frame}

      \begin{frame}{\subonetwo}
        \begin{block}{AP: Adjectival Phrase}
          Made up of at least an Adj
        \end{block}
        \begin{example}
          \begin{enumerate}
            \item \only<2->{[NP }The \only<2->{[AP }tall\only<2->{]} man\only<2->{]}
            \item \only<3->{[VP }is \only<3->{[AP }slender\only<3->{]}.\only<3->{]}
          \end{enumerate}
        \end{example}
        \begin{block}<2->{Distribution}
          \begin{itemize}
            \item<2-> In NPs (1)
            \item<3-> In VPs (2)
          \end{itemize}
        \end{block}
      \end{frame}

      \begin{frame}{\subonetwo}
        \begin{block}{AdvP: Adverbial Phrase}
          Made up of at least an Adv
        \end{block}
        \begin{example}
          \begin{enumerate}
            \item \only<2->{[AP [AdvP }very\only<2->{]} tall\only<2->{]}
          \end{enumerate}
        \end{example}
        \begin{block}<2->{Distribution}
          For now, only in APs
        \end{block}
      \end{frame}

    \subsection{\subonethree}
      \begin{frame}{\subonethree}
        \small
          % This is a reusable table that lists the major syntactic categories
\begin{tabular}{l l l}
  \textbf{Category}         & \textbf{Structure}  & \textbf{Example} \\
  \hline
  \multicolumn{3}{c}{\emph{Syntactic Categories}} \\
  S: Sentence               & An NP \& VP         & the man is slender \\
  NP: Noun Phrase           & At least an N       & the dog \\
  VP: Verb Phrase           & At least a V        & ate gumbo \\
  PP: Prepositional Phrase  & A P \& NP           & at the counter \\
  AP: Adjectival Phrase     & At least an Adj     & tall \\
  AdvP: Adverbial Phrase    & At least an Adv     & very \\
  \multicolumn{3}{c}{\emph{Lexical Categories}} \\
  Det: Determiner           &                     & the \\
  N: Noun                   &                     & dog \\
  V: Verb                   &                     & ate \\
  P: Preposition            &                     & at \\
  Adj: Adjective            &                     & tall \\
  Adv: Adverb               &                     & very
\end{tabular}

      \end{frame}

    \subsection{\subonefour}
      \begin{frame}{\subonefour}
        \begin{block}{Try these}
          \textcite{dawson_language_2016}, chapter 5 exercises 17 and 19
        \end{block}
      \end{frame}

\end{document}
