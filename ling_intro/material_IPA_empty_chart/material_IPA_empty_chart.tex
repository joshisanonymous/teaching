%%%%%%%%%%%%%%%%%%%%%%%%%%%%%%%%%%%%%%%%%%%%%%%%%%%%%%%%%%
% This compiles both the English IPA consonant and vowel %
% charts (_consonants and _vowels) into an actual        %
% document.                                              %
%                                                        %
% Compile with XeLaTeX                                   %
%                                                        %
% -Joshua McNeill (joshua dot mcneill at uga dot edu)    %
%%%%%%%%%%%%%%%%%%%%%%%%%%%%%%%%%%%%%%%%%%%%%%%%%%%%%%%%%%

\documentclass{article}
  % Read in standard preamble (cosmetic stuff)
  %%%%%%%%%%%%%%%%%%%%%%%%%%%%%%%%%%%%%%%%%%%%%%%%%%%%%%%%%%%%%%%%
% This is a standard preamble used in for all materials        %
% documents. It basically contains cosmetic settings.          %
%                                                              %
% Joshua McNeill                                               %
% joshua dot mcneill at uga dot edu                            %
%%%%%%%%%%%%%%%%%%%%%%%%%%%%%%%%%%%%%%%%%%%%%%%%%%%%%%%%%%%%%%%%

% Packages and settings
\usepackage{fontspec}
  \setmainfont{Charis SIL}
\usepackage{hyperref}
  \hypersetup{colorlinks=true,
              allcolors=blue}


  % Packages and settings
  \usepackage{adjustbox}
  \usepackage{tikz}
  \usepackage{hyperref}
    \hypersetup{colorlinks=true,
                allcolors=blue}

  % Document information
  \title{Empty IPA charts}
  \date{}

  %% Custom commands
  % Space between heading and chart
  \newcommand{\headspace}{0.4cm}

\begin{document}
  \maketitle
  CONSONANTS

  \vspace{\headspace}

  \begin{adjustbox}{width=\textwidth}
    % Read in the consonants IPA chart
    %%%%%%%%%%%%%%%%%%%%%%%%%%%%%%%%%%%%%%%%%%%%%%%%%%%%%%%%%%
% This creates an English IPA chart for consonants       %
%                                                        %
% Compiled from material_IPA_en_chart.tex when a         %
% standalone document is needed                          %
%                                                        %
% -Joshua McNeill (joshua dot mcneill at uga dot edu)    %
%%%%%%%%%%%%%%%%%%%%%%%%%%%%%%%%%%%%%%%%%%%%%%%%%%%%%%%%%%

\begin{tabular}{| l | c c | c c | c c | c c | c c | c c | c c | c c | c c | c c | c c |}
  \hline
  & \multicolumn{2}{c}{Bilabial} & \multicolumn{2}{c}{Labiodental} & \multicolumn{2}{c}{Dental} & \multicolumn{2}{c}{Alveolar} & \multicolumn{2}{c}{Postalveolar} & \multicolumn{2}{c}{Retroflex} & \multicolumn{2}{c}{Palatal} & \multicolumn{2}{c}{Velar} & \multicolumn{2}{c}{Uvular} & \multicolumn{2}{c}{Pharyngeal} & \multicolumn{2}{c}{Glottal} \\
  \hline
  Plosive/Stop &   &   &   &   &   &   &   &   &   &   & & & &   &   &   & & & & &   & \\
  \hline
  Nasal        &   &   &   &   &   &   &   &   &   &   & & & &   &   &   & & & & &   & \\
  \hline
  Trill        &   &   &   &   &   &   &   &   &   &   & & & &   &   &   & & & & &   & \\
  \hline
  Tap/Flap     &   &   &   &   &   &   &   &   &   &   & & & &   &   &   & & & & &   & \\
  \hline
  Fricative    &   &   &   &   &   &   &   &   &   &   & & & &   &   &   & & & & &   & \\
  \hline
  \begin{tabular}{@{} l @{}}
    Lateral \\
    fricative
  \end{tabular}&   &   &   &   &   &   &   &   &   &   & & & &   &   &   & & & & &   & \\
  \hline
  Approximant  &   &   &   &   &   &   &   &   &   &   & & & &   &   &   & & & & &   & \\
  \hline
  \begin{tabular}{@{} l @{}}
    Lateral\\
    approximant
  \end{tabular}&   &   &   &   &   &   &   &   &   &   & & & &   &   &   & & & & &   & \\
  \hline
\end{tabular}

  \end{adjustbox}

  {\tiny *Symbols on the right side of a box represent voiced sounds and on the left voiceless sounds.}

  \vspace{2cm}

  \parbox[l]{0.5\textwidth}{
    OTHER SYMBOLS

    \vspace{\headspace}

    \begin{adjustbox}{width=0.4\textwidth}
      \begin{tabular}{c l}
           & Voiced labial-velar approximant \\
           & Voiceless affricate \\
           & Voiced affricate
      \end{tabular}
    \end{adjustbox}
  }
  \parbox[r]{0.5\textwidth}{
    VOWELS

    \vspace{\headspace}

    \begin{adjustbox}{width=0.4\textwidth}
      % Read in the vowels IPA chart
      \input{material_IPA_empty_chart_vowels.tex}
    \end{adjustbox}
  }

  \begin{flushright}
    {\tiny
      Symbols on the right side of a node represent rounded vowels and on the left unrounded.

      Vowels from top to bottom are close/high to mid-close/high to mid-open/low to open/low vowels.

      Vowels not on a line are near-high vowels.

      Vowels from left to right are front to back.

      [ ] is a mid central unrounded vowel.
    }
  \end{flushright}
\end{document}
