%%%%%%%%%%%%%%%%%%%%%%%%%%%%%%%%%%%%%
%                                   %
% Compile with XeLaTeX and biber    %
%                                   %
% Questions or comments:            %
%                                   %
% joshua dot mcneill at uga dot edu %
%                                   %
%%%%%%%%%%%%%%%%%%%%%%%%%%%%%%%%%%%%%

\documentclass{beamer}
  % Read in standard preamble (cosmetic stuff)
  %%%%%%%%%%%%%%%%%%%%%%%%%%%%%%%%%%%%%%%%%%%%%%%%%%%%%%%%%%%%%%%%
% This is a standard preamble used in for all slide documents. %
% It basically contains cosmetic settings.                     %
%                                                              %
% Joshua McNeill                                               %
% joshua dot mcneill at uga dot edu                            %
%%%%%%%%%%%%%%%%%%%%%%%%%%%%%%%%%%%%%%%%%%%%%%%%%%%%%%%%%%%%%%%%

% Beamer settings
% \usetheme{Berkeley}
\usetheme{CambridgeUS}
% \usecolortheme{dove}
% \usecolortheme{rose}
\usecolortheme{seagull}
\usefonttheme{professionalfonts}
\usefonttheme{serif}
\setbeamertemplate{bibliography item}{}

% Packages and settings
\usepackage{fontspec}
  \setmainfont{Charis SIL}
\usepackage{hyperref}
  \hypersetup{colorlinks=true,
              allcolors=blue}
\usepackage{graphicx}
  \graphicspath{{../../figures/}}
\usepackage{soul}
  \setstcolor{red}
\usepackage[normalem]{ulem}
\usepackage{enumerate}
\usepackage{tikz}
  \usetikzlibrary{trees}

% Document information
\author{M. McNeill}
\title[FREN1001]{Français 1001}
\institute{\url{joshua.mcneill@uga.edu}}
\date{}

%% Custom commands
% Lexical items
\newcommand{\lexi}[1]{\textit{#1}}
% Gloss
\newcommand{\gloss}[1]{`#1'}
\newcommand{\tinygloss}[1]{{\tiny`#1'}}
% Orthographic representations
\newcommand{\orth}[1]{$\langle$#1$\rangle$}
% Utterances (pragmatics)
\newcommand{\uttr}[1]{`#1'}
% Sentences (pragmatics)
\newcommand{\sent}[1]{\textit{#1}}
% Fixed length underlines
\newcommand{\funderline}[2][4cm]{
  \underline{\makebox[\ifdim\width>#1\width\else#1\fi]{#2}}
}
% Base dir for definitions
\newcommand{\defs}{../definitions}
\newcommand{\activity}[1]{
  \input{./activities/#1.tex}
}


  % Packages and settings

  % Document information
  \subtitle[Solidarité et subjonctif]{La solidarité et... le subjonctif!}

\begin{document}
  % Read in the standard intro slides (title page and table of contents)
  \begin{frame}
    \titlepage
    \tiny{Office: % Basically a variable for office hours location
Zoom (ID 978 2791 8221)
\\
          Office hours: % Basically a variable for office hours
 mercredi 10h15--13h15
}
  \end{frame}

  \begin{frame}{Le bénévolat}
    \begin{columns}
      \column{0.6\textwidth}
        \begin{enumerate}
          \item \underline{\uncover<2->{ a }} un endroit où on stocke et distribue les aliments aux personnes qui en on besoin
          \item \underline{\uncover<3->{ e }} une association créée pour aider les autres et non pas pour gagner de l'argent
          \item \underline{\uncover<4->{ f }} une personne qui a dû quitter son pays et se retrouve dans un autre pays
          \item \underline{\uncover<5->{ b }} une personne qui fait don de ses services pour aider les autres
          \item \underline{\uncover<6->{ c }} une action avec l'objectif de recevoir de l'argent pour une cause
          \item \underline{\uncover<7->{ d }} une action de bénévole
        \end{enumerate}
      \column{0.4\textwidth}
        \begin{description}
          \item[a.] une banque alimentaire
          \item[b.] un/e bénévole
          \item[c.] une collecte de fonds
          \item[d.] un événement
          \item[e.] un organisme à but non lucratif
          \item[f.] un/e réfugié/e
        \end{description}
    \end{columns}
  \end{frame}

  \begin{frame}{Le subjonctif avec des expressions d'émotion}
    \begin{enumerate}
      \item Elle est contente que vous \underline{\uncover<2->{fassiez}} (faire) don des vêtements pour son association.
      \item C'est dommage que tu ne \underline{\uncover<3->{puisses}} (pouvoir) pas participer.
      \item Il est malheureux que les SDF ne \underline{\uncover<4->{soient}} (être) pas servis.
      \item Je regrette que nous \underline{\uncover<5->{partions}} (partir) si tôt.
      \item C'est bon que tu \underline{\uncover<6->{viennes}} (venir) avec nous.
      \item Ils sont ravis qu'on \underline{\uncover<7->{ait}} (avoir) un examen.
    \end{enumerate}
  \end{frame}

  \begin{frame}{}
    \begin{center}
      \Large Quiz
    \end{center}
  \end{frame}

  \begin{frame}{Que d'émotions!}
    Avec un/e partenaire, imagine que ton professeur fait ces annonces.
    Quelle est ta réaction?
    \begin{description}
      \item[] \textbf{Modèle:} \emph{Vous aurez un examen vendredi.}
      \item[E1:] \alert{C'est dommange qu}'on ait un examen vendredi.
      \item[E2:] \alert{Je suis étonné/e} qu'on ait un autre examen.
    \end{description}
    \begin{enumerate}
      \item Il n'y aura pas de cours demain.
      \item Tout le monde ira au restaurant ensemble ce week-end.
      \item Je vous achèterai un souvenir en France cet été.
      \item Vous n'aurez pas d'examen final.
      \item Les résultats du dernier examen sont excellents.
      \item Vous faites beaucoup de progrès en français.
    \end{enumerate}
  \end{frame}

  \begin{frame}{Posters et slogans}
    En groupes de 3 ou 4, vous allez faire des posters numériques.
    Imaginez une manifestation sur le campus du thème que le prof vous donne.
    Utilisez le Google Slides lié sur eLC pour faire un poster pour chaque personne dans votre groupe.
    Les posters auront des slogans en français.
    Soyez créatifs!
    \begin{columns}[t]
      \column{0.55\textwidth}
        \begin{description}
          \item[] \textbf{Modèle:} \emph{Des slogans pour les transports en commun}
          \item[E1:] Vive le tramway et le métro!
          \item[E2:] À bas les grosses voitures!
          \item[E3:] Prenez le train ou vous êtes un crétin!
        \end{description}
      \column{0.45\textwidth}
        \begin{itemize}
          \item[] Des constructions utiles:
          \item (Les verbes à l'impératif)
          \item À bas ... \gloss{Down with}
          \item Plus de ... \gloss{No more}
          \item Vive ...
          \item Non à ...
          \item Oui à ...
        \end{itemize}
    \end{columns}
  \end{frame}

  \begin{frame}{}
    \begin{center}
      \Large Questions?
    \end{center}
  \end{frame}
\end{document}
