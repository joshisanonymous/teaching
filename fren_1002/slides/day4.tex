%%%%%%%%%%%%%%%%%%%%%%%%%%%%%%%%%%%%%
%                                   %
% Compile with XeLaTeX and biber    %
%                                   %
% Questions or comments:            %
%                                   %
% joshua dot mcneill at uga dot edu %
%                                   %
%%%%%%%%%%%%%%%%%%%%%%%%%%%%%%%%%%%%%

\documentclass{beamer}
  % Read in standard preamble (cosmetic stuff)
  %%%%%%%%%%%%%%%%%%%%%%%%%%%%%%%%%%%%%%%%%%%%%%%%%%%%%%%%%%%%%%%%
% This is a standard preamble used in for all slide documents. %
% It basically contains cosmetic settings.                     %
%                                                              %
% Joshua McNeill                                               %
% joshua dot mcneill at uga dot edu                            %
%%%%%%%%%%%%%%%%%%%%%%%%%%%%%%%%%%%%%%%%%%%%%%%%%%%%%%%%%%%%%%%%

% Beamer settings
% \usetheme{Berkeley}
\usetheme{CambridgeUS}
% \usecolortheme{dove}
% \usecolortheme{rose}
\usecolortheme{seagull}
\usefonttheme{professionalfonts}
\usefonttheme{serif}
\setbeamertemplate{bibliography item}{}

% Packages and settings
\usepackage{fontspec}
  \setmainfont{Charis SIL}
\usepackage{hyperref}
  \hypersetup{colorlinks=true,
              allcolors=blue}
\usepackage{graphicx}
  \graphicspath{{../../figures/}}
\usepackage{soul}
  \setstcolor{red}
\usepackage[normalem]{ulem}
\usepackage{enumerate}
\usepackage{tikz}
  \usetikzlibrary{trees}

% Document information
\author{M. McNeill}
\title[FREN1001]{Français 1001}
\institute{\url{joshua.mcneill@uga.edu}}
\date{}

%% Custom commands
% Lexical items
\newcommand{\lexi}[1]{\textit{#1}}
% Gloss
\newcommand{\gloss}[1]{`#1'}
\newcommand{\tinygloss}[1]{{\tiny`#1'}}
% Orthographic representations
\newcommand{\orth}[1]{$\langle$#1$\rangle$}
% Utterances (pragmatics)
\newcommand{\uttr}[1]{`#1'}
% Sentences (pragmatics)
\newcommand{\sent}[1]{\textit{#1}}
% Fixed length underlines
\newcommand{\funderline}[2][4cm]{
  \underline{\makebox[\ifdim\width>#1\width\else#1\fi]{#2}}
}
% Base dir for definitions
\newcommand{\defs}{../definitions}
\newcommand{\activity}[1]{
  \input{./activities/#1.tex}
}


  % Packages and settings

  % Document information
  \subtitle[Meubles et objets indirects]{Les meubles et les pronoms compléments d'objet indirect}

\begin{document}
  % Read in the standard intro slides (title page and table of contents)
  \begin{frame}
    \titlepage
    \tiny{Office: % Basically a variable for office hours location
Zoom (ID 978 2791 8221)
\\
          Office hours: % Basically a variable for office hours
 mercredi 10h15--13h15
}
  \end{frame}

  \begin{frame}{À qui?}
    % demander, dire, écrire, expliquer, montrer, parler, répondre, téléphoner
    % acheter, apporter, donner, emprunter, offrir, prêter, remettre, rendre
    \begin{columns}
      \column{0.5\textwidth}
        \scriptsize
        On fait les choses suivantes à qui ou pour qui?
        \begin{enumerate}
          \item 
          \item<2->[$\to$] 
        \end{enumerate}
      \column{0.5\textwidth}
        \begin{minipage}[t][0.6\textheight]{\linewidth}
          \begin{center}
            % picture of group
            \only<1-2>{
              \includegraphics[scale=0.5]{} % pic of individual
            }
          \end{center}
        \end{minipage}
    \end{columns}
  \end{frame}

  \begin{frame}{}
    \begin{center}
      \Large Quiz
    \end{center}
  \end{frame}

  \begin{frame}{Rarement, souvent ou jamais?}
    Avec un/e partenaire, demande-lui avec quelle fréquence il/elle fait les choses suivantes.
    Utilise des adverbes comme \lexi{rarement}, \lexi{souvent}, \lexi{jamais} ou bien d'autres.
    \begin{description}
      \item[\textbf{Modèle:}] \emph{prêter tes vêtements à ta/ton colocataire}
      \item[E1:] Est-ce que tu prêtes tes vêtements à ta/ton colocataire?
      \item[E2:] Oui, je lui prête souvent mes vêtements!
    \end{description}
    \begin{columns}[t]
      \column{0.5\textwidth}
        \begin{enumerate}
          \item rendre les devoirs au professeur
          \item expliquer tes problèmes à ta mère
          \item téléphoner à tes parents
          \item offrir des cadeaux à ton père
        \end{enumerate}
      \column{0.5\textwidth} 
        \begin{enumerate}
          \setcounter{enumi}{4}
          \item demander de l'argent à tes parents
          \item emprunter des vêtements à ton/ta meilleur/e ami
          \item acheter des bonbons pour tes nièces et tes neveux
          \item emprunter de l'argent à tes amis
        \end{enumerate}
    \end{columns}
  \end{frame}

  \begin{frame}{Qu'est-ce qu'on peut offrir?}
    Les personnes suivantes ont acheté un nouvel appartement.
    Avec un/e partenaire, discutez ce que vous pourriez offrir comme cadeau.
    \begin{description}
      \item[\textbf{Modèle:}] \emph{Ma sœur n'a pas grand-chose aux murs.}
      \item[E1:] On peut lui offrir une belle affiche.
      \item[E2:] Oui, ou on peut lui offrir une photo de famille.
    \end{description}
    \begin{columns}[t]
      \column{0.5\textwidth}
        \begin{enumerate}
          \item Mes parents aiment bien les films.
          \item Mon oncle adore faire la cuisine.
          \item Ma tante adore travailler dans le jardin.
          \item Ma cousine aime lire.
        \end{enumerate}
      \column{0.5\textwidth} 
        \begin{enumerate}
          \setcounter{enumi}{4}
          \item Mes grands-parents aiment la musique.
          \item Mon cousin n'a pas de colocataire.
          \item Mes amis ont une belle terrasse.
        \end{enumerate}
    \end{columns}
  \end{frame}

  \begin{frame}{}
    \begin{center}
      \Large Questions?
    \end{center}
  \end{frame}
\end{document}
