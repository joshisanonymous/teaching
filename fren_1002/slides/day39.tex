%%%%%%%%%%%%%%%%%%%%%%%%%%%%%%%%%%%%%
%                                   %
% Compile with XeLaTeX and biber    %
%                                   %
% Questions or comments:            %
%                                   %
% joshua dot mcneill at uga dot edu %
%                                   %
%%%%%%%%%%%%%%%%%%%%%%%%%%%%%%%%%%%%%

\documentclass{beamer}
  % Read in standard preamble (cosmetic stuff)
  %%%%%%%%%%%%%%%%%%%%%%%%%%%%%%%%%%%%%%%%%%%%%%%%%%%%%%%%%%%%%%%%
% This is a standard preamble used in for all slide documents. %
% It basically contains cosmetic settings.                     %
%                                                              %
% Joshua McNeill                                               %
% joshua dot mcneill at uga dot edu                            %
%%%%%%%%%%%%%%%%%%%%%%%%%%%%%%%%%%%%%%%%%%%%%%%%%%%%%%%%%%%%%%%%

% Beamer settings
% \usetheme{Berkeley}
\usetheme{CambridgeUS}
% \usecolortheme{dove}
% \usecolortheme{rose}
\usecolortheme{seagull}
\usefonttheme{professionalfonts}
\usefonttheme{serif}
\setbeamertemplate{bibliography item}{}

% Packages and settings
\usepackage{fontspec}
  \setmainfont{Charis SIL}
\usepackage{hyperref}
  \hypersetup{colorlinks=true,
              allcolors=blue}
\usepackage{graphicx}
  \graphicspath{{../../figures/}}
\usepackage{soul}
  \setstcolor{red}
\usepackage[normalem]{ulem}
\usepackage{enumerate}
\usepackage{tikz}
  \usetikzlibrary{trees}

% Document information
\author{M. McNeill}
\title[FREN1001]{Français 1001}
\institute{\url{joshua.mcneill@uga.edu}}
\date{}

%% Custom commands
% Lexical items
\newcommand{\lexi}[1]{\textit{#1}}
% Gloss
\newcommand{\gloss}[1]{`#1'}
\newcommand{\tinygloss}[1]{{\tiny`#1'}}
% Orthographic representations
\newcommand{\orth}[1]{$\langle$#1$\rangle$}
% Utterances (pragmatics)
\newcommand{\uttr}[1]{`#1'}
% Sentences (pragmatics)
\newcommand{\sent}[1]{\textit{#1}}
% Fixed length underlines
\newcommand{\funderline}[2][4cm]{
  \underline{\makebox[\ifdim\width>#1\width\else#1\fi]{#2}}
}
% Base dir for definitions
\newcommand{\defs}{../definitions}
\newcommand{\activity}[1]{
  \input{./activities/#1.tex}
}


  % Packages and settings

  % Document information
  \subtitle[Subjonctif c. indicatif]{Le subjonctif contre l'indicatif}

\begin{document}
  % Read in the standard intro slides (title page and table of contents)
  \begin{frame}
    \titlepage
    \tiny{Office: % Basically a variable for office hours location
Zoom (ID 978 2791 8221)
\\
          Office hours: % Basically a variable for office hours
 mercredi 10h15--13h15
}
  \end{frame}

  \begin{frame}{Suite des posters et slogans}
    Finissons l'activité de la dernière classe.
    \begin{columns}[t]
      \column{0.55\textwidth}
        \begin{description}
          \item[] \textbf{Modèle:} \emph{Des slogans pour les transports en commun}
          \item[E1:] Vive le tramway et le métro!
          \item[E2:] À bas les grosses voitures!
          \item[E3:] Prenez le train ou vous êtes un crétin!
        \end{description}
      \column{0.45\textwidth}
        \begin{itemize}
          \item[] Des constructions utiles:
          \item (Les verbes à l'impératif)
          \item À bas ... \gloss{Down with}
          \item Plus de ... \gloss{No more}
          \item Vive ...
          \item Non à ...
          \item Oui à ...
        \end{itemize}
    \end{columns}
  \end{frame}

  \begin{frame}{}
    \begin{center}
      \Large Quiz
    \end{center}
  \end{frame}

  \begin{frame}{Plus de conjugaisons au subjonctif}
    \begin{enumerate}
      \item Il est bon que vous \underline{\uncover<2->{travailliez}} (travailler) sérieusement.
      \item Il vaudrait mieux que je \underline{\uncover<3->{combatte}} (combattre) l'injustice.
      \item Je veux qu'il \underline{\uncover<4->{aide}} (aider) nos voisins.
      \item Il faut que tu \underline{\uncover<5->{choisisses}} (choisir) le nom de l'association.
      \item Je suis inquiète que nous \underline{\uncover<6->{allions}} (aller) avoir un examen.
      \item C'est surprenant que le subjonctif \underline{\uncover<7->{soit}} (être) tellement facile.
    \end{enumerate}
  \end{frame}

  \begin{frame}{Le subjonctif ou non?}
    \begin{enumerate}
      \item Il est vrai qu'on \underline{\uncover<2->{doit}} (devoir) aider les autres.
      \item<3->[$\to$] l'indicatif
      \item Elle doute qu'ils \underline{\uncover<4->{organisent}} (organiser) bien la manifestation.
      \item<5->[$\to$] le subjonctif
      \item Je pense que vous \underline{\uncover<6->{avez}} (avoir) raison.
      \item<7->[$\to$] l'indicatif
      \item Tu es content que tu \underline{\uncover<8->{comprends}} (comprendre) cette leçon.
      \item<9->[$\to$] l'indicatif
      \item Ils sont fâchés que leur ami ne \underline{\uncover<10->{tienne}} pas de stand.
      \item<11->[$\to$] le subjonctif
      \item Nous préférons que nous \underline{\uncover<12->{luttons}} (lutter) pour une bonne cause.
      \item<13->[$\to$] l'indicatif
    \end{enumerate}
  \end{frame}

  \begin{frame}{Des réactions de groupe}
    En groupes de 3 ou 4, complétez les phrases à tour de rôle \gloss{taking turns}.
    Les autres membres du groupe vont réagir avec des conseils, des émotions, etc.
    \begin{columns}
      \column{0.55\textwidth}
        \begin{description}
          \item[] \textbf{Modèle:} \emph{Après mes études, je veux...}
          \item[E1:] Après mes études, je veux travailler en Belgique
          \item[E2:] Il faudra bien parler français.
          \item[E3:] Il faut que tu apprennes le flamand, aussi.
          \item[E4:] Je suis surprise que tu veuilles travailler en Belgique.
        \end{description}
      \column{0.45\textwidth}
        \begin{enumerate}
          \item Le week-end, j'aime...
          \item Dimanche matin, j'aime mieux...
          \item Pour les vacances, je préfère...
          \item Le semestre prochain, je...
          \item Cet été, je désire...
          \item Après mes études, je veux...
        \end{enumerate}
    \end{columns}
  \end{frame}

  \begin{frame}{}
    \begin{center}
      \Large Questions?
    \end{center}
  \end{frame}
\end{document}
