%%%%%%%%%%%%%%%%%%%%%%%%%%%%%%%%%%%%%
%                                   %
% Compile with XeLaTeX and biber    %
%                                   %
% Questions or comments:            %
%                                   %
% joshua dot mcneill at uga dot edu %
%                                   %
%%%%%%%%%%%%%%%%%%%%%%%%%%%%%%%%%%%%%

\documentclass{beamer}
  % Read in standard preamble (cosmetic stuff)
  %%%%%%%%%%%%%%%%%%%%%%%%%%%%%%%%%%%%%%%%%%%%%%%%%%%%%%%%%%%%%%%%
% This is a standard preamble used in for all slide documents. %
% It basically contains cosmetic settings.                     %
%                                                              %
% Joshua McNeill                                               %
% joshua dot mcneill at uga dot edu                            %
%%%%%%%%%%%%%%%%%%%%%%%%%%%%%%%%%%%%%%%%%%%%%%%%%%%%%%%%%%%%%%%%

% Beamer settings
% \usetheme{Berkeley}
\usetheme{CambridgeUS}
% \usecolortheme{dove}
% \usecolortheme{rose}
\usecolortheme{seagull}
\usefonttheme{professionalfonts}
\usefonttheme{serif}
\setbeamertemplate{bibliography item}{}

% Packages and settings
\usepackage{fontspec}
  \setmainfont{Charis SIL}
\usepackage{hyperref}
  \hypersetup{colorlinks=true,
              allcolors=blue}
\usepackage{graphicx}
  \graphicspath{{../../figures/}}
\usepackage{soul}
  \setstcolor{red}
\usepackage[normalem]{ulem}
\usepackage{enumerate}
\usepackage{tikz}
  \usetikzlibrary{trees}

% Document information
\author{M. McNeill}
\title[FREN1001]{Français 1001}
\institute{\url{joshua.mcneill@uga.edu}}
\date{}

%% Custom commands
% Lexical items
\newcommand{\lexi}[1]{\textit{#1}}
% Gloss
\newcommand{\gloss}[1]{`#1'}
\newcommand{\tinygloss}[1]{{\tiny`#1'}}
% Orthographic representations
\newcommand{\orth}[1]{$\langle$#1$\rangle$}
% Utterances (pragmatics)
\newcommand{\uttr}[1]{`#1'}
% Sentences (pragmatics)
\newcommand{\sent}[1]{\textit{#1}}
% Fixed length underlines
\newcommand{\funderline}[2][4cm]{
  \underline{\makebox[\ifdim\width>#1\width\else#1\fi]{#2}}
}
% Base dir for definitions
\newcommand{\defs}{../definitions}
\newcommand{\activity}[1]{
  \input{./activities/#1.tex}
}


  % Packages and settings

  % Document information
  \subtitle[\lexi{Écrire}, \lexi{lire}, \lexi{dire}]{Les verbes \lexi{écrire}, \lexi{lire} et \lexi{dire}}

\begin{document}
  % Read in the standard intro slides (title page and table of contents)
  \begin{frame}
    \titlepage
    \tiny{Office: % Basically a variable for office hours location
Zoom (ID 978 2791 8221)
\\
          Office hours: % Basically a variable for office hours
 mercredi 10h15--13h15
}
  \end{frame}

  \begin{frame}{}
    \begin{center}
      \begin{tabular}{l | l l | l l}
  \multicolumn{5}{c}{écrire \gloss{to write}} \\
  \hline
      & \multicolumn{2}{l |}{singulier} & \multicolumn{2}{l}{pluriel} \\
  \hline
  1re & j'         & écris              & nous        & écri\alert{v}ons \\
  2e  & tu         & écris              & vous        & écri\alert{v}ez \\
  \hline
  3e  & il (masc)  &                    & ils (masc)  & \\
      & elle (fem) & écrit              & elles (fem) & écri\lexi{v}ent \\
      & on         &                    &             & \\
  \hline
  \multicolumn{5}{c}{Impératifs $\to$ écris, écrivez, écrivons} \\
  \multicolumn{5}{c}{Passé composé $\to$ j'ai écrit} \\
  \multicolumn{5}{c}{Futur simple $\to$ j'écrirai}
\end{tabular}

    \end{center}
    C'est bien pour d'autres verbes $\to$ \lexi{d\alert{écrire}}
  \end{frame}

  \begin{frame}{}
    \begin{center}
      \begin{tabular}{l | l l | l l}
  \multicolumn{5}{c}{lire \gloss{to write}} \\
  \hline
      & \multicolumn{2}{l |}{singulier} & \multicolumn{2}{l}{pluriel} \\
  \hline
  1re & je         & lis                & nous        & li\alert{s}ons \\
  2e  & tu         & lis                & vous        & li\alert{s}ez \\
  \hline
  3e  & il (masc)  &                    & ils (masc)  & \\
      & elle (fem) & lit                & elles (fem) & li\lexi{s}ent \\
      & on         &                    &             & \\
  \hline
  \multicolumn{5}{c}{Impératifs $\to$ lis, lisez, lisons} \\
  \multicolumn{5}{c}{Passé composé $\to$ j'ai \alert{lu}} \\
  \multicolumn{5}{c}{Futur simple $\to$ je lirai}
\end{tabular}

    \end{center}
  \end{frame}

  \begin{frame}{}
    \begin{center}
      \begin{tabular}{l | l l | l l}
  \multicolumn{5}{c}{dire \gloss{to say}} \\
  \hline
      & \multicolumn{2}{l |}{singulier} & \multicolumn{2}{l}{pluriel} \\
  \hline
  1re & je         & dis                & nous        & di\alert{s}ons \\
  2e  & tu         & dis                & vous        & di\alert{t}e\alert{s} \\
  \hline
  3e  & il (masc)  &                    & ils (masc)  & \\
      & elle (fem) & dit                & elles (fem) & di\lexi{s}ent \\
      & on         &                    &             & \\
  \hline
  \multicolumn{5}{c}{Impératifs $\to$ dis, dites, disons} \\
  \multicolumn{5}{c}{Passé composé $\to$ j'ai dit} \\
  \multicolumn{5}{c}{Futur simple $\to$ je dirai}
\end{tabular}

    \end{center}
  \end{frame}

  \begin{frame}{On est d'accord}
    Comment disent-ils <<oui>>? \alert{oui, da, ja, sí, yes}
      \begin{enumerate}
        \item Peter et Helmut sont allemands.
        \item<2->[$\to$] Ils disent <<ja>>.
        \item<3-> Louis-Jean est haïtien.
        \item<4->[$\to$] Il dit <<oui>>.
        \item<5-> Moi, je suis russe.
        \item<6->[$\to$] Je dis <<da>>.
        \item<7-> Isabel et moi sommes mexicaines.
        \item<8->[$\to$] Nous disons <<sí>>.
        \item<9-> George et toi, vous êtes américains.
        \item<10->[$\to$] Vous dites <<yes>>.
      \end{enumerate}
  \end{frame}

  \begin{frame}{}
    \begin{center}
      \Large Quiz
    \end{center}
  \end{frame}

  \begin{frame}{Trouvez une personne}
    \scriptsize
    Circulez dans la salle pour trouver une personne pour qui la phrase est vraie.
    Écrivez les noms pour que vous vous les rappelez.
    N'écrivez pas un nom pour plus d'une phrase.
    \begin{description}
      \item[] \textbf{Modèle:} \emph{lit le journal tous les jours}
      \item[E1:] Est-ce que tu lis le journal tous les jours?
      \item[E2:] Oui, je le lis. Je lis le New York Times.
      \item[] OU
      \item[E2:] Non, je ne le lis pas.
    \end{description}
    \vspace{12pt}
    Trouvez une personne qui...
    \begin{columns}[t]
      \column{0.5\textwidth}
        \begin{enumerate}
          \item lit le journal tous les jours
          \item écrit un blog
          \item dit toujours la vérité
          \item écrit pour le journal de l'université
        \end{enumerate}
      \column{0.5\textwidth}
        \begin{enumerate}
          \setcounter{enumi}{4}
          \item a lu une biographie l'année passée
          \item va écrire un essai cette semaine
          \item lit son horoscope quelquefois
          \item a écrit un poème
        \end{enumerate}
    \end{columns}
  \end{frame}

  \begin{frame}{}
    \begin{center}
      \Large Questions?
    \end{center}
  \end{frame}
\end{document}
