%%%%%%%%%%%%%%%%%%%%%%%%%%%%%%%%%%%%%
%                                   %
% Compile with XeLaTeX and biber    %
%                                   %
% Questions or comments:            %
%                                   %
% joshua dot mcneill at uga dot edu %
%                                   %
%%%%%%%%%%%%%%%%%%%%%%%%%%%%%%%%%%%%%

\documentclass{beamer}
  % Read in standard preamble (cosmetic stuff)
  %%%%%%%%%%%%%%%%%%%%%%%%%%%%%%%%%%%%%%%%%%%%%%%%%%%%%%%%%%%%%%%%
% This is a standard preamble used in for all slide documents. %
% It basically contains cosmetic settings.                     %
%                                                              %
% Joshua McNeill                                               %
% joshua dot mcneill at uga dot edu                            %
%%%%%%%%%%%%%%%%%%%%%%%%%%%%%%%%%%%%%%%%%%%%%%%%%%%%%%%%%%%%%%%%

% Beamer settings
% \usetheme{Berkeley}
\usetheme{CambridgeUS}
% \usecolortheme{dove}
% \usecolortheme{rose}
\usecolortheme{seagull}
\usefonttheme{professionalfonts}
\usefonttheme{serif}
\setbeamertemplate{bibliography item}{}

% Packages and settings
\usepackage{fontspec}
  \setmainfont{Charis SIL}
\usepackage{hyperref}
  \hypersetup{colorlinks=true,
              allcolors=blue}
\usepackage{graphicx}
  \graphicspath{{../../figures/}}
\usepackage{soul}
  \setstcolor{red}
\usepackage[normalem]{ulem}
\usepackage{enumerate}
\usepackage{tikz}
  \usetikzlibrary{trees}

% Document information
\author{M. McNeill}
\title[FREN1001]{Français 1001}
\institute{\url{joshua.mcneill@uga.edu}}
\date{}

%% Custom commands
% Lexical items
\newcommand{\lexi}[1]{\textit{#1}}
% Gloss
\newcommand{\gloss}[1]{`#1'}
\newcommand{\tinygloss}[1]{{\tiny`#1'}}
% Orthographic representations
\newcommand{\orth}[1]{$\langle$#1$\rangle$}
% Utterances (pragmatics)
\newcommand{\uttr}[1]{`#1'}
% Sentences (pragmatics)
\newcommand{\sent}[1]{\textit{#1}}
% Fixed length underlines
\newcommand{\funderline}[2][4cm]{
  \underline{\makebox[\ifdim\width>#1\width\else#1\fi]{#2}}
}
% Base dir for definitions
\newcommand{\defs}{../definitions}
\newcommand{\activity}[1]{
  \input{./activities/#1.tex}
}


  % Packages and settings

  % Document information
  \subtitle[Écran et prépositions]{L'écran et les prépositions de temps}

\begin{document}
  % Read in the standard intro slides (title page and table of contents)
  \begin{frame}
    \titlepage
    \tiny{Office: % Basically a variable for office hours location
Zoom (ID 978 2791 8221)
\\
          Office hours: % Basically a variable for office hours
 mercredi 10h15--13h15
}
  \end{frame}

  \begin{frame}{Quel genre est-ce que tu préfères?}
    \scriptsize
    Avec un/e partenaire, classe les genres d'émissions selon tes préférences, puis compare ta liste avec la liste de ton/ta partenaire.
    Discutes-en. Pourquoi est-ce que tu as ces préférences?
    \begin{description}
      \item[] \textbf{Modèle:}
      \item[E1:] J'aime surtout les séries dramatiques, mais je regarde aussi d'autres séries. Je regarde les séries humoristiques et les dessins animés. Et toi?
      \item[E2:] Moi, j'adore les séries policières mais aussi les émissions de téléréalité.
      \item[E1:] Pourquoi est-ce que tu adores les émissions de téléréalité? Quelle série est ta préférée?
    \end{description}
    \begin{columns}[t]
      \column{0.5\textwidth}
        \begin{itemize}
          \item les dessins animés
          \item les émissions de sport
          \item les émissions de téléréalité
          \item les feuilletons
          \item les informations
        \end{itemize}
      \column{0.5\textwidth}
        \begin{itemize}
          \item les jeux télévisés
          \item les séries dramatiques
          \item les séries humoristiques
          \item les séries policières
          \item les séries de suspense
        \end{itemize}
    \end{columns}
    \uncover<2->{
      \vspace{0.3cm}
      \alert{Quel genre est-ce que la classe préfère dans l'ensemble \gloss{as a whole}?}
    }
  \end{frame}

  \begin{frame}{}
    \begin{center}
      \Large Quiz
    \end{center}
  \end{frame}

  \begin{frame}{Une idée pour une émission}
    Inventons une émission ensemble dans notre genre préféré!
    Nous allons décrire l'émission (l'époque, les personnages, la situation, etc.).
    Chaque personne va donner une phrase pour décrire l'émission qui suit la phrase de la dernière personne.
    \begin{description}
      \item[] \textbf{Modèle:} \emph{les séries policières}
      \item[E1:] Ça commence dans l'année 1975 à Boston.
      \item[E2:] Il y a un policier qui étudie un vieux cas.
      \item[E3:] Le cas concerne un assassinat.
      \item[E4:] Lorsqu'il l'étudie, il remarque qu'il y a un détail que les autres ont manqué.
    \end{description}
  \end{frame}

  \begin{frame}{En même temps \gloss{At the same time}}
    Qu'est-ce qui se passe pendant les situations suivantes?
    \begin{description}
      \item[] \textbf{Modèle:} \emph{Pendant qu'elle regarde le journal télévisé le matin, ma mère...}
      \item[E1:] boit son café.
    \end{description}
    \begin{enumerate}
      \item Quand je fais du sport, ...
      \item Pendant que mon professeur déjeune, ...
      \item Quand je parle au téléphone, ...
      \item Pendant que je suis en cours de français, ...
      \item Quand mes parents sont dans la voiture, ils ...
      \item Pendant que mon colocataire fait ses devoirs, ...
    \end{enumerate}
  \end{frame}

  \begin{frame}{Le passé et le futur}
    Avec un/e partenaire, discute de ce que tu as fait ou tu feras aux époques suivantes.
    \begin{description}
      \item[] \textbf{Modèle:} \emph{Quand j'étais petit/e, ...}
      \item[E1:] Quand j'étais petite, nous passions toujours l'été dans le Maine.
      \item[E2:] Moi, je suivais souvent des cours d'été quand j'étais petit.
    \end{description}
    \begin{columns}[t]
      \column{0.5\textwidth}
        \begin{enumerate}
          \item Quand j'allais au lycée, ...
          \item Pendant que mes parents travaillaient, ...
          \item Quand il faisait très chaud, ...
          \item Quand les vacances arrivaient, ...
        \end{enumerate}
      \column{0.5\textwidth}
        \begin{enumerate}
          \setcounter{enumi}{4}
          \item Quand j'aurai mon diplôme, ...
          \item Quand j'aurai 50 ans, ...
          \item Quand je serai riche, ...
          \item Quand je serai en vacances, ...
        \end{enumerate}
    \end{columns}
  \end{frame}

  \begin{frame}{}
    \begin{center}
      \Large Questions?
    \end{center}
  \end{frame}
\end{document}
