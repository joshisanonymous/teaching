%%%%%%%%%%%%%%%%%%%%%%%%%%%%%%%%%%%%%
%                                   %
% Compile with XeLaTeX and biber    %
%                                   %
% Questions or comments:            %
%                                   %
% joshua dot mcneill at uga dot edu %
%                                   %
%%%%%%%%%%%%%%%%%%%%%%%%%%%%%%%%%%%%%

\documentclass{beamer}
  % Read in standard preamble (cosmetic stuff)
  %%%%%%%%%%%%%%%%%%%%%%%%%%%%%%%%%%%%%%%%%%%%%%%%%%%%%%%%%%%%%%%%
% This is a standard preamble used in for all slide documents. %
% It basically contains cosmetic settings.                     %
%                                                              %
% Joshua McNeill                                               %
% joshua dot mcneill at uga dot edu                            %
%%%%%%%%%%%%%%%%%%%%%%%%%%%%%%%%%%%%%%%%%%%%%%%%%%%%%%%%%%%%%%%%

% Beamer settings
% \usetheme{Berkeley}
\usetheme{CambridgeUS}
% \usecolortheme{dove}
% \usecolortheme{rose}
\usecolortheme{seagull}
\usefonttheme{professionalfonts}
\usefonttheme{serif}
\setbeamertemplate{bibliography item}{}

% Packages and settings
\usepackage{fontspec}
  \setmainfont{Charis SIL}
\usepackage{hyperref}
  \hypersetup{colorlinks=true,
              allcolors=blue}
\usepackage{graphicx}
  \graphicspath{{../../figures/}}
\usepackage{soul}
  \setstcolor{red}
\usepackage[normalem]{ulem}
\usepackage{enumerate}
\usepackage{tikz}
  \usetikzlibrary{trees}

% Document information
\author{M. McNeill}
\title[FREN1001]{Français 1001}
\institute{\url{joshua.mcneill@uga.edu}}
\date{}

%% Custom commands
% Lexical items
\newcommand{\lexi}[1]{\textit{#1}}
% Gloss
\newcommand{\gloss}[1]{`#1'}
\newcommand{\tinygloss}[1]{{\tiny`#1'}}
% Orthographic representations
\newcommand{\orth}[1]{$\langle$#1$\rangle$}
% Utterances (pragmatics)
\newcommand{\uttr}[1]{`#1'}
% Sentences (pragmatics)
\newcommand{\sent}[1]{\textit{#1}}
% Fixed length underlines
\newcommand{\funderline}[2][4cm]{
  \underline{\makebox[\ifdim\width>#1\width\else#1\fi]{#2}}
}
% Base dir for definitions
\newcommand{\defs}{../definitions}
\newcommand{\activity}[1]{
  \input{./activities/#1.tex}
}


  % Packages and settings
  % \usepackage{enumitem}

  % Document information
  \subtitle[Corps et subjonctif]{Le corps humain et le subjonctif}

\begin{document}
  % Read in the standard intro slides (title page and table of contents)
  \begin{frame}
    \titlepage
    \tiny{Office: % Basically a variable for office hours location
Zoom (ID 978 2791 8221)
\\
          Office hours: % Basically a variable for office hours
 mercredi 10h15--13h15
}
  \end{frame}

  \begin{frame}{Casser un membre du corps}
    Est-ce que tu as jamais cassé une partie du corps?
    Est-ce que tu en as presque cassé une?
    En groupes de 3 ou 4, raconte un moment où tu as cassé ou presque cassé une partie du corps.
    \begin{description}
      \item[] \textbf{Modèle:}
      \item[E1:] Moi, je n'ai jamais cassé un membre du corps, mais j'ai presque cassé mon doigt.
      \item[E2:] Qu'est-ce qui s'est passé?
      \item[E1:] Quelqu'un a fermé la porte d'une voiture sur mon doigt quand j'avais 11 ans.
      \item[E3:] Est-ce que ça t'a fait mal? \gloss{Did it hurt?}
    \end{description}
  \end{frame}

  \begin{frame}{}
    \begin{center}
      \Large Quiz
    \end{center}
  \end{frame}

  \begin{frame}{Quel mot?}
    Un \orth{s} écrit peut être prononcé /s/ or /z/.
    Quel mot est-ce que je prononce?
    \begin{center}
      \begin{tabular}{l l}
        poisson               & \alert<2->{poison} \\
        \alert<3->{dessert}   & désert \\
        \alert<4->{vous avez} & vous savez \\
        \alert<5->{ils sont}  & ils ont \\
        ils aiment            & \alert<6->{ils s'aiment} \\
      \end{tabular}
    \end{center}
  \end{frame}

  \begin{frame}{Le subjonctif}
    Conjugons les verbes au subjonctif.
    \begin{enumerate}
      \item Il faut \alert{que} vous \underline{\uncover<2->{rendiez}} (rendre) les papiers.
      \item Il est nécessaire \alert{que} tu \underline{\uncover<3->{arrêtes}} (arrêter) la voiture!
      \item Il est urgent \alert{que} nous \underline{\uncover<4->{mangions}} (manger) moins de sucre.
      \item Il faut \alert{que} je \underline{\uncover<5->{finisse}} (finir) mes devoirs.
      \item Il est utile \alert{qu'}ils me \underline{\uncover<6->{paient}} (payer) beaucoup d'argent.
    \end{enumerate}
  \end{frame}

  \begin{frame}{Des obligations}
    \small
    \begin{columns}
      \column{0.6\textwidth}
        \begin{enumerate}
          \item Pour chaque verbe, écris sur un papier une phrase qui explique ce qu'\alert{il faut que tu fasses}. Alors, les phrases vont consister d'une expression de nécessité et le verbe au subjonctif.
          \item[] par ex. \emph{Il faut que je ...}
          \item<2-> Avec un/e partenaire, discutez des phrases que vous avez écrites. Posez des questions!
          \item<2->[] par ex.
          \item<2->[E1:] Il faut que j'écrive un essai.
          \item<2->[E2:] C'est un essai pour quelle classe?
        \end{enumerate}
      \column{0.4\textwidth}
        \begin{enumerate}
          \item lire
          \item préparer
          \item rendre
          \item finir
          \item téléphoner
        \end{enumerate}
    \end{columns}
  \end{frame}

  \begin{frame}{}
    \begin{center}
      \Large Questions?
    \end{center}
  \end{frame}
\end{document}
