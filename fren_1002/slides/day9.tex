%%%%%%%%%%%%%%%%%%%%%%%%%%%%%%%%%%%%%
%                                   %
% Compile with XeLaTeX and biber    %
%                                   %
% Questions or comments:            %
%                                   %
% joshua dot mcneill at uga dot edu %
%                                   %
%%%%%%%%%%%%%%%%%%%%%%%%%%%%%%%%%%%%%

\documentclass{beamer}
  % Read in standard preamble (cosmetic stuff)
  %%%%%%%%%%%%%%%%%%%%%%%%%%%%%%%%%%%%%%%%%%%%%%%%%%%%%%%%%%%%%%%%
% This is a standard preamble used in for all slide documents. %
% It basically contains cosmetic settings.                     %
%                                                              %
% Joshua McNeill                                               %
% joshua dot mcneill at uga dot edu                            %
%%%%%%%%%%%%%%%%%%%%%%%%%%%%%%%%%%%%%%%%%%%%%%%%%%%%%%%%%%%%%%%%

% Beamer settings
% \usetheme{Berkeley}
\usetheme{CambridgeUS}
% \usecolortheme{dove}
% \usecolortheme{rose}
\usecolortheme{seagull}
\usefonttheme{professionalfonts}
\usefonttheme{serif}
\setbeamertemplate{bibliography item}{}

% Packages and settings
\usepackage{fontspec}
  \setmainfont{Charis SIL}
\usepackage{hyperref}
  \hypersetup{colorlinks=true,
              allcolors=blue}
\usepackage{graphicx}
  \graphicspath{{../../figures/}}
\usepackage[normalem]{ulem}
\usepackage{enumerate}

% Document information
\author{M. McNeill}
\title[FREN2001]{Français 2001}
\institute{\url{joshua.mcneill@uga.edu}}
\date{}

%% Custom commands
% Lexical items
\newcommand{\lexi}[1]{\textit{#1}}
% Gloss
\newcommand{\gloss}[1]{`#1'}
\newcommand{\tinygloss}[1]{{\tiny`#1'}}
% Orthographic representations
\newcommand{\orth}[1]{$\langle$#1$\rangle$}
% Utterances (pragmatics)
\newcommand{\uttr}[1]{`#1'}
% Sentences (pragmatics)
\newcommand{\sent}[1]{\textit{#1}}
% Base dir for definitions
\newcommand{\defs}{../definitions}


  % Packages and settings

  % Document information
  \subtitle[Passé composé (révision)]{Révision du passé composé}

\begin{document}
  % Read in the standard intro slides (title page and table of contents)
  \begin{frame}
    \titlepage
    \tiny{Office: % Basically a variable for office hours location
Gilbert 121\\
          Office hours: % Basically a variable for office hours
 lundi, mercredi, vendredi 10:10--11:10
}
  \end{frame}

  \begin{frame}{Annonces}
    \begin{itemize}
      \item L'atelier de conversation vendredi
      \item[] \tinygloss{Conversation workshop Friday}
    \end{itemize}
  \end{frame}

  \begin{frame}{Passez l'histoire!}
    Imaginons une histoire ensemble pour chaque scénario.
    Chaque personne donne une phrase \emph{au passé composé} à partir de celle de la personne précédente.
    \begin{columns}
      \column{0.5\textwidth}
        \begin{description}
          \item[] \textbf{Modèle:}
          \item[] \emph{Jean et Luke au stade hier}
          \item[E1:] Hier, Jean et Luke ont regardé le match.
          \item[E2:] Leur équipe a gagné.
          \item[E3:] Après, ils ont pris des cocas, mais Luke n'aime pas le coca.
        \end{description}
      \column{0.5\textwidth}
        Les scénarios:
        \begin{enumerate}
          \item Jean et Luke sur une montagne hier. Il neigeait.
          \item Kim dans un parc avec sa mère et sa sœur la semaine dernière.
          \item Avant-hier, notre classe à la plage.
        \end{enumerate}
    \end{columns}
  \end{frame}

  \begin{frame}{}
    \begin{center}
      \Large Quiz
    \end{center}
  \end{frame}

  \begin{frame}{Qu'est-ce qu'ils ont fait?}
    En groupes de 3 ou 4, imaginez autant d'activités que possible que ces personnes ont faites aux endroits mentionnés.
    \begin{description}
      \item[] \textbf{Modèle:} \emph{Qu'est-ce que Julie a fait dans le magasin hier?}
      \item[E1:] Elle a travaillé. (Elle est vendeuse.)
      \item[E2:] Elle a acheté une robe.
    \end{description}
    \begin{columns}[t]
      \column{0.5\textwidth}
        Qu'est-ce que...
        \begin{enumerate}
          \item ... vous avez fait dans le quartier hier?
          \item ... les Aubert ont fait au lac l'été dernier?
          \item ... tu as fait dans le potager hier?
          \item ... on a fait dans la forêt il y a deux jours \gloss{2 days ago}?
        \end{enumerate}
      \column{0.5\textwidth}
        \begin{enumerate}
          \setcounter{enumi}{4}
          \item ... Clément a fait à la rivière avant-hier?
          \item ... vos camarades ont fait chez eux le week-end dernier?
          \item ... le prof a fait sur une colline ce matin?
          \item ... tu as fait chez toi hier soir?
        \end{enumerate}
    \end{columns}
  \end{frame}

  \begin{frame}{}
    \begin{center}
      \Large Questions?
    \end{center}
  \end{frame}
\end{document}
