%%%%%%%%%%%%%%%%%%%%%%%%%%%%%%%%%%%%%
%                                   %
% Compile with XeLaTeX and biber    %
%                                   %
% Questions or comments:            %
%                                   %
% joshua dot mcneill at uga dot edu %
%                                   %
%%%%%%%%%%%%%%%%%%%%%%%%%%%%%%%%%%%%%

\documentclass{beamer}
  % Read in standard preamble (cosmetic stuff)
  %%%%%%%%%%%%%%%%%%%%%%%%%%%%%%%%%%%%%%%%%%%%%%%%%%%%%%%%%%%%%%%%
% This is a standard preamble used in for all slide documents. %
% It basically contains cosmetic settings.                     %
%                                                              %
% Joshua McNeill                                               %
% joshua dot mcneill at uga dot edu                            %
%%%%%%%%%%%%%%%%%%%%%%%%%%%%%%%%%%%%%%%%%%%%%%%%%%%%%%%%%%%%%%%%

% Beamer settings
% \usetheme{Berkeley}
\usetheme{CambridgeUS}
% \usecolortheme{dove}
% \usecolortheme{rose}
\usecolortheme{seagull}
\usefonttheme{professionalfonts}
\usefonttheme{serif}
\setbeamertemplate{bibliography item}{}

% Packages and settings
\usepackage{fontspec}
  \setmainfont{Charis SIL}
\usepackage{hyperref}
  \hypersetup{colorlinks=true,
              allcolors=blue}
\usepackage{graphicx}
  \graphicspath{{../../figures/}}
\usepackage{soul}
  \setstcolor{red}
\usepackage[normalem]{ulem}
\usepackage{enumerate}
\usepackage{tikz}
  \usetikzlibrary{trees}

% Document information
\author{M. McNeill}
\title[FREN1001]{Français 1001}
\institute{\url{joshua.mcneill@uga.edu}}
\date{}

%% Custom commands
% Lexical items
\newcommand{\lexi}[1]{\textit{#1}}
% Gloss
\newcommand{\gloss}[1]{`#1'}
\newcommand{\tinygloss}[1]{{\tiny`#1'}}
% Orthographic representations
\newcommand{\orth}[1]{$\langle$#1$\rangle$}
% Utterances (pragmatics)
\newcommand{\uttr}[1]{`#1'}
% Sentences (pragmatics)
\newcommand{\sent}[1]{\textit{#1}}
% Fixed length underlines
\newcommand{\funderline}[2][4cm]{
  \underline{\makebox[\ifdim\width>#1\width\else#1\fi]{#2}}
}
% Base dir for definitions
\newcommand{\defs}{../definitions}
\newcommand{\activity}[1]{
  \input{./activities/#1.tex}
}


  % Packages and settings
  % \usepackage{enumitem}

  % Document information
  \subtitle[Subjonctif]{Surprise! C'est encore le subjonctif.}

\begin{document}
  % Read in the standard intro slides (title page and table of contents)
  \begin{frame}
    \titlepage
    \tiny{Office: % Basically a variable for office hours location
Zoom (ID 978 2791 8221)
\\
          Office hours: % Basically a variable for office hours
 mercredi 10h15--13h15
}
  \end{frame}

  \begin{frame}{Des verbes avec deux formes au subjonctif}
    \begin{enumerate}
      \item Il faut que j'\underline{\uncover<2->{achète}} (acheter) du papier recyclé.
      \item Tu préfères que nous \underline{\uncover<3->{achetions}} (acheter) aussi du papier recyclé?
      \item Ils aiment que vous \underline{\uncover<4->{preniez}} (prendre) les transports en commun.
      \item Vous voulez que je \underline{\uncover<5->{prennes}} (prendre) aussi ces transports?
      \item Ma tante exige qu'on \underline{\uncover<6->{aille}} (aller) au centre de recyclage à pied.
      \item Elle exige que nous y \underline{\uncover<7->{allions}} (aller) à pied.
    \end{enumerate}
  \end{frame}

  \begin{frame}{}
    \begin{center}
      \Large Quiz
    \end{center}
  \end{frame}

  \begin{frame}{Vos préférences et celles du professeur}
    \only<1>{
      Avec un/e partenaire, décidez si vos préférences sonts les mêmes que les préférences de votre professeur.
      Utilisez le subjonctif quand vous parlez du professeur.
    }
    \begin{description}
      \item[] \textbf{Modèle:} \emph{parler toujours français en classe}
      \item[E1:] Je n'aime pas parler français en classe.
      \item[E2:] Moi, j'aime le parler en classe, et le professeur, il préfère \alert{que nous parlions} français en classe.
    \end{description}
    \begin{columns}
      \column{0.5\textwidth}
        \begin{enumerate}
          \item faire tous les devoirs
          \item prendre des notes
          \item venir en classe tous les jours
          \item faire des crêpes
        \end{enumerate}
      \column{0.5\textwidth}
        \begin{enumerate}
          \setcounter{enumi}{4}
          \item aller voir des films français
          \item ne pas boire de vin
          \item apprendre tout le vocabulaire
          \item parler comme des Français
        \end{enumerate}
    \end{columns}
    \only<2>{
      \vspace{0.4cm}
      Pour chaque numéro, trouve une personne qui a répondu par l'affirmative et une qui a répondu par la négative.
      \alert{Écris} les noms sur un papier.
    }
  \end{frame}

  \begin{frame}{Vos idées?}
    En groupes de 3 ou 4, discutez de vos \emph{désirs}, \emph{souhaits} et \emph{préférences} par rapport aux idées suivantes.
    \begin{description}
      \item[] \textbf{Modèle:} \emph{recycler le carton}
      \item[E1:] Je ne recycle rien, mais il est important que les gens recyclent.
      \item[E2:] Oui, je souhaite que plus de gens recyclent le carton et aussi le plastique.
    \end{description}
    \begin{enumerate}
      \item recycler le carton
      \item utiliser les sacs réutilisables
      \item privilégier l'énergie renouvelable
      \item lutter pour l'environnement
    \end{enumerate}
  \end{frame}

  \begin{frame}{}
    \begin{center}
      \Large Questions?
    \end{center}
  \end{frame}
\end{document}
