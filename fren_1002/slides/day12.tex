%%%%%%%%%%%%%%%%%%%%%%%%%%%%%%%%%%%%%
%                                   %
% Compile with XeLaTeX and biber    %
%                                   %
% Questions or comments:            %
%                                   %
% joshua dot mcneill at uga dot edu %
%                                   %
%%%%%%%%%%%%%%%%%%%%%%%%%%%%%%%%%%%%%

\documentclass{beamer}
  % Read in standard preamble (cosmetic stuff)
  %%%%%%%%%%%%%%%%%%%%%%%%%%%%%%%%%%%%%%%%%%%%%%%%%%%%%%%%%%%%%%%%
% This is a standard preamble used in for all slide documents. %
% It basically contains cosmetic settings.                     %
%                                                              %
% Joshua McNeill                                               %
% joshua dot mcneill at uga dot edu                            %
%%%%%%%%%%%%%%%%%%%%%%%%%%%%%%%%%%%%%%%%%%%%%%%%%%%%%%%%%%%%%%%%

% Beamer settings
% \usetheme{Berkeley}
\usetheme{CambridgeUS}
% \usecolortheme{dove}
% \usecolortheme{rose}
\usecolortheme{seagull}
\usefonttheme{professionalfonts}
\usefonttheme{serif}
\setbeamertemplate{bibliography item}{}

% Packages and settings
\usepackage{fontspec}
  \setmainfont{Charis SIL}
\usepackage{hyperref}
  \hypersetup{colorlinks=true,
              allcolors=blue}
\usepackage{graphicx}
  \graphicspath{{../../figures/}}
\usepackage{soul}
  \setstcolor{red}
\usepackage[normalem]{ulem}
\usepackage{enumerate}
\usepackage{tikz}
  \usetikzlibrary{trees}

% Document information
\author{M. McNeill}
\title[FREN1001]{Français 1001}
\institute{\url{joshua.mcneill@uga.edu}}
\date{}

%% Custom commands
% Lexical items
\newcommand{\lexi}[1]{\textit{#1}}
% Gloss
\newcommand{\gloss}[1]{`#1'}
\newcommand{\tinygloss}[1]{{\tiny`#1'}}
% Orthographic representations
\newcommand{\orth}[1]{$\langle$#1$\rangle$}
% Utterances (pragmatics)
\newcommand{\uttr}[1]{`#1'}
% Sentences (pragmatics)
\newcommand{\sent}[1]{\textit{#1}}
% Fixed length underlines
\newcommand{\funderline}[2][4cm]{
  \underline{\makebox[\ifdim\width>#1\width\else#1\fi]{#2}}
}
% Base dir for definitions
\newcommand{\defs}{../definitions}
\newcommand{\activity}[1]{
  \input{./activities/#1.tex}
}


  % Packages and settings

  % Document information
  \subtitle[\lexi{Qui} et \lexi{que}]{Les questions avec \lexi{qui} et \lexi{que}}

\begin{document}
  % Read in the standard intro slides (title page and table of contents)
  \begin{frame}
    \titlepage
    \tiny{Office: % Basically a variable for office hours location
Zoom (ID 978 2791 8221)
\\
          Office hours: % Basically a variable for office hours
 mercredi 10h15--13h15
}
  \end{frame}

  \begin{frame}{Chose ou personne}
    On parle d'une chose ou d'une personne?
    \begin{center}
      \begin{tabular}{l | c | c}
        Question                                          & Chose            & Personne \\
        \hline
        Qu'est-ce que tu regardes?                        & \uncover<2->{X}  & \\
        \uncover<3->{Qui est-ce que tu préfères?}         &                  & \uncover<4->{X} \\
        \uncover<5->{Qui est-ce que tu attends?}          &                  & \uncover<6->{X} \\
        \uncover<7->{Qu'est-ce que tu écoutes?}           & \uncover<8->{X}  & \\
        \uncover<9->{Qu'est-ce que tu prends?}            & \uncover<10->{X} & \\
        \uncover<11->{Avec qui est-ce que tu vas partir?} &                  & \uncover<12->{X} \\
        \uncover<13->{Qu'est-ce que tu écris?}            & \uncover<14->{X} & \\
        \uncover<15->{De qui est-ce que tu parles?}       &                  & \uncover<16->{X}
      \end{tabular}
    \end{center}
  \end{frame}

  \begin{frame}{}
    \begin{center}
      \Large Quiz
    \end{center}
  \end{frame}

  \begin{frame}{Jeopardy}
    \small
    En groupes de 3 ou 4, jouez au Jeopardy.
    Une personne donne une réponse, et les autres posent des questions logiques.
    La première personne à poser une bonne question gagne un point et donne la réponse suivante.
    \begin{columns}
      \column{0.36\textwidth}
        \begin{description}
          \item[] \textbf{Modèle:}
          \item[E1:] De la musique classique.
          \item[E2:] Qu'est-ce que tu écoutes?
        \end{description}
      \column{0.64\textwidth}
        \begin{center}
          Utilise ces verbes pour les questions.
        \end{center}
        \begin{columns}
          \column{0.32\textwidth}
            \begin{itemize}
              \item écrire
              \item visiter
              \item écouter
              \item étudier
              \item chanter
            \end{itemize}
          \column{0.32\textwidth}
            \begin{itemize}
              \item manger
              \item parler
              \item regarder
              \item téléphoner
              \item courir
            \end{itemize}
        \end{columns}
    \end{columns}
  \end{frame}

  \begin{frame}{}
    \begin{center}
      \Large Questions?
    \end{center}
  \end{frame}
\end{document}
