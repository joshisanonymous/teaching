%%%%%%%%%%%%%%%%%%%%%%%%%%%%%%%%%%%%%
%                                   %
% Compile with XeLaTeX and biber    %
%                                   %
% Questions or comments:            %
%                                   %
% joshua dot mcneill at uga dot edu %
%                                   %
%%%%%%%%%%%%%%%%%%%%%%%%%%%%%%%%%%%%%

\documentclass{beamer}
  % Read in standard preamble (cosmetic stuff)
  %%%%%%%%%%%%%%%%%%%%%%%%%%%%%%%%%%%%%%%%%%%%%%%%%%%%%%%%%%%%%%%%
% This is a standard preamble used in for all slide documents. %
% It basically contains cosmetic settings.                     %
%                                                              %
% Joshua McNeill                                               %
% joshua dot mcneill at uga dot edu                            %
%%%%%%%%%%%%%%%%%%%%%%%%%%%%%%%%%%%%%%%%%%%%%%%%%%%%%%%%%%%%%%%%

% Beamer settings
% \usetheme{Berkeley}
\usetheme{CambridgeUS}
% \usecolortheme{dove}
% \usecolortheme{rose}
\usecolortheme{seagull}
\usefonttheme{professionalfonts}
\usefonttheme{serif}
\setbeamertemplate{bibliography item}{}

% Packages and settings
\usepackage{fontspec}
  \setmainfont{Charis SIL}
\usepackage{hyperref}
  \hypersetup{colorlinks=true,
              allcolors=blue}
\usepackage{graphicx}
  \graphicspath{{../../figures/}}
\usepackage[normalem]{ulem}
\usepackage{enumerate}

% Document information
\author{M. McNeill}
\title[FREN2001]{Français 2001}
\institute{\url{joshua.mcneill@uga.edu}}
\date{}

%% Custom commands
% Lexical items
\newcommand{\lexi}[1]{\textit{#1}}
% Gloss
\newcommand{\gloss}[1]{`#1'}
\newcommand{\tinygloss}[1]{{\tiny`#1'}}
% Orthographic representations
\newcommand{\orth}[1]{$\langle$#1$\rangle$}
% Utterances (pragmatics)
\newcommand{\uttr}[1]{`#1'}
% Sentences (pragmatics)
\newcommand{\sent}[1]{\textit{#1}}
% Base dir for definitions
\newcommand{\defs}{../definitions}


  % Packages and settings

  % Document information
  \subtitle[\lexi{Qui} et \lexi{que}]{Les questions avec \lexi{qui} et \lexi{que}}

\begin{document}
  % Read in the standard intro slides (title page and table of contents)
  \begin{frame}
    \titlepage
    \tiny{Office: % Basically a variable for office hours location
Gilbert 121\\
          Office hours: % Basically a variable for office hours
 lundi, mercredi, vendredi 10:10--11:10
}
  \end{frame}

  \begin{frame}{}
    
    \begin{columns}
      \column{0.5\textwidth}
        \small
        \begin{enumerate}
          \item 
        \end{enumerate}
      \column{0.5\textwidth}
        \begin{minipage}[0.6\textheight]{\linewidth}
          \begin{center}
            
          \end{center}
        \end{minipage}
    \end{columns}
  \end{frame}

  \begin{frame}{}
    \begin{center}
      \Large Quiz
    \end{center}
  \end{frame}

  \begin{frame}{Jeopardy}
    \small
    En groupes de 3 ou 4, jouez au Jeopardy.
    Une personne donne une réponse, et les autres posent des questions logiques.
    La première personne à poser une bonne question gagne un point et donne la réponse suivante.
    \begin{columns}
      \column{0.36\textwidth}
        \begin{description}
          \item[] \textbf{Modèle:}
          \item[E1:] De la musique classique.
          \item[E2:] Qu'est-ce que tu écoutes?
        \end{description}
      \column{0.64\textwidth}
        \begin{center}
          Utilise ces verbes pour les questions.
        \end{center}
        \begin{columns}
          \column{0.32\textwidth}
            \begin{itemize}
              \item admirer
              \item aimer
              \item écouter
              \item étudier
            \end{itemize}
          \column{0.32\textwidth}
            \begin{itemize}
              \item manger
              \item parler
              \item regarder
              \item téléphoner
            \end{itemize}
        \end{columns}
    \end{columns}
  \end{frame}

  \begin{frame}{}
    \begin{center}
      \Large Questions?
    \end{center}
  \end{frame}
\end{document}
