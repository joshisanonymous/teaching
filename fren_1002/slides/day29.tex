%%%%%%%%%%%%%%%%%%%%%%%%%%%%%%%%%%%%%
%                                   %
% Compile with XeLaTeX and biber    %
%                                   %
% Questions or comments:            %
%                                   %
% joshua dot mcneill at uga dot edu %
%                                   %
%%%%%%%%%%%%%%%%%%%%%%%%%%%%%%%%%%%%%

\documentclass{beamer}
  % Read in standard preamble (cosmetic stuff)
  %%%%%%%%%%%%%%%%%%%%%%%%%%%%%%%%%%%%%%%%%%%%%%%%%%%%%%%%%%%%%%%%
% This is a standard preamble used in for all slide documents. %
% It basically contains cosmetic settings.                     %
%                                                              %
% Joshua McNeill                                               %
% joshua dot mcneill at uga dot edu                            %
%%%%%%%%%%%%%%%%%%%%%%%%%%%%%%%%%%%%%%%%%%%%%%%%%%%%%%%%%%%%%%%%

% Beamer settings
% \usetheme{Berkeley}
\usetheme{CambridgeUS}
% \usecolortheme{dove}
% \usecolortheme{rose}
\usecolortheme{seagull}
\usefonttheme{professionalfonts}
\usefonttheme{serif}
\setbeamertemplate{bibliography item}{}

% Packages and settings
\usepackage{fontspec}
  \setmainfont{Charis SIL}
\usepackage{hyperref}
  \hypersetup{colorlinks=true,
              allcolors=blue}
\usepackage{graphicx}
  \graphicspath{{../../figures/}}
\usepackage{soul}
  \setstcolor{red}
\usepackage[normalem]{ulem}
\usepackage{enumerate}
\usepackage{tikz}
  \usetikzlibrary{trees}

% Document information
\author{M. McNeill}
\title[FREN1001]{Français 1001}
\institute{\url{joshua.mcneill@uga.edu}}
\date{}

%% Custom commands
% Lexical items
\newcommand{\lexi}[1]{\textit{#1}}
% Gloss
\newcommand{\gloss}[1]{`#1'}
\newcommand{\tinygloss}[1]{{\tiny`#1'}}
% Orthographic representations
\newcommand{\orth}[1]{$\langle$#1$\rangle$}
% Utterances (pragmatics)
\newcommand{\uttr}[1]{`#1'}
% Sentences (pragmatics)
\newcommand{\sent}[1]{\textit{#1}}
% Fixed length underlines
\newcommand{\funderline}[2][4cm]{
  \underline{\makebox[\ifdim\width>#1\width\else#1\fi]{#2}}
}
% Base dir for definitions
\newcommand{\defs}{../definitions}
\newcommand{\activity}[1]{
  \input{./activities/#1.tex}
}


  % Packages and settings

  % Document information
  \subtitle[Exclamations, \lexi{savoir} et \lexi{connaître}]{Les exclamations et les verbes \lexi{savoir} et \lexi{connaître}}

\begin{document}
  % Read in the standard intro slides (title page and table of contents)
  \begin{frame}
    \titlepage
    \tiny{Office: % Basically a variable for office hours location
Zoom (ID 978 2791 8221)
\\
          Office hours: % Basically a variable for office hours
 mercredi 10h15--13h15
}
  \end{frame}

  \begin{frame}{/w/ vs /ɥ/}
    \begin{center}
      Quel son entendez-vous?

      \begin{tabular}{l | c c}
        Mot                          & /w/             & /ɥ/ \\
        \hline
        \uncover<2->{ess\alert{u}yé} &                 & \uncover<2->{X} \\
        \uncover<3->{v\alert{oy}ez}  & \uncover<3->{X} & \\
        \uncover<4->{s\alert{oi}s}   & \uncover<4->{X} & \\
        \uncover<5->{pl\alert{u}ie}  &                 & \uncover<5->{X} \\
        \uncover<6->{s\alert{u}is}   &                 & \uncover<6->{X} \\
      \end{tabular}
    \end{center}
  \end{frame}

  \begin{frame}{!!!}
    Qu'est-ce qu'on dit?
    \begin{columns}
      \column{0.5\textwidth}
        \only<2>{
          \begin{itemize}
            \item Zut!
            \item Imbécile!
            \item Quel idiot!
            \item Incroyable!
          \end{itemize}
        }
        \only<4>{
          \begin{itemize}
            \item Ça va s'arranger.
            \item Ne t'en fais pas!
          \end{itemize}
        }
        \only<6>{
          \begin{itemize}
            \item Ne crie pas!
            \item Ne sois pas furieux!
            \item Sois plus calme!
            \item Ce n'est pas grave.
          \end{itemize}
        }
      \column{0.5\textwidth}
        \begin{minipage}[c][0.8\textheight]{\linewidth}
          \begin{center}
            \only<1-2>{
              \includegraphics[scale=0.4]{vase.jpg}
            }
            \only<3-4>{
              \includegraphics[scale=0.14]{séparation.jpg}
            }
            \only<5-6>{
              \includegraphics[scale=0.14]{colère.jpg}
            }
          \end{center}
        \end{minipage}
    \end{columns}
  \end{frame}

  \begin{frame}{}
    \begin{center}
      \Large Quiz
    \end{center}
  \end{frame}

  \begin{frame}{Sais-tu conjuguer connaître?}
    savoir \alert{faire} (v.), connaître \alert{une chose} (n.)
    \begin{enumerate}
      \item Je \underline{\uncover<2->{connais}} (connaître) bien sa famille.
      \item Ma mère \underline{\uncover<3->{sait}} (savoir) se détendre.
      \item Ils \underline{\uncover<5->{savent}} (savoir) se disputer.
      \item Nous \underline{\uncover<6->{connaissons}} (connaître) Athens.
      \item Vous \underline{\uncover<7->{connaissez}} ce poème?
    \end{enumerate}
  \end{frame}

  \begin{frame}{Explique-nous!}
    \begin{columns}
      \column{0.5\textwidth}
        \begin{enumerate}
          \item \alert{En phrases complètes}, ecris deux choses que tu sais faire et deux choses que tu connais sur un papier (que tu vas rendre à M. McNeill).
          \item<2-> Explique à un/e partenaire comment faire l'une des choses que tu as écrites.
          \item<3-> Explique à la classe.
        \end{enumerate}
      \column{0.5\textwidth}
        \only<2->{
          \begin{description}
            \item[] \textbf{Modèle:}
            \item[E1:] Moi, je sais surfer. D'abord, il faut entrer dans l'eau avec une planche de surf. Ensuite, il est important de trouver une grande vague \gloss{wave}...
          \end{description}
        }
    \end{columns}
  \end{frame}

  \begin{frame}{}
    \begin{center}
      \Large Questions?
    \end{center}
  \end{frame}
\end{document}
