%%%%%%%%%%%%%%%%%%%%%%%%%%%%%%%%%%%%%
%                                   %
% Compile with XeLaTeX and biber    %
%                                   %
% Questions or comments:            %
%                                   %
% joshua dot mcneill at uga dot edu %
%                                   %
%%%%%%%%%%%%%%%%%%%%%%%%%%%%%%%%%%%%%

\documentclass{beamer}
  % Read in standard preamble (cosmetic stuff)
  %%%%%%%%%%%%%%%%%%%%%%%%%%%%%%%%%%%%%%%%%%%%%%%%%%%%%%%%%%%%%%%%
% This is a standard preamble used in for all slide documents. %
% It basically contains cosmetic settings.                     %
%                                                              %
% Joshua McNeill                                               %
% joshua dot mcneill at uga dot edu                            %
%%%%%%%%%%%%%%%%%%%%%%%%%%%%%%%%%%%%%%%%%%%%%%%%%%%%%%%%%%%%%%%%

% Beamer settings
% \usetheme{Berkeley}
\usetheme{CambridgeUS}
% \usecolortheme{dove}
% \usecolortheme{rose}
\usecolortheme{seagull}
\usefonttheme{professionalfonts}
\usefonttheme{serif}
\setbeamertemplate{bibliography item}{}

% Packages and settings
\usepackage{fontspec}
  \setmainfont{Charis SIL}
\usepackage{hyperref}
  \hypersetup{colorlinks=true,
              allcolors=blue}
\usepackage{graphicx}
  \graphicspath{{../../figures/}}
\usepackage[normalem]{ulem}
\usepackage{enumerate}

% Document information
\author{M. McNeill}
\title[FREN2001]{Français 2001}
\institute{\url{joshua.mcneill@uga.edu}}
\date{}

%% Custom commands
% Lexical items
\newcommand{\lexi}[1]{\textit{#1}}
% Gloss
\newcommand{\gloss}[1]{`#1'}
\newcommand{\tinygloss}[1]{{\tiny`#1'}}
% Orthographic representations
\newcommand{\orth}[1]{$\langle$#1$\rangle$}
% Utterances (pragmatics)
\newcommand{\uttr}[1]{`#1'}
% Sentences (pragmatics)
\newcommand{\sent}[1]{\textit{#1}}
% Base dir for definitions
\newcommand{\defs}{../definitions}


  % Packages and settings
  % \usepackage{enumitem}

  % Document information
  \subtitle[Environnement, verbes et subjonctif]{L'environnement, de nouveaux verbes et... plus de subjonctif!}

\begin{document}
  % Read in the standard intro slides (title page and table of contents)
  \begin{frame}
    \titlepage
    \tiny{Office: % Basically a variable for office hours location
Gilbert 121\\
          Office hours: % Basically a variable for office hours
 lundi, mercredi, vendredi 10:10--11:10
}
  \end{frame}

  \begin{frame}{Annonces}
    \begin{itemize}
      \item Le devoir 5 est disponible sur eLC, et il est à rendre le 14 (vendredi).
      \item[] \tinygloss{Homework 5 is available on eLC, and it's due the 14th (Friday).}
    \end{itemize}
  \end{frame}

  \begin{frame}{Des verbes irreguliers}
    réduire combattre éteindre
    \begin{enumerate}
      \item Nous \underline{\uncover<2->{combattons}} (combattre) des infections avec des antibiotiques.
      \item Du sirop \underline{\uncover<3->{réduit}} (réduire) les toux.
      \item Ils \underline{\uncover<4->{éteignent}} (éteindre) leurs ordinateurs tous les jours.
      \item Si vous recyclez, vous \underline{\uncover<5->{réduisez}} (réduire) le gaspillage.
      \item J'\underline{\uncover<6->{éteins}} (éteindre) les flammes parce qu'on n'a pas allumé le feu.
    \end{enumerate}
  \end{frame}

  \begin{frame}{Quel éco-geste?}
    \scriptsize
    Avec un/e partenaire, dites à tour de rôle \gloss{taking turns} quelle est la catégorie de l'éco-geste.
    \begin{description}
      \item[] \textbf{Modèle:}
      \item[E1:] Je consomme certains produits après la Date Limite d'Utilisation.
      \item[E2:] \alert{Tu réduis} la quantité des déchets ménagers.
    \end{description}
    \begin{columns}[t]
      \column{0.6\textwidth}
        Les éco-gestes:
        \begin{enumerate}
          \item Je prends des douches rapides: moins de cinq minutes.
          \item J'ai acheté une voiture électrique.
          \item Je débranche mon ordinateur quand je ne travaille pas.
          \item Je n'achète jamais d'eau en bouteille.
          \item Je vais à la fac à vélo ou à pied.
          \item Je fais attention d'éteindre la lumière quand je quitte une pièce.
        \end{enumerate}
      \column{0.4\textwidth}
        Les catégories:
        \begin{itemize}
          \item réduire la consommation d'énergie
          \item réduire la quantité des déchets ménagers
          \item privilégier les transports propres
        \end{itemize}
    \end{columns}
  \end{frame}

  \begin{frame}{}
    \begin{center}
      \Large Quiz
    \end{center}
  \end{frame}

  \begin{frame}{Encore plus de subjonctif!}
    On utilise aussi le subjonctif avec des expressions de volonté.
    \begin{enumerate}
      \item Elles souhaitent qu'il (part/\alert<2->{parte}).
      \item Il exige qu'on (\alert<3->{attende}/attend) jusqu'à demain.
      \item Le maire veux qu'on (\alert<4->{fasse}/fait) du recyclage.
      \item Je désire que tu (sors/\alert<5->{sortes}) avec tes amis.
      \item Nous demandons que vous (fumez/\alert<6->{fumiez}) moins.
    \end{enumerate}
  \end{frame}

  \begin{frame}{Harmonie ou conflit?}
    \small
    Sur un papier, pour chaque numéro, écris si toi et tes parents partagez les mêmes souhaits, désirs, etc.
    \begin{description}
      \item[] \textbf{Modèle:} \emph{ta future profession : souhaiter}
      \item Je souhaite \alert{être} musicienne. Mes parents souhaitent \alert{que je sois} médecin.
      \item[OU] Mes parents souhaitent \alert{que je sois} architecte et moi aussi \gloss{me too}.
    \end{description}
    \begin{columns}
      \column{0.6\textwidth}
        \begin{enumerate}
          \item tes études : souhaiter
          \item tes projets pour l'été prochain : préférer
          \item ta prochaine voiture : désirer
          \item ton lieu de résidence éventuel : désirer
          \item tes futurs enfants : souhaiter
        \end{enumerate}
      \column{0.4\textwidth}
        \uncover<2->{
          Avec un/e partenaire, partagez et discutez de vos réponses.
        }
    \end{columns}
  \end{frame}

  \begin{frame}{}
    \begin{center}
      \Large Questions?
    \end{center}
  \end{frame}
\end{document}
