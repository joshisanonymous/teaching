%%%%%%%%%%%%%%%%%%%%%%%%%%%%%%%%%%%%%
%                                   %
% Compile with XeLaTeX and biber    %
%                                   %
% Questions or comments:            %
%                                   %
% joshua dot mcneill at uga dot edu %
%                                   %
%%%%%%%%%%%%%%%%%%%%%%%%%%%%%%%%%%%%%

\documentclass{beamer}
  % Read in standard preamble (cosmetic stuff)
  %%%%%%%%%%%%%%%%%%%%%%%%%%%%%%%%%%%%%%%%%%%%%%%%%%%%%%%%%%%%%%%%
% This is a standard preamble used in for all slide documents. %
% It basically contains cosmetic settings.                     %
%                                                              %
% Joshua McNeill                                               %
% joshua dot mcneill at uga dot edu                            %
%%%%%%%%%%%%%%%%%%%%%%%%%%%%%%%%%%%%%%%%%%%%%%%%%%%%%%%%%%%%%%%%

% Beamer settings
% \usetheme{Berkeley}
\usetheme{CambridgeUS}
% \usecolortheme{dove}
% \usecolortheme{rose}
\usecolortheme{seagull}
\usefonttheme{professionalfonts}
\usefonttheme{serif}
\setbeamertemplate{bibliography item}{}

% Packages and settings
\usepackage{fontspec}
  \setmainfont{Charis SIL}
\usepackage{hyperref}
  \hypersetup{colorlinks=true,
              allcolors=blue}
\usepackage{graphicx}
  \graphicspath{{../../figures/}}
\usepackage{soul}
  \setstcolor{red}
\usepackage[normalem]{ulem}
\usepackage{enumerate}
\usepackage{tikz}
  \usetikzlibrary{trees}

% Document information
\author{M. McNeill}
\title[FREN1001]{Français 1001}
\institute{\url{joshua.mcneill@uga.edu}}
\date{}

%% Custom commands
% Lexical items
\newcommand{\lexi}[1]{\textit{#1}}
% Gloss
\newcommand{\gloss}[1]{`#1'}
\newcommand{\tinygloss}[1]{{\tiny`#1'}}
% Orthographic representations
\newcommand{\orth}[1]{$\langle$#1$\rangle$}
% Utterances (pragmatics)
\newcommand{\uttr}[1]{`#1'}
% Sentences (pragmatics)
\newcommand{\sent}[1]{\textit{#1}}
% Fixed length underlines
\newcommand{\funderline}[2][4cm]{
  \underline{\makebox[\ifdim\width>#1\width\else#1\fi]{#2}}
}
% Base dir for definitions
\newcommand{\defs}{../definitions}
\newcommand{\activity}[1]{
  \input{./activities/#1.tex}
}


  % Packages and settings
  % \usepackage{enumitem}

  % Document information
  \subtitle[Maladies, remèdes et subjonctif]{Les maladies, les remèdes et plus de subjonctif}

\begin{document}
  % Read in the standard intro slides (title page and table of contents)
  \begin{frame}
    \titlepage
    \tiny{Office: % Basically a variable for office hours location
Zoom (ID 978 2791 8221)
\\
          Office hours: % Basically a variable for office hours
 mercredi 10h15--13h15
}
  \end{frame}

  \begin{frame}{La dernière maladie}
    Avec un/e partenaire, décris les symptômes de la dernière maladie que tu as eue.
    Ton/ta partenaire doit deviner la maladie (par ex., grippe, rhume, infection) et suggérer une remède.
    \begin{description}
      \item[] \textbf{Modèle:}
      \item[E1:] J'étais fatigué/e, j'avais mal à la tête et je toussais beaucoup.
      \item[E2:] Est-ce que tu avais un rhume?
      \item[E1:] Oui, c'était ça!
      \item[E2:] Pour un rhume, il faut que tu prennes ...
    \end{description}
  \end{frame}

  \begin{frame}{}
    \begin{center}
      \Large Quiz
    \end{center}
  \end{frame}

  \begin{frame}{Le subjonctif irregulier}
    Quelquefois, le subjonctif est irregulier!
    \begin{enumerate}
      \item Il faut \alert{que} vous \underline{\uncover<2->{fassiez}} (faire) votre travail!
      \item Il est important \alert{que} tu \underline{\uncover<3->{saches}} (savoir) faire ton travail.
      \item Il vaut mieux \alert{qu'}elle \underline{\uncover<4->{puisse}} (pouvoir) dormir après son travail.
      \item Il est essentiel \alert{qu'}il ne \underline{\uncover<5->{pleuve}} pas (pleuvoir) aujourd'hui (parce qu'on a enfin terminé le travail).
      \item Il faut \alert{que} je \underline{\uncover<6->{sois}} (être) en bonne santé.
    \end{enumerate}
  \end{frame}

  \begin{frame}{Des maladies fréquentes}
    \begin{columns}
      \column{0.6\textwidth}
      \scriptsize
        En groupes de 3 ou 4, répondez aux questions suivantes pour l'une des maladies à droite (assignée par M. McNeill).
        Une personne doit écrire les réponses pour que vous vous les rappeliez.
        \begin{itemize}
          \item Est-ce qu'il y a un autre nom pour cette maladie?
          \item Quels sont les symptômes?
          \item Quelles sont les remèdes?
          \begin{itemize}
            \scriptsize
            \item (\emph{Il faut qu'on ...})
          \end{itemize}
        \end{itemize}
        \uncover<2->{
          Présentez vos maladies à la classe.
          Pendant que vous écoutez, \alert{écrivez} une réponse à la question suivante sur un papier (en français, dans une phrase complète).
          \begin{itemize}
            \item Quelle est la pire maladie, à ton avis, et pourquoi?
          \end{itemize}
        }
      \column{0.4\textwidth}
        \begin{enumerate}
          \item la grippe
          \item la bronchite
          \item la sinusite
          \item la gastro-entérite
          \item la conjonctivite
          \item la varicelle
        \end{enumerate}
    \end{columns}
  \end{frame}

  \begin{frame}{}
    \begin{center}
      \Large Questions?
    \end{center}
  \end{frame}
\end{document}
