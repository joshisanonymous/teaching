%%%%%%%%%%%%%%%%%%%%%%%%%%%%%%%%%%%%%
%                                   %
% Compile with XeLaTeX and biber    %
%                                   %
% Questions or comments:            %
%                                   %
% joshua dot mcneill at uga dot edu %
%                                   %
%%%%%%%%%%%%%%%%%%%%%%%%%%%%%%%%%%%%%

\documentclass{beamer}
  % Read in standard preamble (cosmetic stuff)
  %%%%%%%%%%%%%%%%%%%%%%%%%%%%%%%%%%%%%%%%%%%%%%%%%%%%%%%%%%%%%%%%
% This is a standard preamble used in for all slide documents. %
% It basically contains cosmetic settings.                     %
%                                                              %
% Joshua McNeill                                               %
% joshua dot mcneill at uga dot edu                            %
%%%%%%%%%%%%%%%%%%%%%%%%%%%%%%%%%%%%%%%%%%%%%%%%%%%%%%%%%%%%%%%%

% Beamer settings
% \usetheme{Berkeley}
\usetheme{CambridgeUS}
% \usecolortheme{dove}
% \usecolortheme{rose}
\usecolortheme{seagull}
\usefonttheme{professionalfonts}
\usefonttheme{serif}
\setbeamertemplate{bibliography item}{}

% Packages and settings
\usepackage{fontspec}
  \setmainfont{Charis SIL}
\usepackage{hyperref}
  \hypersetup{colorlinks=true,
              allcolors=blue}
\usepackage{graphicx}
  \graphicspath{{../../figures/}}
\usepackage[normalem]{ulem}
\usepackage{enumerate}

% Document information
\author{M. McNeill}
\title[FREN2001]{Français 2001}
\institute{\url{joshua.mcneill@uga.edu}}
\date{}

%% Custom commands
% Lexical items
\newcommand{\lexi}[1]{\textit{#1}}
% Gloss
\newcommand{\gloss}[1]{`#1'}
\newcommand{\tinygloss}[1]{{\tiny`#1'}}
% Orthographic representations
\newcommand{\orth}[1]{$\langle$#1$\rangle$}
% Utterances (pragmatics)
\newcommand{\uttr}[1]{`#1'}
% Sentences (pragmatics)
\newcommand{\sent}[1]{\textit{#1}}
% Base dir for definitions
\newcommand{\defs}{../definitions}


  % Packages and settings

  % Document information
  \subtitle[Révision, examen 2]{Révision de l'examen 2 (Ch 8 \& 9)}

\begin{document}
  % Read in the standard intro slides (title page and table of contents)
  \begin{frame}
    \titlepage
    \tiny{Office: % Basically a variable for office hours location
Gilbert 121\\
          Office hours: % Basically a variable for office hours
 lundi, mercredi, vendredi 10:10--11:10
}
  \end{frame}

  \begin{frame}{Le jeu}
    \alert{Le prix:} 2\% sur l'examen pour le/la gagnant/e, 1\% pour la deuxième place \\
    \alert{Les règles:}
    \begin{enumerate}
      \item Pose une question concernante l'examen
      \begin{itemize}
        \item Si personnne ne peut la répondre $\to$ +1
        \item Si quelqu'un peut la répondre $\to$ +0
      \end{itemize}
      \item Si tu poses une question déjà posée $\to$ +0
      \item Si tu poses une question qui ne concerne pas l'examen $\to$ +0
    \end{enumerate}
    \alert{Indice:} Tu peux poser la question en anglais, mais si tu poses la question \emph{en français}, seulement ceux qui la comprendre peut la répondre.
  \end{frame}

  \begin{frame}{Futur simple}
    \begin{enumerate}
      \item Je vais acheter un bateau demain parce que je \underline{\uncover<2->{traverserai}} (traverser) le monde un jour.
      \item Ne t'en fais pas! La voiture est tombée en panne \gloss{broke down}, mais nous en \underline{\uncover<3->{achèterons}} (acheter) une autre.
      \item Popeye et Jack \underline{\uncover<3->{feront}} (faire) de la voile jusqu'à Hawaï.
      \item Vous \underline{\uncover<4->{sortirez}} (sortir) avec vos amis tous les soirs à Honolulu.
    \end{enumerate}
  \end{frame}

  \begin{frame}{Prépositions}
    \begin{enumerate}
      \item 
  \end{frame}

  \begin{frame}{Pronoms relatifs}
    
  \end{frame}

  \begin{frame}{Nouvelles conjugaisons}
    
  \end{frame}

  \begin{frame}{Expressions de nécessité}
    
  \end{frame}
\end{document}
