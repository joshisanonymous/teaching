%%%%%%%%%%%%%%%%%%%%%%%%%%%%%%%%%%%%%
%                                   %
% Compile with XeLaTeX and biber    %
%                                   %
% Questions or comments:            %
%                                   %
% joshua dot mcneill at uga dot edu %
%                                   %
%%%%%%%%%%%%%%%%%%%%%%%%%%%%%%%%%%%%%

\documentclass{beamer}
  % Read in standard preamble (cosmetic stuff)
  %%%%%%%%%%%%%%%%%%%%%%%%%%%%%%%%%%%%%%%%%%%%%%%%%%%%%%%%%%%%%%%%
% This is a standard preamble used in for all slide documents. %
% It basically contains cosmetic settings.                     %
%                                                              %
% Joshua McNeill                                               %
% joshua dot mcneill at uga dot edu                            %
%%%%%%%%%%%%%%%%%%%%%%%%%%%%%%%%%%%%%%%%%%%%%%%%%%%%%%%%%%%%%%%%

% Beamer settings
% \usetheme{Berkeley}
\usetheme{CambridgeUS}
% \usecolortheme{dove}
% \usecolortheme{rose}
\usecolortheme{seagull}
\usefonttheme{professionalfonts}
\usefonttheme{serif}
\setbeamertemplate{bibliography item}{}

% Packages and settings
\usepackage{fontspec}
  \setmainfont{Charis SIL}
\usepackage{hyperref}
  \hypersetup{colorlinks=true,
              allcolors=blue}
\usepackage{graphicx}
  \graphicspath{{../../figures/}}
\usepackage[normalem]{ulem}
\usepackage{enumerate}

% Document information
\author{M. McNeill}
\title[FREN2001]{Français 2001}
\institute{\url{joshua.mcneill@uga.edu}}
\date{}

%% Custom commands
% Lexical items
\newcommand{\lexi}[1]{\textit{#1}}
% Gloss
\newcommand{\gloss}[1]{`#1'}
\newcommand{\tinygloss}[1]{{\tiny`#1'}}
% Orthographic representations
\newcommand{\orth}[1]{$\langle$#1$\rangle$}
% Utterances (pragmatics)
\newcommand{\uttr}[1]{`#1'}
% Sentences (pragmatics)
\newcommand{\sent}[1]{\textit{#1}}
% Base dir for definitions
\newcommand{\defs}{../definitions}


  % Packages and settings

  % Document information
  \subtitle[Technologie et conditionnel]{La technologie et le conditionnel}

\begin{document}
  % Read in the standard intro slides (title page and table of contents)
  \begin{frame}
    \titlepage
    \tiny{Office: % Basically a variable for office hours location
Gilbert 121\\
          Office hours: % Basically a variable for office hours
 lundi, mercredi, vendredi 10:10--11:10
}
  \end{frame}

  \begin{frame}{Projet ou rêve?}
    \small
    Est-ce que c'est un projet ou un rêve?
    \begin{center}
      \begin{tabular}{r l | c c}
                        &                                                   & rêve            & projet \\
        \hline
        \textbf{Modèle} & J'irais en Suisse.                                & X               & \\
        1.              & Claire et moi irions au cinéma à Montréal.        & \uncover<2->{X} & \\
        2.              & Paul achètera un nouvel ordinateur portable.      &                 & \uncover<3->{X} \\
        3.              & Nous travaillerons pour Google.                   &                 & \uncover<4->{X} \\
        4.              & J'achèterais un écran haute définition.           & \uncover<5->{X} & \\
        5.              & J'utiliserai un tableur pour mon compte bancaire. &                 & \uncover<6->{X} \\
        6.              & Vous assisteriez aux cours par appel vidéo.       & \uncover<7->{X} & \\
      \end{tabular}
    \end{center}
  \end{frame}

  \begin{frame}{Conjuguer le conditionnel}
    Donnons les conjugaisons pour les verbes suivants.
    \begin{enumerate}
      \item Je \underline{\uncover<2->{naviguerais}} (naviguer) sur Internet avec mon smartphone.
      \item On \underline{\uncover<3->{composerait}} (composer) les textos avec la voix.
      \item Nous \underline{\uncover<4->{stockerions}} (stocker) tous les fichiers avec des mots de passe.
      \item Ils se \underline{\uncover<5->{connecteraient}} (connecter) à Internet si le WiFi marchait bien.
      \item Vous \underline{\uncover<6->{téléchargeriez}} (télécharger) les épisodes de la série policière.
      \item Tu \underline{\uncover<7->{recevrais}} (recevoir) le montant par e-mail.
    \end{enumerate}
  \end{frame}

  \begin{frame}{}
    \begin{center}
      \Large Quiz
    \end{center}
  \end{frame}

  \begin{frame}{Avec plus d'argent}
    Avec un/e partenaire, imagine que tu as plus d'argent.
    Est-ce que tu est d'accord avec les phrases suivantes?
    Dis à ton/ta partenaire <<oui>> ou <<non>> et conjugue la phrase.
    \begin{description}
      \item[] \textbf{Modèle:} \emph{acheter une nouvelle voiture.}
      \item[E1:] Oui, j'achèterais une nouvelle voiture (si j'avais plus d'argent).
      \item[E2:] Non, je n'achèterais pas de nouvelle voiture.
    \end{description}
    \begin{enumerate}
      \item voyager tout le temps
      \item partager l'argent avec la famille
      \item prêter de l'argent aux amis
      \item dîner dans les meilleurs restaurants
      \item donner de l'argent aux personnes en difficulté
      \item acheter un château en France
      \item ne pas travailler
    \end{enumerate}
  \end{frame}

  \begin{frame}{Trouver une personne}
    \scriptsize
    Pour chaque poste ci-dessous, circule dans la salle pour trouver une personne qui aimerait l'avoir.
    Demande ce qu'il ou elle ferait dans ce poste, et note son nom et sa réponse.
    N'écris pas un nom pour plus d'un poste!
    \begin{description}
      \item[] \textbf{Modèle:} \emph{le/la professeur/e de notre cours de français}
      \item[E1:] Est-ce que tu aimerais être le/la professeur/e de notre cours de français?
      \item[E2:] Oui, j'aimerais être le/la professeur/e de notre cours de français.
      \item[E1:] Qu'est-ce que tu ferais?
      \item[E2:] Je donnerais moins de devoirs!
      \item[] (E1 écrit son nom et sa réponse.)
    \end{description}
    \begin{columns}[t]
      \column{0.5\textwidth}
        \begin{enumerate}
          \item le/la professeur/e de notre cours de français
          \item le/la président/e de notre université
          \item un/e acteur/actrice célèbre
          \item un/e scientifique célèbre
        \end{enumerate}
      \column{0.5\textwidth}
        \begin{enumerate}
          \setcounter{enumi}{4}
          \item le/la directeur/directrice d'une grande société
          \item le/la maire d'Athens
          \item le/la président/e des États-Unis
          \item un/e athlète célèbre
        \end{enumerate}
    \end{columns}
  \end{frame}

  \begin{frame}{}
    \begin{center}
      \Large Questions?
    \end{center}
  \end{frame}
\end{document}
